% !TEX root = lecture.tex
\chapter{De Rham复形}

这一章讨论De Rham复形. 为此,先引入一些记号.
设$\Omega\subset\mathbb R^d$是$d$维连通区域,取$\Omega$内的任一点$a$. 
记$\mathbb M$, $\mathbb S$和$\mathbb K$分别为$d$阶矩阵空间,对称矩阵空间和反对称矩阵空间.
对于矩阵$\tau$, 定义对称部分$\sym\tau=(\tau+\tau^{\intercal})/2$和反对称部分$\skw\tau=(\tau-\tau^{\intercal})/2$. 显然有$\tau=\sym\tau+\skw\tau$.
将两维向量$v=(v_1, v_2)^{\intercal}$按顺时针旋转$90^{\circ}$得到的向量记为$v^{\perp}$,即
$$
v^{\perp}=\begin{pmatrix}
v_2 \\
-v_1
\end{pmatrix}=\begin{pmatrix}
0 & 1 \\
-1 & 0
\end{pmatrix}\begin{pmatrix}
v_1 \\
v_2
\end{pmatrix}.
$$

对于函数$v$, 定义梯度算子$\nabla v=\grad v=\big(\frac{\partial v}{\partial x_1}, \ldots, \frac{\partial v}{\partial x_d}\big)^{\intercal}$.
对于向量函数$v=(v_1, \ldots, v_d)^{\intercal}$, 分别按行和按列定义$\grad v$和$\nabla v$,即
$$
\grad v=\left(\frac{\partial v_i}{\partial x_j}\right)_{d\times d}=\begin{pmatrix}
\frac{\partial v_1}{\partial x_1} & \cdots & \frac{\partial v_1}{\partial x_d} \\
\vdots &  & \vdots \\
\frac{\partial v_d}{\partial x_1} & \cdots & \frac{\partial v_d}{\partial x_d} 
\end{pmatrix},\quad \nabla v=\left(\frac{\partial v_j}{\partial x_i}\right)_{d\times d}=\begin{pmatrix}
\frac{\partial v_1}{\partial x_1} & \cdots & \frac{\partial v_d}{\partial x_1} \\
\vdots &  & \vdots \\
\frac{\partial v_1}{\partial x_d} & \cdots & \frac{\partial v_d}{\partial x_d} 
\end{pmatrix}.
$$
显然有$\grad v=(\nabla v)^{\intercal}$. 定义散度
$$
\div v=\nabla \cdot v=\frac{\partial v_1}{\partial x_1}+\cdots+\frac{\partial v_d}{\partial x_d}.
$$
易知$\div v=\tr(\grad v)=\tr(\nabla v)$. 对于二元函数$v$,定义旋度$\curl v=(\nabla v)^{\intercal}=\left(\frac{\partial v}{\partial x_2}, -\frac{\partial v}{\partial x_1}\right)^{\intercal}$. 对于二元向量函数$v=(v_1, v_2)^{\intercal}$,定义旋度
$$
\rot v=\frac{\partial v_2}{\partial x_1}-\frac{\partial v_1}{\partial x_2}=\div v^{\perp},\quad \curl v=\begin{pmatrix}
\frac{\partial v_1}{\partial x_2} & -\frac{\partial v_1}{\partial x_1} \\
\frac{\partial v_2}{\partial x_2} & -\frac{\partial v_2}{\partial x_1}
\end{pmatrix}.
$$
对于三元向量函数$v=(v_1, v_2, v_3)^{\intercal}$,定义旋度
$$
\curl v=\nabla\times v=\begin{pmatrix}
\frac{\partial v_3}{\partial x_2} - \frac{\partial v_2}{\partial x_3} \\
\frac{\partial v_1}{\partial x_3} - \frac{\partial v_3}{\partial x_1} \\
\frac{\partial v_2}{\partial x_1} - \frac{\partial v_1}{\partial x_2}
\end{pmatrix}.
$$
对于一般的$d$元向量函数$v=(v_1, \ldots, v_d)^{\intercal}$,定义旋度$\curl v=\skw\grad v$.


定义Sobolev空间
\begin{align*}
H^1(\Omega) &=\{v\in L^2(\Omega): \grad v\in L^2(\Omega;\mathbb R^d)\}, \\
H(\div, \Omega) &=\{v\in L^2(\Omega;\mathbb R^d): \div v\in L^2(\Omega)\}, \\
H(\rot, \Omega) &=\{v\in L^2(\Omega;\mathbb R^2): \rot v\in L^2(\Omega)\}, \\
H(\curl, \Omega) &=\{v\in L^2(\Omega;\mathbb R^d): \curl v\in L^2(\Omega;\mathbb K)\}, \quad d\geq3.
\end{align*}


\section{三维de Rham复形}

设$\Omega$是三维连通区域,取$\Omega$内的任一点$a$. 给出三维de Rham复形
\begin{equation}\label{eq:deRhamcomplex3d}
%\resizebox{.9\hsize}{!}{$
\mathbb R\xrightarrow{\hookrightarrow} H^1(\Omega)\xrightarrow{\grad} H(\curl, \Omega)\xrightarrow{\curl} H(\div, \Omega) \xrightarrow{\div} L^2(\Omega)\xrightarrow{}0.
\end{equation}
之所以称为复形,是因为
\begin{align*}    
\grad H^1(\Omega)&\subseteq H(\curl, \Omega)\cap\ker(\curl), \\
\curl H(\curl, \Omega)&\subseteq H(\div, \Omega)\cap\ker(\div), \\
\div H(\div, \Omega) &\subseteq L^2(\Omega),
\end{align*}
其中核空间$\mathbb B\cap\ker(A)=\{v\in \mathbb B : Av=0\}$.

当上式中的$\subseteq$都改成$=$后,则称复形是正合的. 为证明三维de Rham复形在可缩区域上是正合的,引入Poincar\'e算子
\cite{GopalakrishnanDemkowicz2004,Hiptmair1999,ChristiansenHuSande2020}
\begin{align}
\label{poincareopP1}
\mathcal{P}_1u &= \int_0^1u(a+t(x-a))\cdot (x-a)\dd t, \quad\quad u\in C^{\infty}(\Omega;\mathbb R^3), \\
\label{poincareopP2}
\mathcal{P}_2v &= \int_0^1tv(a+t(x-a))\times (x-a)\dd t, \quad \, v\in C^{\infty}(\Omega;\mathbb R^3), \\
\label{poincareopP3}
\mathcal{P}_3w &= (x-a)\int_0^1t^2w(a+t(x-a))\dd t, \quad\;\; w\in C^{\infty}(\Omega).
\end{align}


显然有复形
\begin{equation}\label{eq:deRhamcomplexSmooth3d}
\mathbb R\xleftarrow{\mathcal{P}_0} C^{\infty}(\Omega) \xleftarrow{\mathcal{P}_1} C^{\infty}(\Omega;\mathbb R^3)\xleftarrow{\mathcal{P}_2} C^{\infty}(\Omega;\mathbb R^3) \xleftarrow{\mathcal{P}_3} C^{\infty}(\Omega)\xleftarrow{}0,
\end{equation}
其中$\mathcal{P}_0: C^{\infty}(\Omega)\to\mathbb R$定义为$\mathcal{P}_0w=w-\mathcal{P}_1\grad w$. 直接计算可得
\begin{align*}
\mathcal{P}_0w=w(x)-\int_0^1(\grad w)(a+t(x-a))\cdot (x-a)\dd t=w(x)-\int_0^1\frac{\dd}{\dd t}w(a+t(x-a))\dd t=w(a).
\end{align*}

\begin{lemma}
成立
\begin{align}
\label{eq:poincareidentity3}
\div\mathcal{P}_3w&=w \quad\forall~w\in C^{\infty}(\Omega), \\
\label{eq:poincareidentity2}
\curl\mathcal{P}_2v+\mathcal{P}_3\div v&=v \quad\,\forall~v\in C^{\infty}(\Omega;\mathbb R^3), \\
\label{eq:poincareidentity1}
\grad\mathcal{P}_1u+\mathcal{P}_2\curl u&=u \quad\,\forall~u\in C^{\infty}(\Omega;\mathbb R^3).
\end{align}
\end{lemma}
\begin{proof}
容易验证
$$
(t\frac{\dd}{\dd t})w(a+t(x-a))=((x-a)\cdot\nabla)w(a+t(x-a)).
$$
于是
\begin{align*}    
\div\mathcal{P}_3w&=3\int_0^1t^2w(a+t(x-a))\dd t + \int_0^1t^2((x-a)\cdot\nabla)w(a+t(x-a))\dd t \\
&=3\int_0^1t^2w(a+t(x-a))\dd t + \int_0^1t^3\frac{\dd}{\dd t}w(a+t(x-a))\dd t \\
&=t^3w(a+t(x-a))|_{t=0}^1=w(x).
\end{align*}

由$\curl(v\times(x-a))=-(x-a)\div v+((x-a)\cdot\nabla) v+2v$知,
\begin{align*}
\curl\mathcal{P}_2v&=-\int_0^1t(x-a)\div(v(a+t(x-a)))\dd t + \int_0^1 t((x-a)\cdot\nabla) v(a+t(x-a))\dd t \\
& \quad + \int_0^12tv(a+t(x-a))\dd t \\
&=-\int_0^1t^2(x-a)(\div v)(a+t(x-a))\dd t + \int_0^1 t^2\frac{\dd}{\dd t} v(a+t(x-a))\dd t \\
& \quad + \int_0^12tv(a+t(x-a))\dd t \\
&=-\mathcal{P}_3\div v + t^2v(a+t(x-a))|_{t=0}^1=-\mathcal{P}_3\div v+v.
\end{align*}

对任意的向量$y$成立$(\nabla u)y=-(\curl u)\times y + (y\cdot\nabla)u$,故
\begin{align*}
\grad\mathcal{P}_1u&=\int_0^1t(\nabla u)|_{a+t(x-a)}(x-a)\dd t + \int_0^1 u(a+t(x-a))\dd t \\
&=-\int_0^1t(\curl u)|_{a+t(x-a)}\times(x-a)\dd t + \int_0^1 t\frac{\dd}{\dd t}u(a+t(x-a))\dd t \\
& \quad + \int_0^1 u(a+t(x-a))\dd t \\
&=-\mathcal{P}_2\curl u + tu(a+t(x-a))|_{t=0}^1=-\mathcal{P}_2\curl u+u.
\end{align*}
\end{proof}

\begin{lemma}
对于$u\in C^{\infty}(\Omega;\mathbb R^3)$, $v\in C^{\infty}(\Omega;\mathbb R^3)$和$w\in C^{\infty}(\Omega)$,有
\begin{align}
\label{poincareP3bound}
\|\mathcal{P}_3w\|_0&\leq \frac{2}{3}h_{\Omega}\|w\|_0, \\
\label{poincareP2bound}
\|\mathcal{P}_2v\|_0&\leq 2h_{\Omega}\|v\|_0, \\
\label{poincareP1bound}
|\mathcal{P}_1u|_{1}&\leq \|u\|_0+2h_{\Omega}\|\curl u\|_0, \\
% \|\mathcal{P}_1u\|_{H^1(\Omega)/\mathbb R}&\leq C\|u\|_{H(\curl,\Omega)},
\label{poincareP1boundLp}
\|\mathcal{P}_1u\|_{L^p(\Omega)}&\leq \frac{p}{(p-3)}h_{\Omega}\|u\|_{L^p(\Omega)},
\end{align}
其中$p>3$.
\end{lemma}
\begin{proof}
设$s=t^{1/\alpha}$ ($\alpha>1/3$), 则
$$
\mathcal{P}_3w= \alpha (x-a)\int_0^1s^{3\alpha-1}w(a+s^{\alpha}(x-a))\dd s.
$$
令$y_s=a+s^{\alpha}(x-a)$, $\Omega_s=\{a+s^{\alpha}(x-a): x\in\Omega\}\subseteq\Omega$.
从而
\begin{align*}
\|\mathcal{P}_3w\|_0^2&\leq\alpha^2h_{\Omega}^2\int_{\Omega}\dd x\int_0^1s^{6\alpha-2}w^2(a+s^{\alpha}(x-a))\dd s = \alpha^2h_{\Omega}^2\int_0^1s^{6\alpha-2}\dd s\int_{\Omega}w^2(a+s^{\alpha}(x-a))\dd x \\
&= \alpha^2h_{\Omega}^2\int_0^1s^{6\alpha-2}\dd s\int_{\Omega}w^2(y_s)\dd x= \alpha^2h_{\Omega}^2\int_0^1s^{3\alpha-2}\dd s\int_{\Omega_s}w^2(y_s)\dd y_s \\
&\leq \alpha^2h_{\Omega}^2\int_0^1s^{3\alpha-2}\dd s\int_{\Omega}w^2(y_s)\dd y_s=\frac{\alpha^2}{3\alpha-1}h_{\Omega}^2\|w\|_0^2.
\end{align*}
当$\alpha=\frac{2}{3}$时,$\frac{\alpha^2}{3\alpha-1}$取到最小值$\frac{4}{9}$,故有
$$
\|\mathcal{P}_3w\|_0\leq \frac{2}{3}h_{\Omega}\|w\|_0.
$$

设$s=t^{1/\alpha}$ ($\alpha>1$), 则
$$
\mathcal{P}_2v = \int_0^1tv(a+t(x-a))\times (x-a)\dd t= \alpha\int_0^1s^{2\alpha-1}v(a+s^{\alpha}(x-a))\times (x-a)\dd s.
$$
令$y_s=a+s^{\alpha}(x-a)$, $\Omega_s=\{a+s^{\alpha}(x-a): x\in\Omega\}\subseteq\Omega$.
从而
\begin{align*}
\|\mathcal{P}_2v\|_0^2&\leq\alpha^2h_{\Omega}^2\int_{\Omega}\dd x\int_0^1s^{4\alpha-2}|v(a+s^{\alpha}(x-a))|^2\dd s = \alpha^2h_{\Omega}^2\int_0^1s^{4\alpha-2}\dd s\int_{\Omega}|v(a+s^{\alpha}(x-a))|^2\dd x \\
&= \alpha^2h_{\Omega}^2\int_0^1s^{4\alpha-2}\dd s\int_{\Omega}|v(y_s)|^2\dd x= \alpha^2h_{\Omega}^2\int_0^1s^{\alpha-2}\dd s\int_{\Omega_s}|v(y_s)|^2\dd y_s \\
&\leq \alpha^2h_{\Omega}^2\int_0^1s^{\alpha-2}\dd s\int_{\Omega}|v(y_s)|^2\dd y_s=\frac{\alpha^2}{\alpha-1}h_{\Omega}^2\|v\|_0^2.
\end{align*}
当$\alpha=2$时,$\frac{\alpha^2}{\alpha-1}$取到最小值$4$,故有
$$
\|\mathcal{P}_2v\|_0\leq 2h_{\Omega}\|v\|_0.
$$

由\eqref{eq:poincareidentity1}得 
$$
\|\grad\mathcal{P}_1u\|_0=\|u -\mathcal{P}_2\curl u\|_0\leq \|u\|_0+\|\mathcal{P}_2\curl u\|_0\leq \|u\|_0+2h_{\Omega}\|\curl u\|_0.
$$
设$s=t^{1/\alpha}$ ($\alpha>(p-1)/(p-3)$), 则
$$
\mathcal{P}_1u = \int_0^1u(a+t(x-a))\cdot (x-a)\dd t= \alpha\int_0^1s^{\alpha-1}u(a+s^{\alpha}(x-a))\cdot (x-a)\dd s.
$$
令$y_s=a+s^{\alpha}(x-a)$, $\Omega_s=\{a+s^{\alpha}(x-a): x\in\Omega\}\subseteq\Omega$.
从而
\begin{align*}
\|\mathcal{P}_1u\|_{L^p(\Omega)}^p&\leq\alpha^ph_{\Omega}^p\int_{\Omega}\dd x\int_0^1s^{p\alpha-p}|u(a+s^{\alpha}(x-a))|^p\dd s \\
&= \alpha^ph_{\Omega}^p\int_0^1s^{p\alpha-p}\dd s\int_{\Omega}|u(a+s^{\alpha}(x-a))|^p\dd x \\
&= \alpha^ph_{\Omega}^p\int_0^1s^{p\alpha-p}\dd s\int_{\Omega}|u(y_s)|^p\dd x= \alpha^ph_{\Omega}^p\int_0^1s^{(p-3)\alpha-p}\dd s\int_{\Omega_s}|u(y_s)|^p\dd y_s \\
&\leq \alpha^ph_{\Omega}^p\int_0^1s^{(p-3)\alpha-p}\dd s\int_{\Omega}|u(y_s)|^p\dd y_s=\frac{\alpha^p}{(p-3)\alpha-(p-1)}h_{\Omega}^p\|u\|_{L^p(\Omega)}^p.
\end{align*}
当$\alpha=\frac{p}{p-3}$时,$\frac{\alpha^p}{(p-3)\alpha-(p-1)}$取到最小值$\alpha^p$,故有
$$
\|\mathcal{P}_1u\|_{L^p(\Omega)}\leq \frac{p}{(p-3)}h_{\Omega}\|u\|_{L^p(\Omega)}.
$$
\end{proof}

\begin{remark}
算子$\mathcal{P}_1$不是$L^2$有界的. 事实上,设$\Omega\subseteq\mathbb R^3$包含坐标原点,取$q(x)=x/|x|^2$, 则$q(x)\in\mathbb L^2(\Omega;\mathbb R^3)$, 但是$\mathcal{P}_1q$没定义.
\end{remark}

\begin{theorem}
由式\eqref{poincareopP1}-\eqref{poincareopP3}定义的Poincar\'e算子$\mathcal{P}_1$, $\mathcal{P}_2$和$\mathcal{P}_3$可唯一的延拓成Sobolev空间上的有界线性算子,即
\begin{align}
\label{poincareP3sobolevbound}
\|\mathcal{P}_3w\|_{H(\div)}&\lesssim \|w\|_0 \quad\quad\;\;\;\forall~w\in L^2(\Omega), \\
\label{poincareP2sobolevbound}
\|\mathcal{P}_2v\|_{H(\curl)}&\lesssim \|v\|_{H(\div)} \quad\;\forall~v\in H(\div,\Omega), \\
\label{poincareP1sobolevbound}
|\mathcal{P}_1u|_{1}&\lesssim \|u\|_{H(\curl)} \quad\forall~u\in H(\curl,\Omega).
% \|\mathcal{P}_1u\|_{H^1(\Omega)/\mathbb R}&\leq C\|u\|_{H(\curl,\Omega)},
\end{align}
同时成立复形
\begin{equation}\label{eq:deRhamcomplexPoincare3d}
%\resizebox{.9\hsize}{!}{$
\mathbb R\xleftarrow{\mathcal{P}_0} H^1(\Omega)\xleftarrow{\mathcal{P}_1} H(\curl, \Omega)\xleftarrow{\mathcal{P}_2} H(\div, \Omega) \xleftarrow{\mathcal{P}_3} L^2(\Omega)\xleftarrow{}0.
\end{equation}
且有恒等式  
\begin{align}
\label{eq:poincareidentityP3}
\div\mathcal{P}_3w&=w \quad\forall~w\in L^{2}(\Omega), \\
\label{eq:poincareidentityP2}
\curl\mathcal{P}_2v+\mathcal{P}_3\div v&=v \quad\,\forall~v\in H(\div, \Omega), \\
\label{eq:poincareidentityP1}
\grad\mathcal{P}_1u+\mathcal{P}_2\curl u&=u \quad\,\forall~u\in H(\curl,\Omega), \\
\label{eq:poincareidentityP0}
\mathcal{P}_0w+\mathcal{P}_1\grad w&=w \quad\,\forall~w\in H^1(\Omega).
\end{align}  
\end{theorem}
\begin{proof}
利用\eqref{eq:poincareidentity3}-\eqref{eq:poincareidentity1}和\eqref{poincareP3bound}-\eqref{poincareP1bound}可得
$$
\|\mathcal{P}_3w\|_{H(\div)}^2=\|\mathcal{P}_3w\|_{0}^2+\|w\|_0^2\leq(1+\frac{4}{9}h_{\Omega}^2)\|w\|_0^2 \quad \forall~w\in C^{\infty}(\Omega),
$$
$$
\|\mathcal{P}_2v\|_{H(\curl)}^2=\|\mathcal{P}_2v\|_{0}^2+\|v-\mathcal{P}_3\div v\|_0^2\leq(2+4h_{\Omega}^2)\|v\|_0^2 +\frac{8}{9}h_{\Omega}^2\|\div v\|_0^2 \quad \forall~v\in C^{\infty}(\Omega;\mathbb R^3).
$$
从而由稠密性可得\eqref{poincareP3sobolevbound}-\eqref{poincareP1sobolevbound}.

同样利用稠密性,由\eqref{eq:poincareidentity3}-\eqref{eq:poincareidentity1}和\eqref{poincareP3sobolevbound}-\eqref{poincareP1sobolevbound}可得\eqref{eq:poincareidentityP3}-\eqref{eq:poincareidentityP1}.
\end{proof}

\begin{theorem}
设三维区域$\Omega$是可缩的,则
三维de Rham复形\eqref{eq:deRhamcomplex3d}
\begin{equation*}
\mathbb R\xrightarrow{\hookrightarrow} H^1(\Omega)\xrightarrow{\grad} H(\curl, \Omega)\xrightarrow{\curl} H(\div, \Omega) \xrightarrow{\div} L^2(\Omega)\xrightarrow{}0
\end{equation*}
是正合的.
\end{theorem}
\begin{proof}
由恒等式\eqref{eq:poincareidentityP3}-\eqref{eq:poincareidentityP0}直接可得.
\end{proof}