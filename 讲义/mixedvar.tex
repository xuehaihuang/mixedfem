% !TEX root = lecture.tex
\chapter{混合变分问题}





% ------------------------------------------------
% 散度定理  本质守恒
% ------------------------------------------------
\section{混合变分问题举例}
\subsection*{Poisson方程}
对$ f \in L^2(\Omega)$, 考虑 Poisson 方程齐次 Dirichlet 边值问题
\begin{equation}
\begin{aligned}
-\Delta u & = f,   \qquad \text{in} \quad \Omega , \nonumber \\
u & = 0,           \qquad \text{on} \quad \partial \Omega. \nonumber
\end{aligned}
\end{equation}
其变分形式为, 找 $u \in V = H_0^1(\Omega) $, 使得
\begin{equation}\label{Possion}
( \nabla u, \nabla v ) = ( f, v ), \quad \forall v \in V. 
\end{equation}
对应的有限元方法为, 找 $ u \in V_h $, 使得
\begin{equation}\label{Possion_fem}
( \nabla_h u, \nabla_h v ) = ( f, v ), \quad \forall v \in V_h.  
\end{equation}
其中有限元空间$V_h$是$V$的某种分片多项式近似. 另一方面, 许多偏微分方程都是由物理过程决定的, 对上述泊松方程, 通过引入一个新的变量$\boldsymbol{\sigma}$ 可以分解为如下形式
\begin{subequations}
\begin{align}
\qquad \qquad \qquad \qquad \qquad \qquad
\boldsymbol\sigma &= \nabla u, &&\text{in} \quad\Omega \qquad \qquad \qquad \qquad \qquad \qquad \label{Poisson1}\\
-\div\boldsymbol\sigma &= f , &&\text{in} \quad\Omega \label{Poisson2}\\
u & = 0,               &&  \text{on} \quad \partial \Omega. \label{Poisson3}
\end{align}
\end{subequations}
其中\eqref{Poisson1}称为本构方程, \eqref{Poisson2}为守恒方程. 本例中可以理解标量势函数$u$(温度, 电势等)决定了通量$\boldsymbol{\sigma}$; 通量的散度(通过分布积分, 对应于净流出量)对应于源项$f$, 即
\[
-\int_K \div \boldsymbol \sigma \text{d}x = -\int_{\partial K}\boldsymbol \sigma\cdot\boldsymbol{n}\,\text{d}s=\int_{K}f\,\text{d}x,
\]
其中$K$是$\Omega$的任意子区域. 对于标准有限元方法\eqref{Possion_fem}, 上述守恒相当于测试函数空间$V_h$包含分片间断函数, 显然不成立. 而混合形式\eqref{Poisson1}-\eqref{Poisson3}对应的混合有限元方法则可以满足守恒律.

\par 混合形式\eqref{Poisson1}-\eqref{Poisson3}可以定义多种变分问题:
\begin{formula}
找 $\boldsymbol{\sigma} \in H(\div, \Omega), u \in L^2(\Omega)$, 使得
\begin{equation}
\begin{aligned}\label{MixPoisson1}
(\boldsymbol\sigma, v) + (\div \boldsymbol\tau, u) &= 0,  &&\forall \boldsymbol\tau \in H(\div,\Omega),\\ 
-(\div \boldsymbol\sigma, v) &= (f, v),  &&\forall  v \in L^2(\Omega)
\end{aligned}
\end{equation}
\end{formula}
变分问题\eqref{MixPoisson1}称为Possion方程的混合对偶格式. 通过分部积分可以将微分算子转移到$u, v $上, 得到与Primal变分形式\eqref{Possion}等价的混合变分问题\eqref{MixPoisson2}.
\begin{formula}
找 $\boldsymbol{\sigma} \in L_0^2(\Omega; \mathbb{R}^d), u \in H_0^1(\Omega)$, 使得
\begin{equation}\label{MixPoisson2}
\begin{aligned}
(\boldsymbol\sigma, \boldsymbol\tau) - (\boldsymbol\tau, \nabla u) &= 0, \qquad\;\;\forall\boldsymbol\tau\in L_0^2(\Omega;\mathbb{R}^d),\\ 
(\boldsymbol\sigma, \nabla v) &= (f, v), \quad \forall v\in H_0^1(\Omega).
\end{aligned}
\end{equation}
\end{formula}
若将\eqref{MixPoisson1}中的内积理解为$H^{-1}(\Omega)$与$H_0^1(\Omega)$上的对偶对, 则可得到
\begin{formula}
找 $\boldsymbol{\sigma} \in L_0^2(\Omega; \mathbb{R}^d) , u \in H_0^1(\Omega)$, 使得
\begin{equation}
\begin{aligned}
(\boldsymbol\sigma, \boldsymbol\tau) + \langle\div \boldsymbol\tau, u\rangle  &= 0, \qquad\;\;\forall\boldsymbol\tau\in L_0^2(\Omega;\mathbb{R}^d),\\ 
-\langle\div \boldsymbol\sigma, v\rangle&= (f, v), \quad \forall v\in H_0^1(\Omega).
\end{aligned}
\end{equation}
\end{formula}



\subsection*{Stokes方程}
考虑Stokes方程:
\begin{subequations}
\begin{align}
-\Delta \boldsymbol{u} - \grad p &=\boldsymbol{f},  &&\text{in} \quad \Omega, \label{Stokes1} \\
\div \boldsymbol{u} & = 0,    &&\text{in} \quad \Omega,  \label{Stokes2} \\ 
\boldsymbol{u} & =\boldsymbol{0}, &&\text{on} \quad \partial \Omega \label{Stokes3} .
\end{align}
\end{subequations}
其中$\boldsymbol u$表示流体速度,$p$表示压力,\eqref{Stokes1}为本构方程, \eqref{Stokes2}为守恒方程. 通过引入伪应力$\boldsymbol{\sigma}$, 可以将\eqref{Stokes1}-\eqref{Stokes3}改写为
\begin{subequations}
\begin{align}
\boldsymbol\sigma &= \nabla\boldsymbol u,    && \text{in} \quad \Omega \label{Stokes_sigma1}\\
-\div\boldsymbol\sigma - \nabla p &=f  && \text{in} \quad \Omega, \label{Stokes_sigma2}\\
\div\boldsymbol u &=0,              && \text{in} \quad \Omega \label{Stokes_sigma3}\\
\boldsymbol{u} &= \boldsymbol{0} && \text{on} \quad \partial \Omega . \label{Stokes_sigma4}
\end{align}
\end{subequations}
注意到$\tr\boldsymbol\sigma= \div\boldsymbol u = 0$,即$\boldsymbol{\sigma}$属于无迹张量空间$\mathbb{T}$.
\par \eqref{Stokes1}-\eqref{Stokes3}对应的变分问题为
\begin{formula}
找$ \boldsymbol{u} \in H_0^1(\Omega; \mathbb{R}^d), p \in L_0^2(\Omega)$, 使得
\begin{equation}\label{Stokes_fem}
\begin{aligned}
(\nabla \boldsymbol u,\nabla \boldsymbol v) + (\div\boldsymbol v,p) &= (\boldsymbol f,\boldsymbol v),\quad  \forall \boldsymbol v \in H_0^1(\Omega;\mathbb{R}^d),\\
(\div\boldsymbol u,q) &= 0,\qquad \quad \forall q\in L_0^2(\Omega).
\end{aligned}
\end{equation}
\end{formula}
变分问题\eqref{Stokes_fem}说明, 与椭圆型偏微分方程(eg. Possion方程的primal形式)不同, Stokes方程对应的变分问题本身即为鞍点问题. \eqref{Stokes_sigma1}-\eqref{Stokes_sigma4}对应的变分问题为
\begin{formula}
找 $\boldsymbol{\sigma} \in H(\div, \Omega; \mathbb{T}), u \in H_0(\div, \Omega)), p \in L_0^2(\Omega)$, 使得
\begin{equation}
\begin{aligned}
(\boldsymbol\sigma,\boldsymbol\tau)+(\div\boldsymbol\tau,\boldsymbol u) + (\div\boldsymbol u,q)&=0,\qquad  &&  \forall\,\boldsymbol{\tau}\in H(\div,\Omega;\mathbb{T}),q\in L_0^2(\Omega)\\
-(\div\boldsymbol{\sigma},v)+(\div\boldsymbol v,p) &=(\boldsymbol f,\boldsymbol v),\qquad   && \forall\,  \boldsymbol v\in H_0(\div,\Omega).  \nonumber
\end{aligned}
\end{equation}
\end{formula}
实际上, $\boldsymbol{\sigma}$ 的光滑性要求可以降低, 对应的分部式混合变分问题为
\begin{formula}
找 $\boldsymbol{\sigma} \in H^{-1}(\curl\div,\Omega;\mathbb{T}), u \in H_0(\div, \Omega)), p \in L_0^2(\Omega)$, 使得
\begin{equation}
\begin{aligned}
(\boldsymbol\sigma,\boldsymbol\tau)+\langle\div\boldsymbol\tau,\boldsymbol u\rangle+ (\div\boldsymbol u,q)&=0,\qquad &&\forall\,\boldsymbol{\tau}\in H^{-1}(\curl\div,\Omega;\mathbb{T}),q\in L_0^2(\Omega)\\
-\langle\div\boldsymbol\sigma,\boldsymbol v \rangle+(\div\boldsymbol v,p) &=(\boldsymbol f,\boldsymbol v),\qquad &&\forall\,\boldsymbol v\in H_0(\div,\Omega), \nonumber
\end{aligned}
\end{equation}
其中$H^{-1}(\curl\div,\Omega;\mathbb{T}) = \{\boldsymbol\tau\in L^2(\Omega;\mathbb{T}):\div\boldsymbol\tau\in (H_0(\div,\Omega))')\}$, $\langle \cdot, \cdot \rangle$ 表示 $ (H_0(\div, \Omega))' $ 与 $H_0(\div, \Omega)$间的对偶对.
\end{formula}
在Stokes方程的基础上, 有更贴合物理性质的应变Stokes方程:
\begin{equation}\label{st_Stokes}
\begin{aligned}
-\div(\varepsilon(\boldsymbol u) )-\grad p&=\boldsymbol{f},  && \text{in} \quad \Omega, \\ 
\div \boldsymbol{u}&=0, && \text{in} \quad \Omega , \\ 
\boldsymbol{u}  &=\boldsymbol{0} && \text{on} \quad \partial \Omega ,
\end{aligned}
\end{equation}
其中$\varepsilon(\boldsymbol u)=\tfrac12(\grad\boldsymbol u+\grad\boldsymbol u^{\intercal})$为梯度的对称部分. 进一步地, 引入$\boldsymbol{\sigma} = \varepsilon(\boldsymbol u)$, 对应的方程为
\begin{equation}\label{st_Stokes_sigma}
\begin{aligned}
\boldsymbol{\sigma} & = \varepsilon(\boldsymbol u),  && \text{in} \quad \Omega, \\ 
-\div \boldsymbol{\sigma} - \grad p &=\boldsymbol{f},  && \text{in} \quad \Omega, \\ 
\div \boldsymbol{u}&=0, && \text{in} \quad \Omega ,\\ 
\boldsymbol{u}  &=\boldsymbol{0} && \text{on} \quad \partial \Omega .
\end{aligned}
\end{equation}
方程\eqref{st_Stokes}, \eqref{st_Stokes_sigma}对应的变分问题分别为
\begin{formula}
找 $\boldsymbol{u} \in H_0^1(\Omega;\mathbb{R}^d), p \in L_0^2(\Omega)$, 使得
\begin{equation}
\begin{aligned}
(\varepsilon(\boldsymbol u),\varepsilon(\boldsymbol u))+(\div\boldsymbol{v},p)&=(\boldsymbol{f},\boldsymbol{v}),\qquad \forall\,\boldsymbol{v}\in H^{1}_{0}(\Omega;\mathbb{R}^d),\\
(\div\boldsymbol{u},q)&=0,\qquad \qquad\forall\,q\in L_0^{2}(\Omega). \nonumber
\end{aligned}
\end{equation}
\end{formula}
\begin{formula}
{找 $\boldsymbol{\sigma} \in H(\div, \Omega; \mathbb{S} \cap \mathbb{T}), u \in H_0(\div, \Omega)), p \in L_0^2(\Omega)$, 使得
\begin{equation}
\begin{aligned}
(\boldsymbol\sigma,\boldsymbol\tau)+(\div\boldsymbol\tau,\boldsymbol u) - (\div\boldsymbol u,q)&=0,\qquad  &&  \forall\,\boldsymbol{\tau}\in H(\div,\Omega;\mathbb{S} \cap \mathbb{T}),q\in L_0^2(\Omega)\\
(\div\boldsymbol{\sigma},v)-(\div\boldsymbol v,p) &=-(\boldsymbol f,\boldsymbol v),\qquad   && \forall\,  \boldsymbol v\in H_0(\div,\Omega).  \nonumber
\end{aligned}
\end{equation}}
\end{formula}



\subsection*{线弹性方程}
线弹性方程为:
\begin{subequations}
\begin{align}
-\div \boldsymbol\sigma(\boldsymbol{u}) &= \boldsymbol f, && \text{in } \Omega, \label{linear_ela1} \\
\boldsymbol\sigma(\boldsymbol{u}) &=2\mu\,\varepsilon(\boldsymbol{u})+\lambda \div \boldsymbol{u} I && \text{in } \Omega, \label{linear_ela2} \\
\boldsymbol u &=0, &&\text{on } \partial \Omega. \label{linear_ela3} 
\end{align}
\end{subequations}
其中 \eqref{linear_ela1}为守恒方程, \eqref{linear_ela2}为本构方程, $\boldsymbol u$ 是位移向量,$\varepsilon(\boldsymbol u)$ 是应变, $\boldsymbol{\sigma}(\boldsymbol u)$ 是应力张量, $\mu$是正常数, $\lambda$ 称为Lamé 系数.若引入$p = \lambda \div\boldsymbol u$,则线弹性方程可以改写为
\begin{equation}\label{linear_eal_p}
\begin{aligned}
-\div (2 \mu \varepsilon(\boldsymbol u) + p I ) &= \boldsymbol f,  &&\text{in } \Omega , \\
p &= \lambda \div \boldsymbol u \quad  &&\text{in } \Omega , \\
\boldsymbol u  &= 0 , && \text{on } \partial \Omega. 
\end{aligned}
\end{equation}
\eqref{linear_eal_p}对应的混合变分问题为
\begin{formula}
找 $ \boldsymbol u \in  H^1_0(\Omega;\mathbb{R}^d), p \in L_0^2(\Omega)$, 使得
\begin{subequations}
\begin{align}
2\mu(\varepsilon(\boldsymbol{u}),\varepsilon(\boldsymbol{v}))+(\div\boldsymbol{v},p)&=(f,\boldsymbol{v}), \label{linear_ela_p_weak1} \qquad 
\forall\boldsymbol{v}\in H^1_0(\Omega;\mathbb{R}^d) ,    \\
(\div\boldsymbol{u},q)-\frac{1}{\lambda}(p,q)&=0,\qquad \qquad\forall\,q\in L_0^{2}(\Omega). \label{linear_ela_p_weak2}
\end{align}
\end{subequations}
\end{formula}
\eqref{linear_ela_p_weak1}-\eqref{linear_ela_p_weak2}通常被称为 Lamé 系统. 当$\lambda \to \infty $时, 其数值方法常会出现闭锁(locking)现象. 注意到当没有$\frac{1}{\lambda}(p,q)$这一项时,即为应变Stokes方程, 即应变Stokes方程是线弹性方程的极限问题, 使用应变Stokes方程的有限元方法可以解决闭锁现象.
\par 进一步地, 对线弹性方程\eqref{linear_ela1}-\eqref{linear_ela3}改写. 由$\operatorname{tr}\boldsymbol\sigma = 2\mu \div \boldsymbol u + d \lambda \div \boldsymbol u$,得 $\div \boldsymbol u=\frac{1}{d\lambda+2\mu}\operatorname{tr}\boldsymbol\sigma$, 即 $\boldsymbol\sigma =2\mu\varepsilon(\boldsymbol u)+\frac{\lambda}{d\lambda+2\mu}\operatorname{tr}\boldsymbol\sigma I$, 故有
\begin{equation}
\varepsilon(u) = \frac{1}{2\mu}\Big(\boldsymbol\sigma- \frac{\lambda}{d\lambda+2\mu}\operatorname{tr}\boldsymbol\sigma I\Big). \nonumber
\end{equation}
为了简化记号,引入算子$A$:
\begin{equation}
A\boldsymbol\sigma=\frac{1}{2\mu}\Big(\boldsymbol\sigma - \frac{\lambda}{d\lambda+2\mu}\operatorname{tr}\boldsymbol\sigma I\Big)
= \frac{1}{2\mu}\,\dev\boldsymbol\sigma + \frac{1}{d(2\mu+d\lambda)}\,\operatorname{tr}\boldsymbol\sigma I, \nonumber
\end{equation}
其中$\dev\boldsymbol\sigma = \boldsymbol\sigma-\frac{1}{d}\operatorname{tr}\boldsymbol\sigma I$,则对应的方程为
\begin{equation}
\begin{aligned}
A\boldsymbol{\sigma} & = \varepsilon(\boldsymbol u),  && \text{in } \Omega , \\
-\div \boldsymbol{\sigma} &= \boldsymbol f, && \text{in } \Omega, \\
\boldsymbol u &= 0 , &&\text{on } \partial \Omega. \nonumber
\end{aligned}
\end{equation}
\newline 其对应的混合变分问题为
\begin{formula}
找 $ \boldsymbol \sigma \in H(\div , \Omega; \mathbb{S}), \boldsymbol u \in L^2(\Omega; \mathbb{R}^d)$, 使得
\begin{equation}
\begin{aligned}
(A\boldsymbol\sigma,\boldsymbol\tau)+(\div\boldsymbol \tau,\boldsymbol u) &=0 ,\qquad && \forall\boldsymbol\tau\in H(\div,\Omega;\mathbb{S}),\\
-(\div\boldsymbol\sigma,\boldsymbol v)&=(\boldsymbol f,\boldsymbol v), \quad && \forall \boldsymbol v \in L^2(\Omega;\mathbb{R}^d).  \nonumber
\end{aligned}
\end{equation}
\end{formula}

\begin{formula}
找 $ \boldsymbol \sigma \in L^2(\Omega; \mathbb{S}), \boldsymbol u \in H_0^1(L^2(\Omega; \mathbb{R}^d)$, 使得
\begin{equation}
\begin{aligned}
(A\boldsymbol\sigma,\boldsymbol\tau) - ( \boldsymbol \tau,\nabla \boldsymbol u) &=0 ,\qquad &&\forall\boldsymbol\tau\in L^2(\Omega;\mathbb{S}),\\
(\boldsymbol\sigma, \nabla \boldsymbol v) & =(\boldsymbol f,\boldsymbol v), \quad &&\forall \boldsymbol v \in H_0^1(\Omega;\mathbb{R}^d).  \nonumber
\end{aligned}
\end{equation}
\end{formula}

\begin{formula}
找 $ \boldsymbol \sigma \in H^{-1}(\curl\div,\Omega;\mathbb{S}), \boldsymbol u \in H_0(\div, \Omega)$, 使得
\begin{equation}
\begin{aligned}
(A\boldsymbol\sigma,\boldsymbol\tau)+\langle\div\boldsymbol \tau,\boldsymbol u\rangle &=0 ,\qquad  &&\forall\boldsymbol\tau\in H^{-1}(\curl\div,\Omega;\mathbb{S}),\\
-\langle\div\boldsymbol\sigma,\boldsymbol v\rangle &=(\boldsymbol f,\boldsymbol v), \quad  &&\forall \boldsymbol v \in H_0(\div,\Omega), \nonumber 
\end{aligned}
\end{equation}
其中$ H^{-1}(\curl\div,\Omega;\mathbb{S}):=\{\boldsymbol\tau\in L^2(\Omega;\mathbb{S}):\div\boldsymbol\tau\in (H_0(\div,\Omega))' \}$, $\langle \cdot, \cdot \rangle$ 为 $( H_0(\div, \Omega))'$ 与  $H_0(\div, \Omega)$的对偶对.
\end{formula}

\begin{formula}
找 $ \boldsymbol \sigma \in H^{-1}(\div \div,\Omega;\mathbb{S}), \boldsymbol u \in H_0(\curl, \Omega)$, 使得
\begin{equation}
\begin{aligned}
(A\boldsymbol\sigma,\boldsymbol\tau)+\langle\div\boldsymbol \tau,\boldsymbol u\rangle &=0 ,\qquad &&\forall\boldsymbol\tau\in H^{-1}(\div\div,\Omega;\mathbb{S}),\\
-\langle\div\boldsymbol\sigma,\boldsymbol v\rangle &=(\boldsymbol f,\boldsymbol v), \quad &&\forall \boldsymbol v \in H_0(\curl,\Omega), \nonumber
\end{aligned}
\end{equation}
其中$ H^{-1}(\div\div,\Omega;\mathbb{S}):=\{\boldsymbol\tau\in L^2(\Omega;\mathbb{S}):\div\boldsymbol\tau\in (H_0(\curl,\Omega))' \}$,  $\langle \cdot, \cdot \rangle$ 为 $( H_0(\curl, \Omega))'$ 与  $H_0(\curl, \Omega)$的对偶对.
\end{formula}



\subsection*{双旋度方程}
双旋度方程为
\begin{subequations}
\begin{align}
\operatorname{curl}^2 \boldsymbol u& = \boldsymbol f,   && \text{in } \Omega  \label{curl1} \\
\div \boldsymbol u&=0,   && \text{in } \Omega \label{curl2} \\
\boldsymbol u\times\boldsymbol{n} &=0 ,  && \text{on } \partial \Omega \label{curl3} ,
\end{align}
\end{subequations}
约束\eqref{curl2}可以理解为$\boldsymbol u $与$H_0^1(\Omega)$的函数在$L^2$意义下正交: $(u ,\nabla v) = 0, \forall v \in H_0^1(\Omega)$. 其变分问题为
\begin{formula}
找 $ \boldsymbol u \in H_0(\curl, \Omega), p \in H_0^1(\Omega)$, 使得
\begin{equation}
\begin{aligned}
(\curl \boldsymbol u,\curl \boldsymbol v)+( \boldsymbol v,\nabla p)&=(\boldsymbol f,\boldsymbol v),  &&\forall \boldsymbol v\in H_0(\curl,\Omega),\\
(\boldsymbol{u},\nabla q) & = 0 ,  && \forall q\in H_0^1(\Omega).  \nonumber
\end{aligned}
\end{equation}
\end{formula}





\subsection*{重调和方程}
齐次Dirchlet边界的重调和方程为
\begin{equation}\label{biharmonic}
\begin{aligned}
\Delta^2 u &=f,  && \text{in } \Omega, \\
u = \partial_n u & =0, &&\text{on } \partial \Omega .
\end{aligned}
\end{equation}
四阶算子$\Delta^2 = \div\div\nabla\nabla$, 通过引入变量$\boldsymbol\sigma$, 得
\begin{equation}
\begin{aligned}
\boldsymbol \sigma &=\nabla^2 u,   \quad && \text{in } \Omega, \\
\div\div \boldsymbol \sigma &=f,  && \text{in } \Omega, \\
u = \partial_n u & =0, &&\text{on } \partial \Omega . 
\end{aligned}
\end{equation}
\newline 其对应的变分问题为:
\begin{formula}
找 $ \boldsymbol \sigma \in L^2(\Omega; \mathbb{S}), u \in H_0^2(\Omega) $, 使得
\begin{equation}\label{biharmonic_formula1}
\begin{aligned}
(\boldsymbol\sigma, \boldsymbol\tau) - (\nabla^2 u, \boldsymbol\tau) &= 0,\qquad  &&\forall
\boldsymbol\tau \in L^2(\Omega; \mathbb S), \\
(\boldsymbol\sigma, \nabla^2 v) &= (f,v),\quad  && \forall v\in H_0^2(\Omega). 
\end{aligned}
\end{equation}
\end{formula}

\begin{formula}
找 $ \boldsymbol \sigma \in H(\div\div,\Omega;\mathbb{S}), u \in L^2(\Omega) $, 使得
\begin{equation}\label{biharmonic_formula2}
\begin{aligned}
(\boldsymbol\sigma, \boldsymbol\tau) - (\div\div\boldsymbol\tau,u) &= 0,\qquad && \forall
\boldsymbol\tau \in H(\div\div,\Omega;\mathbb{S}), \\
(\div\div\boldsymbol\sigma,  v) &= (f,v),\quad  && \forall v\in L^2(\Omega).
\end{aligned}
\end{equation}
\end{formula}


\begin{formula}
找 $ \boldsymbol \sigma \in H(\div, \Omega; \mathbb{S}) , u \in H_0^1(\Omega) $, 使得
\begin{equation}\label{biharmonic_formula3}
\begin{aligned}
(\boldsymbol\sigma, \boldsymbol\tau) + (\nabla u, \operatorname{div}\boldsymbol\tau) &= 0,
\qquad  && \forall	\boldsymbol \tau \in H(\operatorname{div}, \Omega; \mathbb S),\\
-(\operatorname{div}\boldsymbol\sigma, \nabla v) &= (f,v),
\quad && \forall v \in H^1_0(\Omega).
\end{aligned}
\end{equation}
\end{formula}
变分问题\eqref{biharmonic_formula1}称为双调和方程的Primal形式; \eqref{biharmonic_formula2}为对偶形式; \eqref{biharmonic_formula3}对前两者间空间的光滑性做了平衡, 但该变分问题适定性会出现问题,解决的方法是将内积使用$\langle \operatorname{div}\boldsymbol\sigma, \nabla v\rangle_{(H_0(\operatorname{curl}))'\times H_0(\operatorname{curl})}$代替. 另一种分部式变分问题为
\begin{formula}
找 $ \boldsymbol \sigma \in H^{-1}(\operatorname{div}\operatorname{div},\Omega; \mathbb S), u \in H_0^1(\Omega)$, 使得
\begin{equation}
\begin{aligned}
(\boldsymbol \sigma, \boldsymbol \tau)
- \langle \operatorname{div}\operatorname{div}\boldsymbol\tau, u\rangle
&= 0,
\qquad  && \forall \boldsymbol \tau \in H^{-1}(\operatorname{div}\operatorname{div},\Omega; \mathbb S),\\
\langle \operatorname{div}\operatorname{div}\boldsymbol\sigma, v\rangle
&= (f,v),
\quad  && \forall v \in H^1_0(\Omega).   \nonumber
\end{aligned}
\end{equation}
\end{formula}


双调和方程\eqref{biharmonic}也可以改写为
\begin{equation}
\begin{aligned}
\sigma & =-\Delta u,    &&\text{in } \Omega,  \\
-\Delta \sigma &=f,       &&\text{in } \Omega\\
u =  \partial_n u &=0,  &&\text{on } \partial \Omega
\end{aligned}
\end{equation}
该格式对应的变分问题\eqref{biharmonic_formula5}在$\Omega$为凸区域时与原形式等价.
\begin{formula}
找 $ \sigma \in H^1(\Omega), u \in H_0^1(\Omega)$, 使得
\begin{equation}\label{biharmonic_formula5}
\begin{aligned}
(\sigma,\tau) - (\nabla \tau, \nabla u) &= 0,\qquad\qquad && \forall\tau\in H^1(\Omega)\\
(\nabla \sigma, \nabla v) &= (f, v),\qquad  && \forall  v\in H_0^1(\Omega).
\end{aligned}
\end{equation}
\end{formula}

这种格式在适定性上存在一些问题,但可以理解为:
\begin{formula}
找 $ \sigma \in H^{-1}(\Delta, \Omega), u \in H_0^1(\Omega) $, 使得 
\begin{equation}
\begin{aligned}
(\sigma,\tau) + \langle\Delta\tau,u\rangle&= 0,\qquad\qquad&&  \forall\sigma\in H^{-1}(\Delta,\Omega)\\
- \langle\Delta\sigma,v\rangle &= (f, v),\qquad &&  \forall  v\in H_0^1(\Omega). \nonumber
\end{aligned}  
\end{equation}
\end{formula}

另一方面,我们也可以将原方程重写为一阶系统:
\begin{equation}
\begin{aligned}
\boldsymbol \phi &= \nabla u,     &&\text{in } \Omega , \\
\boldsymbol \sigma &= \nabla \boldsymbol \phi,  &&\text{in } \Omega , \\
\boldsymbol q & = \div \boldsymbol \sigma ,  &&\text{in } \Omega , \\
\div \boldsymbol q & = f             &&\text{in } \Omega , \\
u &= \boldsymbol \phi \cdot \boldsymbol n = 0 , \quad  &&\text{on } \partial \Omega , \\
\end{aligned}
\end{equation}
对应的混合变分问题为
\begin{formula}
找 $ \boldsymbol \phi \in L^2(\Omega; \mathbb{R}^d), \boldsymbol \sigma \in H(\div, \Omega; \mathbb{M}), \boldsymbol q \in H(\div, \Omega), u \in L^2(\Omega)$, 使得
\begin{equation}
\begin{aligned}
(\boldsymbol \sigma, \boldsymbol \tau)+(\div \boldsymbol\tau, \boldsymbol\phi) - ( \boldsymbol \phi, \boldsymbol p) - (\div \boldsymbol p,u) &=0  \quad && \forall \boldsymbol \tau\in H(\div,\Omega;\mathbb{M}), \boldsymbol p \in H(\div,\Omega) ,\\
(\div \boldsymbol \sigma, \boldsymbol \psi)-(\boldsymbol q, \boldsymbol \psi)-(\div \boldsymbol q,v) &= -(f,v) \quad && \forall \boldsymbol \psi\in L^2(\Omega;\mathbb{R}^d), v\in L^2(\Omega).
\end{aligned}
\end{equation}
\end{formula}







\subsection*{$m$ 重调和方程}
一般地, 对$m$重调和方程
\begin{equation}
(-\Delta)^m u =f,  \nonumber  
\end{equation}
通过引入$\boldsymbol \sigma $ 可以改写为
\begin{equation}
\begin{aligned}
\boldsymbol \sigma &=\nabla^m u,\\
(-\div)^m \boldsymbol \sigma&=f.  \nonumber
\end{aligned}
\end{equation}
对应的混合变分问题为
\begin{formula}
找$ \boldsymbol \sigma \in  H(\div^m,\Omega;\mathbb{S}), u \in L^2(\Omega)$, 使得
\begin{equation}
\begin{aligned}
(\boldsymbol\sigma, \boldsymbol\tau) - ((-\div)^{m}\boldsymbol\tau,u) &= 0,\qquad\forall
\boldsymbol\tau \in H(\div^m,\Omega;\mathbb{S}), \\
((-\div)^{m}\boldsymbol\sigma,  v) &= (f,v),\quad\forall v\in L^2(\Omega).  \nonumber
\end{aligned}
\end{equation}
\end{formula}

\subsection*{四阶奇异摄动方程}
奇异摄动问题为
\begin{equation}
\begin{aligned}
\varepsilon^2\Delta^2 u-\Delta u &= f ,   &&\text{in } \Omega , \\
u =  \partial_n u &=0,  &&\text{on } \partial  \Omega .  \nonumber
\end{aligned}
\end{equation}
当$\varepsilon$趋向$0$时, 该问题趋于Possion方程, 此时边界条件有多余, 会导致边界层现象. 引入$\boldsymbol \sigma$, 上述问题可以改写为
\begin{equation}
\begin{aligned}
\boldsymbol \sigma & = \varepsilon^2 \nabla^2 u,  \quad &&\text{in } \Omega, \\ 
\div \div \boldsymbol \sigma - \Delta u & = f,  &&\text{in } \Omega, \\ 
u =  \partial_n u &=0,  &&\text{on } \partial  \Omega .  \nonumber
\end{aligned}
\end{equation}
对应的混合变分问题为\eqref{Singular_per_formula1}, 该方法可以避免边界层现象.
\begin{formula}
找 $ \boldsymbol \sigma \in H^{-1}(\div \div, \Omega; \mathbb{S}), u \in H_0^1(\Omega)$, 使得
\begin{equation}\label{Singular_per_formula1}
\begin{aligned}
\varepsilon^{-2}(\boldsymbol \sigma, \boldsymbol \tau) - \langle\operatorname{div}\operatorname{div}\boldsymbol \tau, u\rangle &= 0 ,\quad &&\forall\tau\in H^{-1}(\div\div,\Omega,\mathbb{S})\\
\langle\operatorname{div}\operatorname{div} \boldsymbol \sigma, v\rangle + (\nabla u, \nabla v) &= ( f, v ), \quad && \forall v\in H_0^{1}(\Omega).
\end{aligned}
\end{equation}
\end{formula}





\subsection*{应变梯度问题}
应变梯度问题为
\begin{equation}
\begin{aligned}
\div (\iota^2\Delta-I)\boldsymbol \sigma(\boldsymbol u) &= \boldsymbol f,     \\
\boldsymbol \sigma(\boldsymbol u)&=2\mu \varepsilon(\boldsymbol u)+\lambda \div \boldsymbol u I. \nonumber
\end{aligned}
\end{equation}
引入新变量$p$, 上述方程改写为
\begin{equation}
\begin{aligned}
 p = \lambda\div \boldsymbol u,\\
\div (\iota^2\Delta-I)(2\mu \varepsilon(\boldsymbol u)+p I) = \boldsymbol f. \label{strain_gra}
\end{aligned}
\end{equation}
问题\eqref{strain_gra}对应的混合问题为
\begin{formula}
找 $ \boldsymbol  u \in H_0^2(\Omega; \mathbb{R}^d), p \in H_0^1(\Omega) \cap L_0^2(\Omega) $, 使得
\begin{equation}
\begin{aligned}
2\mu (\varepsilon( \boldsymbol u),\varepsilon(\boldsymbol v))_{\iota}+ (\div \boldsymbol v,p)_{\iota} &=(\boldsymbol f,\boldsymbol v), \quad &&\forall \boldsymbol v\in H^2_0(\Omega;\mathbb{R}^d),\\
(\div \boldsymbol u,q)_{\iota} - \tfrac1\lambda(p,q)_{\iota} & =0,  \quad && \forall q\in H_0^1(\Omega)\cap L_0^2(\Omega). \nonumber
\end{aligned}
\end{equation}
其中内积$(p, q)_{\iota} := \iota^2 (\nabla p, \nabla q) + (p, q)$.
\end{formula}

或者,类似于线弹性方程的做法:
\begin{subequations}
\begin{align}
\iota^{-2}\bar{A}\bar{\boldsymbol\sigma} &= \nabla\varepsilon(\boldsymbol u),\\
\div\div\bar{\boldsymbol\sigma}-\div\varepsilon(\boldsymbol u)&=f.
\end{align}
\end{subequations}
得到混合格式:
\begin{subequations}
\begin{align}
(\iota^{-2}\bar{A}\bar{\boldsymbol\sigma},\bar{\boldsymbol\tau}) - \langle\div\div\bar{\boldsymbol\tau},\boldsymbol u\rangle &= 0 \quad \bar{\boldsymbol \tau}\in H^{-1}(\div\div,\Omega;\mathbb{S}\otimes\mathbb{R}^d),\\
\langle\div\div\bar{\boldsymbol\sigma},\boldsymbol v\rangle +  (\varepsilon(\boldsymbol u),\varepsilon(\boldsymbol v)) & = (f,\boldsymbol v) \quad \boldsymbol u\in H_0^1(\Omega;\mathbb{R}^d).
\end{align}
\end{subequations}


% \section{Reissner-Mindlin板}

% \section{Hodge-Laplace方程}

% \section{Inf-sup 条件:Babu\v{s}ka 理论}
\section{Babu\v{s}ka 理论}

算子方程$Au = f$的适定性包括存在性、唯一性和稳定性三个方面.


\subsection{抽象变分问题}

给定变分问题:找$u\in U$使得
\begin{equation}\label{variationalproblemabstract}
a(u, v) = \langle f, v\rangle\quad\forall~v\in V.
\end{equation}
这里, 
$
a(\cdot, \cdot): U\times V\to\mathbb R
$是Banach空间$U$和$V$上的双线性形式, 即关于每个变量都是线性的.
基于双线性形式$a(\cdot, \cdot)$, 如下定义两个线性算子
$A: U\to V'$ 和 $A': V\to U'$:
\[
\langle Au, v\rangle=\langle u, A'v\rangle=a(u, v).
\]

变分问题~\eqref{variationalproblemabstract} 等价于如下算子方程: 给定$f\in V'$, 找$u\in U$, 使得在对偶空间$V'$里满足
\begin{equation}\label{eq:operatorequation}
 Au = f .
\end{equation}
% 等价地, 
% \[
% a(u, v) = \langle f, v\rangle \quad\forall~ v \in V.
% \]

% To begin with, we have to assume both $A$ and $A'$ are continuous which can be derived
% from the continuity of the bilinear form. 
假定双线性形式$a(\cdot, \cdot)$是连续的, 即
\begin{equation}\label{abounded}
a(u, v) \leq  C\|u\|_U\|v\|_V\quad \forall~u\in U,  v\in V.
\end{equation}
将使得上述不等式成立的最小的常数记作$\|a\|$, 即
$$
\|a\|=\sup_{u\in U, v\in V}\frac{a(u, v)}{\|u\|_U\|v\|_V}.
$$
在连续性条件 \eqref{abounded} 下, 易知$A$和$A'$都是有界算子, 且$\|A\| = \|A'\| = \|a\|$.
为了讨论算子方程 \eqref{eq:operatorequation} 的存在性和唯一性, 引入以下两个inf-sup条件:
\begin{equation}\label{infsup1}
\inf_{v\in V}\sup_{u\in U}\frac{a(u, v)}{\|u\|_U\|v\|_V}=\alpha_1>0,
\end{equation}
\begin{equation}\label{infsup2}
\inf_{u\in U}\sup_{v\in V}\frac{a(u, v)}{\|u\|_U\|v\|_V}=\alpha_2>0.
\end{equation}

\begin{theorem}[\cite{Babuska1971}]
假设双线性形式$a(\cdot, \cdot)$是连续的, 即 \eqref{abounded} 成立. 算子方程 \eqref{eq:operatorequation} 是适定的, 当且仅当 \eqref{infsup1} 和 \eqref{infsup2} 成立. 在 \eqref{infsup1} 和 \eqref{infsup2} 成立的前提下, 有
\[
\|A^{-1}\|=\|(A')^{-1}\|=\alpha_{1}^{-1}=\alpha_{2}^{-1}=\alpha^{-1},
\]
故算子方程 \eqref{eq:operatorequation} 的解满足
\[
\|u\|_U\leq\frac{1}{\alpha}\|f\|_{V'}.
\]
\end{theorem}
\begin{proof}
Inf-sup条件 \eqref{infsup1} 的算子形式为 
$$
\|A'v\|_{U'}\geq \alpha_1\|v\|_V\quad \forall~v\in V,
$$
这等价于算子$A$是满射. 类似地, inf-sup条件 \eqref{infsup2} 的算子形式为
$$
\|Au\|_{V'}\geq \alpha_2\|u\|_U\quad \forall~u\in U,
$$ 
这等价于算子$A$是单射. 因此, inf-sup条件 \eqref{infsup1} 和 \eqref{infsup2} 等价于
$A : U\to V$是同构算子. 进一步, 由开映射定理知, 算子$A^{-1}$是有界的,  且易知$A^{-1}$的范数等于$\alpha_{2}^{-1}$. 关于$A'$的结论类似可证.
\end{proof}

下面以 inf-sup 条件 \eqref{infsup1} 为例来说明如何验证 inf-sup 条件. 
% 显然, inf-sup条件\eqref{infsup1}成立的一个充分条件是
% \begin{equation}\label{infsupequiv1}
% \textrm{对任意的 }~v\in V,\; \textrm{总存在非零 }~u \in U, \textrm{ s.t. } a(u, v)\geq\alpha\|u\|_U\|v\|_V.
% \end{equation}
% 以下定理给出inf-sup条件\eqref{infsup1}的另一种等价刻画, 这样要验证inf-sup条件\eqref{infsup1}只需构造一个合适的函数.
\begin{theorem}\label{thm:infsupequiv2}
Inf-sup条件 \eqref{infsup1} 等价于: 对任意的 $v\in V$ 存在
$u\in U$, 使得
\begin{equation}\label{infsupequiv2}
a(u, v)\geq C_1\|v\|_V^2,\quad\quad \|u\|_U\leq C_2\|v\|_V.
\end{equation}
\end{theorem}
\begin{proof}
显然 \eqref{infsupequiv2} 意味着 \eqref{infsup1}, 其中$\alpha_1\geq C_1/C_2$. 现在来证明 \eqref{infsup1} 意味着 \eqref{infsupequiv2}. 对任意的$v\in V$, 由Hahn-Banach定理的推论知, 存在$f\in V'$满足$f(v) =\|v\|_V^2$, 以及 $\|f\|_{V'} = \|v\|_V$. 由于$A$是满射, 由开映射定理知, 总可以找到$\tilde u\in U$满足$A\tilde u = f$, 以及 $\inf\limits_{w\in\ker(A)}\|\tilde u+w\|_U\leq\alpha_E^{-1}\|f\|_{V'}=\alpha_E^{-1}\|v\|_V$. 于是, 存在$u\in U$和$C\geq1$满足
\[
\|u\|_U\leq C\alpha_E^{-1}\|v\|_V, \quad  Au=f.
\]
因此, $a(u, v)=\langle Au, v\rangle=f(v)=\|v\|_V^2$. 如果Banach空间$U$是自反的, 还可以要求$C=1$.
\end{proof}

% 显然, inf-sup条件\eqref{infsup1}成立的一个充分条件是
% 以下定理给出inf-sup条件\eqref{infsup1}的另一种等价刻画, 
这样要验证inf-sup条件 \eqref{infsup1} 只需构造一个合适的函数.

由定理~\ref{thm:infsupequiv2} 可知, inf-sup条件 \eqref{infsup1} 也等价于 %\eqref{infsupequiv1}.
\begin{equation*}%\label{infsupequiv1}
\textrm{对任意的 }~v\in V,\; \textrm{总存在非零 }~u \in U, \textrm{ s.t. } a(u, v)\geq\alpha\|u\|_U\|v\|_V.
\end{equation*}

\begin{remark}[\cite{Lau1978}]
设$K$是(实)Banach空间$X$的一个非空子集. 对任意的$x\in X$,
称$y\in K$是$x$在 $K$上的\textbf{最佳逼近}, 如果
\[
\|x - y\|= \inf\{\|x - z\|: z\in K\}.
\]
如果任何一点$x\in X$在$K$上都有(唯一)最佳逼近, 则称集合$K$是\textbf{可逼近的}(\textbf{Chebyshev集}). 易知, 自反空间$X$中的闭凸集合$K$都是可逼近的. 进一步, 如果范数是严格凸的, 则$K$是Chebyshev集. 但是, 如果$X$不是自反的或$K$不是凸的, 那么以上结论通常不成立.
\end{remark}


满足 \eqref{infsupequiv2} 的$u$依赖于$v$. 一种特殊情形是当$U=V$时$u = v$,  相应的结果就是Lax-Milgram引理.

\begin{lemma}[Lax-Milgram引理, \cite{LaxMilgram1954}]\label{laxmilgramlemma}
设$V\times V$上的双线性形式$a(\cdot, \cdot)$满足
\begin{enumerate}
\item 连续性: $a(u, v)\leq\beta\|u\|_V\|v\|_V$;
\item 强制性: $a(u, u)\geq\alpha\|u\|_V^2$,
\end{enumerate}
则对任意的$f\in V'$, 存在唯一的$u\in V$满足
\[
a(u, v) = \langle f, v\rangle,
\]
且
\[
\|u\|_V\leq\frac{1}{\alpha}\|f\|_{V'}.
\]
\end{lemma}

最简单的情形是双线性形式$a(\cdot, \cdot)$在$V$上对称正定的. 此时, $a(\cdot, \cdot)$定义了一个新的内积, Lax-Milgram引理就是Riesz表示定理.



% \subsection{变分问题的离散}

% 给定变分问题:找$u\in U$使得
% \begin{equation}\label{variationalproblemabstract}
% a(u, v) = \langle f, v\rangle\quad\forall~v\in V.
% \end{equation}

\subsection{变分问题的离散:Ritz方法}
假设$a(\cdot,\cdot)$是对称正定的, 且$U=V$. 抽象Galerkin变分问题 \eqref{variationalproblemabstract} 的Ritz变分问题为:找$u\in U$, 使得
\begin{equation}\label{ritzvariationalproblemabstract}
J(u)=\min_{w\in U}J(w),
\end{equation}
其中$\displaystyle J(u)=\frac{1}{2}a(u, u)-\langle f, u\rangle$.

因为$U$通常是无穷维的函数空间, 直接求解Ritz变分问题 \eqref{ritzvariationalproblemabstract} 非常困难, 为此需要近似求解 \eqref{ritzvariationalproblemabstract}. 设有限维子空间$U_h\subset U$是$U$的协调离散, 称$U_h$为\textbf{试探函数空间}. Ritz变分问题 \eqref{ritzvariationalproblemabstract} 的近似问题是:
找$u_h\in U_h$, 使得
\begin{equation}\label{ritzmethod}
J(u_h)=\min_{w_h\in U_h}J(w_h).
\end{equation}
离散问题 \eqref{ritzmethod} 称为\textbf{Ritz方法}.

设$U_h$的一组基函数为$\varphi_1,\ldots, \varphi_N$, 即$U_h=\textrm{span}\{\varphi_1,\ldots, \varphi_N\}$. 对任意的$w_h\in U_h$, 有线性表示
$$
w_h=\sum_{i=1}^Nc_i\varphi_i,
$$
其中系数$c_1,\ldots, c_n$是常数. 于是
$$
J(w_h)=\frac{1}{2}\sum_{i,j=1}^Na(\varphi_i, \varphi_j)c_ic_j - \sum_{i=1}^N\langle f, \varphi_i\rangle c_i.
$$
令$A=(a_{ij})_{N\times N}$, $b=(b_i)_{N\times1}$, $c=(c_i)_{N\times1}$, 其中$a_{ij}=a(\varphi_i, \varphi_j)$, $b_i=\langle f, \varphi_i\rangle$, 则
$$
J(w_h)=\frac{1}{2}c^{\intercal}Ac - c^{\intercal}b.
$$
由此, Ritz方法 \eqref{ritzmethod} 等价于以$c$为自变量的二次函数的极值问题
$$
\min_{c\in \mathbb R^N}\frac{1}{2}c^{\intercal}Ac - c^{\intercal}b.
$$
设$u_h=(\varphi_1, \ldots,\varphi_N)c^{*}$, 其中$c^{*}\in\mathbb R^N$. 于是向量$c^{*}$满足线性代数方程组
$$
Ac^{*}=b.
$$

\subsection{变分问题的离散:Galerkin方法}

考虑变分问题 \eqref{variationalproblemabstract} 在有限维子空间 $U_h\subset U$ 和 $V_h\subset V$ 上的协调离散, 称 $V_h$ 为\textbf{检验函数空间}.
变分问题 \eqref{variationalproblemabstract} 的离散方法为:找 $u_h\in U_h$ 使得
\begin{equation}\label{discetemethodabstract}
a(u_h, v_h) = \langle f, v_h\rangle\quad\forall~ v_h\in V_h.
\end{equation}
若 $U_h\neq V_h$, 称离散方法 \eqref{discetemethodabstract} 为\textbf{Petrov-Galerkin方法};若 $U_h=V_h$, 称离散方法 \eqref{discetemethodabstract} 为\textbf{Galerkin方法}.
离散问题 \eqref{discetemethodabstract} 的存在唯一性等价于以下离散 inf-sup 条件:
\begin{equation}\label{disceteinfsupabstract}
\inf_{v_h\in V_h}\sup_{u_h\in U_h}\frac{a(u_h, v_h)}{\|u_h\|_U\|v_h\|_V}=\inf_{u_h\in U_h}\sup_{v_h\in V_h}\frac{a(u_h, v_h)}{\|u_h\|_U\|v_h\|_V}=\alpha_h>0.
\end{equation}

选取有限维空间 $U_h$ 和 $V_h$ 的基函数, 设 $U_h=\textrm{span}\{\varphi_1,\ldots, \varphi_N\}$, $V_h=\textrm{span}\{\psi_1,\ldots, \psi_N\}$.
设$\displaystyle u_h=\sum_{j=1}^Nc_j^{*}\varphi_j$, 其中$c^{*}=(c_1^{*}, \ldots, c_N^{*})^{\intercal}\in\mathbb R^N$. 
代入离散方法 \eqref{discetemethodabstract}, 并取 $v_h=\psi_i$ ($i=1,\ldots, N$), 得
$$
\sum_{j=1}^Na(\varphi_j, \psi_i)c_j^{*}=\langle f, \psi_i\rangle, \quad i=1,\ldots, N.
$$
令$A=(a_{ij})_{N\times N}$, $b=(b_i)_{N\times1}$, 其中$a_{ij}=a(\varphi_j, \psi_i)$, $b_i=\langle f, \psi_i\rangle$.
由此, 离散方法 \eqref{discetemethodabstract} 转化成线性代数方程组
$$
Ac^{*}=b.
$$
考虑到适定性, 要求系数矩阵$A$是可逆的. 为确保一致稳定性, 要求常数 $\alpha_h$ 关于 $h$ 一致有下界, 且下界为正.

在双线性形式 $a(\cdot,\cdot)$ 是对称正定及 $U_h=V_h$ 的情形下, Ritz 方法 \eqref{ritzmethod} 和 Galerkin 方法 \eqref{discetemethodabstract} 是等价的. 如果双线性形式 $a(\cdot,\cdot)$ 不是对称的, 离散方法 \eqref{discetemethodabstract} 仍可以用. 

利用inf-sup条件可以建立离散方法 \eqref{discetemethodabstract} 的误差估计. 协调离散的关键性质是Garlerkin正交性
\[
a(u-u_h, v_h) = 0\quad \forall~v_h\in V_h.
\]
引入投影算子 $P_h : U\to U_h$, 
\begin{equation*}
a(P_hu, v_h)=a(u,u_h)\qquad\forall~u\in U, v_h\in V_h.
\end{equation*}
显然, $u_h = P_hu$, 并且 $P_h$ 限制在 $U_h$ 上是恒等算子.
\begin{lemma}
成立
\begin{equation*}
A_hP_hu = I'Au\quad\forall~u\in U,
\end{equation*}
其中 $I': V'\to V_h'$ 是嵌入算子 $I: V_h\to V$ 的对偶,  算子 $A_h: U_h\to V_h'$ 定义为
\[
\langle A_hu_h, v_h\rangle=a(u_h, v_h)\quad\forall~u_h\in U_h, v_h\in V_h.
\]
\end{lemma} 
\begin{proof}
由 $P_h$ 和 $I$ 的定义, 有
\begin{align*}
a(P_hu, v_h)=a(u, v_h)=a(u, Iv_h)=\langle Au, Iv_h\rangle_{V'\times V}=\langle I'Au, v_h\rangle_{V_h'\times V_h}.
\end{align*}
结合 $A_h$ 的定义, 即得证.
\end{proof}

\begin{lemma}
投影算子 $P_h : U\to U_h$ 是有界的:
\begin{equation}\label{eq:Phbounded}
\|P_hu\|_U\leq \frac{\|a\|}{\alpha_h}\|u\|_U\quad\forall~u\in U.
\end{equation}
\end{lemma}
\begin{proof}
由离散 inf-sup 条件 \eqref{disceteinfsupabstract} 和双线性形式 $a(\cdot, \cdot)$ 的有界性 \eqref{abounded}, 有
\[
\|P_hu\|_U\leq\frac{1}{\alpha_h}\sup_{v_h\in V_h}\frac{a(P_hu, v_h)}{\|v_h\|_V}=\frac{1}{\alpha_h}\sup_{v_h\in V_h}\frac{a(u, v_h)}{\|v_h\|_V}\leq \frac{1}{\alpha_h}\sup_{v\in V}\frac{a(u, v)}{\|v\|_V}\leq\frac{\|a\|}{\alpha_h}\|u\|_U.
\]
证毕.
\end{proof}

\begin{lemma}
设 $U_h$ 是 $U$ 的非零子空间. 成立等式
\begin{equation}\label{eq:PhIPhequal}
\|I-P_h\|=\|P_h\|.
\end{equation}
\end{lemma}
\begin{proof}
由假设 $U_h$ 是 $U$ 的非零子空间可知, $P_h\neq 0$ 且 $P_h\neq I$. 又 $P_h^2=P_h$, 应用文献 \cite{XuZikatanov2003} 中的引理~5 可得等式~\eqref{eq:PhIPhequal}.
\end{proof}

\begin{theorem}[C\'ea引理]
设双线性形式$a(\cdot, \cdot)$满足 \eqref{abounded}, \eqref{infsup1}, \eqref{infsup2} 和 \eqref{disceteinfsupabstract}. 变分问题 \eqref{variationalproblemabstract} 存在唯一解 $u\in U$, 离散问题 \eqref{discetemethodabstract} 存在唯一解 $u_h\in U_h$, 并成立
\[
\|u-u_h\|_U\leq\frac{\|a\|}{\alpha_h}\inf_{v_h\in U_h}\|u-v_h\|_U.
\]
\end{theorem}
\begin{proof}
在定理假设下, 对于 $f\in V'$, 变分问题 \eqref{variationalproblemabstract} 和离散问题 \eqref{discetemethodabstract} 都是适定的. 
% 引入投影算子$P_h : U\to U_h$, 定义为$P_hu = u_h$.
% 显然, $P_h$限制在$U_h$上是恒等算子. 易知$P_h$的算子表达式为$P_h = A_h^{-1}I'A$, 其中$I': V'\to V_h'$是嵌入算子$I: V_h\to V$的对偶,  算子$A_h: U_h\to V_h'$定义为
% \[
% \langle A_hu_h, v_h\rangle=a(u_h, v_h)\quad\forall~u_h\in U_h, v_h\in V_h.
% \]
% 如下证明$P_h$是有界线性算子, 且$\|P_h\| \leq\|a\|/\alpha_h$:
% \[
% \|u_h\|_U\leq\frac{1}{\alpha_h}\sup_{v_h\in V_h}\frac{a(u_h, v_h)}{\|v_h\|_V}=\frac{1}{\alpha_h}\sup_{v_h\in V_h}\frac{a(u, v_h)}{\|v_h\|_V}\leq \frac{1}{\alpha_h}\sup_{v\in V}\frac{a(u, v)}{\|v\|_V}\leq\frac{\|a\|}{\alpha_h}\|u\|_U.
% \]
对任意的 $w_h\in U_h$, 注意到 $P_hw_h = w_h$, 有
\[
\|u-u_h\|_U=\|(I-P_h)(u-w_h)\|_U\leq\|I-P_h\|\|u-w_h\|_U.
\]
应用等式~\eqref{eq:PhIPhequal} 和 \eqref{eq:Phbounded}, 
即可完成证明.
\end{proof}



% \section{鞍点系统的 Inf-sup 条件:Brezzi 理论}
\section{鞍点系统:Brezzi 理论}

\subsection{混合变分问题}
首先考虑一个抽象混合变分问题。设 $V$ 和 $P$ 为两个 Banach 空间。对于给定的 $(f, g)\in V'\times P'$,求 $(u, p)\in V \times P$ 使得
\begin{subequations}\label{abstractmixed}
\begin{align}
a(u, v)+ b(v, p) & =\langle f, v\rangle \quad \forall~v\in V, \label{abstractmixed1}\\
b(u, q) & =\langle g, q\rangle  \quad \forall~q\in P.\label{abstractmixed2}
\end{align}
\end{subequations}
现引入线性算子
\[
A : V \to V',\quad \textrm{ 定义为 }\; \langle Au, v\rangle = a(u, v)
\]
与
\[
B: V\to P', \quad B' : P \to V',\quad \textrm{ 满足 }\; \langle Bv, q\rangle = \langle v, B'q\rangle = b(v, q).
\]
抽象混合变分问题 \eqref{abstractmixed} 对应的算子方程为
\begin{align}
Au+ B'p & =f, \label{abstractmixedoperator1}\\
Bu & =g, \label{abstractmixedoperator2}
\end{align}
或简记为
\begin{equation*}%\label{abstractmixedoperatormatrix}
\begin{pmatrix}
A &  B' \\
B & 0
\end{pmatrix}
\begin{pmatrix}
u \\ p
\end{pmatrix}
=
\begin{pmatrix}
f \\ g
\end{pmatrix}.
\end{equation*}


\subsection{Inf-sup 条件}

我们将研究该抽象混合变分问题的适定性。

首先假设所有双线性形式均连续,即算子 $A, B, B'$ 连续:双线性形式 $a(\cdot, \cdot)$ 与 $b(\cdot, \cdot)$ 满足连续性条件
\begin{equation}\label{acontinuous}
a(u, v)\leq C\|u\|_V\|v\|_V \quad \forall~ u, v\in V,
\end{equation}
\begin{equation}\label{bcontinuous}
b(v, q)\leq C\|v\|_V\|q\|_P \quad \forall~ v\in V, q\in P.
\end{equation}

第二个方程 \eqref{abstractmixedoperator2} 的可解性等价于算子 $B$ 为满射(或算子 $B'$ 为单射且值域 $R(B')$ 闭),这又等价于如下 inf-sup 条件
\begin{equation}\label{infsup}
\inf_{q\in P}\sup_{v\in V}\frac{b(v, q)}{\|v\|_V\|q\|_P}=\beta>0.
\end{equation}
在条件 \eqref{infsup} 下,$B : V/\ker(B)\to P'$ 为同构映射。因此对任意 $g\in P'$,存在 $u_1\in V/\ker(B)$ 使得 $Bu_1 = g$ 且 $\|u_1\|_V\leq \beta^{-1}\|g\|_{P'}$。

确定唯一的 $u_1$ 后,我们将测试函数 $v$ 限制在 $\ker(B)$ 上考虑 \eqref{abstractmixed1}。由于对 $v\in \ker(B)$ 有 $\langle v, B'q\rangle = \langle Bv, q\rangle = 0$,得到如下变分问题:求 $u_0\in \ker(B)$ 使得
\begin{equation}\label{abstractkernelproblem}
a(u_0, v) = \langle f, v\rangle- a(u_1, v)\quad \forall~v \in \ker(B).
\end{equation}
变分问题 \eqref{abstractkernelproblem} 的适定性等价于双线性形式 $a(u, v)$ 在空间 $Z = \ker(B)$ 上满足两个 inf-sup 条件:
\begin{equation}\label{infsupkernel}
\inf_{u\in Z}\sup_{v\in Z}\frac{a(u, v)}{\|u\|_V\|v\|_V}=\inf_{v\in Z}\sup_{u\in Z}\frac{a(u, v)}{\|u\|_V\|v\|_V}=\alpha>0.
\end{equation}

如此确定唯一的 $u = u_0 + u_1$ 后,通过求解
\[
B'p = f - Au
\]
得到 $p$。由于 $u_0$ 是 \eqref{abstractkernelproblem} 的解,右端项 $f - Au \in \ker(B)^{\perp}$。因此我们要求 $B': P\to \ker(B)^{\perp}$ 为同构映射,这也等价于条件 \eqref{infsup}。

\begin{theorem}
设双线性形式 $a(\cdot, \cdot)$ 和 $b(\cdot, \cdot)$ 连续(即 \eqref{acontinuous}-\eqref{bcontinuous} 成立),则混合变分问题 \eqref{abstractmixed1}-\eqref{abstractmixed2} 适定的充要条件是 \eqref{infsup} 与 \eqref{infsupkernel} 同时成立。当这两个条件满足时,我们有稳定性估计
\[
\|u\|_V+\|p\|_P\lesssim \|f\|_{V'}+\|g\|_{P'}.
\]
\end{theorem}

以下关于算子 $B$ 的 inf-sup 条件的刻画非常实用。其验证过程可转化为构造一个合适的函数。该证明与定理~\ref{thm:infsupequiv2} 的证明类似,故在此从略。
\begin{theorem}
Inf-sup 条件 \eqref{infsup} 等价于:对任意 $q\in P$,存在 $v\in V$ 使得
\begin{equation*}%\label{infsupequiv2}
b(v, q)\geq C_1\|q\|_P^2,\quad\textrm{且 } \quad\|v\|_V\leq C_2\|q\|_P.
\end{equation*}
\end{theorem}

需要指出的是,一般而言,构造符合要求的 $v = v(q)$(尤其是对其范数 $\|v\|_V$ 的控制)可能并非易事。


\subsection{协调离散}

我们考虑混合问题的有限元逼近:求 $u_h\in V_h$ 和 $p_h\in P_h$ 使得
\begin{align}
a(u_h, v_h)+ b(v_h, p_h) & =\langle f, v_h\rangle \quad \forall~v_h\in V_h, \label{abstractmfem1}\\
b(u_h, q_h) & =\langle g, q_h\rangle  \quad \forall~q_h\in P_h.\label{abstractmfem2}
\end{align}
考虑协调有限元情形 $V_h\subset V$ 和 $P_h\subset P$。记 $B_h: V_h\to P_h'$,并令 $Z_h = \ker(B_h)$。回顾 $Z = \ker(B)$,在应用于 Stokes 方程时 $B = -\div$,因此 $Z$ 称为无散空间,$Z_h$ 称为离散无散空间。

\begin{remark}
一般而言 $Z_h \not\subset Z$,即离散无散函数未必严格无散。只需比较 $B_hu_h = 0$(在 $(P_h)'$ 中)
\[
\langle B_hu_h, q_h\rangle=0\quad\forall~q_h\in P_h,
\]
与 $Bu_h=0$(在 $P'$ 中)
\[
\langle Bu_h, q\rangle=0\quad\forall~q\in P.
\]
\end{remark}

有限元逼近的离散 inf-sup 条件为:
\begin{equation}\label{discreteinfsupkernel}
\inf_{u_h\in Z_h}\sup_{v_h\in Z_h}\frac{a(u_h, v_h)}{\|u_h\|_V\|v_h\|_V}=\inf_{v_h\in Z_h}\sup_{u_h\in Z_h}\frac{a(u_h, v_h)}{\|u_h\|_V\|v_h\|_V}=\alpha_h>0,
\end{equation}
\begin{equation}\label{discreteinfsup}
\inf_{q_h\in P_h}\sup_{v_h\in V_h}\frac{b(v_h, q_h)}{\|v_h\|_V\|q_h\|_P}=\beta_h>0.
\end{equation}

\begin{theorem}
若 \eqref{acontinuous}-\eqref{bcontinuous}、\eqref{infsup}、\eqref{infsupkernel}、\eqref{discreteinfsupkernel} 及 \eqref{discreteinfsup} 成立,则离散问题适定,且成立误差估计
\[
\|u-u_h\|_V+\|p-p_h\|_P\leq C\inf_{v_h\in V_h, q_h\in P_h}(\|u-v_h\|_V+\|p-q_h\|_P).
\]
\end{theorem}

\begin{exe}
设 $U = V\times P$,并用定义在 $U$ 上的单个双线性形式重写混合变分形式。随后应用 Babu\v{s}ka 理论证明上述定理,并明确写出常数 $C$ 如何依赖于各 inf-sup 条件中的常数。
\end{exe}


\subsection{Fortin 算子}
注意到连续层面的 inf-sup 条件 \eqref{infsup} 蕴含:对任意 $q_h\in P_h$,存在 $v \in V$ 使得
$$
b(v, q_h)\geq\beta\|v\|_V\|q_h\|_P, \quad \|v\|_V\leq C\|q_h\|_P.
$$
为建立离散 inf-sup 条件,我们需要构造满足相应性质的 $v_h\in V_h$。一种方法是利用所谓的 Fortin 算子 \cite{Fortin1977},通过 $v$ 构造出所需的 $v_h$。

\begin{defn}[Fortin 算子]
线性算子 $\Pi_h: V\to V_h$ 称为 Fortin 算子,若其满足:
\begin{enumerate}
\item 对任意 $q_h\in P_h$,$b(\Pi_hv, q_h) = b(v, q_h)$;
\item $|\Pi_hv|_V\leq C|v|_V$。
\end{enumerate}
\end{defn}

\begin{theorem}
若 inf-sup 条件 \eqref{infsup} 成立且存在 Fortin 算子 $\Pi_h$,则离散 inf-sup 条件 \eqref{discreteinfsup} 成立。
\end{theorem}
\begin{proof}
由 inf-sup 条件 \eqref{infsup} 可得
\[
\beta\leq \inf_{q_h\in P_h}\sup_{v\in V}\frac{b(v, q_h)}{\|v\|_V\|q_h\|_P}\leq C\inf_{q_h\in P_h}\sup_{v\in V}\frac{b(\Pi_hv, q_h)}{\|\Pi_hv\|_V\|q_h\|_P}\leq C\inf_{q_h\in P_h}\sup_{v_h\in V_h}\frac{b(v_h, q_h)}{\|v_h\|_V\|q_h\|_P}
\]
因此离散 inf-sup 条件 \eqref{discreteinfsup} 成立,且可取 $\beta_h=\beta/C$。
\end{proof}


% \section{鞍点系统的 Inf-sup 条件:Zulehner 理论}
\section{鞍点系统:Zulehner 理论}

We refer to \cite{Zulehner2011} for details in this section.

We shall consider an abstract mixed variational
problem first. Let $V$ and $P$ be two Hilbert spaces. For given $(f, g)\in V'\times P'$, find
$(u, p)\in V \times P$ such that
\begin{align}
a(u, v)+ b(v, p) & =f(v) \quad \forall~v\in V, \label{abstractmixed3}\\
b(u, q)-c(p,q) & =g(q)  \quad \forall~q\in P,\label{abstractmixed4}
\end{align}
where $a$, $b$, and $c$ are bounded bilinear forms on $V\times V$ , $V\times P$, and $P\times P$, respectively.
Assume in this section that $a$ and $c$ are symmetric, i.e.,
\[
a(w,v)=a(v,w)\quad\forall~w,v\in V,\quad c(p,q)=c(q,p)\quad\forall~p,q\in P,
\]
and $a$ and $c$ are nonnegative, i.e.,
\[
a(v,v)\geq0\quad\forall~v\in V, \quad c(q,q)\geq0\quad\forall~q\in P.
\]
Thus \eqref{abstractmixed3}-\eqref{abstractmixed4} is a symmetric and indefinite problem.

The mixed variational problem \eqref{abstractmixed3}-\eqref{abstractmixed4} in $V$ and $P$ can also be written as a variational problem on the product space $X=V\times P$. Find $x=(u,p)\in X$ such that
\begin{equation}\label{eq:BxyFy}
\mathcal B(x, y) =\mathcal F(y) \quad\forall~y\in X
\end{equation}
with
\[
\mathcal B(x, y)=a(u,v)+b(v,p)+b(u,q)-c(p,q) ,\quad\mathcal F(y)=f(v)+g(q)
\]
for $y=(v,q)$.
Since we have assumed that the bilinear forms $a$, $b$, and $c$ are bounded, there is a
constant $\overline{C}$ such that
\begin{equation}\label{eq:Bsupsup}
\sup_{y\in X}\sup_{x\in X}\frac{\mathcal B(x, y)}{\|x\|_X\|y\|_X}\leq \overline{C}.
\end{equation}
By the Babu\v{s}ka theory, problem \eqref{eq:BxyFy} is well-posed if and only if there is a constant $\underline{C}>0$ such that
\begin{equation}\label{eq:Binfsup}
\inf_{y\in X}\sup_{x\in X}\frac{\mathcal B(x, y)}{\|x\|_X\|y\|_X}\geq \underline{C}.
\end{equation}
Then the following estimate holds
\[
\frac{1}{\overline{C}}\|\mathcal F\|_{X'}\leq \|x\|_X\leq \frac{1}{\underline{C}}\|\mathcal F\|_{X'}.
\]

Let us introduce linear operators
\[
A : V \to V',\quad \textrm{ as } \langle Au, v\rangle = a(u, v),
\]
\[
B: V\to P', \quad B' : P \to V',\quad \textrm{ as } \langle Bv, q\rangle = \langle v, B'q\rangle = b(v, q).
\]
and
\[
C : P \to P',\quad \textrm{ as } \langle Cp, q\rangle = c(p, q).
\]

Written in the operator form, the problem becomes
\begin{align}
Au+ B'p & =f, \label{abstractmixedoperator3}\\
Bu-Cp & =g, \label{abstractmixedoperator4}
\end{align}
or in short
\begin{equation*}%\label{abstractmixedoperatormatrix}
\begin{pmatrix}
A &  B' \\
B & -C
\end{pmatrix}
\begin{pmatrix}
u \\ p
\end{pmatrix}
=
\begin{pmatrix}
f \\ g
\end{pmatrix}.
\end{equation*}
In a similar way, we associate a linear operator $\mathcal A\in L(X, X')$ to the bilinear form $\mathcal B$, given by
\[
\langle\mathcal Ax, y\rangle = \mathcal B(x,y).
\]
Then problem \eqref{eq:BxyFy}, which is equivalent to \eqref{abstractmixed3}-\eqref{abstractmixed4}, reads
\begin{equation}\label{eq:operatormathcalAF}
\mathcal Ax=\mathcal F.
\end{equation}
In operator notation the conditions \eqref{eq:Bsupsup} and \eqref{eq:Binfsup} can be written in the
following form
\begin{equation}\label{eq:operatormathcalAsupinf}
\underline{C}\|y\|_X\leq\|\mathcal A y\|_{X'}\leq \overline{C}\|y\|_X \quad \forall~y\in X.
\end{equation}

\begin{lemma}\label{lem:zulehnerrhs}
Let $f\in V'$ and $g\in P'$. Let $\mathcal F\in X'$ be given by $\mathcal F(v, q)=f(v)+g(q)$. Then
\[
\|\mathcal F\|_{X'}^2=\|f\|_{V'}^2+\|g\|_{P'}^2.
\]
\end{lemma}
\begin{proof}
It holds
\begin{align*}
\|\mathcal F\|_{X'}^2&=\sup_{(v,q)\in X}\frac{(f(v)+g(q))^2}{\|(v,q)\|_X^2}=\sup_{(v,q)\in X}\frac{(f(v)+g(q))^2}{\|v\|_V^2+\|q\|_P^2} \\
&\leq\sup_{(v,q)\in X}\frac{(\|f\|_{V'}\|v\|_{V}+\|g\|_{P'}\|q\|_{P})^2}{\|v\|_V^2+\|q\|_P^2}\leq \|f\|_{V'}^2+\|g\|_{P'}^2.
\end{align*}
Since both $V$ and $P$ are Hilbert spaces, by the Riesz-isomorphism, there exist $w\in V$ and $p\in P$ such that
\[
f(v)=(w, v)_V \quad \forall~v\in V, \quad  \|f\|_{V'}=\|w\|_{V}, 
\]
\[
g(q)=(p, q)_P \quad \forall~q\in P, \quad \|g\|_{P'}=\|p\|_{P}.
\]
Hence
\[
\frac{(f(w)+g(p))^2}{\|w\|_V^2+\|p\|_P^2} = \|w\|_{V}^2+\|p\|_{P}^2=\|f\|_{V'}^2+\|g\|_{P'}^2.
\]
This end the proof.
\end{proof}

As a consequence of an estimate of the form \eqref{eq:operatormathcalAsupinf} in $X$ we obtain two simple estimates, one in $V$ and one in $P$.
\begin{theorem}
If \eqref{eq:operatormathcalAsupinf} holds for constants $\underline{C}, \overline{C}>0$, then
\begin{equation}\label{eq:zulehnerinfsup1}
\underline{C}^2\|v\|_V^2\leq \|Av\|_{V'}^2 + \|Bv\|_{P'}^2 \leq \overline{C}^2\|v\|_V^2\quad \forall~v\in V,
\end{equation}
\begin{equation}\label{eq:zulehnerinfsup2}
\underline{C}^2\|q\|_P^2\leq \|Cq\|_{P'}^2 + \|B'q\|_{V'}^2 \leq \overline{C}^2\|q\|_P^2\quad \forall~q\in P.
\end{equation}
\end{theorem}
\begin{proof}
For $y=(v, q)$, we have
\[
\|\mathcal Ay\|_{X'}=\sup_{(w,r)\in X}\frac{\mathcal B((v,q), (w,r))}{\|(w,r)\|_{X}}
=\sup_{(w,r)\in X}\frac{a(v,w)+b(w,q)+b(v,r)-c(q,r)}{\|(w,r)\|_{X}}.
\]
Since
\[
a(v,w)+b(w,q)=\langle Av+B'q , w\rangle, \quad b(v,r)-c(q,r)=\langle Bv-Cq, r\rangle,
\]
we obtain from Lemma~\ref{lem:zulehnerrhs} that
\[
\|\mathcal Ay\|_{X'}^2 = \|Av+B'q\|_{V'}^2 + \|Bv-Cq\|_{P'}^2.
\]
Then the estimates \eqref{eq:zulehnerinfsup1} and \eqref{eq:zulehnerinfsup2} immediately follow from \eqref{eq:operatormathcalAsupinf} for $q=0$ and for $v=0$, respectively.
\end{proof}

So, \eqref{eq:zulehnerinfsup1} and \eqref{eq:zulehnerinfsup2} are necessary conditions for \eqref{eq:operatormathcalAsupinf}. Next we will show that \eqref{eq:zulehnerinfsup1} and \eqref{eq:zulehnerinfsup2}, not necessarily with the same constants, are also sufficient.

\begin{theorem}
If there are constants $\underline{C}_v, \overline{C}_v, \underline{C}_p, \overline{C}_p>0$ such that
\begin{equation*}
\underline{C}_v^2\|v\|_V^2\leq \|Av\|_{V'}^2 + \|Bv\|_{P'}^2 \leq \overline{C}_v^2\|v\|_V^2\quad \forall~v\in V,
\end{equation*}
\begin{equation*}
\underline{C}_p^2\|q\|_P^2\leq \|Cq\|_{P'}^2 + \|B'q\|_{V'}^2 \leq \overline{C}_p^2\|q\|_P^2\quad \forall~q\in P,
\end{equation*}
then there are constants $\underline{C}, \overline{C}>0$ such that
\[
\underline{C}\|y\|_X\leq\|\mathcal A y\|_{X'}\leq \overline{C}\|y\|_X \quad \forall~y\in X,
\]
where $\underline{C}$ and $\overline{C}>0$ depend only on $\underline{C}_v$, $\overline{C}_v$, $\underline{C}_p$ and $\overline{C}_p$.
\end{theorem}
\begin{proof}
For $y=(v, q)$, we have
\begin{align*}
\|\mathcal Ay\|_{X'}^2  & = \|Av+B'q\|_{V'}^2 + \|Bv-Cq\|_{P'}^2 \leq 2(\|Av\|_{V'}^2 + \|B'q\|_{V'}^2 + \|Bv\|_{P'}^2 + \|Cq\|_{P'}^2) \\
&\leq 2(\overline{C}_v^2\|v\|_V^2 + \overline{C}_p^2\|q\|_P^2)\leq 2\max\{\overline{C}_v^2, \overline{C}_p^2\}\|y\|_X^2,
\end{align*}
which proves the upper bound in \eqref{eq:operatormathcalAsupinf} with $\overline{C}^2=2\max\{\overline{C}_v^2, \overline{C}_p^2\}$.

For showing a lower bound, we start with the following estimate based on the triangle inequality in $X'$
\begin{align*}
\|\mathcal Ay\|_{X'}  & = \left(\|Av+B'q\|_{V'}^2 + \|Bv-Cq\|_{P'}^2\right)^{1/2} \geq \left(\|B'q\|_{V'}^2 + \|Bv\|_{P'}^2\right)^{1/2} - \left(\|Av\|_{V'}^2 + \|Cq\|_{P'}^2\right)^{1/2} \\
&=\frac{\left(\|B'q\|_{V'}^2 + \|Bv\|_{P'}^2\right) - \left(\|Av\|_{V'}^2 + \|Cq\|_{P'}^2\right)}{\left(\|B'q\|_{V'}^2 + \|Bv\|_{P'}^2\right)^{1/2} + \left(\|Av\|_{V'}^2 + \|Cq\|_{P'}^2\right)^{1/2}}
\end{align*}
for $y=(v, q)$.
A second lower bound follows from
\[
\|\mathcal Ay\|_{X'}=\sup_{(w,r)\in X}\frac{\mathcal B((v,q), (w,r))}{\|(w,r)\|_{X}}\geq \frac{\mathcal B((v,q), (v,-q))}{\|(v,-q)\|_{X}} = \frac{a(v,v)+c(q, q)}{\|y\|_{X}}.
\]
Since
\[
a(v,w)^2\leq a(v,v)a(w,w)\leq a(v,v)\|Aw\|_{V'}\|w\|_{V}\leq\overline{C}_va(v,v)\|w\|_{V}^2,
\]
we have
\[
\|Av\|_{V'}^2=\sup_{w\in V}\frac{a(v,w)^2}{\|w\|_V^2}\leq\overline{C}_va(v,v).
\]
Analogously, we obtain
\[
\|Cq\|_{P'}^2\leq\overline{C}_pc(q,q).
\]
Hence
\begin{align*}
a(v,v) + c(q,q)  & \geq \frac{1}{\overline{C}_v}\|Av\|_{V'}^2 + \frac{1}{\overline{C}_p}\|Cq\|_{P'}^2\geq \frac{1}{\max\{\overline{C}_v, \overline{C}_p\}}(\|Av\|_{V'}^2+\|Cq\|_{P'}^2).
\end{align*}
With this estimate and
\begin{align*}
\|y\|_X&\leq\frac{1}{\min\{\underline{C}_v,\underline{C}_p\}}\left(\|B'q\|_{V'}^2 + \|Bv\|_{P'}^2+\|Av\|_{V'}^2 + \|Cq\|_{P'}^2\right)^{1/2} \\
&\leq\frac{1}{\min\{\underline{C}_v,\underline{C}_p\}}\left(\|B'q\|_{V'}^2 + \|Bv\|_{P'}^2\right)^{1/2} + \frac{1}{\min\{\underline{C}_v,\underline{C}_p\}}\left(\|Av\|_{V'}^2 + \|Cq\|_{P'}^2\right)^{1/2},
\end{align*}
we obtain for the second lower bound
\[
\|\mathcal Ay\|_{X'}\geq \frac{\min\{\underline{C}_v,\underline{C}_p\}}{\max\{\overline{C}_v, \overline{C}_p\}}\frac{\|Av\|_{V'}^2+\|Cq\|_{P'}^2}{\left(\|B'q\|_{V'}^2 + \|Bv\|_{P'}^2\right)^{1/2} + \left(\|Av\|_{V'}^2 + \|Cq\|_{P'}^2\right)^{1/2}}.
\]
Taking a weighted sum of the two lower bounds we immediately obtain
\begin{align*}
\|\mathcal Ay\|_{X'}&=\frac{\min\{\underline{C}_v,\underline{C}_p\}}{\min\{\underline{C}_v,\underline{C}_p\}+2\max\{\overline{C}_v, \overline{C}_p\}}\|\mathcal Ay\|_{X'}+\frac{2\max\{\overline{C}_v, \overline{C}_p\}}{\min\{\underline{C}_v,\underline{C}_p\}+2\max\{\overline{C}_v, \overline{C}_p\}}\|\mathcal Ay\|_{X'} \\
&\geq\frac{\min\{\underline{C}_v,\underline{C}_p\}}{\min\{\underline{C}_v,\underline{C}_p\}+2\max\{\overline{C}_v, \overline{C}_p\}}\frac{\|B'q\|_{V'}^2 + \|Bv\|_{P'}^2 + \|Av\|_{V'}^2 + \|Cq\|_{P'}^2}{\left(\|B'q\|_{V'}^2 + \|Bv\|_{P'}^2\right)^{1/2} + \left(\|Av\|_{V'}^2 + \|Cq\|_{P'}^2\right)^{1/2}} \\
&\geq\frac{\min\{\underline{C}_v,\underline{C}_p\}}{\sqrt{2}\min\{\underline{C}_v,\underline{C}_p\}+2\sqrt{2}\max\{\overline{C}_v, \overline{C}_p\}}\left(\|B'q\|_{V'}^2 + \|Bv\|_{P'}^2 + \|Av\|_{V'}^2 + \|Cq\|_{P'}^2\right)^{1/2}.
\end{align*}
Hence
\begin{align*}
\|\mathcal Ay\|_{X'}&\geq \frac{\min\{\underline{C}_v,\underline{C}_p\}}{\sqrt{2}\min\{\underline{C}_v,\underline{C}_p\}+2\sqrt{2}\max\{\overline{C}_v, \overline{C}_p\}}\left(\underline{C}_v^2\|v\|_V^2+\underline{C}_p^2\|q\|_P^2\right)^{1/2}\\
&\geq \frac{\left(\min\{\underline{C}_v,\underline{C}_p\}\right)^2}{\sqrt{2}\min\{\underline{C}_v,\underline{C}_p\}+2\sqrt{2}\max\{\overline{C}_v, \overline{C}_p\}}\|y\|_X,
\end{align*}
which concludes the proof by choosing $\underline{C}=\dfrac{\left(\min\{\underline{C}_v,\underline{C}_p\}\right)^2}{\sqrt{2}\min\{\underline{C}_v,\underline{C}_p\}+2\sqrt{2}\max\{\overline{C}_v, \overline{C}_p\}}$.
\end{proof}
% \begin{proof}
% For $y=(v, q)$, we have
% \begin{align*}
% \|\mathcal Ay\|_{X'}^2  & = \|Av+B'q\|_{V'}^2 + \|Bv-Cq\|_{P'}^2 \leq 2(\|Av\|_{V'}^2 + \|B'q\|_{V'}^2 + \|Bv\|_{P'}^2 + \|Cq\|_{P'}^2) \\
% &\leq 2(\overline{C}_v^2\|v\|_V^2 + \overline{C}_p^2\|q\|_P^2)\leq 2\max\{\overline{C}_v^2, \overline{C}_p^2\}\|y\|_X^2,
% \end{align*}
% which proves the upper bound in \eqref{eq:operatormathcalAsupinf} with $\overline{C}^2=2\max\{\overline{C}_v^2, \overline{C}_p^2\}$.

% For showing a lower bound, we start with the following estimate based on the triangle inequality in $X'$
% \begin{align*}
% \|\mathcal Ay\|_{X'}  & = \left(\|Av+B'q\|_{V'}^2 + \|Bv-Cq\|_{P'}^2\right)^{1/2} \geq \left(\|B'q\|_{V'}^2 + \|Bv\|_{P'}^2\right)^{1/2} - \left(\|Av\|_{V'}^2 + \|Cq\|_{P'}^2\right)^{1/2} \\
% &=(\eta-\xi)\|y\|_{X}
% \end{align*}
% for $y=(v, q)$ with
% \[
% \xi=\frac{\left(\|Av\|_{V'}^2 + \|Cq\|_{P'}^2\right)^{1/2}}{\|(v, q)\|_{X}}, \quad \eta=\frac{\left(\|B'q\|_{V'}^2 + \|Bv\|_{P'}^2\right)^{1/2}}{\|(v, q)\|_{X}}.
% \]
% A second lower bound follows from
% \[
% \|\mathcal Ay\|_{X'}=\sup_{(w,r)\in X}\frac{\mathcal B((v,q), (w,r))}{\|(w,r)\|_{X}}\geq \frac{\mathcal B((v,q), (v,-q))}{\|(v,-q)\|_{X}} = \frac{a(v,v)+c(q, q)}{\|y\|_{X}}.
% \]
% Since
% \[
% a(v,w)^2\leq a(v,v)a(w,w)\leq a(v,v)\|Aw\|_{V'}\|w\|_{V}\leq\overline{C}_va(v,v)\|w\|_{V}^2,
% \]
% we have
% \[
% \|Av\|_{V'}^2=\sup_{w\in V}\frac{a(v,w)^2}{\|w\|_V^2}\leq\overline{C}_va(v,v).
% \]
% Analogously, we obtain
% \[
% \|Cq\|_{P'}^2\leq\overline{C}_pc(q,q).
% \]
% Hence
% \begin{align*}
% a(v,v) + c(q,q)  & \geq \frac{1}{\overline{C}_v}\|Av\|_{V'}^2 + \frac{1}{\overline{C}_p}\|Cq\|_{P'}^2\geq \frac{1}{\max\{\overline{C}_v, \overline{C}_p\}}(\|Av\|_{V'}^2+\|Cq\|_{P'}^2)=\frac{\sqrt{2}}{\overline{C}}\xi^2\|(v,q)\|_X^2.
% \end{align*}
% With this estimate we obtain for the second lower bound
% \[
% \|\mathcal Ay\|_{X'}\geq \frac{\sqrt{2}}{\overline{C}}\xi^2\|y\|_X.
% \]
% Taking the maximum of the two lower bounds we immediately obtain
% \[
% \|\mathcal Ay\|_{X'}\geq \varphi(\xi, \eta)\|y\|_X \quad\textrm{ with } \varphi(\xi, \eta)=\max\left\{\eta-\xi, \frac{\sqrt{2}}{\overline{C}}\xi^2\right\}.
% \]
% Observe that
% \[
% 0<\min\{\underline{C}_v^2,\underline{C}_p^2\}\leq \xi^2+\eta^2\leq \max\{\overline{C}_v^2,\overline{C}_p^2\}.
% \]
% Let
% \[
% \underline{C}=\min\{\varphi(\xi, \eta):  \xi^2+\eta^2\geq \min\{\underline{C}_v^2,\underline{C}_p^2\}, \xi, \eta\geq0\}.
% \]
% Since $\underline{C}$ is achieved when $\eta-\xi=\frac{\sqrt{2}}{\overline{C}}\xi^2$ and $\xi^2+\eta^2=\min\{\underline{C}_v^2,\underline{C}_p^2\}$, we have $\underline{C}>0$,
% %Elementary calculations show that (why?)
% %\[
% %\min\{\varphi(\xi, \eta):  \xi^2+\eta^2\geq \min\{\underline{C}_v^2,\underline{C}_p^2\}, \xi, \eta\geq0\}\geq\frac{3-\sqrt{5}}{2\sqrt{2}}\frac{\min\{\underline{C}_v^2,\underline{C}_p^2\}}{\overline{C}}=\underline{C},
% %\]
% which concludes the proof.
% \end{proof}

In the following two lemmas it will be shown that conditions \eqref{eq:zulehnerinfsup1} and \eqref{eq:zulehnerinfsup2} of the
last theorem can be replaced by two other conditions which will turn out to be easier to work with.

\begin{lemma}
If there are constants $\underline{\gamma}_v, \overline{\gamma}_v>0$ such that
\begin{equation}\label{eq:zulehnerinfsup3}
\underline{\gamma}_v\|v\|_V^2\leq a(v,v)+\|Bv\|_{P'}^2\leq\overline{\gamma}_v\|v\|_V^2\quad\forall~v\in V,
\end{equation}
then \eqref{eq:zulehnerinfsup1} is satisfied with constants $\underline{C}_v, \overline{C}_v>0$ that depend only on $\underline{\gamma}_v, \overline{\gamma}_v$.

And, vice versa, if there are constants $\underline{C}_v, \overline{C}_v>0$ such that \eqref{eq:zulehnerinfsup1} is satisfied, then \eqref{eq:zulehnerinfsup3} is satisfied with constants $\underline{\gamma}_v, \overline{\gamma}_v>0$ that depend only on $\underline{C}_v, \overline{C}_v$.
\end{lemma}
\begin{proof}
Assume that \eqref{eq:zulehnerinfsup3} is satisfied. Then we have
\[
a(v,w)^2\leq a(v,v)a(w,w)\leq \overline{\gamma}_va(v,v)\|w\|_V^2,
\]
which imples
\[
\|Av\|_{V'}^2\leq \overline{\gamma}_va(v,v).
\]
Therefore
\[
\|Av\|_{V'}^2 + \|Bv\|_{P'}^2\leq \overline{\gamma}_va(v,v)+\|Bv\|_{P'}^2\leq \max\{\overline{\gamma}_v, 1\}(a(v,v)+\|Bv\|_{P'}^2)\leq \max\{\overline{\gamma}_v, 1\}\overline{\gamma}_v\|v\|_{V}^2.
\]
This shows the upper bound in \eqref{eq:zulehnerinfsup1} for $\overline{C}_v^2=\max\{\overline{\gamma}_v, 1\}\overline{\gamma}_v$. 

For the lower bound observe that, for all $\varepsilon>0$,
\begin{equation}\label{eq:20200920-1}
a(v,v)\leq\|Av\|_{V'}\|v\|_V\leq\frac{1}{2\varepsilon}\|Av\|_{V'}^2 + \frac{\varepsilon}{2}\|v\|_V^2,
\end{equation}
which imples
\[
\underline{\gamma}_v\|v\|_V^2\leq a(v,v)+\|Bv\|_{P'}^2\leq\frac{1}{2\varepsilon}\|Av\|_{V'}^2 + \frac{\varepsilon}{2}\|v\|_V^2+\|Bv\|_{P'}^2,
\]
and, therefore,
\[
\left(\underline{\gamma}_v-\frac{\varepsilon}{2}\right)\|v\|_V^2 \leq \frac{1}{2\varepsilon}\|Av\|_{V'}^2 + \|Bv\|_{P'}^2 \leq \max\{\frac{1}{2\varepsilon},1\}(\|Av\|_{V'}^2 + \|Bv\|_{P'}^2).
\]
For $\varepsilon=\underline{\gamma}_v$, we obtain the lower bound in \eqref{eq:zulehnerinfsup1}  with $\underline{C}_v=\min\{\underline{\gamma}_v,1/2\}\underline{\gamma}_v$.

Now assume that \eqref{eq:zulehnerinfsup1} is satisfied. Then we have (see the proof of the last theorem)
\[
\|Av\|_{V'}^2\leq\overline{C}_va(v,v),
\]
and, therefore,
\[
a(v,v)+\|Bv\|_{P'}^2\geq\overline{C}_v^{-1}\|Av\|_{V'}^2+\|Bv\|_{P'}^2\geq \min\{1, \overline{C}_v^{-1}\}(\|Av\|_{V'}^2+\|Bv\|_{P'}^2)\geq \min\{1, \overline{C}_v^{-1}\}\underline{C}_v^2\|v\|_{V}^2,
\]
showing the lower bound in \eqref{eq:zulehnerinfsup3} for $\underline{\gamma}_v=\min\{1, \overline{C}_v^{-1}\}\underline{C}_v^2$.

For the upper bound we use \eqref{eq:20200920-1} for $\varepsilon=1/2$ and obtain
\[
a(v, v)+\|Bv\|_{P'}^2\leq \|Av\|_{V'}^2+\frac{1}{4}\|v\|_V^2+\|Bv\|_{P'}^2\leq \left(\overline{C}_v^2+\frac{1}{4}\right)\|v\|_V^2.
\]
So, the upper bound in \eqref{eq:20200920-1} is satisfied for $\overline{\gamma}_v=\overline{C}_v^2+1/4$.
\end{proof}

Completely analogously, we have the following lemma.
\begin{lemma}
If there are constants $\underline{\gamma}_p, \overline{\gamma}_p>0$ such that
\begin{equation}\label{eq:zulehnerinfsup4}
\underline{\gamma}_p\|q\|_P^2\leq c(q,q)+\|B'q\|_{V'}^2\leq\overline{\gamma}_p\|q\|_P^2\quad\forall~q\in P,
\end{equation}
then \eqref{eq:zulehnerinfsup2} is satisfied with constants $\underline{C}_p, \overline{C}_p>0$ that depend only on $\underline{\gamma}_p, \overline{\gamma}_p$.

And, vice versa, if there are constants $\underline{C}_p, \overline{C}_p>0$ such that \eqref{eq:zulehnerinfsup2} is satisfied, then \eqref{eq:zulehnerinfsup4} is satisfied with constants $\underline{\gamma}_p, \overline{\gamma}_p>0$ that depend only on $\underline{C}_p, \overline{C}_p$.
\end{lemma}

By summarizing the results of the last two theorems and lemmas we finally obtain the following theorem.
\begin{theorem}
If there are constants $\underline{\gamma}_v, \overline{\gamma}_v, \underline{\gamma}_p, \overline{\gamma}_p>0$ such that 
\begin{equation}\label{eq:zulehnerinfsup5}
\underline{\gamma}_v\|v\|_V^2\leq a(v,v)+\|Bv\|_{P'}^2\leq\overline{\gamma}_v\|v\|_V^2\quad\forall~v\in V,
\end{equation}
\begin{equation}\label{eq:zulehnerinfsup6}
\underline{\gamma}_p\|q\|_P^2\leq c(q,q)+\|B'q\|_{V'}^2\leq\overline{\gamma}_p\|q\|_P^2\quad\forall~q\in P,
\end{equation}
then
\begin{equation}\label{eq:zulehnerinfsup7}
\underline{C}\|y\|_X\leq \|\mathcal A y\|_{X'}\leq\overline{C}\|y\|_X^2\quad\forall~y\in X
\end{equation}
is satisfied with constants $\underline{C}, \overline{C} > 0$ that depend only on $\underline{\gamma}_v, \overline{\gamma}_v, \underline{\gamma}_p, \overline{\gamma}_p$. And, vice versa, if
the estimates \eqref{eq:zulehnerinfsup7} are satisfied with constants $\underline{C}, \overline{C} > 0$, then the estimates \eqref{eq:zulehnerinfsup5} and \eqref{eq:zulehnerinfsup6} are satisfied with constants
$\underline{\gamma}_v, \overline{\gamma}_v, \underline{\gamma}_p, \overline{\gamma}_p>0$ that depend only on $\underline{C}, \overline{C}$.
\end{theorem}

\begin{remark}
In the case $C=0$ (i.e., $c(q,q)\equiv0$), the lower estimate in condition \eqref{eq:zulehnerinfsup6} has the special form
\begin{equation}\label{eq:zulehnerinfsup4reduce}
\underline{\gamma}_p\|q\|_P^2\leq \|B'q\|_{V'}^2\quad\forall~q\in P.
\end{equation}
From the lower estimate in \eqref{eq:zulehnerinfsup5} it immediately follows that
\begin{equation}\label{eq:zulehnerinfsup3reduce}
\underline{\gamma}_v\|v\|_V^2\leq a(v,v)\quad\forall~v\in \ker B=\{v\in V: Bv=0\}.
\end{equation}
On the other hand, from \eqref{eq:zulehnerinfsup4reduce} and \eqref{eq:zulehnerinfsup3reduce} the lower estimate in \eqref{eq:zulehnerinfsup5} easily follows, using
the fact that \eqref{eq:zulehnerinfsup4reduce} implies
\[
\underline{\gamma}_p\|v\|_V^2\leq \|Bv\|_{P'}^2\quad\forall~v\in (\ker B)^{\perp},
\]
where $ (\ker B)^{\perp}$ denotes the orthogonal complement of $\ker B$. 
So we have recovered a classical result by Brezzi \cite{Brezzi1974,BoffiBrezziFortin2013}. Let $a$ and $b$ be bounded bilinear forms and $c\equiv0$. Then problem \eqref{eq:BxyFy} is well posed if and only if $a$ is coercive on $\ker B$ (see \eqref{eq:zulehnerinfsup3reduce}) and the inf-sup condition for $b$ is satisfied (see \eqref{eq:zulehnerinfsup4reduce}).
\end{remark}



