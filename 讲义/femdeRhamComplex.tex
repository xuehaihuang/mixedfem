% !TEX root = lecture.tex
\chapter{有限元de Rham复形}



\section{多项式复形}

In this section, we shall consider polynomial spaces on a simply connected domain $D\subset\mathbb R^d$ with $d=2,3$. Without loss of generality, we assume $D$ contains the origin $\bs0$ of coordinate system.

Let $\mathbb P_k(D)$ stand for the set of all polynomials in $D$ with the total degree no more than $k$, and $\mathbb P_k(D; \mathbb{X})$ denotes the tensor or vector version, where $\mathbb{X}$ is $\mathbb M$, $\mathbb{S}$, $\mathbb{K}$ or $\mathbb{R}^d$. Let $\mathbb H_k(D):=\mathbb P_k(D)\backslash \mathbb P_{k-1}(D)$. It holds
\begin{equation}\label{eq:Hkxgrad}
\bs x\cdot\nabla q=kq\quad\forall~q\in\mathbb H_k(D),
\end{equation}
\begin{equation}\label{eq:Hkdiv}
\div(\bs x q)=(k+d)q\quad\forall~q\in\mathbb H_k(D).
\end{equation}
Define operator $\pi_{0}: \mathcal C^0(D)\to \mathbb R$ as
\[
\pi_{0}v:=v(\bs0).
\]


\subsection{Polynomial complexes in two dimensions}
\begin{lemma}
The polynomial complex
\begin{equation}\label{eq:deRhamcomplexPoly2d}
%\resizebox{.9\hsize}{!}{$
\mathbb R\xrightarrow{\subset} \mathbb P_{k+1}(D)\xrightarrow{\curl} \mathbb P_k(D;\mathbb R^2) \xrightarrow{\div} \mathbb P_{k-1}(D)\xrightarrow{}0
\end{equation}
is exact. Here integer $k\geq1$.
\end{lemma}
\begin{proof}
Due to \eqref{eq:Hkdiv}, we have $\div(\bs x \mathbb P_{k-1}(D) )= \mathbb P_{k-1}(D)$. Thus $\div\mathbb P_k(D;\mathbb R^2) = \mathbb P_{k-1}(D)$.
Furthermore by direct calculation
\[
\dim\mathbb P_k(D;\mathbb R^2)=\dim \curl \, \mathbb P_{k+1}(D)+\dim \mathbb P_{k-1}(D),
\]
thus the complex \eqref{eq:deRhamcomplexPoly2d} is exact.
\end{proof}

By rotating the complex \eqref{eq:deRhamcomplexPoly2d}, we get the following exact polynomial complex
\begin{equation*}
%\resizebox{.9\hsize}{!}{$
\mathbb R\xrightarrow{\subset} \mathbb P_{k+1}(D)\xrightarrow{\grad} \mathbb P_k(D;\mathbb R^2) \xrightarrow{\rot} \mathbb P_{k-1}(D)\xrightarrow{}0.
\end{equation*}


\begin{lemma}
The polynomial complex
\begin{equation}\label{eq:KoszulcomplexPoly2d}
%\resizebox{.9\hsize}{!}{$
0\xrightarrow{\subset}\mathbb P_{k-1}(D) \xrightarrow{\boldsymbol x} \mathbb P_k(D;\mathbb R^2) \xrightarrow{\boldsymbol x^{\perp}\cdot} \mathbb P_{k+1}(D)\xrightarrow{\pi_{0}}\mathbb R\xrightarrow{}0
\end{equation}
is exact.
\end{lemma}

Those two complexes \eqref{eq:deRhamcomplexPoly2d} and \eqref{eq:KoszulcomplexPoly2d} are connected as
\begin{equation}\label{eq:deRhamcomplexPolydouble2d}
%\resizebox{.93\hsize}{!}{$
\xymatrix{
\mathbb R\ar@<0.4ex>[r]^-{\subset} & \; \mathbb P_{k+1}(D)\; \ar@<0.4ex>[r]^-{\curl}\ar@<0.4ex>[l]^-{\pi_0}  & \; \mathbb P_k(D;\mathbb R^2) \ar@<0.4ex>[r]^-{\div}\; \ar@<0.4ex>[l]^-{\boldsymbol x^{\bot}\cdot} & \; \mathbb P_{k-1}(D)  \; \ar@<0.4ex>[r]^-{} \ar@<0.4ex>[l]^-{\boldsymbol x}
& 0 \ar@<0.4ex>[l]^-{\supset} }.
%$}
\end{equation}
Unlike the Koszul complex for vectors functions, we do not have the identity property applied to homogenous polynomials. Fortunately decomposition of polynomial spaces using Koszul and differential operators still holds.

First of all, we have the decomposition
\[
\mathbb P_{k+1}(D) = \boldsymbol x^{\perp}\cdot\mathbb P_k(D;\mathbb R^2)\oplus\mathbb R.
\]

\begin{lemma}\label{lem:symmpolyspacedirectsum}
It holds
 \begin{equation}\label{eq:vector2polyspacedecomp2}
\mathbb P_{k}(D;\mathbb R^2)=\curl\mathbb P_{k+1}(D)\oplus\mathbb P_{k-1}(D)\bs x.
\end{equation}
%And $\div\boldsymbol \div: \mathbb C_k^{\oplus}(D;\mathbb S)\to\mathbb P_{k-2}(D;\mathbb R^2)$ is a bijection.
\end{lemma}
\begin{proof}
Assume $q\in\mathbb P_{k-1}(D)$ satisfies $\boldsymbol xq\in\curl\mathbb P_{k+1}(D)$, which means
\[
\div(\boldsymbol  xq)=0.
\]
Then it follows from \eqref{eq:Hkdiv} that
$q=0$.
Hence $\curl\mathbb P_{k+1}(D)\cap\mathbb P_{k-1}(D)\bs x=0$.
Therefore we obtain the decomposition by the fact $\dim\mathbb P_{k}(D;\mathbb R^2)=\dim\curl\mathbb P_{k+1}(D)+\dim(\mathbb P_{k-1}(D)\bs x)$.
\end{proof}

Similarly, we have
\begin{equation*}
%\resizebox{.93\hsize}{!}{$
\xymatrix{
\mathbb R\ar@<0.4ex>[r]^-{\subset} & \; \mathbb P_{k+1}(D)\; \ar@<0.4ex>[r]^-{\grad}\ar@<0.4ex>[l]^-{\pi_0}  & \; \mathbb P_k(D;\mathbb R^2) \ar@<0.4ex>[r]^-{\rot}\; \ar@<0.4ex>[l]^-{\boldsymbol x\cdot} & \; \mathbb P_{k-1}(D)  \; \ar@<0.4ex>[r]^-{} \ar@<0.4ex>[l]^-{\boldsymbol x^{\bot}}
& 0 \ar@<0.4ex>[l]^-{\supset} },
%$}
\end{equation*}
\[
\mathbb P_{k+1}(D) = \boldsymbol x\cdot\mathbb P_k(D;\mathbb R^2)\oplus\mathbb R,
\]
\[
\mathbb P_{k}(D;\mathbb R^2)=\nabla\mathbb P_{k+1}(D)\oplus\mathbb P_{k-1}(D)\bs x^{\perp}.
\]


\subsection{Polynomial complexes in three dimensions}


\begin{lemma}
The polynomial complex
\begin{equation}\label{eq:deRhamcomplex3dPoly}
%\resizebox{.9\hsize}{!}{$
\mathbb R\xrightarrow{\subset} \mathbb P_{k+1}(D)\xrightarrow{\grad} \mathbb P_{k}(D;\mathbb R^3)\xrightarrow{\curl}\mathbb P_{k-1}(D;\mathbb R^3) \xrightarrow{\div} \mathbb P_{k-2}(D)\xrightarrow{}0
%$}
\end{equation}
is exact.
\end{lemma}
\begin{proof}
Again it follows from \eqref{eq:Hkdiv} that $\div\mathbb P_k(D;\mathbb R^2) = \mathbb P_{k-1}(D)$, and
\begin{align*}
\dim\mathbb P_k(D;\mathbb R^3)\cap\ker(\div)&=\dim \mathbb P_k(D; \mathbb R^3)-\dim \mathbb P_{k-1}(D) \\
&=\frac{1}{2}(k+1)(k+2)(k+3)-\frac{1}{6}k(k+1)(k+2)=\frac{1}{6}(k+1)(k+2)(2k+9).
\end{align*}

For any $\boldsymbol v\in\mathbb P_{k+1}(D; \mathbb R^3)\cap\ker(\curl)$, the line integral
\[
q=\int_{\bs 0}^{\bs x}\bs v\cdot\dd\bs x \in \mathbb P_{k+2}(D)
\]
is path independent, thus it is well-defined and $\boldsymbol v=\grad q$. Thus $\grad\mathbb P_{k+2}(D)=\mathbb P_{k+1}(D; \mathbb R^3)\cap\ker(\curl)$, and
\begin{align*}
\dim\curl\mathbb P_{k+1}(D; \mathbb R^3)&=\dim \mathbb P_{k+1}(D; \mathbb R^3)-\dim \mathbb P_{k+1}(D; \mathbb R^3)\cap\ker(\curl) \\
&=\dim \mathbb P_{k+1}(D; \mathbb R^3)-\dim\grad\mathbb P_{k+2}(D) \\
&=\frac{1}{2}(k+2)(k+3)(k+4)-\frac{1}{6}(k+3)(k+4)(k+5)+1 \\
&=\frac{1}{6}(k+1)(k+2)(2k+9).
\end{align*}

Noting that $\dim\curl\mathbb P_{k+1}(D; \mathbb R^3)=\dim\mathbb P_k(D;\mathbb R^3)\cap\ker(\div)$ and $\curl\mathbb P_{k+1}(D; \mathbb R^3)\subseteq\mathbb P_k(D;\mathbb R^3)\cap\ker(\div)$, we obtain $\curl\mathbb P_{k+1}(D; \mathbb R^3)=\mathbb P_k(D;\mathbb R^3)\cap\ker(\div)$.
Thus the complex \eqref{eq:deRhamcomplex3dPoly} is exact.
\end{proof}

%\LC{Add the backwards complex using Poincare operators first. Then the following decomposition follows naturally.}
%



\begin{lemma}\label{lem:Koszul}
The polynomial complex
%\begin{equation}\label{eq:divdivKoszulcomplexPoly}
%%\resizebox{.9\hsize}{!}{$
%0\autorightarrow{$\subset$}{}\mathbb P_{k-2}(\Omega) \autorightarrow{$\boldsymbol x\boldsymbol x^{\intercal}$}{} \mathbb P_k(\Omega; \mathbb S) \autorightarrow{$\boldsymbol x^{\perp}$}{} \mathbb P_k(\Omega; \mathbb S)\boldsymbol x^{\perp}\autorightarrow{}{}0
%\end{equation}
\begin{equation}\label{eq:deRhamvKoszulcomplex3dPoly}
%\resizebox{.922\hsize}{!}{$
0\xrightarrow{\subset}\mathbb P_{k-1}(D) \xrightarrow{\boldsymbol x} \mathbb P_k(D; \mathbb R^3) \xrightarrow{\boldsymbol x\times} \mathbb P_{k+1}(D; \mathbb R^3)\xrightarrow{\boldsymbol x\cdot} \mathbb P_{k+2}(D)\xrightarrow{\pi_{0}}\mathbb {R}\xrightarrow{}0
%$}
\end{equation}
is exact.
\end{lemma}
\begin{proof}
It is easy to see that $\pi_0\mathbb P_{k+2}(D)=\mathbb R$ and $\boldsymbol x\cdot\mathbb P_{k+1}(D; \mathbb R^3)=\mathbb P_{k+2}(D)\cap\ker(\pi_0)$. And
\begin{align*}
\dim\mathbb P_{k+1}(D; \mathbb R^3)\cap\ker(\bs x\cdot)&=\dim\mathbb P_{k+1}(D; \mathbb R^3)-\dim\bs x\cdot\mathbb P_{k+1}(D; \mathbb R^3) \\
&=\dim\mathbb P_{k+1}(D; \mathbb R^3)-\dim\mathbb P_{k+2}(D)\cap\ker(\pi_0) \\
&=\frac{1}{2}(k+2)(k+3)(k+4) - \frac{1}{6}(k+3)(k+4)(k+5) + 1 \\
&=\frac{1}{6}(k+1)(k+2)(2k+9).
\end{align*}

For any $\boldsymbol v\in\mathbb P_{k}(D; \mathbb R^3)\cap\ker(\bs x\times)$, we have $\bs x\times\boldsymbol v=\boldsymbol0$. Thus $\bs v$ is parallel to $\bs x$, which implies $\bs v\in\bs x\mathbb P_{k-1}(D)$. That is $\mathbb P_{k}(D; \mathbb R^3)\cap\ker(\bs x\times)=\bs x\mathbb P_{k-1}(D)$. And
\begin{align*}
\dim\bs x\times\mathbb P_{k}(D; \mathbb R^3)&=\dim\mathbb P_{k}(D; \mathbb R^3)-\dim\mathbb P_{k}(D; \mathbb R^3)\cap\ker(\bs x\times) \\
&=\dim\mathbb P_{k}(D; \mathbb R^3)-\dim\bs x\mathbb P_{k-1}(D) \\
&=\frac{1}{2}(k+1)(k+2)(k+3) - \frac{1}{6}k(k+1)(k+2) \\
&=\frac{1}{6}(k+1)(k+2)(2k+9).
\end{align*}

Noting that $\dim\bs x\times\mathbb P_{k}(D; \mathbb R^3)=\dim\mathbb P_{k+1}(D; \mathbb R^3)\cap\ker(\bs x\cdot)$ and $\bs x\times\mathbb P_{k}(D; \mathbb R^3)\subseteq\mathbb P_{k+1}(D; \mathbb R^3)\cap\ker(\bs x\cdot)$, we obtain $\bs x\times\mathbb P_{k}(D; \mathbb R^3)=\mathbb P_{k+1}(D; \mathbb R^3)\cap\ker(\bs x\cdot)$.
Thus the complex \eqref{eq:deRhamvKoszulcomplex3dPoly} is exact.
\end{proof}

Those two complexes \eqref{eq:deRhamcomplex3dPoly} and \eqref{eq:deRhamvKoszulcomplex3dPoly} are connected as
\begin{equation}\label{eq:divdivcomplex3dPolydouble}
%\resizebox{.92\hsize}{!}{$
\xymatrix{
\mathbb R\ar@<0.4ex>[r]^-{\subset} & \mathbb P_{k+2}(D)\ar@<0.4ex>[r]^-{\grad}\ar@<0.4ex>[l]^-{\pi_{0}} & \mathbb P_{k+1}(D; \mathbb R^3)\ar@<0.4ex>[r]^-{\curl}\ar@<0.4ex>[l]^-{\boldsymbol x\cdot}  & \mathbb P_k(D; \mathbb R^3) \ar@<0.4ex>[r]^-{\div}\ar@<0.4ex>[l]^-{\boldsymbol x\times} & \mathbb P_{k-1}(D)  \ar@<0.4ex>[r]^-{} \ar@<0.4ex>[l]^-{\boldsymbol x}
& 0 \ar@<0.4ex>[l]^-{\supset} }.
%$}
\end{equation}

It is easy to see that
\[
\mathbb P_{k+2}(D)= \boldsymbol x\cdot\mathbb P_{k+1}(D; \mathbb R^3)\oplus\mathbb R.
\]
We then move to the space $\mathbb P_{k+1}(D; \mathbb R^3)$.
\begin{lemma}
We have the decomposition
\begin{equation}\label{eq:vector3polyspacedecomp1}
\mathbb P_{k+1}(D; \mathbb R^3)=\grad\mathbb P_{k+2}(D)\oplus\boldsymbol x\times\mathbb P_k(D; \mathbb R^3).
\end{equation}
\end{lemma}
\begin{proof}
Since the dimension of space in the left hand side is the summation of the dimension of the two spaces in the right hand side in \eqref{eq:vector3polyspacedecomp1}, we only need to prove that the sum in \eqref{eq:vector3polyspacedecomp1} is the direct sum.

For any $\boldsymbol v=\nabla q$ with $q\in\mathbb P_{k+2}(D)$ satisfying $\boldsymbol v\in\boldsymbol x\times\mathbb P_k(D; \mathbb R^3)$, it follows $(\boldsymbol x\cdot\nabla)q=0$. Then we get from \eqref{eq:Hkxgrad} that $q$ is constant. Hence $\bs v=\bs0$.
\end{proof}

\begin{lemma}
We have the decomposition
\begin{equation}\label{eq:vector3polyspacedecomp2}
\mathbb P_{k}(D; \mathbb R^3)=\curl\mathbb P_{k+1}(D;\mathbb R^3)\oplus \boldsymbol x\mathbb P_{k-1}(D).
\end{equation}
\end{lemma}
\begin{proof}
 Noting that the dimension of space in the left hand side is the summation of the dimension of the two spaces in the right hand side in \eqref{eq:vector3polyspacedecomp2}, we only need to prove the direct sum.

For any $\boldsymbol v=\bs x q$ with $q\in\mathbb P_{k-1}(D)$ satisfying $\boldsymbol v\in\curl\mathbb P_{k+1}(D;\mathbb R^3)$, it follows $\div(\bs x q)=0$. 
Applying \eqref{eq:Hkdiv}, we get $q=0$. Thus $\bs v=\bs0$.
\end{proof}

\section{Simplicial approximations of $H(\div)$}
The material in this section comes from \cite{BoffiBrezziFortin2013}.
Let $\Gamma:=\partial\Omega$.
Define
\[
H(\div,\Omega):=\{\bs v\in \bs L^2(\Omega;\mathbb R^d): \div\bs v\in L^2(\Omega)\}.
\]
with squared  norm $\|\bs v\|_{\div, \Omega}^2:=\|\bs v\|_{0, \Omega}^2+\|\div\bs v\|_{0, \Omega}^2$.
\begin{lemma}
Let $\bs v\in H(\div,\Omega)$. We have $\bs v\cdot\bs n|_{\Gamma}\in H^{-1/2}(\Gamma)$ and the Green's formula
\[
(\div\bs v, w)_{\Omega} + (\bs v, \grad w)_{\Omega}=\langle\bs v\cdot\bs n, w\rangle_{H^{-1/2}(\Gamma)\times H^{1/2}(\Gamma)}\quad\forall~w\in H^1(\Omega).
\]
\end{lemma}

\begin{lemma}
The trace operator $\bs v\in H(\div,\Omega)\to\bs v\cdot\bs n|_{\Gamma}\in H^{-1/2}(\Gamma)$ is surjective.
\end{lemma}

Now introduce $H(\div)$-conforming finite elements on simplexes.
Assume $K\subset \mathbb R^d$ is a simplex with $d=2,3$. 

In two dimensions,
thanks to \eqref{eq:vector2polyspacedecomp2}, define the shape function space
\[
V_{k,\ell}^d(K):=\curl\mathbb P_{k+1}(K)\oplus \boldsymbol x\mathbb P_{\ell-1}(K)=\mathbb P_{k}(K; \mathbb R^2) + \boldsymbol x\mathbb P_{\ell-1}(K),
\]
where $\ell=k, k+1$ for $k\geq1$, and $\ell=1$ for $k=0$. 
The degrees of freedom $\mathcal N_{k,\ell}^d(K)$ are given by
\begin{align}
(\boldsymbol v\cdot\boldsymbol  n, q)_e & \quad\forall~q\in\mathbb P_{k}(e),  e\in\mathcal E(K),\label{Hdivfem2ddof1}\\
(\boldsymbol v, \boldsymbol q)_K & \quad\forall~\boldsymbol q\in\nabla\mathbb P_{\ell-1}(K)\oplus\boldsymbol x^{\perp}\mathbb P_{k-2}(K). \label{Hdivfem2ddof2}
\end{align}


In three dimensions,
thanks to \eqref{eq:vector3polyspacedecomp2}, define the shape function space
\[
V_{k,\ell}^d(K):=\curl\mathbb P_{k+1}(K;\mathbb R^3)\oplus \boldsymbol x\mathbb P_{\ell-1}(K)=\mathbb P_{k}(K; \mathbb R^3) + \boldsymbol x\mathbb P_{\ell-1}(K),
\]
where $\ell=k, k+1$ for $k\geq1$, and $\ell=1$ for $k=0$. 
The degrees of freedom $\mathcal N_{k,\ell}^d(K)$ are given by
\begin{align}
(\boldsymbol v\cdot\boldsymbol  n, q)_F & \quad\forall~q\in\mathbb P_{k}(F),  F\in\mathcal F(K),\label{Hdivfem3ddof1}\\
(\boldsymbol v, \boldsymbol q)_K & \quad\forall~\boldsymbol q\in\nabla\mathbb P_{\ell-1}(K)\oplus\boldsymbol x\times\mathbb P_{k-2}(K; \mathbb R^3). \label{Hdivfem3ddof2}
\end{align}

Apparently $\bs v\cdot \bs n|_F\in\mathbb P_{k}(F)$ for any $\bs v\in V_{k,\ell}^d(K)$ and $F\in\mathcal F(K)$.
And $\div V_{k,\ell}^d(K)=\mathbb P_{\ell-1}(K)$.




\begin{lemma}\label{lem:unisovlenHdivfem}
The finite element triple $(K, V_{k,\ell}^d(K), \mathcal N_{k,\ell}^d(K))$ is well-defined.
\end{lemma}
\begin{proof}
We only prove the unisolvence of the finite element triple $(K, V_{k,\ell}^d(K), \mathcal N_{k,\ell}^d(K))$ in three dimensions, since the proof is similar for two dimensions. It is easy to check that the number of the degrees of freedom \eqref{Hdivfem3ddof1}-\eqref{Hdivfem3ddof2}  is equal to $\dim V_{k,\ell}^d(K)$.


Take any $\boldsymbol v\in V_{k,\ell}^d(K)$ and
suppose all the degrees of freedom \eqref{Hdivfem3ddof1}-\eqref{Hdivfem3ddof2} vanish. 
By the vanishing degrees of freedom \eqref{Hdivfem3ddof1}, we get $\bs v\in H_0(\div, K)$. 
Together with the vanishing degrees of freedom \eqref{Hdivfem3ddof2} and the fact $\div V_{k,\ell}^d(K)=\mathbb P_{\ell-1}(K)$, we obtain $\div\bs v=0$, and then $\bs v\in\curl\mathbb P_{k+1}(K;\mathbb R^3)$.
It follows from \eqref{eq:vector3polyspacedecomp1} and the vanishing degrees of freedom \eqref{Hdivfem3ddof2} that 
\begin{equation}\label{eq:20200917}
(\boldsymbol v, \boldsymbol q)_K =0 \quad\forall~\boldsymbol q\in\mathbb P_{k-1}(K; \mathbb R^3). 
\end{equation}

For ease of presentation, denote four faces in $\mathcal F(K)$ by $F_i$, which is opposite to the $i$th vertex of $K$, and by $\bs n_i$ the outward unit normal vector of $F_i$ for $i=1,2,3,4$. 
Let $\bs t_i$ be the unit tangential vector of the edge from vertex $4$ to vertex $i$. 
The set of three vectors $\{\bs t_1, \bs t_2, \bs t_3\}$ forms a basis of $\mathbb R^3$ although they may not be orthogonal in general. Then we have
\[
\bs v=\sum_{i=1}^3\frac{\bs v\cdot\bs n_i}{\bs t_i\cdot\bs n_i}\bs t_i.
\]
Since $\bs v\cdot\bs n_i|_{F_i}=0$, there exists $p\in\mathbb P_{k-1}(K)$ satisfying $\bs v\cdot\bs n_i=\lambda_ip$. Taking $\bs q=\bs n_ip$ in \eqref{eq:20200917} gives
\[
(\lambda_ip, p)_K =0.
\]
Hence $p=0$ and $\boldsymbol v\cdot\bs n_i=0$. As a result, $\boldsymbol v=\bs 0$.
\end{proof}

We can see that the finite element triple $(K, V_{k,k+1}^d(K), \mathcal N_{k,k+1}^d(K))$ is the Raviart-Thomas element \cite{RaviartThomas1977,Nedelec1980}, and $(K, V_{k,k}^d(K), \mathcal N_{k,k}^d(K))$ is the Brezzi-Douglas-Marini element \cite{BrezziDouglasMarini1986,BrezziDouglasDuranFortin1987}.

%\newpage
Next we present the explicit expression of the $H(\div)$ bubble functions for $k\geq2$. Let
\[
\mathring V_{k}^d(K):=\{\bs v\in\mathbb P_{k}(K;\mathbb R^3): \bs v\cdot\bs n|_{\partial K}=0\},\quad
V_{k,b}^d(K):=\sum_{1\leq i<j\leq 4}\lambda_i\lambda_j\mathbb P_{k-2}(K)\bs t_{ij},
\]
where $\bs t_{ij}:=\bs x_j-\bs x_i$ with $\bs x_1, \bs x_2, \bs x_3, \bs x_4$ being the four vertices of the tetrahedron $K$.
\begin{lemma}
It holds for $k\geq2$ that
\begin{equation}\label{eq:Hdivbubbleidentity}
\mathring V_{k}^d(K)=V_{k,b}^d(K).
\end{equation}
\end{lemma}
\begin{proof}
It is easy to see that $V_{k,b}^d(K)\subseteq\mathring V_{k}^d(K)$. Next let's prove $\mathring V_{k}^d(K)\subseteq V_{k,b}^d(K)$. Take any $\bs v\in\mathring V_{k}^d(K)$. Since $\{\bs t_{14}, \bs t_{24}, \bs t_{34}\}$ forms a basis of $\mathbb R^3$, we can express 
$$\bs v=q_1\bs t_{14}+q_2\bs t_{24}+q_3\bs t_{34},$$ 
where $q_1, q_2, q_3\in\mathbb P_k(K)$. Noting that $\bs v\cdot\bs n|_{\partial K}=0$, we get
\[
q_i\bs t_{i4}\cdot\bs n_i|_{F_i}=0\quad\textrm{ for } i=1,2,3,
\quad
(q_1+q_2+q_3)\bs t_{14}\cdot\bs n_4|_{F_4}=0.
\]
Thus there exist $\tilde q_i\in\mathbb P_{k-1}(K)$ such that $q_i=\tilde q_i\lambda_i$ for $i=1,2,3$. Then $(\tilde q_1\lambda_1+\tilde q_2\lambda_2+\tilde q_3\lambda_3)|_{F_4}=0$. And there exists $\tilde q_4\in\mathbb P_{k-1}(K)$ satisfying
\begin{equation}\label{eq:20201002-1}
\sum_{i=1}^3\tilde q_i\lambda_i = \tilde q_1\lambda_1+\tilde q_2\lambda_2+\tilde q_3\lambda_3=\tilde q_4\lambda_4.
\end{equation}
For each $i=1,2,3$, express $\tilde q_i$ as
\[
\tilde q_i=\sum_{j=1}^4q_{ij}\lambda_j=q_{i4}\lambda_4+q_{ii}\lambda_i+\sum_{j=1, j\neq i}^3q_{ij}\lambda_j,
\]
where $q_{i4}\in\sum\limits_{0\leq k_1,k_2,k_3,k_4\leq k-2
\atop k_1+k_2+k_3+k_4=k-2}\textrm{span}\{\lambda_1^{k_1}\lambda_2^{k_2}\lambda_3^{k_3}\lambda_4^{k_4}\}$, $q_{ii}\in\textrm{span}\{\lambda_i^{k-2}\}$, $q_{ij}\in\sum\limits_{0\leq k_1,k_2,k_3\leq k-2
\atop k_1+k_2+k_3=k-2}\textrm{span}\{\lambda_1^{k_1}\lambda_2^{k_2}\lambda_3^{k_3}\}$.
Then \eqref{eq:20201002-1} becomes
\[
(q_{12}+q_{21})\lambda_1\lambda_2+(q_{13}+q_{31})\lambda_1\lambda_3+(q_{23}+q_{32})\lambda_2\lambda_3
+ q_{11}\lambda_1^2+ q_{22}\lambda_2^2 + q_{33}\lambda_3^2=\tilde q_4\lambda_4-\lambda_4\sum_{i=1}^3\lambda_iq_{i4}.
\]
Hence $q_{11}=q_{22}=q_{33}=0$, and
\[
(q_{12}+q_{21})\lambda_1\lambda_2+(q_{13}+q_{31})\lambda_1\lambda_3+(q_{23}+q_{32})\lambda_2\lambda_3=0.
\]
By this identity, we get $\lambda_1|(q_{23}+q_{32})$, $\lambda_2|(q_{31}+q_{13})$ and $\lambda_3|(q_{12}+q_{21})$.
Hence there exist $p_1, p_2\in \mathbb P_{k-3}(K)$ such that
\begin{equation}\label{eq:20201002-2}
q_{23}+q_{32}=\lambda_1p_1, \quad q_{31}+q_{13}=\lambda_2p_2, \quad q_{12}+q_{21}=-\lambda_3(p_1+p_2).
\end{equation}
Therefore
\begin{align*}
\bs v&=\sum_{i=1}^3\tilde q_i\lambda_i\bs t_{i4} =\sum_{i=1}^3q_{i4}\lambda_4\lambda_i\bs t_{i4} + \sum_{i=1}^3\sum_{j=1, j\neq i}^3q_{ij}\lambda_j\lambda_i\bs t_{i4}  \\
&=\sum_{i=1}^3q_{i4}\lambda_4\lambda_i\bs t_{i4} + \lambda_1\lambda_2(q_{12}\bs t_{14}+q_{21}\bs t_{24}) + \lambda_1\lambda_3(q_{13}\bs t_{14}+q_{31}\bs t_{34}) + \lambda_2\lambda_3(q_{23}\bs t_{24}+q_{32}\bs t_{34}) \\
&=\sum_{i=1}^3q_{i4}\lambda_4\lambda_i\bs t_{i4}  + \lambda_1\lambda_2q_{12}\bs t_{12} + \lambda_1\lambda_3q_{13}\bs t_{13}  + \lambda_2\lambda_3q_{23}\bs t_{23}\\
&\quad + \lambda_1\lambda_2(q_{12}+q_{21})\bs t_{24} + \lambda_1\lambda_3(q_{13}+q_{31})\bs t_{34}+ \lambda_2\lambda_3(q_{23}+q_{32})\bs t_{34}. 
\end{align*}
On the other hand, it follows from \eqref{eq:20201002-2} that
\begin{align*}
&\lambda_1\lambda_2(q_{12}+q_{21})\bs t_{24} + \lambda_1\lambda_3(q_{13}+q_{31})\bs t_{34}+ \lambda_2\lambda_3(q_{23}+q_{32})\bs t_{34} \\
=&\lambda_1\lambda_2\lambda_3\big(-(p_1+p_2)\bs t_{24} +p_2\bs t_{34}+p_1\bs t_{34} \big)=\lambda_1\lambda_2\lambda_3(p_1+p_2)\bs t_{32}.
\end{align*}
Finally \eqref{eq:Hdivbubbleidentity} holds from the last two identities.
\end{proof}


Since
\[
\resizebox{\textwidth}{!}{$ %
\dim\textrm{span}\{\lambda_1\lambda_4\mathbb P_{k-2}(K)\bs t_{14}, \lambda_2\lambda_4\mathbb P_{k-2}(K)\bs t_{24}, \lambda_3\lambda_4\mathbb P_{k-2}(K)\bs t_{34}\}\cap\textrm{span}\{\lambda_2\lambda_3\mathbb P_{k-2}(K)\bs t_{23}\}=\dim\mathbb P_{k-3}(K), 
$}
\]
\[
\resizebox{\textwidth}{!}{$ %
\dim\textrm{span}\{\lambda_1\lambda_4\mathbb P_{k-2}(K)\bs t_{14}, \lambda_2\lambda_4\mathbb P_{k-2}(K)\bs t_{24}, \lambda_3\lambda_4\mathbb P_{k-2}(K)\bs t_{34}, \lambda_2\lambda_3\mathbb P_{k-2}(K)\bs t_{23}\}\cap\textrm{span}\{\lambda_1\lambda_3\mathbb P_{k-2}(K)\bs t_{13}, \lambda_1\lambda_2\mathbb P_{k-2}(K)\bs t_{12}\}=\dim\mathbb P_{k-3}(K) = 3\dim\mathbb P_{k-3}(K)-\mathbb P_{k-4}(K), 
$}
\]
We have from \eqref{eq:Hdivbubbleidentity} that
\begin{align*}
\dim\mathring V_{k}^d(K)=6\dim\mathbb P_{k-2}(K)-\dim\mathbb P_{k-3}(K)-(3\dim\mathbb P_{k-3}(K)-\mathbb P_{k-4}(K))=\frac{1}{2}(k^2-1)(k+2).
\end{align*}
We can also acquire $\dim\mathring V_{k}^d(K)$ from the unisolvence of the BDM element.


\XH{Discuss more on RT element in arbitrary dimension}.
The geometric decomposition for Raviart-Thomas element in arbitrary dimension is (cf. \cite{ArnoldFalkWinther2009})
$$
\mathbb P_k(K;\mathbb R^d)+\mathbb P_k(K)\bs x=\Oplus_{i=0}^d\mathbb P_{k}(F_{i^*})\frac{1}{|K|}(\bs x-\bs x_i) \oplus \Oplus_{i=1}^d\mathbb P_{k-1}(K)\lambda_i(\bs x-\bs x_i),
$$
where $\bs x_i$ is the Cartesian coordinate of vertex $\texttt{v}_i$, $F_{i^*}\in\Delta_{d-1}(K)$ is opposite to $\texttt{v}_i$.



\section{Simplicial approximations of $H(\curl)$}

Assume $d=3$.
Define
\[
H(\curl,\Omega):=\{\bs v\in \bs L^2(\Omega;\mathbb R^3): \curl\bs v\in \bs L^2(\Omega;\mathbb R^3)\}.
\]
with squared  norm $\|\bs v\|_{\curl, \Omega}^2:=\|\bs v\|_{0, \Omega}^2+\|\curl\bs v\|_{0, \Omega}^2$.

The results on the trace of $H(\curl,\Omega)$ can be found in \cite{BuffaCostabelSheen2002,BuffaCiarlet2001,BuffaCiarlet2001a}.
 For any regular vector field $\bs v$ in $\Omega$, we define the tangential trace $\gamma_{t}(\bs v):=\bs n\times \bs v|_{\Gamma}$, and the projection on the tangential plane $\pi_{t}(\bs v):=\bs n\times\bs v\times  \bs n|_{\Gamma}$. Let
 \[
 \bs H^{-1/2}(\div_{\Gamma}, \Gamma):=\{\bs\lambda\in V_{\pi}': \div_{\Gamma}\bs\lambda\in H^{-1/2}(\Gamma)\},
 \]
 \[
 \bs H^{-1/2}(\curl_{\Gamma}, \Gamma):=\{\bs\lambda\in V_{\gamma}': \curl_{\Gamma}\bs\lambda\in H^{-1/2}(\Gamma)\},
 \]
 where $V_{\gamma}:=\gamma_{t}(\bs H^{1/2}(\Gamma))$ and $V_{\pi}:=\pi_{t}(\bs H^{1/2}(\Gamma))$.
Spaces $ \bs H^{-1/2}(\div_{\Gamma}, \Gamma)$ and $ \bs H^{-1/2}(\curl_{\Gamma}, \Gamma)$ are dual to each other with the pivot space $\bs L_t^2(\Gamma):=\{\bs \lambda\in\bs L^2(\Gamma;\mathbb R^3): \bs \lambda\cdot\bs n=0\}$.

If the surface $\Gamma$ was regular, then
\[
V_{\gamma}=V_{\pi}=\{\bs\lambda\in\bs H^{1/2}(\Gamma, \mathbb R^3): \bs\lambda\cdot\bs n=0\}.
\]
In the case of piecewise regular surfaces, the spaces $V_{\gamma}$ and $V_{\pi}$ are different (see \cite{BuffaCiarlet2001}).

\begin{lemma}
The trace operators $\gamma_{t}:  H(\curl,\Omega)\to\bs H^{-1/2}(\div_{\Gamma}, \Gamma)$ and $\pi_{t}:  H(\curl,\Omega)\to\bs H^{-1/2}(\curl_{\Gamma}, \Gamma)$ are linear, continuous and surjective. And we have the 
Green's formula
\[
(\curl\bs v, \bs\phi)_{\Omega} - (\bs v, \curl\bs\phi)_{\Omega}=\langle\gamma_t\bs v, \pi_t\bs \phi\rangle\quad\forall~\bs v, \bs \phi\in H(\curl,\Omega).
\]
\end{lemma}

Now introduce $H(\curl)$-conforming finite elements on tetrahedrons.
Assume $K\subset \mathbb R^3$ is a tetrahedron.
Thanks to \eqref{eq:vector3polyspacedecomp1}, define the shape function space
\[
V_{k,\ell}^c(K):=\grad\mathbb P_{k+1}(K)\oplus \boldsymbol x\times\mathbb P_{\ell-1}(K; \mathbb R^3)=\mathbb P_{k}(K; \mathbb R^3) + \boldsymbol x\times\mathbb P_{\ell-1}(K; \mathbb R^3),
\]
where $\ell=k, k+1$ for $k\geq1$, and $\ell=1$ for $k=0$. 
The degrees of freedom $\mathcal N_{k,\ell}^c(K)$ are given by
\begin{align}
(\boldsymbol v\cdot\boldsymbol t, q)_e & \quad\forall~q\in\mathbb P_{k}(e),  e\in\mathcal E(K),\label{Hcurlfem3ddof1}\\
(\boldsymbol n\times\boldsymbol v\times\boldsymbol  n, \bs q)_F & \quad\forall~q\in\curl_F\mathbb P_{
\ell-1}(F)\oplus\bs x\mathbb P_{k-2}(F),  F\in\mathcal F(K),\label{Hcurlfem3ddof2}\\
(\boldsymbol v, \boldsymbol q)_K & \quad\forall~\boldsymbol q\in\curl\mathbb P_{\ell-2}(K; \mathbb R^3)\oplus\boldsymbol x\mathbb P_{k-3}(K). \label{Hcurlfem3ddof3}
\end{align}
Since $\ell=k, k+1$, we have $\curl_F\mathbb P_{
\ell-1}(F)\oplus\bs x\mathbb P_{k-2}(F)=\mathbb P_{
\ell-2}(F;\mathbb R^2)+\bs x\mathbb P_{k-2}(F)$ and $\curl\mathbb P_{\ell-2}(K; \mathbb R^3)\oplus\boldsymbol x\mathbb P_{k-3}(K)=\mathbb P_{
\ell-3}(F;\mathbb R^3)+\boldsymbol x\mathbb P_{k-3}(K)$.

Apparently $\bs v\cdot \bs t|_e\in\mathbb P_{k}(e)$ for any $\bs v\in V_{k,\ell}^c(K)$ and $e\in\mathcal E(K)$.
And $\curl V_{k,\ell}^c(K)=\curl\mathbb P_{\ell}(K;\mathbb R^3)$.




\begin{lemma}\label{lem:unisovlenHdivfem}
The finite element triple $(K, V_{k,\ell}^c(K), \mathcal N_{k,\ell}^c(K))$ is well-defined.
\end{lemma}
\begin{proof}
It is easy to check that the number of the degrees of freedom \eqref{Hcurlfem3ddof1}-\eqref{Hcurlfem3ddof3}  is equal to $\dim V_{k,\ell}^c(K)$.


Take any $\boldsymbol v\in V_{k,\ell}^c(K)$ and
suppose all the degrees of freedom \eqref{Hcurlfem3ddof1}-\eqref{Hcurlfem3ddof3} vanish. 
By the vanishing degrees of freedom \eqref{Hcurlfem3ddof1}-\eqref{Hcurlfem3ddof2}, we get $\bs v\in H_0(\curl, K)$. 
Now consider $\curl\bs v\in H_0(\div, K)\cap\cap \mathbb P_{\ell-1}(K;\mathbb R^3)$.
By the vanishing degrees of freedom \eqref{Hcurlfem3ddof3}, we get 
\[
(\curl\boldsymbol v, \boldsymbol q)_K = \quad\forall~\boldsymbol q\in\mathbb P_{\ell-2}(K; \mathbb R^3).
\]
Then by the well-posedness of the finite element triple $(K, V_{\ell-1,\ell-1}^d(K), \mathcal N_{\ell-1,\ell-1}^d(K))$, we obtain $\curl\boldsymbol v=0$. Hence $\boldsymbol v\in \grad\mathbb P_{k+1}(K)\cap H_0(\curl, K)$. Then there exists $w\in\mathbb P_{k-3}(K)$ such that $\bs v=\nabla(b_Kw)$, where $b_K:=\lambda_1\lambda_2\lambda_3\lambda_4$ is the bubble function.
It follows from the vanishing degrees of freedom \eqref{Hcurlfem3ddof3} that
\[
(b_Kw, \div\boldsymbol q)_K = \quad\forall~\boldsymbol q\in\boldsymbol x\mathbb P_{k-3}(K).
\]
Due to \eqref{eq:vector3polyspacedecomp2},  it holds $\div(\boldsymbol x\mathbb P_{k-3}(K))=\mathbb P_{k-3}(K)$. As a result, $w=0$ and then $\boldsymbol v=\bs 0$.
\end{proof}

We can see that the finite element triple $(K, V_{k,k+1}^c(K), \mathcal N_{k,k+1}^c(K))$ is the first kind N\'ed\'elec element \cite{Nedelec1980}, and $(K, V_{k,k}^c(K), \mathcal N_{k,k}^c(K))$ is the second kind N\'ed\'elec  element \cite{Nedelec1986}.

Next we present the explicit expression of the $H(\curl)$ bubble functions for $k\geq3$. Let
\[
\mathring V_{k}^c(K):=\{\bs v\in\mathbb P_{k}(K;\mathbb R^3): \bs v\times\bs n|_{\partial K}=0\},\quad
V_{k,b}^c(K):=\sum_{i=1}^4b_{F_i}\mathbb P_{k-3}(K)\nabla\lambda_{i}.
\]
\begin{lemma}
It holds for $k\geq3$ that
\begin{equation}\label{eq:Hcurlbubbleidentity}
\mathring V_{k}^c(K)=V_{k,b}^c(K).
\end{equation}
\end{lemma}
\begin{proof}
It is easy to see that $V_{k,b}^c(K)\subseteq\mathring V_{k}^c(K)$. Next let's prove $\mathring V_{k}^c(K)\subseteq V_{k,b}^c(K)$. Take any $\bs v\in\mathring V_{k}^c(K)$. Since $\{\nabla\lambda_{1}, \nabla\lambda_{2}, \nabla\lambda_{3}\}$ forms a basis of $\mathbb R^3$, we can express 
$$\bs v=q_1\nabla\lambda_{1}+q_2\nabla\lambda_{2}+q_3\nabla\lambda_{3},$$ 
where $q_1, q_2, q_3\in\mathbb P_k(K)$. Noting that $\bs v\times\bs n|_{\partial K}=0$, we get
\[
(q_2\nabla\lambda_{2}\times\bs n_1+q_3\nabla\lambda_{3}\times\bs n_1)|_{F_1}= 
(q_1\nabla\lambda_{1}\times\bs n_2+q_3\nabla\lambda_{3}\times\bs n_2)|_{F_2}= 
(q_1\nabla\lambda_{1}\times\bs n_3+q_2\nabla\lambda_{2}\times\bs n_3)|_{F_3}=0,
\]
\begin{equation}\label{eq:20201002-3}
(q_1\nabla\lambda_{1}\times\bs n_4+q_2\nabla\lambda_{2}\times\bs n_4+q_3\nabla\lambda_{3}\times\bs n_4)|_{F_4}=0.
\end{equation}
Hence $\lambda_2\lambda_3|q_1$, $\lambda_1\lambda_3|q_2$ and $\lambda_1\lambda_2|q_3$. This means
there exist $\tilde q_1, \tilde q_2, \tilde q_3\in\mathbb P_{k-2}(K)$ such that $q_1=\tilde q_1\lambda_2\lambda_3$, $q_2=\tilde q_2\lambda_1\lambda_3$ and $q_3=\tilde q_3\lambda_1\lambda_2$.  %Then
%\[
%\bs v=\tilde q_1\lambda_2\lambda_3\nabla\lambda_{1}+\tilde q_2\lambda_1\lambda_3\nabla\lambda_{2}+\tilde q_3\lambda_1\lambda_2\nabla\lambda_{3}.
%\]
Since $\nabla\lambda_{1}\times\bs n_4+\nabla\lambda_{2}\times\bs n_4+\nabla\lambda_{3}\times\bs n_4=\bs0$,
it follows from~\eqref{eq:20201002-3} that
\[
((q_1-q_3)\nabla\lambda_{1}\times\bs n_4+(q_2-q_3)\nabla\lambda_{2}\times\bs n_4)|_{F_4}=0.
\]
Consequently $\lambda_4|(\tilde q_1\lambda_3-\tilde q_3\lambda_1)$ and $\lambda_4|(\tilde q_2\lambda_3-\tilde q_3\lambda_2)$. 
Thus there exist $p_{ij}\in\mathbb P_{k-3}(K)$ ($i=1,2,3, j=1,2$) such that
\[
\tilde q_1=\lambda_1p_{11}+\lambda_4p_{12},
\quad \tilde q_2=\lambda_2p_{21}+\lambda_4p_{22}, \quad \tilde q_3=\lambda_3p_{31}+\lambda_4p_{32}.
\]
Moreover we have $\lambda_4|(p_{11}-p_{31})$ and $\lambda_4|(p_{21}-p_{31})$.
Therefore we acquire from $\nabla\lambda_{3}=-\nabla\lambda_{1}-\nabla\lambda_{2}-\nabla\lambda_{4}$ that
\begin{align*}
\bs v&=\tilde q_1\lambda_2\lambda_3\nabla\lambda_{1}+\tilde q_2\lambda_1\lambda_3\nabla\lambda_{2}+\tilde q_3\lambda_1\lambda_2\nabla\lambda_{3} \\
&=b_{F_4}(p_{11}\nabla\lambda_{1}+p_{21}\nabla\lambda_{2}+p_{31}\nabla\lambda_{3}) +p_{12}b_{F_1}\nabla\lambda_{1}+p_{22}b_{F_2}\nabla\lambda_{2}+p_{32}b_{F_3}\nabla\lambda_{3} \\
&=b_{F_4}((p_{11}-p_{31})\nabla\lambda_{1}+(p_{21}-p_{31})\nabla\lambda_{2}) - p_{31}b_{F_4}\nabla\lambda_{4} +p_{12}b_{F_1}\nabla\lambda_{1}+p_{22}b_{F_2}\nabla\lambda_{2}+p_{32}b_{F_3}\nabla\lambda_{3},
\end{align*}
which ends the proof.
\end{proof}

We have from \eqref{eq:Hcurlbubbleidentity} that
\begin{align*}
\dim\mathring V_{k}^c(K)=4\dim\mathbb P_{k-3}(K)-\dim\mathbb P_{k-4}(K)=\frac{1}{2}(k^2-1)(k-2).
\end{align*}
We can also acquire $\dim\mathring V_{k}^c(K)$ from the unisolvence of the second kind N\'ed\'elec element.

\section{Finite element de Rham complex}
Recall that a Hilbert complex is a sequence of Hilbert spaces connected by a sequence of linear operators satisfying the property: the composition of two consecutive operators is vanished. A Hilbert complex is exact means the range of each map is the kernel of the succeeding map. As $\Omega$ is topologically trivial, the following de Rham Complex of $\Omega$ is exact
\begin{equation}\label{eq:derham}
0\xrightarrow{} H^1(\Omega)\xrightarrow{\grad}\boldsymbol H(\curl;\Omega)\xrightarrow{\curl}\boldsymbol H(\div;\Omega)\xrightarrow{\div}L^2(\Omega) \xrightarrow{}0.
\end{equation}
