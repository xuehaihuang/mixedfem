\documentclass[tikz,border=1mm]{standalone}
\usepackage{tikz,tikz-3dplot}
\usepackage{amsmath}
\begin{document}

\tdplotsetmaincoords{80}{-35}
\begin{tikzpicture}[line join = round, line cap = round,tdplot_main_coords,scale=3]
%\tdplotsetrotatedcoords{20}{60}{180}
%\pgfmathsetmacro{\factor}{1/sqrt(2)};
\coordinate [label=right: $1$] (P1) at (1,0,0); 
\coordinate [label=left: $2$] (P2) at (0,1,0); 
\coordinate [label=above: $3$] (P3) at (0,0,1);
\coordinate [label=below: $0$] (P4) at (0,0,-0.1); 

%\filldraw[black] (P2) circle (0.75pt);
%\draw[->] (0,0) -- (3,0,0) node[right] {$x$};
%\draw[->] (0,0) -- (0,3,0) node[above] {$y$};
%\draw[->] (0,0) -- (0,0,3) node[below left] {$z$};
%\foreach \i in {A,B,C,D}
%    \draw[dashed] (0,0)--(\i);
%\begin{scope}[scale=2]
\draw[dashed] (P1)--(P2);
\draw[-, fill=white!30, opacity=.5] (P2)--(P3)--(P4)--cycle;
\draw[-, fill=white!30, opacity=.5] (P1) --(P4)--(P3)--cycle;
%\node[left] at (0.1,0.3,0.3) {$F_j$};

\draw[thick, ->, >=stealth] (P4) -- (P2);
\draw[thick, ->, >=stealth] (P4) -- (P1);
\draw[thick, ->, >=stealth] (P4) -- (P3);
\node[right] at (0,0,0.5) {$\boldsymbol{t}_{03}$};
%\node[below] at (0,0.5,0) {$t_2$};
\node[below] at (-0.45,0,0.1) {$\boldsymbol{t}_{02}$};
\node[below] at (0.45,0,-0.05) {$\boldsymbol{t}_{01}$};

\draw[thick, ->, >=stealth] (0.25,0,0.4) -- (0.25,-0.55,0.725); %n2
\draw[thick, dashed] (0.175,0.175,0) -- (0.175,0.175,-0.135); %n1
\draw[thick, ->, >=stealth] (0.175,0.175,-0.135) -- (0.175,0.175,-0.35); %n1
\draw[thick, ->, >=stealth] (-0.25,0,0.425) -- (-0.25,0.55,0.55); %n1
\node[right] at (0.25,-0.55,0.725) {$\boldsymbol n_{F_2}$};
\node[below] at (0.175,0.175,-0.35) {$\boldsymbol n_{F_3}$};
\node[left] at (-0.25,0.55,0.55) {$\boldsymbol n_{F_1}$};%\node[right] at (0.5,0,-0.5) {$n_3$};

%\draw[-, fill=purple!30, opacity=.5] (P2)--(P4)--(P3)--cycle;
%\end{scope}

\end{tikzpicture}

\end{document}