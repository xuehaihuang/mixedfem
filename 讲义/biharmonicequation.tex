% !TEX root = lecture.tex
\chapter{重调和方程混合元方法}

The biharmonic equation with homogenous Dirichlet
boundary condition is given by
\begin{equation}\label{eq:biharmonic}
\begin{cases}
\Delta^2u=f\quad \textrm{in }\Omega, \\
u|_{\partial\Omega}=\partial_{n}u|_{\partial\Omega}=0.
\end{cases}
\end{equation}
The biharmonic equation
\eqref{eq:biharmonic} arises in areas of continuum mechanics, including linear elasticity theory and the solution of Stokes flows.
In two dimensionas, the above problem is the Kirchhoff model describing the deflection of an elastic thin plate subject to a vertical
load $f$ \cite{FengShi1996,Reddy2006}.

Let $f\in H^{-2}(\Omega)$. The variational problem for the biharmonic equation
\eqref{eq:biharmonic} is to find $u\in H_0^2(\Omega)$ satisfying
\begin{equation}\label{eq:biharmonicvariatonalform}
a(u, v)=\langle f, v\rangle \quad\forall~v\in H_0^2(\Omega),
\end{equation}
where $a(u, v):=(\nabla^2u, \nabla^2v)$.
The variational problem~\eqref{eq:biharmonicvariatonalform} is wellposed since
\[
\|v\|_2\lesssim |v|_2\quad\forall~v\in H_0^2(\Omega).
\]

By the elliptic regularity theory for non-smooth domains (cf. \cite{BlumRannacher1980,Grisvard1985,Dauge1988,Grisvard1992}), there exists $\alpha\in (\frac{1}{2}, 1]$ such that for $f\in H^{-2+\alpha}(\Omega)$, the solution $u$ of problem~\eqref{eq:biharmonicvariatonalform}
belongs to $H^{2+\alpha}(\Omega)$ and
\[
\|u\|_{H^{2+\alpha}(\Omega)}\lesssim \|f\|_{H^{-2+\alpha}(\Omega)}.
\]
When $\Omega$ is convex, we can take $\alpha=1$, i.e.,
\[
\|u\|_{H^{3}(\Omega)}\lesssim \|f\|_{H^{-1}(\Omega)}.
\]





\section{Hellan-Herrmann-Johnson Mixed Method}

In this section, assume $f\in H^{-1}(\Omega)$.
Introducing $\bs\sigma:=-\nabla^2 u$, rewrite the biharmonic equation \eqref{eq:biharmonic}
\begin{equation}\label{eq:biharmonic2variables}
\begin{cases}
\bs\sigma =-\nabla^2 u\qquad\;\;\; \textrm{in }\Omega, \\
\div\div\bs\sigma=-f\quad \textrm{in }\Omega, \\
u|_{\partial\Omega}=\partial_{n}u|_{\partial\Omega}=0.
\end{cases}
\end{equation}
Define the Hilbert space (cf. \cite{PechsteinSchoberl2011})
\[
\boldsymbol{H}^{-1}(\div{\div },\Omega; \mathbb{S}):=\{\boldsymbol{\tau}\in \boldsymbol{L}^{2}(\Omega; \mathbb{S}): \div \div\boldsymbol{\tau}\in H^{-1}(\Omega)\}
\]
with squared norm $
\|\boldsymbol{\tau}\|_{\boldsymbol{H}^{-1}(\div\div)}^2:=\|\boldsymbol{\tau}\|_{0}^2+\|\div \div\boldsymbol{\tau}\|_{-1}^2
$.

The Hellan-Herrmann-Johnson (HHJ) mixed formulation~\cite{KrendlRafetsederZulehner2016,Hellan1967,Herrmann1967,Johnson1973} of the biharmonic equation \eqref{eq:biharmonic2variables} in two dimensions is to find $(\boldsymbol{\sigma} , u)\in \boldsymbol{H}^{-1}(\div\div,\Omega; \mathbb{S})\times H_0^1(\Omega)$ such that
\begin{align}
a(\boldsymbol\sigma, \boldsymbol\tau)+ b(\boldsymbol\tau, u) & =0 \quad\quad\quad\;\; \forall~\boldsymbol\tau\in\boldsymbol{H}^{-1}(\div\div,\Omega; \mathbb{S}), \label{eq:hhjmixedformulation1}\\
b(\boldsymbol\sigma, v) & =-\langle f, v\rangle  \quad \forall~v\in H_0^1(\Omega). \label{eq:hhjmixedformulation2}
\end{align}
where
\[
a(\boldsymbol\sigma, \boldsymbol\tau):=(\boldsymbol\sigma, \boldsymbol\tau),\quad b(\boldsymbol\tau, v):=\langle \div\div\boldsymbol\tau, v\rangle.
\]

Obviously
\[
a(\boldsymbol\sigma, \boldsymbol\tau)\leq\|\boldsymbol\sigma\|_0\|\boldsymbol\tau\|_0\leq\|\boldsymbol\sigma\|_{\boldsymbol{H}^{-1}(\div\div)}\|\boldsymbol\tau\|_{\boldsymbol{H}^{-1}(\div\div)} \quad\forall~\boldsymbol\sigma, \boldsymbol\tau\in \boldsymbol{H}^{-1}(\div\div,\Omega; \mathbb{S}),
\]
\[
b(\boldsymbol\tau, v)\leq\|\div\div\boldsymbol\tau\|_{-1}|v|_1\leq \|\boldsymbol\tau\|_{\boldsymbol{H}^{-1}(\div\div)}|v|_1 \quad\forall~\boldsymbol\tau\in \boldsymbol{H}^{-1}(\div\div,\Omega; \mathbb{S}), v\in H_0^1(\Omega).
\]
Define $B: \boldsymbol{H}^{-1}(\div\div,\Omega; \mathbb{S})\to H^{-1}(\Omega)$ by
\[
\langle B\bs\tau, v\rangle=b(\bs\tau, v)=\langle \div\div\boldsymbol\tau, v\rangle.
\]
Since $\ker B=\{\boldsymbol{\tau}\in \boldsymbol{L}^{2}(\Omega; \mathbb{S}): \div\div\boldsymbol{\tau}=0\}$, we have the coercivity on the kernel $\ker B$
\[
a(\boldsymbol\tau, \boldsymbol\tau)=\|\boldsymbol\tau\|_0^2=\|\boldsymbol\tau\|_{\boldsymbol{H}^{-1}(\div\div)}^2\quad\forall~\boldsymbol\tau\in\ker B.
\]

Given a scalar function $v$, we can embed it into the symmetric tensor space as $\iota (v) = v \boldsymbol I_{2\times 2}$. Following the proofs in \cite{BoffiBrezziFortin2013,BrezziRaviart1977}, we can see that for $v\in H_0^1(\Omega)$,
\[
b(-\iota(v), v)=-\langle \div\div(\iota(v)), v\rangle=-\langle \Delta v, v\rangle=|v|_1^2,
\]
\[
\|-\iota(v)\|_{\boldsymbol{H}^{-1}(\div\div)}^2=\|\iota(v)\|_0^2+|v|_1^2\lesssim |v|_1^2.
\]
Therefore we obtain the following inf-sup condition
\[
\sup_{\bs\tau\in\boldsymbol{H}^{-1}(\div\div,\Omega; \mathbb{S})}\frac{b(\boldsymbol\tau, v)}{\|\boldsymbol\tau\|_{\boldsymbol{H}^{-1}(\div\div)}}\geq\frac{b(-\iota(v), v)}{\|-\iota(v)\|_{\boldsymbol{H}^{-1}(\div\div)}}\gtrsim |v|_1.
\]

Then by the Brezzi Theory, the HHJ mixed formulation \eqref{eq:hhjmixedformulation1}-\eqref{eq:hhjmixedformulation2} is well-posed.

\begin{lemma}[Corollary 2.3 in \cite{KrendlRafetsederZulehner2016}]
The problem \eqref{eq:biharmonicvariatonalform} and the mixed formulation \eqref{eq:hhjmixedformulation1}-\eqref{eq:hhjmixedformulation2} are fully equivalent, i.e., if $u\in H_0^2(\Omega)$ solves problem \eqref{eq:biharmonicvariatonalform}, then $\bs\sigma=-\nabla^2 u\in \boldsymbol{H}^{-1}(\div\div,\Omega; \mathbb{S})$ and $(\bs\sigma, u)$ solves the mixed formulation \eqref{eq:hhjmixedformulation1}-\eqref{eq:hhjmixedformulation2}. And, vice versa, if $(\bs\sigma, u)\in \boldsymbol{H}^{-1}(\div\div,\Omega; \mathbb{S})\times H_0^1(\Omega)$ solves the mixed formulation \eqref{eq:hhjmixedformulation1}-\eqref{eq:hhjmixedformulation2}, then $u\in H_0^2(\Omega)$ and $u$ solves problem \eqref{eq:biharmonicvariatonalform}.
\end{lemma}
\begin{proof}
Both problems are uniquely solvable. Therefore, it suffices to show that $(\bs\sigma, u)$ with $\bs\sigma=-\nabla^2 u$ solves the mixed formulation \eqref{eq:hhjmixedformulation1}-\eqref{eq:hhjmixedformulation2}, if $u$ solves problem \eqref{eq:biharmonicvariatonalform}. Assume $u\in H_0^2(\Omega)$ is the solution of problem \eqref{eq:biharmonicvariatonalform}.
Then, obviously, $\bs\sigma\in\boldsymbol{L}^{2}(\Omega; \mathbb{S})$ and
\[
(\bs\sigma, \nabla^2v)=-\langle f, v\rangle\quad\forall~v\in H_0^2(\Omega),
\]
which implies that $\div\div\bs\sigma=-f\in H^{-1}(\Omega)$ in the distributional sense. Therefore, $\bs\sigma\in \boldsymbol{H}^{-1}(\div\div,\Omega; \mathbb{S})$ and \eqref{eq:hhjmixedformulation2} immediately follows.

By the definition of $\div\div\bs\tau$ in the distributional sense we have
\[
\langle \div\div\boldsymbol\tau, v\rangle=(\boldsymbol\tau, \nabla^2v)\quad\forall~v\in C_0^{\infty}(\Omega).
\]
Since $C_0^{\infty}(\Omega)$ is dense in $H_0^2(\Omega)$, it follows for $v=u$ that
\[
\langle \div\div\boldsymbol\tau, u\rangle=(\boldsymbol\tau, \nabla^2u)=-(\boldsymbol\tau, \bs\sigma),
\]
which shows \eqref{eq:hhjmixedformulation1}.
\end{proof}

Let
\[
H_{n,0}^{1/2}(\Gamma):=\{\partial_{n}\varphi: \varphi\in H^2(\Omega)\cap H_0^1(\Omega)\},
\]
which is equipped with norm
\[
\|q\|_{H_{n,0}^{1/2}(\Gamma)}:=\inf_{\varphi\in H^2(\Omega)\cap H_0^1(\Omega), \partial_{n}\varphi=q}\|\varphi\|_{H^2(\Omega)}.
\]
Define
\[
H_{n}^{-1/2}(\Gamma):=\left(H_{n,0}^{1/2}(\Gamma)\right)'.
\]
\begin{lemma}[Theorem 3.34 in \cite{Sinwel2009}]
Let $\Omega\subset\mathbb R^n$ with $n=2,3$ be a bounded domain with smooth or Lipschitz boundary
$\Gamma=\partial\Omega$.
\begin{enumerate}[(a)]
\item The trace operator $\bs\tau\in\boldsymbol{H}^{-1}(\div\div,\Omega; \mathbb{S})\to (\boldsymbol{n}^{\intercal}\bs\tau\boldsymbol{n})|_{\Gamma}\in H_{n}^{-1/2}(\Gamma)$ is surjective.
It holds
\[
\|\boldsymbol{n}^{\intercal}\bs\tau\boldsymbol{n}\|_{H_{n}^{-1/2}(\Gamma)}\lesssim \|\boldsymbol\tau\|_{\boldsymbol{H}^{-1}(\div\div)}\quad\forall~\boldsymbol\tau\in\boldsymbol{H}^{-1}(\div\div).
\]
\item For any $g\in H_{n}^{-1/2}(\Gamma)$, there exists some $\boldsymbol\tau\in\boldsymbol{H}^{-1}(\div\div)$ such that
\[
(\boldsymbol{n}^{\intercal}\bs\tau\boldsymbol{n})|_{\Gamma}=g\quad\textrm{and}\quad \|\boldsymbol\tau\|_{\boldsymbol{H}^{-1}(\div\div)}\lesssim \|g\|_{H_{n}^{-1/2}(\Gamma)}.
\]
\end{enumerate}
\end{lemma}


\begin{lemma}[Theorem 2.1 in \cite{PechsteinSchoberl2011} and Theorem 3.36 in \cite{Sinwel2009}]\label{lem:Hdivdivcontinuous}
Let $\Omega$, $\Omega_1$ and $\Omega_2$ are three open and bounded domains in $\mathbb R^n$. Suppose
$\bar{\Omega}=\bar{\Omega}_1\cup \bar{\Omega}_2$ and $\Omega_1\cap \Omega_2=\varnothing$. Let $\bs\tau\in \bs L^2(\Omega;\mathbb S)$ satisfy $\bs\tau|_{\Omega_1}\in \bs{C}^{0}(\bar{\Omega}_1;\mathbb S)\cap \bs{H}^1(\Omega_1;\mathbb S)$, $\bs\tau|_{\Omega_2}\in \bs{C}^{0}(\bar{\Omega}_2;\mathbb S)\cap \bs{H}^1(\Omega_2;\mathbb S)$, $\boldsymbol{t}^{\intercal}(\bs\tau|_{\Omega_1})\boldsymbol{n}\in H^{1/2}(\partial\Omega_1)$ and $\boldsymbol{t}^{\intercal}(\bs\tau|_{\Omega_2})\boldsymbol{n}\in H^{1/2}(\partial\Omega_2)$.
If the normal-normal component $\boldsymbol{n}^{\intercal}\bs\tau\boldsymbol{n}$ is continuous across the interface $\partial\Omega_1\cap\partial\Omega_2$, then $\bs\tau\in \boldsymbol{H}^{-1}(\div\div,\Omega; \mathbb{S})$.
\end{lemma}
\begin{remark}
The condition $\boldsymbol{t}^{\intercal}(\bs\tau|_{\Omega_1})\boldsymbol{n}\in H^{1/2}(\partial\Omega_1)$ is an inner-elemental continuity
constraint for the tangential stress vector $\boldsymbol{t}^{\intercal}(\bs\tau|_{\Omega_1})\boldsymbol{n}$ at vertices in 2D and at edges in 3D.
If we slightly weaken the regularity, the constraint disappears, and much simpler
elements can be used.
\end{remark}

\subsection{HHJ method}
By Lemma~\ref{lem:Hdivdivcontinuous}, it is natural to find the approximation of the moment $\bs\sigma$ in the finite-dimensional subspace of
\[
\Sigma^{\div\div}:=\{\bs\tau\in\bs L^2(\Omega;\mathbb S):\bs\tau\in \bs H^1(K;\mathbb S)\quad\forall~K\in\mathcal T_h\textrm{ and }\llbracket \boldsymbol{n}^{\intercal}\bs\tau\boldsymbol{n}\rrbracket|_e=0\textrm{ for each } e\in\mathcal E_h^i\}.
\]
For any $\bs\tau\in\Sigma^{\div\div}$ and $v\in C_0^{\infty}(\Omega)$, we have
\begin{align*}
\langle \div\div\boldsymbol\tau, v\rangle&=(\boldsymbol\tau, \nabla^2v)=-\sum_{K\in\mathcal T_h}(\div\boldsymbol\tau, \nabla v)_K+\sum_{K\in\mathcal T_h}(\boldsymbol\tau\boldsymbol{n}, \nabla v)_{\partial K} \\
&=-\sum_{K\in\mathcal T_h}(\div\boldsymbol\tau, \nabla v)_K+\sum_{K\in\mathcal T_h}(\boldsymbol{t}^{\intercal}\bs\tau\boldsymbol{n}, \partial_t v)_{\partial K}+\sum_{K\in\mathcal T_h}(\boldsymbol{n}^{\intercal}\bs\tau\boldsymbol{n}, \partial_{n} v)_{\partial K}  \\
&=-\sum_{K\in\mathcal T_h}(\div\boldsymbol\tau, \nabla v)_K+\sum_{K\in\mathcal T_h}(\boldsymbol{t}^{\intercal}\bs\tau\boldsymbol{n}, \partial_t v)_{\partial K}.
\end{align*}

Define finite element spaces
\begin{align*}
&\Sigma_h^{\div\div}:=\{\bs\tau\in\bs L^2(\Omega;\mathbb S):\bs\tau\in \mathbb P_{k-1}(K;\mathbb S)\quad\forall~K\in\mathcal T_h\textrm{ and }\llbracket \boldsymbol{n}^{\intercal}\bs\tau\boldsymbol{n}\rrbracket|_e=0\textrm{ for each } e\in\mathcal E_h^i\},
\\
&V_h:=\left\{v\in H_0^1(\Omega): v|_K\in \mathbb P_{k}(K)\quad\forall~K\in\mathcal T_h\right\}
\end{align*}
with $k\geq1$. Given $K\in\mathcal T_h$, the local degrees of freedom of the space $\boldsymbol\Sigma_h$ are
\begin{itemize}
\item $\int_e(\boldsymbol{n}^{\intercal}\bs\tau\boldsymbol{n})v\dd s\quad\forall~v\in \mathbb P_{k-1}(e), e\in\mathcal E_h(K)$;
\item $\int_K\bs\tau:\bs\varsigma\dx\quad\forall~\bs\varsigma\in \mathbb P_{k-2}(K, \mathbb S)$.
\end{itemize}
The local degrees of freedom of the space $V_h$ are
\begin{itemize}
\item $w(p)$ for each vertex $p$ of $K$;
\item $\int_ewv\dd s\quad\forall~v\in \mathbb P_{k-2}(e), e\in\mathcal E_h(K)$;
\item $\int_Kwv\dx\quad\forall~v\in \mathbb P_{k-3}(K)$.
\end{itemize}
The basis functions for symmetric tensors %corresponding to the degrees of freedom 
are
\begin{itemize}
\item Corresponding to the degrees of freedom on edge with vertices $p_i$ and $p_j$: 
$$
\lambda_i^{\ell}\lambda_j^{k-1-\ell}\sym(\boldsymbol{t}_i\otimes\boldsymbol{t}_j) \textrm{ for } \ell=0,1,\cdots, k-1;
$$
\item Corresponding to the interior degrees of freedom: 
$$
\lambda_1^{\alpha_1}\lambda_2^{\alpha_2}\lambda_3^{\alpha_3}\lambda_{\ell}\sym(\boldsymbol{t}_i\otimes \boldsymbol{t}_j) \textrm{ with } \alpha_1+\alpha_2+\alpha_3=k-2,
$$
where $(ij\ell)$ is a circular permutation of $(123)$.
\end{itemize}

The Hellan-Herrmann-Johnson mixed method for the mixed formulation \eqref{eq:hhjmixedformulation1}-\eqref{eq:hhjmixedformulation2} is to find
$(\boldsymbol{\sigma}_h, u_h)\in \Sigma_h^{\div\div}\times V_h$ such that
\begin{align}
a(\boldsymbol\sigma_h, \boldsymbol\tau_h)+ b_h(\boldsymbol\tau_h, u_h) & =0 \quad\quad\quad\quad \forall~\boldsymbol\tau_h\in\Sigma_h^{\div\div}, \label{eq:hhjmfem1}\\
b_h(\boldsymbol\sigma_h, v_h) & =-\langle f, v_h\rangle  \quad \forall~v_h\in V_h, \label{eq:hhjmfem2}
\end{align}
where
\[
b_h(\boldsymbol\sigma_h, v_h):=-\sum_{K\in\mathcal T_h}(\div\boldsymbol\sigma_h, \nabla v_h)_K+\sum_{K\in\mathcal T_h}(\boldsymbol{t}^{\intercal}\boldsymbol\sigma_h\boldsymbol{n}, \partial_t v_h)_{\partial K}.
\]
Applying integration by parts, it holds
\[
b_h(\boldsymbol\sigma_h, v_h)=\sum_{K\in\mathcal T_h}(\boldsymbol\sigma_h, \nabla^2v_h)_K-\sum_{K\in\mathcal T_h}(\boldsymbol{n}^{\intercal}\boldsymbol\sigma_h\boldsymbol{n}, \partial_{n} v_h)_{\partial K}.
\]

Introduce mesh dependent norms
\begin{align*}
\|\boldsymbol{\tau}\|_{0,h}^2&:=\|\boldsymbol{\tau}\|_0^2+\sum_{e\in\mathcal{E}_h}h_e\|\boldsymbol{n}^{\intercal}\bs\tau\boldsymbol{n}\|_{0,e}^2\quad\forall~\bs\tau\in\Sigma^{\div\div},\\
\|v\|_{2,h}^2&:=\sum_{K\in\mathcal{T}_h}|v|_{2,K}^2+\sum_{e\in\mathcal{E}_h}h_e^{-1}\|\llbracket\partial_{n_e}v\rrbracket\|_{0,e}^2\quad\forall~v\in V:=H^2(\mathcal T_h)\cap H_0^1(\Omega).
\end{align*}

To derive the error estimates of the HHJ mixed method \eqref{eq:hhjmfem1}-\eqref{eq:hhjmfem2},  we need some interpolation operators. First, define $I_h^{\div\div}: \Sigma^{\div\div}\to \Sigma_h^{\div\div}$ in the following way (cf. \cite{BabuvskaOsbornPitkaranta1980,FalkOsborn1980,Comodi1989,BoffiBrezziFortin2013}): given $\boldsymbol{\tau}\in\Sigma^{\div\div}$, for any element $K\in \mathcal{T}_h$ and any edge $e$ of $K$,
\[
\int_e\boldsymbol{n}^{\intercal}(\boldsymbol{\tau}-I_h^{\div\div}\boldsymbol{\tau})\boldsymbol{n}\,q\dd s=0 \quad \forall~q\in \mathbb P_{k-1}(e),
\]
\[
\int_K(\boldsymbol{\tau}-I_h^{\div\div}\boldsymbol{\tau}):\boldsymbol{\varsigma}\dx=0 \quad \forall~\boldsymbol{\varsigma}\in \mathbb{P}_{k-2}(K, \mathbb{S}).
\]
It admits (cf. \cite{BabuvskaOsbornPitkaranta1980,FalkOsborn1980,Comodi1989,BoffiBrezziFortin2013})
\begin{equation}\label{eq:bpi}
b_h(\boldsymbol{\tau}-I_h^{\div\div}\boldsymbol{\tau}, v_h)=0 \quad \forall~\boldsymbol{\tau}\in \Sigma^{\div\div}, v_h\in V_h.
\end{equation}
Next, define $P_h: V\to V_h$ in the following way (cf. \cite{BabuvskaOsbornPitkaranta1980,FalkOsborn1980,Comodi1989,Stenberg1991}) : given $w\in V$, for any element $K\in \mathcal{T}_h$, any vertex $p$ of $K$ and any edge $e$ of $K$,
\[
P_hw(p)=w(p),
\]
\[
\int_e(w-P_hw)v\dd s=0 \quad \forall~v\in \mathbb P_{k-2}(e),
\]
\[
\int_K(w-P_hw)v\dx=0 \quad \forall~v\in \mathbb P_{k-3}(K).
\]
According to the definition of $P_h$,
we have~(cf. \cite{BabuvskaOsbornPitkaranta1980,FalkOsborn1980,Comodi1989,BoffiBrezziFortin2013})
\begin{equation}\label{eq:bp0}
b_h(\boldsymbol{\tau}_h, v-P_hv)=0 \quad \forall~ \boldsymbol{\tau}_h\in \Sigma_h^{\div\div}, v\in V.
\end{equation}
The error estimates for interpolation operators $I_h^{\div\div}$ and $P_h$ are summarized in the following lemma(cf. \cite{BabuvskaOsbornPitkaranta1980,FalkOsborn1980,Comodi1989,Stenberg1991}).
\begin{lemma}\label{lem:hhjinterpolation}
For all $v\in H^{m+3}(K)$, $\boldsymbol{\tau}\in \boldsymbol{H}^{m+1}(\Omega, \mathbb{S})$ with $m$ a non-negative integer and
all $K\in\mathcal{T}_h$, we have the estimates
\begin{align*}
\left\|\boldsymbol{\tau}-I_h^{\div\div}\boldsymbol{\tau}\right\|_{0,K}+h_K^{1/2}\left|\boldsymbol{\tau}-I_h^{\div\div}\boldsymbol{\tau}\right|_{0,\partial K} &\lesssim
h_K^{\min\{m+1,k\}}|\boldsymbol{\tau}|_{m+1,K}, \\
\|v-P_hv\|_{0,K}+h_K\left|v-P_hv\right|_{1,K}+h_K^2\left|v-P_hv\right|_{2,K}\qquad\;\;\;&
 \\
 +h_K^{3/2}\left\|\boldsymbol{\nabla}(v-P_hv)\right\|_{0,\partial K} &\lesssim
h^{\min\{m+2,k\}+1}_K|v|_{m+3,K}.
\end{align*}
\end{lemma}

\begin{lemma}[Lemma~4.2 in \cite{HuangHuangXu2011}]
There holds the following inf-sup condition
\begin{equation}\label{eq:HHJinfsup}
\|v_h\|_{2,h}\lesssim \sup_{\bs\tau_h\in\Sigma_h^{\div\div}}\frac{b_h(\bs\tau_h, v_h)}{\|\bs\tau_h\|_{0,h}}\quad\forall~v_h\in V_h.
\end{equation}
\end{lemma}
\begin{proof}
Let $\bs\tau_h\in\Sigma_h^{\div\div}$ such that for any $K\in\mathcal T_h$ and $e\in\mathcal E_h$,
\begin{align*}
\boldsymbol{n}^{\intercal}\bs\tau\boldsymbol{n}&=\frac{-1}{h_e}\llbracket\partial_{n_e}v_h\rrbracket\quad\textrm{ on  } e,
\\
\int_K\bs\tau_h:\bs\varsigma\dx&=\int_K\nabla^2v_h:\bs\varsigma\dx\quad\forall~\bs\varsigma\in \mathbb P_{k-2}(K;\mathbb S).
\end{align*}
It follows from the scaling argument
\begin{equation}\label{eq:temp20180429-1}
\|\bs\tau_h\|_{0,h}\lesssim \|v_h\|_{2,h}.
\end{equation}
And we also have
\[
b_h(\bs\tau_h, v_h)=\sum_{K\in\mathcal T_h}(\boldsymbol\tau_h, \nabla^2v_h)_K+\sum_{K\in\mathcal T_h}(\boldsymbol{n}^{\intercal}\bs\tau_h\boldsymbol{n}, \partial_{n} v_h)_{\partial K}=\|v_h\|_{2,h}^2,
\]
which together with \eqref{eq:temp20180429-1} implies \eqref{eq:HHJinfsup}.
\end{proof}

Apparently both the bilinear forms $a(\cdot, \cdot)$ and $b_h(\cdot, \cdot)$ are continuous with respect to the
mesh dependent norms $\|\cdot\|_{0,h}$ and $\|\cdot\|_{2,h}$. And by the inverse inequality, we have
\[
\|\bs\tau_h\|_{0,h}^2\lesssim \|\bs\tau_h\|_0^2=a(\bs\tau_h,\bs\tau_h)\quad\forall~\bs\tau_h\in\Sigma_h^{\div\div}.
\]
Then
by Brezzi's theory, we have the following inf-sup condition
\begin{equation}\label{eq:HHJinfsup2}
\|\widetilde{\bs\sigma}_h\|_{0,h}+\|\widetilde{u}_h\|_{2,h}\lesssim \sup_{\bs\tau_h\in\Sigma_h^{\div\div}, v_h\in V_h}\frac{a(\widetilde{\boldsymbol\sigma}_h, \boldsymbol\tau_h)+ b_h(\boldsymbol\tau_h, \widetilde{u}_h)+b_h(\widetilde{\boldsymbol\sigma}_h, v_h)}{\|\bs\tau_h\|_{0,h}+\|v_h\|_{2,h}}
\end{equation}
for any $\widetilde{\bs\sigma}_h\in\Sigma_h^{\div\div}$ and $\widetilde{u}_h\in V_h$.

The well-posedness of the HHJ mixed method \eqref{eq:hhjmfem1}-\eqref{eq:hhjmfem2} follows from the inf-sup condition \eqref{eq:HHJinfsup2}.

\begin{theorem}
Let $(\boldsymbol{\sigma} , u)\in \boldsymbol{H}^{-1}(\div\div,\Omega; \mathbb{S})\times H_0^1(\Omega)$ be the solution of the HHJ mixed formulation \eqref{eq:hhjmixedformulation1}-\eqref{eq:hhjmixedformulation2}, and $(\boldsymbol{\sigma}_h, u_h)\in\Sigma_h^{\div\div}\times V_h$ be the HHJ mixed finite element method \eqref{eq:hhjmfem1}-\eqref{eq:hhjmfem2}.
Assume $\boldsymbol{\sigma}\in \boldsymbol{H}^{m+1}(\Omega, \mathbb{S})$ and $u\in H^{m+3}(\Omega)$ for
some non-negative integer $m$.
Then
\begin{equation}\label{eq:hhjenergyerror}
\|\bs\sigma-\bs\sigma_h\|_{0,h}+\|P_hu-u_h\|_{2,h}\lesssim h^{\min\{m+1,k\}}\|\bs\sigma\|_{m+1},%(\|\bs\sigma\|_{m+1}+\|u\|_{m+3}).
\end{equation}
\[
\|u-u_h\|_{2,h}\lesssim h^{\min\{m+1,k-1\}}\|u\|_{m+3}.%(\|\bs\sigma\|_{m+1}+\|u\|_{m+3}).
\]
\end{theorem}
\begin{proof}
Using integration by parts, we get from problem \eqref{eq:biharmonic2variables}
\begin{align*}
a(\boldsymbol\sigma, \boldsymbol\tau_h)+ b_h(\boldsymbol\tau_h, u) & =0 \quad\quad\quad\quad \forall~\boldsymbol\tau_h\in\Sigma_h^{\div\div}, \\
b_h(\boldsymbol\sigma, v_h) & =-\langle f, v_h\rangle  \quad \forall~v_h\in V_h.
\end{align*}
Subtracting \eqref{eq:hhjmfem1} and \eqref{eq:hhjmfem2} from the last two inequalities, we achieve the following orthogonality
\begin{align}
a(\boldsymbol\sigma-\boldsymbol\sigma_h, \boldsymbol\tau_h)+ b_h(\boldsymbol\tau_h, u-u_h) & =0 \quad\quad \forall~\boldsymbol\tau_h\in\Sigma_h^{\div\div}, \label{eq:temp20180502-1}\\
b_h(\boldsymbol\sigma-\boldsymbol\sigma_h, v_h) & =0  \quad\quad \forall~v_h\in V_h. \label{eq:temp20180502-2}
\end{align}
Then from \eqref{eq:bpi} and \eqref{eq:bp0}, it holds for any $\boldsymbol\tau_h\in\Sigma_h^{\div\div}$ and $v_h\in V_h$
\begin{align*}
&\,a(I_h^{\div\div}\boldsymbol\sigma-\boldsymbol\sigma_h, \boldsymbol\tau_h)+ b_h(\boldsymbol\tau_h, P_hu-u_h)+ b_h(I_h^{\div\div}\boldsymbol\sigma-\boldsymbol\sigma_h, v_h)  \\
=&\,a(I_h^{\div\div}\boldsymbol\sigma-\boldsymbol\sigma, \boldsymbol\tau_h)+ b_h(\boldsymbol\tau_h, P_hu-u)+ b_h(I_h^{\div\div}\boldsymbol\sigma-\boldsymbol\sigma, v_h) \\
=&\,a(I_h^{\div\div}\boldsymbol\sigma-\boldsymbol\sigma, \boldsymbol\tau_h).
\end{align*}
Taking $\widetilde{\bs\sigma}_h=I_h^{\div\div}\boldsymbol{\sigma}-\boldsymbol{\sigma}_h$ and $\widetilde{u}_h=P_hu-u_h$ in \eqref{eq:HHJinfsup2},
we have
\begin{align*}
&\,\|I_h^{\div\div}\boldsymbol{\sigma}-\boldsymbol{\sigma}_h\|_{0,h}+\|P_hu-u_h\|_{2,h}\\
\lesssim&\, \sup_{\bs\tau_h\in\Sigma_h^{\div\div}, v_h\in V_h}\frac{a(I_h^{\div\div}\boldsymbol\sigma-\boldsymbol\sigma_h, \boldsymbol\tau_h)+ b_h(\boldsymbol\tau_h, P_hu-u_h)+ b_h(I_h^{\div\div}\boldsymbol\sigma-\boldsymbol\sigma_h, v_h)}{\|\bs\tau_h\|_{0,h}+\|v_h\|_{2,h}} \\
\lesssim &\,\sup_{\bs\tau_h\in\Sigma_h^{\div\div}, v_h\in V_h}\frac{a(I_h^{\div\div}\boldsymbol\sigma-\boldsymbol\sigma, \boldsymbol\tau_h)}{\|\bs\tau_h\|_{0,h}+\|v_h\|_{2,h}}\lesssim \|\boldsymbol\sigma-I_h^{\div\div}\boldsymbol\sigma\|_0,
\end{align*}
which combined with the triangle inequality gives
\[
\|\boldsymbol{\sigma}-\boldsymbol{\sigma}_h\|_{0,h}+\|P_hu-u_h\|_{2,h}\lesssim \|\boldsymbol\sigma-I_h^{\div\div}\boldsymbol\sigma\|_{0,h}.
\]
Finally we ends the proof by applying Lemma~\ref{lem:hhjinterpolation}.
\end{proof}


To derive the error estimate of $|u-u_h|_{1}$, we present the second discrete inf-sup condition. Define $(\div\div)_h:\Sigma_h^{\div\div}\to V_h$ by
$$
((\div\div)_h\boldsymbol{\tau}_h, v_h)=b_h(\bs\tau_h, v_h).
$$
Then equip space $\Sigma_h^{\div\div}$ with the squared mesh-dependent norm
$$
\|\boldsymbol{\tau}_h\|_{H_h^{-1}(\div\div)}^2:=\|\boldsymbol{\tau}_h\|_{0}^2+ \|(\div\div)_h\boldsymbol{\tau}_h\|_{-1,h}^2\quad\textrm{ with } \|w\|_{-1,h}:=\sup_{v_h\in V_h}\frac{(w, v_h)}{|v_h|_1}.
$$
By \eqref{eq:bpi} and \eqref{eq:temp20180502-2}, we have
$$
(\div\div)_h(I_h^{\div\div}\sigma)=(\div\div)_h\sigma=(\div\div)_h\sigma_h.
$$

\begin{lemma}
We have $(\div\div)_h\Sigma_h^{\div\div}=V_h$, and
the discrete inf-sup condition
\begin{equation}\label{eq:HHJdiscreteinfsup2}
\|v_h\|_{1}\lesssim \sup_{\bs\tau_h\in\Sigma_h^{\div\div}}\frac{b_h(\bs\tau_h, v_h)}{\|\bs\tau_h\|_{H_h^{-1}(\div\div)}}\quad\forall~v_h\in V_h.
\end{equation}
\end{lemma}
\begin{proof}
For $v_h\in V_h$, set $\bs\tau_h=-I_h^{\div\div}(v_hI)$. By \eqref{eq:bpi},
$$
b_h(\bs\tau_h, v_h)=-b_h(v_hI, v_h)=|v_h|_1^2.
$$
And
$$
\|\boldsymbol{\tau}_h\|_{H_h^{-1}(\div\div)}\lesssim\|\boldsymbol{\tau}_h\|_{0}+  \sup_{v_h\in V_h}\frac{b_h(\bs\tau_h, v_h)}{|v_h|_1}\lesssim |v_h|_1.
$$
Thus the discrete inf-sup condition \eqref{eq:HHJdiscreteinfsup2} holds. Combine the discrete inf-sup condition \eqref{eq:HHJdiscreteinfsup2} and the fact $(\div\div)_h\Sigma_h^{\div\div}\subseteq V_h$ to get $(\div\div)_h\Sigma_h^{\div\div}=V_h$.
\end{proof}

Both the bilinear forms $a(\cdot, \cdot)$ and $b_h(\cdot, \cdot)$ are continuous with respect to norms $\|\cdot\|_{H_h^{-1}(\div\div)}$ and $|\cdot|_{1}$. And by the inverse inequality, we have
\[
\|\bs\tau_h\|_{H_h^{-1}(\div\div)}^2=\|\bs\tau_h\|_0^2=a(\bs\tau_h,\bs\tau_h)\quad\forall~\bs\tau_h\in\Sigma_h^{\div\div}\ker((\div\div)_h).
\]
Then
by Brezzi's theory, we have the following inf-sup condition
\begin{equation}\label{eq:HHJinfsup3}
\|\widetilde{\bs\sigma}_h\|_{H_h^{-1}(\div\div)}+|\widetilde{u}_h|_{1}\lesssim \sup_{\bs\tau_h\in\Sigma_h^{\div\div}, v_h\in V_h}\frac{a(\widetilde{\boldsymbol\sigma}_h, \boldsymbol\tau_h)+ b_h(\boldsymbol\tau_h, \widetilde{u}_h)+b_h(\widetilde{\boldsymbol\sigma}_h, v_h)}{\|\bs\tau_h\|_{H_h^{-1}(\div\div)}+|v_h|_{1}}
\end{equation}
for any $\widetilde{\bs\sigma}_h\in\Sigma_h^{\div\div}$ and $\widetilde{u}_h\in V_h$.


\begin{theorem}
Let $(\boldsymbol{\sigma} , u)\in \boldsymbol{H}^{-1}(\div\div,\Omega; \mathbb{S})\times H_0^1(\Omega)$ be the solution of the HHJ mixed formulation \eqref{eq:hhjmixedformulation1}-\eqref{eq:hhjmixedformulation2}, and $(\boldsymbol{\sigma}_h, u_h)\in\Sigma_h^{\div\div}\times V_h$ be the HHJ mixed finite element method \eqref{eq:hhjmfem1}-\eqref{eq:hhjmfem2}.
Assume $\boldsymbol{\sigma}\in \boldsymbol{H}^{m+1}(\Omega, \mathbb{S})$ and $u\in H^{m+3}(\Omega)$ for
some non-negative integer $m$.
Then
% \begin{equation*}%\label{eq:hhjenergyerror2}
% \|\bs\sigma-\bs\sigma_h\|_{H_h^{-1}(\div\div)}\lesssim h^{\min\{m+1,k\}}\|\bs\sigma\|_{m+1},%(\|\bs\sigma\|_{m+1}+\|u\|_{m+3}).
% \end{equation*}
\[
|u-u_h|_{1}\lesssim h^{\min\{m+1,k\}}\|u\|_{m+3}.%(\|\bs\sigma\|_{m+1}+\|u\|_{m+3}).
\]
\end{theorem}
\begin{proof}
Taking $\widetilde{\bs\sigma}_h=\boldsymbol{\Pi}_h\boldsymbol{\sigma}-\boldsymbol{\sigma}_h$ and $\widetilde{u}_h=P_hu-u_h$ in \eqref{eq:HHJinfsup3},
we have
\begin{align*}
&\,\|\boldsymbol{\Pi}_h\boldsymbol{\sigma}-\boldsymbol{\sigma}_h\|_{H_h^{-1}(\div\div)}+|P_hu-u_h|_{1}\\
\lesssim&\, \sup_{\bs\tau_h\in\Sigma_h^{\div\div}, v_h\in V_h}\frac{a(\boldsymbol{\Pi}_h\boldsymbol\sigma-\boldsymbol\sigma_h, \boldsymbol\tau_h)+ b_h(\boldsymbol\tau_h, P_hu-u_h)+ b_h(\boldsymbol{\Pi}_h\boldsymbol\sigma-\boldsymbol\sigma_h, v_h)}{\|\bs\tau_h\|_{H_h^{-1}(\div\div)}+|v_h|_{1}} \\
\lesssim &\,\sup_{\bs\tau_h\in\Sigma_h^{\div\div}, v_h\in V_h}\frac{a(\boldsymbol{\Pi}_h\boldsymbol\sigma-\boldsymbol\sigma, \boldsymbol\tau_h)}{\|\bs\tau_h\|_{H_h^{-1}(\div\div)}+|v_h|_{1}}\lesssim \|\boldsymbol\sigma-\boldsymbol{\Pi}_h\boldsymbol\sigma\|_0,
\end{align*}
which combined with the triangle inequality gives
\[
\|\bs\sigma-\bs\sigma_h\|_{0}+|P_hu-u_h|_{1}\lesssim \|\boldsymbol\sigma-\boldsymbol{\Pi}_h\boldsymbol\sigma\|_{0,h}.
\]
Finally we end the proof by applying Lemma~\ref{lem:hhjinterpolation}.
\end{proof}


Using the usual duality argument, we can additionally derive a superconvergent error estimate between $u_h$ and $P_hu$ in $H^1$ norm.
Let $(\widetilde{\boldsymbol{\sigma}},\widetilde{u})$ be the solution of the
auxiliary problem:
\begin{equation}\label{eq:hhjdual1}
\left\{
\begin{array}{ll}
\widetilde{\boldsymbol{\sigma}}=-\nabla^2\widetilde{u} & \text{in}\ \Omega, \\
\div\div\widetilde{\boldsymbol{\sigma}}
=\Delta(P_hu-u_h) & \text{in}\ \Omega, \\
\widetilde{u}=\partial_{n}\widetilde{u}=0 & \text{on}\ \partial
\Omega.
\end{array}
\right.
\end{equation}
Since $\triangle(P_hu-u_h)\not\in
L^2(\Omega)$,
the second equation of \eqref{eq:hhjdual1} is interpreted by the following
relation
\begin{equation}
\int_\Omega (\div\widetilde{\boldsymbol{\sigma}})\cdot
\nabla v\dx=\int_\Omega \nabla(P_hu-u_h)\cdot
\nabla v\dx\quad \forall\,v\in H^1_0(\Omega).
\label{hhjdual2}
\end{equation}
We assume that
$\widetilde{u}\in H^3(\Omega)\cap H_0^2(\Omega)$ with
the bound
\begin{equation}
\label{hhjregularity}
\|\widetilde{\boldsymbol{\sigma}}\|_{1}+\|\widetilde{u}\|_{3}\lesssim |P_hu-u_h|_{1}.
\end{equation}
When $\Omega$ is a convex bounded polygonal domain, the regularity result \eqref{hhjregularity} has been obtained in \cite{Dauge1988,Grisvard1992}.

\begin{theorem}
Let $(\boldsymbol{\sigma} , u)\in \boldsymbol{H}^{-1}(\div\div,\Omega; \mathbb{S})\times H_0^1(\Omega)$ be the solution of the HHJ mixed formulation \eqref{eq:hhjmixedformulation1}-\eqref{eq:hhjmixedformulation2}, and $(\boldsymbol{\sigma}_h, u_h)\in \Sigma_h^{\div\div}\times V_h$ be the HHJ mixed finite element method \eqref{eq:hhjmfem1}-\eqref{eq:hhjmfem2}.
Assume the regularity condition \eqref{hhjregularity} holds, $\boldsymbol{\sigma}\in \boldsymbol{H}^{m+1}(\Omega, \mathbb{S})$ and $u\in H^{m+3}(\Omega)$ for
some non-negative integer $m$.
Then
\begin{equation}\label{eq:hhjH1supererror}
|P_hu-u_h|_{1}\lesssim h^{\min\{m+1,k\}+1}(\|\bs\sigma\|_{m+1}+\delta_{k1}\|f\|_0).%(\|\bs\sigma\|_{m+1}+\|u\|_{m+3}).
\end{equation}
% \[
% |u-u_h|_{1}\lesssim h^{\min\{m+1,k\}}(h\|\bs\sigma\|_{m+1}+\|u\|_{m+2}+\delta_{k1}h\|f\|_0).%(\|\bs\sigma\|_{m+1}+\|u\|_{m+3}).
% \]
\end{theorem}
\begin{proof}
Taking $v=P_hu-u_h$ in \eqref{hhjdual2}, we get from \eqref{eq:bpi}-\eqref{eq:bp0} and \eqref{eq:temp20180502-1}
\begin{align*}
|P_hu-u_h|_1^2&=(\div\widetilde{\boldsymbol{\sigma}}, \nabla(P_hu-u_h))=-b_h(\widetilde{\boldsymbol{\sigma}}, P_hu-u_h)\\
&=-b_h(I_h^{\div\div}\widetilde{\boldsymbol{\sigma}}, P_hu-u_h)=-b_h(I_h^{\div\div}\widetilde{\boldsymbol{\sigma}}, u-u_h) \\
&=a(\boldsymbol\sigma-\boldsymbol\sigma_h, I_h^{\div\div}\widetilde{\boldsymbol{\sigma}}).
\end{align*}
Using \eqref{eq:hhjdual1}, \eqref{eq:temp20180502-2} and \eqref{eq:bp0}, it holds
\begin{align*}
a(\boldsymbol\sigma-\boldsymbol\sigma_h, \widetilde{\boldsymbol{\sigma}})&=-a(\boldsymbol\sigma-\boldsymbol\sigma_h, \nabla^2\widetilde{u})=-(\boldsymbol\sigma-\boldsymbol\sigma_h, \nabla^2\widetilde{u})\\
&=-b_h(\boldsymbol\sigma-\boldsymbol\sigma_h, \widetilde{u})=-b_h(\boldsymbol\sigma-\boldsymbol\sigma_h, \widetilde{u}-P_h\widetilde{u}) \\
&=-b_h(\boldsymbol\sigma, \widetilde{u}-P_h\widetilde{u}).
\end{align*}
Hence
\begin{align}
|P_hu-u_h|_1^2&=a(\boldsymbol\sigma-\boldsymbol\sigma_h, I_h^{\div\div}\widetilde{\boldsymbol{\sigma}})=a(\boldsymbol\sigma-\boldsymbol\sigma_h, I_h^{\div\div}\widetilde{\boldsymbol{\sigma}}-\widetilde{\boldsymbol{\sigma}})+a(\boldsymbol\sigma-\boldsymbol\sigma_h, \widetilde{\boldsymbol{\sigma}}) \notag\\
&=a(\boldsymbol\sigma-\boldsymbol\sigma_h, I_h^{\div\div}\widetilde{\boldsymbol{\sigma}}-\widetilde{\boldsymbol{\sigma}})-b_h(\boldsymbol\sigma, \widetilde{u}-P_h\widetilde{u}).\label{eq:temp20180502-3}
\end{align}

If $k=1$, we acquire from \eqref{eq:temp20180502-3}, \eqref{eq:hhjmixedformulation2}, Lemma~\ref{lem:hhjinterpolation} and \eqref{eq:hhjenergyerror}
\begin{align*}
|P_hu-u_h|_1^2&=a(\boldsymbol\sigma-\boldsymbol\sigma_h, I_h^{\div\div}\widetilde{\boldsymbol{\sigma}}-\widetilde{\boldsymbol{\sigma}})+(f, \widetilde{u}-P_h\widetilde{u}) \\
&\lesssim \|\boldsymbol\sigma-\boldsymbol\sigma_h\|_0\|I_h^{\div\div}\widetilde{\boldsymbol{\sigma}}-\widetilde{\boldsymbol{\sigma}}\|_0 + \|f\|_0\|\widetilde{u}-P_h\widetilde{u}\|_0
\\
&\lesssim h\|\boldsymbol\sigma-\boldsymbol\sigma_h\|_0\|\widetilde{\boldsymbol{\sigma}}\|_1 +h^2\|f\|_0\|\widetilde{u}\|_2\lesssim (h\|\boldsymbol\sigma-\boldsymbol\sigma_h\|_0+h^2\|f\|_0)(\|\widetilde{\boldsymbol{\sigma}}\|_1 +\|\widetilde{u}\|_2) \\
&\lesssim h^2(\|\boldsymbol\sigma\|_1+\|f\|_0)(\|\widetilde{\boldsymbol{\sigma}}\|_1 +\|\widetilde{u}\|_2)
\end{align*}
Then it follows from \eqref{hhjregularity}
\[
|P_hu-u_h|_1\lesssim h^2(\|\boldsymbol\sigma\|_1+\|f\|_0).
\]

If $k\geq2$, we get from \eqref{eq:temp20180502-3}, \eqref{eq:bp0}, Lemma~\ref{lem:hhjinterpolation},
\begin{align*}
|P_hu-u_h|_1^2&=a(\boldsymbol\sigma-\boldsymbol\sigma_h, I_h^{\div\div}\widetilde{\boldsymbol{\sigma}}-\widetilde{\boldsymbol{\sigma}})-b_h(\boldsymbol\sigma, \widetilde{u}-P_h\widetilde{u}) \\
&=a(\boldsymbol\sigma-\boldsymbol\sigma_h, I_h^{\div\div}\widetilde{\boldsymbol{\sigma}}-\widetilde{\boldsymbol{\sigma}})-b_h(\boldsymbol\sigma-I_h^{\div\div}\boldsymbol\sigma, \widetilde{u}-P_h\widetilde{u}) \\
&\lesssim \|\boldsymbol\sigma-\boldsymbol\sigma_h\|_0\|I_h^{\div\div}\widetilde{\boldsymbol{\sigma}}-\widetilde{\boldsymbol{\sigma}}\|_0+\|\boldsymbol\sigma-I_h^{\div\div}\boldsymbol\sigma\|_{0,h}\|\widetilde{u}-P_h\widetilde{u}\|_{2,h} \\
&\lesssim h\|\boldsymbol\sigma-\boldsymbol\sigma_h\|_0\|\widetilde{\boldsymbol{\sigma}}\|_1+h\|\boldsymbol\sigma-I_h^{\div\div}\boldsymbol\sigma\|_{0,h}\|\widetilde{u}\|_{3} \\
&\lesssim h(\|\boldsymbol\sigma-\boldsymbol\sigma_h\|_0+\|\boldsymbol\sigma-I_h^{\div\div}\boldsymbol\sigma\|_{0,h})(\|\widetilde{\boldsymbol{\sigma}}\|_1+\|\widetilde{u}\|_{3}).
\end{align*}
Then it follows from \eqref{hhjregularity} and \eqref{eq:hhjenergyerror}
\begin{align*}
|P_hu-u_h|_1\lesssim h(\|\boldsymbol\sigma-\boldsymbol\sigma_h\|_0+\|\boldsymbol\sigma-I_h^{\div\div}\boldsymbol\sigma\|_{0,h})\lesssim h^{\min\{m+1,k\}+1}\|\bs\sigma\|_{m+1}.
\end{align*}
\end{proof}


\subsection{Postprocessing}
We will construct a new superconvergent approximation to deflection $u$ in virtue of the optimal result of moment in \eqref{eq:hhjenergyerror} and the superconvergent result \eqref{eq:hhjH1supererror} in this section. For any integer $l\geq1$,
let $I_h^{l}$ be the Lagrange interpolation operator onto the element-wise $l$-th Lagrange element space (cf. \cite{Ciarlet1978,BrennerScott2008}) with respect to $\mathcal{T}_h$. Denote
\[
V_h^{\ast}:=\left\{v\in L^2(\Omega): v|_K\in \mathbb P_{k+1}(K)\quad \forall\,K\in\mathcal
{T}_h\right\}.
\]
With this space, define a new approximation $u_h^{\ast}\in V_h^{\ast}$ to $u$ piecewisely as a solution of the following problem: for any $K\in\mathcal{T}_h$,
\begin{equation}\label{eq:postprocess1}
u_h^{\ast}(\texttt{v}_i)=u_h(\texttt{v}_i) \textrm{ for } i=1,2,3,
\end{equation}
\begin{align}
\int_K\nabla^2u_h^{\ast}:\nabla^2v\dx=-\int_K\boldsymbol{\sigma}_h:\nabla^2v\dx \label{eq:postprocess2}
\end{align}
for any $v\in \mathbb P_{k+1}(K)$ with $v(\texttt{v}_i)=0~(i=1,2,3)$,
where $\{\texttt{v}_i\}_{i=1}^3$ are three vertices of $K$.
By \eqref{eq:postprocess1} and error estimate of $I_h^1$, we have
\begin{equation}\label{eq:postproestimate1}
|I_h^1(I_h^{k+1}u-u_h^{\ast})|_{1}=|I_h^1(P_hu-u_h)|_{1}\lesssim |P_hu-u_h|_{1}.
\end{equation}
Denote $z:=(I-I_h^1)(I_h^{k+1}u-u_h^{\ast})$. It is easy to see that $I_h^1z=0$, $z\in V_h^{\ast}$, and $z(\texttt{v})=0$ for each vertex $\texttt{v}$ of triangulation $\mathcal{T}_h$. Thus we have from error estimate of $I_h^1$ that
\begin{equation}\label{eq:z1}
\|z\|_{0, K}+h_K|z|_{1,K}=\|z-I_h^1z\|_{0, K}+h_K|z-I_h^1z|_{1,K}\lesssim h_K^2|z|_{2,K}.
\end{equation}

\begin{theorem}\label{thm:uuhstar1}
Assume the regularity condition \eqref{hhjregularity} holds, $\boldsymbol{\sigma}\in \boldsymbol{H}^{m+1}(\Omega, \mathbb{S})$ and $u\in H^{m+3}(\Omega)$ for
some non-negative integer $m$. Then
\[
|u-u_h^{\ast}|_{1,h}\lesssim h^{\min\{m+1,k\}+1}(\|\boldsymbol{\sigma}\|_{m+1}+\|u\|_{m+3}+\delta_{k1}\|f\|_0).
\]
\end{theorem}
\begin{proof}
From \eqref{eq:postprocess2} with $v=z$ and the first equation of problem~\eqref{eq:biharmonic2variables}, it follows that
\[
\int_K\nabla^2(u-u_h^{\ast}):\nabla^2z\dx=-\int_K(\boldsymbol{\sigma}-\boldsymbol{\sigma}_h):\nabla^2z\dx.
\]
Noting the definition of $z$, we have
\begin{align*}
&\int_K\nabla^2z:\nabla^2z\dx \\
=&\int_K\nabla^2(I_h^{k+1}u-u_h^{\ast}):\nabla^2z\dx \\
=&\int_K\nabla^2(I_h^{k+1}u-u):\nabla^2z\dx+\int_K\nabla^2(u-u_h^{\ast}):\nabla^2z\dx \\
=&\int_K\nabla^2(I_h^{k+1}u-u):\nabla^2z\dx-\int_K(\boldsymbol{\sigma}-\boldsymbol{\sigma}_h):\nabla^2z\dx.
\end{align*}
On the other hand, by the Cauchy-Schwarz inequality,
\[
|z|_{2,K}^2\lesssim  |I_h^{k+1}u-u|_{2,K}|z|_{2,K} + \|\boldsymbol{\sigma}-\boldsymbol{\sigma}_h\|_{0,K}|z|_{2,K},
\]
which together with \eqref{eq:z1} gives
\[
|z|_{1,K}\lesssim h_K|z|_{2,K}\lesssim h_K|I_h^{k+1}u-u|_{2,K} + h_K\|\boldsymbol{\sigma}-\boldsymbol{\sigma}_h\|_{0,K}.
\]
Then we have from the triangle inequality and \eqref{eq:postproestimate1} that
\begin{align*}
|u-u_h^{\ast}|_{1,h}\leq & |u-I_h^{k+1}u|_{1,h}+|I_h^1(I_h^{k+1}u-u_h^{\ast})|_{1} +|z|_{1,h} \\
\lesssim & |u-I_h^{k+1}u|_{1,h} + |P_hu-u_h|_{1} + h(|u-I_h^{k+1}u|_{2,h}+\|\boldsymbol{\sigma}-\boldsymbol{\sigma}_h\|_{0}).
\end{align*}
Finally, the required result follows readily from the error estimate of $I_h^{k+1}$, \eqref{eq:hhjH1supererror} and \eqref{eq:hhjenergyerror}.
\end{proof}


\subsection{Hybridization}

We will hybridize the HHJ mixed formulation \eqref{eq:hhjmixedformulation1}-\eqref{eq:hhjmixedformulation2} in this subsection \cite{ArnoldBrezzi1985,HuangHuang2016,HuangHuang2019}. Introduce two finite element spaces
\begin{align*}
\Sigma_h&:=\{\boldsymbol{\tau}_h\in L^{2}(\Omega ; \mathbb{S}): \boldsymbol{\tau}_h|_T\in \mathbb P_{k-1}(T ; \mathbb{S})  \text { for each }  T \in \mathcal{T}_{h}\}, \\
M_h&:=\{\mu_h\in L^2(\mathcal{E}_{h}): \mu_h|_e\in \mathbb P_{k-1}(e) \text { for each } e \in \mathring{\mathcal{E}}_{h}, \text { and } \mu_h=0 \text { on } \mathcal{E}_{h}\backslash\mathring{\mathcal{E}}_{h}\}.
\end{align*}
Equip the multiplier space $M_h$ with squared norm
$$
\|\mu_h\|_{\alpha,h}^2:=\sum_{T\in\mathcal T_h}\sum_{e\in\mathcal{E}(T)}h_e^{-2\alpha}\|\mu_h\|_{0,e}^2,\quad \alpha=\pm1/2.
$$

The hybridization of the HHJ mixed formulation \eqref{eq:hhjmixedformulation1}-\eqref{eq:hhjmixedformulation2} is to find $(\boldsymbol{\sigma}_h, u_h,\lambda_h)\in \Sigma_h\times V_h\times M_h$ such that
\begin{align}
a(\boldsymbol\sigma_h, \boldsymbol\tau_h)+ b_h(\boldsymbol\tau_h; u_h,\lambda_h) & =0 \quad\quad\quad\;\;\;\, \forall~\boldsymbol\tau_h\in \Sigma_h, \label{eq:hybirdmfem1}\\
b_h(\boldsymbol\sigma_h; v_h,\mu_h) & =-(f, v_h)  \quad \forall~v_h\in V_h, \mu_h\in M_h. \label{eq:hybirdmfem2}
\end{align}
where
\[
b_h(\boldsymbol\tau_h; u_h,\lambda_h):=(\boldsymbol{\tau}_h, \nabla_h^2u_h) + \sum_{T\in \mathcal{T}_{h}}(\boldsymbol{n}^{\intercal}\boldsymbol{\tau}_h\boldsymbol{n}_e, \lambda_h-\partial_{n_e}u_h)_{\partial T},
\]
where $\nabla_h$ is the piecewise counterpart of $\nabla$ with respect to $\mathcal T_h$.
Here $\boldsymbol{n}$ is the unit outward normal to $\partial T$, and $\boldsymbol{n}_e$ is a fixed unit normal vector of edge $e$.

\begin{lemma}
There holds the following inf-sup condition
\begin{equation}\label{eq:HHJdiscreteinfsup3}
\|\nabla_h^2v_h\|_{0}+\|\mu_h-\partial_{n_e}v_h\|_{1/2,h}\lesssim \sup_{\bs\tau_h\in\Sigma_h}\frac{b_h(\bs\tau_h;v_h,\mu_h)}{\|\bs\tau_h\|_{0,h}}\quad\forall~v_h\in V_h,\mu_h\in M_h.
\end{equation}
\end{lemma}
\begin{proof}
Let $\bs\tau_h\in\Sigma_h$ such that
\begin{align*}
(\boldsymbol{n}^{\intercal}\boldsymbol{\tau}_h\boldsymbol{n}_e)|_{e}&=\frac{1}{h_e}(\mu_h-\partial_{n_e}v_h),\qquad e\in\mathcal E(T), T\in\mathcal T_h,
\\
\int_T\bs\tau_h:\bs\varsigma\dx&=\int_T\nabla^2v_h:\bs\varsigma\dx\quad\forall~\bs\varsigma\in \mathbb P_{k-2}(T;\mathbb S), T\in\mathcal T_h.
\end{align*}
Then
\begin{equation*}
\|\bs\tau_h\|_{0,h}\lesssim \|\nabla_h^2v_h\|_{0} + \|\mu_h-\partial_{n_e}v_h\|_{1/2,h},
\end{equation*}
\[
b_h(\bs\tau_h;v_h,\mu_h)=\|\nabla_h^2v_h\|_{0}^2+\|\mu_h-\partial_{n_e}v_h\|_{1/2,h}^2.
\]
Therefore the inf-sup \eqref{eq:HHJdiscreteinfsup3} holds.
\end{proof}

Thanks to the discrete inf-sup conditions \eqref{eq:HHJdiscreteinfsup3} and the fact $\|\boldsymbol{\tau}_h\|_{0,h}\lesssim \|\boldsymbol{\tau}_h\|_{0}$ for $\boldsymbol{\tau}_h\in\Sigma_h$, the well-posedness of the mixed finite element method \eqref{eq:hybirdmfem1}-\eqref{eq:hybirdmfem2} follows from the Babu{\v{s}}ka-Brezzi theory \cite{BoffiBrezziFortin2013}.
\begin{theorem}
The hybridized mixed finite element method \eqref{eq:hybirdmfem1}-\eqref{eq:hybirdmfem2} is well-posed. We have the discrete stability 
\begin{align}
\notag
&\quad \|\widetilde{\boldsymbol{\sigma}}_h\|_{0,h}+\|\nabla_h^2\widetilde{u}_h\|_{0}+\|\widetilde{\lambda}_h-\partial_{n_e}\widetilde{u}_h\|_{1/2,h} \\
\label{eq:discretestability3}
&\lesssim\sup_{\boldsymbol{\tau}_h\in\Sigma_h, v_h\in V_h,\mu\in M_h}\frac{A_h(\widetilde{\boldsymbol{\sigma}}_h,\widetilde{u}_h,\widetilde{\lambda}_h;\boldsymbol{\tau}_h,v_h,\mu_h)}{\|\boldsymbol{\tau}_h\|_{0,h}+\|\nabla_h^2v_h\|_{0}+\|\mu_h-\partial_{n_e}v_h\|_{1/2,h}}    
\end{align}
for any $\widetilde{\boldsymbol{\sigma}}_h\in\Sigma_h$, $\widetilde{u}_h\in V_h$ and $\widetilde{\lambda}_h\in M_h$, where
$$
A_h(\widetilde{\boldsymbol{\sigma}}_h,\widetilde{u}_h,\widetilde{\lambda}_h;\boldsymbol{\tau}_h,v_h,\mu_h):=(\widetilde{\boldsymbol{\sigma}}_h,\boldsymbol{\tau}_h)+ b_h(\boldsymbol\tau_h; \widetilde{u}_h,\widetilde{\lambda}_h)+ b_h(\widetilde{\boldsymbol\sigma}_h; v_h,\mu_h).
$$
\end{theorem}


\begin{theorem}
Let $(\boldsymbol{\sigma} , u)\in \boldsymbol{H}^{-1}(\div\div,\Omega; \mathbb{S})\times H_0^1(\Omega)$ be the solution of the HHJ mixed formulation \eqref{eq:hhjmixedformulation1}-\eqref{eq:hhjmixedformulation2}, and $(\boldsymbol{\sigma}_h,u_h,\lambda_h)$ be the solution of the mixed finite element method \eqref{eq:hybirdmfem1}-\eqref{eq:hybirdmfem2}.
Assume $\boldsymbol{\sigma}\in \boldsymbol{H}^{m+1}(\Omega, \mathbb{S})$ and $u\in H^{m+3}(\Omega)$ for
some non-negative integer $m$.
Then
\begin{align}
\label{eq:errorestimate3}\|\partial_{n_e}(P_hu-u_h)-(Q_{\mathcal{E}_h}^{k-1}(\partial_{n_e}u)-\lambda_h)\|_{1/2,h}&\lesssim h^{\min\{m+1,k\}}\|\bs\sigma\|_{m+1}, \\
\label{eq:errorestimate4}
\|\partial_{n_e}u_h-\lambda_h\|_{-1/2,h}&\lesssim   h^{\min\{m+2,k\}}\|u\|_{m+3}.
\end{align}
When $\Omega$ is convex, we have
\begin{equation}\label{eq:errorestimate5}
\|Q_{\mathcal{E}_h}^{k-1}(\partial_{n_e}u)-\lambda_h\|_{-1/2,h}\lesssim h^{\min\{m+1,k\}+1}(\|\bs\sigma\|_{m+1}+\delta_{k1}\|f\|_0).
\end{equation}
\end{theorem}
\begin{proof}
Apply \eqref{eq:bpi}, we can show that
\begin{align*}	
&\quad\; A_h(I_h^{\div\div}\boldsymbol{\sigma}-\boldsymbol{\sigma}_h, P_hu-u_h, Q_{\mathcal{E}_h}^{k-1}(\partial_{n_e}u)-\lambda_h;\boldsymbol{\tau}_h, v_h,\mu_h) =(I_h^{\div\div}\boldsymbol{\sigma}-\boldsymbol{\sigma},\boldsymbol{\tau}_h)
\end{align*}
holds for any $\boldsymbol{\tau}_h\in\boldsymbol{\Sigma}_h$, $v_h\in V_h$ and $\mu\in M_h$.

It follows from	the stability results \eqref{eq:discretestability3} that
\begin{align*}
\|\partial_{n_e}(P_hu-u_h)-(Q_{\mathcal{E}_h}^{k-1}(\partial_{n_e}u)-\lambda_h)\|_{1/2,h}&\lesssim \|\boldsymbol{\sigma}-I_h^{\div\div}\boldsymbol{\sigma}\|_0,
\end{align*}
which together with the interpolation error estimate of $I_h^{\div\div}$ indicates \eqref{eq:errorestimate3}. And \eqref{eq:errorestimate4} follows from the triangle inequality, \eqref{eq:errorestimate3} and the interpolation error estimates of $P_h$ and $Q_{\mathcal{E}_h}^{k-1}$. Finally \eqref{eq:errorestimate5} follows from \eqref{eq:errorestimate3}, \eqref{eq:hhjH1supererror} and the inverse inequality.
\end{proof}


Define $\nabla_w^2: V_h\times M_h\to\Sigma_h$ by 
$$
(\nabla_w^2(v_h,\mu_h), \boldsymbol\tau_h)=b_h(\boldsymbol\tau_h; v_h,\mu_h)\quad\forall~\boldsymbol\tau_h\in\Sigma_h.
$$
Then the hybridized mixed finite element method \eqref{eq:hybirdmfem1}-\eqref{eq:hybirdmfem2} can be recast as: find $(u_h,\lambda_h)\in V_h\times M_h$ such that
$$
(\nabla_w^2(u_h,\lambda_h), \nabla_w^2(v_h,\mu_h))=(f, v_h)\quad\forall~v_h\in V_h, 
\mu_h\in M_h.
$$


\subsection{Equivalence between HHJ method and modified Morley element method}
Recall the Morley element space
\begin{align*}
V_h^M := \Big\{&v_h\in L^2(\Omega): v_h|_K\in \mathbb P_2(K),\; \forall~K \in
\mathcal{T}_h; \displaystyle\int_e\llbracket\partial_{n} v_h\rrbracket\dd s = 0,\; \forall~e \in
\mathcal{E}_h; \\
&
\quad\; v_h \textrm{ is continous at each vertex } \texttt{v}\in \mathcal N_h; v_h(\texttt{v})=0, \;\forall~ \texttt{v}\in\mathcal N_h \cap \partial\Omega
 \Big\}.
\end{align*}
A modified Morley element method is to find
$w_h\in V_h^M$ such that
\begin{equation}\label{eq:modiMorley4Biharmonic}
(\nabla_h^2 w_h, \nabla_h^2 v_h)=(f, I_h^1v_h)\quad\forall~v_h\in V_h^M.
\end{equation}

\begin{theorem}
Let $w_h\in V_h^M$ be the solution of Morley element method \eqref{eq:modiMorley4Biharmonic}. Then $(-\nabla_h^2w_h$, $I_h^1w_h$, $Q_{\mathcal E_h}^0(\partial_{n_e}w_h))\in\Sigma_h\times V_h\times M_h$ is the solution of the hybridized mixed finite element method~\eqref{eq:hybirdmfem1}-\eqref{eq:hybirdmfem2} with $k=1$.
\end{theorem}
\begin{proof}
Choose $v_h\in V_h^M$ such that $v_h$ vanishes at all vertices of $\mathcal T_h$, then $I_h^1v_h=0$.
Applying integration by parts on the left hand side of \eqref{eq:modiMorley4Biharmonic}, we get
$$
\sum_{e\in\mathcal E_h^i}([n^{\intercal}(\nabla_h^2w_h)n],\partial_nv_h)_e=\sum_{T\in\mathcal T_h}(n^{\intercal}(\nabla_h^2w_h)n,\partial_nv_h)_{\partial T}=0.
$$
As a result, $[n^{\intercal}(\nabla_h^2w_h)n]_e=0$ for all $e\in\mathcal E_h^i$, that is $\nabla_h^2w_h\in \Sigma_h^{\div\div}$.

For $v_h\in V_h^M$, by \eqref{eq:bp0} and \eqref{eq:modiMorley4Biharmonic},
\begin{align*}  
b_h(\nabla_h^2w_h; I_h^1v_h,\mu_h)&=b_h(\nabla_h^2w_h, I_h^1v_h)=b_h(\nabla_h^2w_h, v_h) \\
&=(\nabla_h^2w_h, \nabla_h^2v_h) - \sum_{T\in \mathcal{T}_{h}}(n^{\intercal}(\nabla_h^2w_h)n, \partial_{n}u_h)_{\partial T} \\
&=(\nabla_h^2w_h, \nabla_h^2v_h)=-(f, I_h^1v_h).
\end{align*}
Then $(-\nabla_h^2w_h$, $I_h^1w_h$, $Q_{\mathcal E_h}^0(\partial_{n_e}w_h))$ satisfies \eqref{eq:hybirdmfem2}

It follows from \eqref{eq:bp0} that
$$(-\nabla_h^2w_h, \boldsymbol\tau_h)+ b_h(\boldsymbol\tau_h; I_h^1w_h,Q_{\mathcal E_h}^0(\partial_{n_e}w_h)) = b_h(\boldsymbol\tau_h, w_h-I_h^1w_h)=0,
$$
which ends the proof.
\end{proof}

