% !TEX root = lecture.tex
\chapter{Stokes方程的混合有限元方法}

\section{Stokes方程}
设$\Omega\subset\mathbb{R}^{d}$ ($d\geq2$)是有界区域.
考虑具有齐次 Dirichlet边界条件下的  Stokes 方程
\begin{equation}\label{eq-Stokes2}
\left\{
\begin{array}{l}
-\Delta \boldsymbol{u} - \nabla p=\boldsymbol{f}(x),\qquad \,\boldsymbol{x}\in\Omega,\\
\div \boldsymbol{u}=0, \qquad \,\qquad \qquad \,\boldsymbol{x}\in\Omega,\\
\boldsymbol{u}=0,
\quad \qquad \qquad \qquad\;\;\; \boldsymbol{x}\in\partial\Omega,
\end{array}
\right.
\end{equation}
其中$\boldsymbol{u}$ 表示流体速度, $p$ 表示压力.
问题(\ref{eq-Stokes2})的混合变分问题为 : 找 $\boldsymbol{u}\in H_0^{1}(\Omega;\mathbb{R}^{d}), p\in L_{0}^{2}(\Omega)$,使得
\begin{align}
(\nabla\boldsymbol{u},\nabla\boldsymbol{v}) + (\div\boldsymbol{v},p) &=(\boldsymbol{f},\boldsymbol{v}),\quad \boldsymbol{v}\in  H_0^{1}(\Omega;\mathbb{R}^{d}),\label{BI2}\\
(\div\boldsymbol{u},q) &=0, \quad\quad\quad\, q\in L_{0}^{2}(\Omega).\label{BI02}
\end{align}

对于上述混合变分问题(\ref{BI2})-(\ref{BI02}), 根据 Brezzi 定理可以得到下面的适定性结果.
\begin{theorem}\label{Tem01}
% 若 (\ref{BI5})和 (\ref{BT2}) 成立, 则
混合变分问题(\ref{BI2})-(\ref{BI02})存在唯一的解
 $\boldsymbol{u}\in H_0^{1}(\Omega; \mathbb{R}^{d}), p\in L_{0}^{2}(\Omega)$, 且
\begin{equation*}
\|\boldsymbol{u}\|_{1}+\|p\|_{0}\lesssim \|\boldsymbol{f}\|_{-1}.
\end{equation*}
\end{theorem}
\begin{prf}
为了得到混合变分问题(\ref{BI2})-(\ref{BI02})解的适定性, 我们来验证  Brezzi 定理的两个条件.
\begin{enumerate}
\item 强制性:
由Poinc\'{a}re 不等式可得强制性
\begin{align}\label{BI5}
\|\boldsymbol{v}\|^{2}_{1}
\lesssim (\nabla\boldsymbol{v},\nabla\boldsymbol{v}),\quad \boldsymbol{v}\in H_0^{1}(\Omega; \mathbb{R}^{d}).
\end{align}
\item Inf-sup 条件
\begin{align}\label{BT2}
\|q\|_{0}\lesssim\sup _{\boldsymbol{v} \in H_0^{1}(\Omega; \mathbb{R}^{d})}
\frac{(\div\boldsymbol{v}, q)}{\|\boldsymbol{v}\|_{1}},\quad\, q\in L_{0}^{2}(\Omega).
\end{align}
由$\div H_0^{1}(\Omega; \mathbb{R}^{d})=L_{0}^{2}(\Omega)$可知, 存在$\boldsymbol{u}\in H_0^{1}(\Omega; \mathbb{R}^{d})$满足
\begin{align*}
\div{\boldsymbol{u}}=q\in L_{0}^{2}(\Omega),\quad \text{且}\quad\|\boldsymbol{u}\|_1\lesssim \|q\|_0.
\end{align*}
于是$\|q\|_0\|\boldsymbol{u}\|_1\lesssim \|q\|_0^{2}=(\div{\boldsymbol{u}},q)$.
进而有
\begin{align}\label{BW2}
\|q\|_{0}\lesssim \frac{(\div{\boldsymbol{u}},q)}{\|\boldsymbol{u}\|_1}\lesssim\sup _{\boldsymbol{v} \in H_0^{1}(\Omega; \mathbb{R}^{d})}
\frac{(\div\boldsymbol{v}, q)}{\|\boldsymbol{v}\|_{1}},\quad\, q\in L_{0}^{2}(\Omega).
\end{align}
即inf-sup 条件成立.
\end{enumerate}
\end{prf}
事实上,inf-sup条件\eqref{BT2}等价于$\div H_0^{1}(\Omega; \mathbb{R}^{d})=L_{0}^{2}(\Omega)$.


\section{Stokes方程的协调混合有限元方法}
设$ V_h \subset H_0^{1}(\Omega; \mathbb{R}^{d})$ 和 $P_h \subset  L_{0}^{2}(\Omega)$为两个有限元空间,则混合变分问题(\ref{BI2})-(\ref{BI02})的有限元离散为 : 找 $\boldsymbol{u}_h \in  V_h$ 和 $p_h \in P_h$满足
\begin{align}
(\nabla\boldsymbol{u}_h,\nabla\boldsymbol{v}) + (\div\boldsymbol{v},p_h) &=(\boldsymbol{f},\boldsymbol{v}),\quad \boldsymbol{v}\in   V_h,\label{CO1}\\
(\div\boldsymbol{u}_h,q) &=0, \quad\quad\quad\, q\in P_h.\label{CO2}
\end{align}
为了得到混合元方法(\ref{CO1})-(\ref{CO2})解的适定性, 我们需要验证Brezzi 定理的两个条件.
\begin{enumerate}[label=(\alph*)]
\item 强制性:
\begin{align}\label{u3}
\|\boldsymbol{v}\|^{2}_{1}
\lesssim (\nabla\boldsymbol{v},\nabla\boldsymbol{v}),\quad \boldsymbol{v}\in  V_h.\end{align}
\item 离散inf-sup 条件
\begin{equation}\label{u4}
\|q\|_{0}\lesssim\sup _{\boldsymbol{v} \in  V_h}
\frac{(\div\boldsymbol{v}, q)}{\|\boldsymbol{v}\|_{1}},\quad\, q\in P_h.
\end{equation}
\end{enumerate}

由于$ V_h \subset H_0^{1}(\Omega; \mathbb{R}^{d})$,连续情形强制性条件\eqref{BI5}意味着离散强制性条件\eqref{u3}.

下面考虑离散inf-sup条件\eqref{u4}. 连续情形,inf-sup条件\eqref{BT2}等价于$\div H_0^{1}(\Omega; \mathbb{R}^{d})=L_{0}^{2}(\Omega)$. 但是,离散inf-sup条件\eqref{u4}并不等价于$\div V_h=P_h$.

\begin{lemma}
记$Q_h :L_0^{2}(\Omega)\rightarrow  P_h$是$L^2$正交投影算子.
假设$\|\boldsymbol{v}\|_1\eqsim \|Q_h\div\boldsymbol{v}\|_0$对所有的$\boldsymbol{v}\in V_h/\ker(\div)$成立, 则离散inf-sup条件\eqref{u4}等价于$Q_h\div V_h=P_h$.
\end{lemma}
\begin{prf}
先来证明离散inf-sup条件\eqref{u4}意味着$Q_h\div V_h=P_h$. 显然$Q_h\div V_h\subseteq P_h$. 若存在$q\in P_h$满足$q$与$Q_h\div V_h$正交,则与离散inf-sup条件\eqref{u4}矛盾,故有$Q_h\div V_h=P_h$.

再证$Q_h\div V_h=P_h$意味着离散inf-sup条件\eqref{u4}. 对任意的$q\in P_h$, 存在$\boldsymbol{v}\in V_h/\ker(\div)$使得$Q_h\div\boldsymbol{v}=q$. 由假设条件可知$\|\boldsymbol{v}\|_1\eqsim \|Q_h\div\boldsymbol{v}\|_0=\|q\|_0$. 于是
\begin{equation*}
\|q\|_0\|\boldsymbol{v}\|_1\lesssim \|q\|_0^2=(Q_h\div\boldsymbol{v}, q)=(\div\boldsymbol{v}, q).
\end{equation*}
故离散inf-sup条件\eqref{u4}成立.
\end{prf}

% 回顾连续性的 Inf-sup条件, 满足:
% \begin{align}
% \|q\|_{0}\lesssim\sup _{\boldsymbol{v} \in H_0^{1}(\Omega; \mathbb{R}^{d})}
% \frac{(\div\boldsymbol{v}, q)}{\|\boldsymbol{v}\|_{1}},\quad \forall q\in L_0^{2}(\Omega).\,\Leftrightarrow\,
% \div H_0^{1}(\Omega,\mathbb{R}^{2})=L_0^{2}(\Omega).
% \end{align}
% 但对任一空间 $Q_{h}$,\,$ V_h$ 而非上面的 $L_0^{2}(\Omega), \,H_0^{1}(\Omega; \mathbb{R}^{d})$ 来说,
% \begin{align*}
% \|q\|_{0}\lesssim\sup _{\boldsymbol{v} \in  V_h} \frac{(\div\boldsymbol{v}, q)}{\|\boldsymbol{v}\|_{1}},\quad \forall q\in Q_h.
% &\,\nRightarrow\,\div  V_h=Q_h.\\
% \|q\|_{0}\lesssim\sup _{\boldsymbol{v} \in  V_h} \frac{(\div\boldsymbol{v}, q)}{\|\boldsymbol{v}\|_{1}},\quad \forall q\in Q.&\,\Leftarrow\,\div  V_h=Q_h.
% \end{align*}
% 所以作 $\tilde{Q}_h :L_0^{2}(\Omega)\rightarrow  Q_h$ 的 $L^2$ 正交投影, 满足:
% $$ (\tilde{Q}_h \boldsymbol{v},q)=(\boldsymbol{v},q),\quad\forall\, q\in L_0^{2}(\Omega).$$
% 则有
% \begin{align*}
% \|q\|_{0}
% \lesssim\sup _{\boldsymbol{v} \in  V_h} \frac{(\tilde{Q}_h\div\boldsymbol{v}, q)}{\|\boldsymbol{v}\|_{1}},
% \quad \forall q\in Q_h.\,\Leftrightarrow\,\tilde{Q}_h\div  V_h=Q_h.\\
% \Rightarrow
% \|q\|_{0}
% \lesssim\sup _{\boldsymbol{v} \in  V_h} \frac{(\tilde{Q}_h \div\boldsymbol{v}, q)}{\|\boldsymbol{v}\|_{1}}
% =\sup _{\boldsymbol{v} \in  V_h}
% \frac{(\div\boldsymbol{v}, q)}{\|\boldsymbol{v}\|_{1}}
% \quad \forall q\in Q_h.
% \end{align*}

因此,$\div V_h=P_h$意味着离散inf-sup条件\eqref{u4}成立,但是
离散inf-sup条件\eqref{u4}成立并不推出$\div V_h=P_h$. 如果有限元空间$ V_h$和$P_h$满足$\div V_h=P_h$,由方程\eqref{CO2}可得$\div\boldsymbol{u}_h=0$,此时称混合元方法(\ref{CO1})-(\ref{CO2})是\textbf{divergence-free}的,或\textbf{质量守恒}的. 

% 综上构造的有限元空间要想满足 离散的 Inf-sup 条件, 当不满足 $\div  V_h=Q_h $ 时(一般的 Stokes 方程的有限元空间都不满足), 则必须满足:
% \begin{align*}
% \|q\|_{0}
% \lesssim\sup _{\boldsymbol{v} \in  V_h}
% \frac{(\div\boldsymbol{v}, q)}{\|\boldsymbol{v}\|_{1}},
% \quad \forall q\in Q_h.\,\Leftrightarrow\,\tilde{Q}_h\div  V_h=Q_h.
% \end{align*}
构建的有限元空间 $ V_h$ 和 $P_h$ 必须满足离散inf-sup 条件\eqref{u4}. Fortin\cite{Fortin1977}在1977 年提出了一个简单而实用的判别准则,称之为 Fortin 准则.
\begin{lemma}[Fortin 准则]\label{lem01}
离散inf-sup 条件(\ref{u4})等价于
存在一个有界线性算子 $\Pi_h :H_0^{1}(\Omega; \mathbb{R}^{d})\rightarrow V_h$满足
\begin{align}
 (\div(\boldsymbol{v}-\Pi_h\boldsymbol{v}), q) &=0,\qquad\quad  \boldsymbol{v}\in H_0^{1}(\Omega; \mathbb{R}^{d}),q\in P_{h} ,\label{k1}\\
 \|\Pi_h \boldsymbol{v}\|_{1} &\leq C\|\boldsymbol{v}\|_{1}, \quad\;  \boldsymbol{v}\in H_0^{1}(\Omega; \mathbb{R}^{d}).\label{k2}
\end{align}
算子$\Pi_h$称为Fortin算子. 这里, 正常数 $C$ 可能与网格剖分 $\mathcal{T}_h$ 有关.
\end{lemma}
\begin{prf}
先证明存在Fortin算子意味着离散inf-sup 条件(\ref{u4})成立. 对于$q\in P_h$, 由inf-sup 条件(\ref{BT2})和\eqref{k1}-\eqref{k2} 可得
\begin{align*}
\|q\|_{0}\lesssim\sup _{\boldsymbol{v} \in H_0^{1}(\Omega; \mathbb{R}^{d})}
\frac{(\div\boldsymbol{v}, q)}{\|\boldsymbol{v}\|_{1}}=\sup _{\boldsymbol{v} \in H_0^{1}(\Omega; \mathbb{R}^{d})}
\frac{(\div(\Pi_h\boldsymbol{v}), q)}{\|\boldsymbol{v}\|_{1}}\lesssim\sup _{\boldsymbol{v} \in H_0^{1}(\Omega; \mathbb{R}^{d})}
\frac{(\div(\Pi_h\boldsymbol{v}), q)}{\|\Pi_h\boldsymbol{v}\|_{1}}\leq \sup _{\boldsymbol{v} \in V_h}
\frac{(\div\boldsymbol{v}, q)}{\|\boldsymbol{v}\|_{1}}.
\end{align*}
% 结合inf-sup 条件(\ref{BT2})和 Fortin可得
% \begin{align*}
% \|q\|_{0}
% &\lesssim\sup _{\boldsymbol{v} \in H_0^{1}(\Omega; \mathbb{R}^{d})}
% \frac{(\div\boldsymbol{v}, q)}{\|\boldsymbol{v}\|_{1}}\|q\|_{0}\\
% &\lesssim\sup _{\boldsymbol{v} \in H_0^{1}(\Omega; \mathbb{R}^{d})}
% \frac{(\div\Pi_h\boldsymbol{v}, q)}{\|\boldsymbol{v}\|_{1}}\\
% &\lesssim\sup _{\boldsymbol{v} \in H_0^{1}(\Omega; \mathbb{R}^{d})}
% \frac{(\div\Pi_h\boldsymbol{v}, q)}{\|\Pi_h\boldsymbol{v}\|_{1}}\\
% &=\sup _{\boldsymbol{v}_h \in \Pi_h H_0^{1}(\Omega; \mathbb{R}^{d})}
% \frac{(\div\boldsymbol{v}_h, q)}{\|\boldsymbol{v}_h\|_{1}}\\
% &\lesssim\sup _{\boldsymbol{v} \in  V_h}
% \frac{(\div\boldsymbol{v}, q)}{\|\boldsymbol{v}\|_{1}},\quad \forall q\in Q_{h}.
% \end{align*}
故离散inf-sup 条件(\ref{u4})成立.

再证明离散inf-sup 条件(\ref{u4})成立意味着存在Fortin算子. 
% 当$\boldsymbol{v}\in V_h$时,取$\Pi_{h,1}\boldsymbol{v}=\boldsymbol{v}$显然满足\eqref{k1}-\eqref{k2}.
% 记$H_0^{1}(\Omega; \mathbb{R}^{d})/V_h$为$V_h$在$H_0^{1}(\Omega; \mathbb{R}^{d})$中关于$H^1$内积的正交补.
对于$\boldsymbol{v}\in H_0^{1}(\Omega; \mathbb{R}^{d})$, $Q_h(\div\boldsymbol{v})\in P_h$. 由离散inf-sup条件(\ref{u4})知,存在$\boldsymbol{v}_h\in V_h$满足
\begin{equation*}
Q_h(\div\boldsymbol{v}_h)=Q_h(\div\boldsymbol{v}),\quad \|\boldsymbol{v}_h\|_1\lesssim \|Q_h(\div\boldsymbol{v})\|_0\lesssim \|\boldsymbol{v}\|_1.
\end{equation*}
记$\Pi_{h}\boldsymbol{v}=\boldsymbol{v}_h$, 则$\Pi_{h}$满足\eqref{k1}-\eqref{k2}.
% 最后将两部分合并,对$\boldsymbol{v}\in H_0^{1}(\Omega; \mathbb{R}^{d})$, 存在$H^1$正交分解$\boldsymbol{v}=\boldsymbol{v}_1+\boldsymbol{v}_2$,其中$\boldsymbol{v}_1\in V_h$, $\boldsymbol{v}_2\in H_0^{1}(\Omega; \mathbb{R}^{d})/V_h$,令$\Pi_h\boldsymbol{v}=\Pi_{h,1}\boldsymbol{v}_1+\Pi_{h,2}\boldsymbol{v}_2$. 可知$\Pi_h$是Fortin算子. 
\end{prf}

%  当我们利用 Fortin 准则来验证离散的 inf-sup 条件时,首先需要建立一个有界线性算子
%  $\Pi_h :H_0^{1}(\Omega; \mathbb{R}^{d})\rightarrow  V_h$, 使其满足:
%  \begin{align}
% (\div(\boldsymbol{v}-\Pi_h\boldsymbol{v}),q) &=0,
% \qquad \forall q\in Q_{h} ,\,\forall \boldsymbol{v}\in H_0^{1}(\Omega; \mathbb{R}^{d}),\label{o1}\\
%  \|\Pi_h\boldsymbol{v} \|_{1}  &\lesssim\|\boldsymbol{v}\|_{1},
%  \qquad\, \forall \boldsymbol{v}\in H_0^{1}(\Omega; \mathbb{R}^{d}).\label{o2}
% \end{align}
一般情况下分两步来构造Fortin算子$\Pi_h$.
先构造两个有界线性算子 $\Pi_1,\, \Pi_2:H_0^{1}(\Omega; \mathbb{R}^{d})\rightarrow  V_h$, 使其满足:
 \begin{align}
|\Pi_{1}\boldsymbol{v}|_{1}^2+\sum_{T\in\mathcal T_h}h_T^{-2}\|\boldsymbol{v}-\Pi_{1}\boldsymbol{v}\|_{0,T}^2 &\lesssim |\boldsymbol{v}|_1^2, \qquad\quad\qquad\qquad\; \boldsymbol{v}\in H_0^{1}(\Omega; \mathbb{R}^{d}), \label{fortin:pi1}\\
 \|\Pi_2\boldsymbol{v} \|_{0,T}  &\lesssim\|\boldsymbol{v} \|_{0,\omega_T}+h_T|\boldsymbol{v} |_{1,\omega_T},
 \quad\, \boldsymbol{v}\in H_0^{1}(\Omega; \mathbb{R}^{d}), \label{fortin:pi21}\\
(\div(\boldsymbol{v}-\Pi_2\boldsymbol{v}),q)&=0,
\quad\;\; q\in P_{h}, \boldsymbol{v}\in H_0^{1}(\Omega; \mathbb{R}^{d}),\label{fortin:pi22}
\end{align}
其中$\omega_T$是$\mathcal T_h$中所有与$T$相交非空的单形的并集.
再如下定义有界线性算子$\Pi_{h}: H_0^{1}(\Omega; \mathbb{R}^{d})\rightarrow  V_{h}$:
\begin{equation}\label{PihPi12}
\Pi_{h}\boldsymbol{v}=\Pi_{1}\boldsymbol{v}
+\Pi_{2}(\boldsymbol{v}-\Pi_{1}\boldsymbol{v}),\quad \, \boldsymbol{v}\in H_0^{1}(\Omega; \mathbb{R}^{d}).
\end{equation}
这里有界线性算子$\Pi_{1}$通常用于保证收敛阶, 有界线性算子 $\Pi_{2}$用于保证离散inf-sup 条件.

\begin{lemma}\label{lem:fortinoperator}
假设\eqref{fortin:pi1}-\eqref{fortin:pi22}成立,则由式\eqref{PihPi12}定义的$\Pi_{h}$是Fortin算子,即满足 (\ref{k1}) 和 (\ref{k2}).
\end{lemma}
\begin{prf}	
利用逆不等式、\eqref{fortin:pi21}和\eqref{fortin:pi1},
\begin{equation*}
|\Pi_{2}(\boldsymbol{v}-\Pi_{1}\boldsymbol{v})|_1^2\lesssim \sum_{T\in\mathcal T_h}(h_T^{-2}\|\boldsymbol{v}-\Pi_{1}\boldsymbol{v}\|_{0,T}^2 + |\boldsymbol{v}-\Pi_{1}\boldsymbol{v}|_{1,T}^2)\lesssim |\boldsymbol{v}|_1^2, \quad\boldsymbol{v}\in H_0^{1}(\Omega; \mathbb{R}^{d}).
\end{equation*}
再由\eqref{fortin:pi1}可得
\begin{equation*}
\|\Pi_{h}\boldsymbol{v}\|_{1}\leq\|\Pi_{1}\boldsymbol{v}\|_1
+\|\Pi_{2}(\boldsymbol{v}-\Pi_{1}\boldsymbol{v})\|_1\lesssim \|\Pi_{1}\boldsymbol{v}\|_1+|\boldsymbol{v}|_1\lesssim \|\boldsymbol{v}\|_1, \quad \boldsymbol{v}\in H_0^{1}(\Omega; \mathbb{R}^{d}).
\end{equation*}
故(\ref{k2})成立.
对于$\boldsymbol{v}\in H_0^{1}(\Omega; \mathbb{R}^{d})$和$q\in P_h$, 利用\eqref{fortin:pi22}可得
\begin{equation*}
(\div(\boldsymbol{v}-\Pi_h\boldsymbol{v}),q)
=(\div(\boldsymbol{v}-\Pi_{1}\boldsymbol{v}
-\Pi_{2}(\boldsymbol{v}-\Pi_{1}\boldsymbol{v})),q)=0.
\end{equation*}
从而(\ref{k1})成立.
\end{prf}

\begin{theorem}\label{thm:stokeserrorestimate}
设$(\boldsymbol{u}, p)\in H_0^{1}(\Omega; \mathbb{R}^{d})\times L_{0}^{2}(\Omega)$是Stokes方程\eqref{eq-Stokes2}的解, $(\boldsymbol{u}_h, p_h)\in V_h\times P_h$是混合元方法(\ref{CO1})-(\ref{CO2})的解,则有误差估计
\begin{equation}\label{stokeserroruh}	
|\boldsymbol{u}-\boldsymbol{u}_h|_1\lesssim |\boldsymbol{u}-\Pi_{h}\boldsymbol{u}|_1 + \inf_{q\in P_h}\sup_{\boldsymbol{v}\in V_h}\frac{(\div\boldsymbol{v},p-q)}{|\boldsymbol{v}|_1},
\end{equation}
\begin{equation}\label{stokeserrorph}	
\|p-p_h\|_0\lesssim |\boldsymbol{u}-\Pi_{h}\boldsymbol{u}|_1 + \inf_{q\in P_h}\|p-q\|_0.
\end{equation}
进一步,若有$\div V_h=P_h$,则
\begin{equation}\label{stokeserroruhu}	
|\boldsymbol{u}-\boldsymbol{u}_h|_1\lesssim |\boldsymbol{u}-\Pi_{h}\boldsymbol{u}|_1.
\end{equation}
\end{theorem}
\begin{prf}
将(\ref{CO1})-(\ref{CO2})减去(\ref{BI2})-(\ref{BI02})可得
误差方程
\begin{align}
(\nabla(\boldsymbol{u}-\boldsymbol{u}_h),\nabla\boldsymbol{v}) + (\div\boldsymbol{v},p-p_h) &=0,\quad \boldsymbol{v}\in   V_h,\label{StokesErrorEqn1}\\
(\div(\boldsymbol{u}-\boldsymbol{u}_h),q) &=0, \quad q\in P_h.\label{StokesErrorEqn2}
\end{align}
由\eqref{k1}和误差方程\eqref{StokesErrorEqn2}可知,
\begin{equation}
(\div(\Pi_h\boldsymbol{u}-\boldsymbol{u}_h),q) =0, \quad q\in P_h.\label{StokesErrorEqn3}
\end{equation}
将误差方程\eqref{StokesErrorEqn1}中的$\boldsymbol{v}$用$\Pi_{h}\boldsymbol{u}-\boldsymbol{u}_h$代入,有
\begin{equation*}
|\Pi_{h}\boldsymbol{u}-\boldsymbol{u}_h|_1^2=-(\nabla(\boldsymbol{u}-\Pi_{h}\boldsymbol{u}),\nabla(\Pi_{h}\boldsymbol{u}-\boldsymbol{u}_h))-(\div(\Pi_{h}\boldsymbol{u}-\boldsymbol{u}_h),p-p_h).
\end{equation*}
利用\eqref{StokesErrorEqn3}可得
\begin{equation*}
|\Pi_{h}\boldsymbol{u}-\boldsymbol{u}_h|_1^2=-(\nabla(\boldsymbol{u}-\Pi_{h}\boldsymbol{u}),\nabla(\Pi_{h}\boldsymbol{u}-\boldsymbol{u}_h))-(\div(\Pi_{h}\boldsymbol{u}-\boldsymbol{u}_h),p-q).
\end{equation*}
从而
\begin{equation*}
|\Pi_{h}\boldsymbol{u}-\boldsymbol{u}_h|_1\lesssim |\boldsymbol{u}-\Pi_{h}\boldsymbol{u}|_1 + \inf_{q\in P_h}\sup_{\boldsymbol{v}\in V_h}\frac{(\div\boldsymbol{v},p-q)}{|\boldsymbol{v}|_1}
\end{equation*}
进一步利用三角不等式可得\eqref{stokeserroruh}.

由离散inf-sup 条件\eqref{u4},
\begin{equation*}
\|q-p_h\|_{0}\lesssim\sup _{\boldsymbol{v} \in  V_h}
\frac{(\div\boldsymbol{v}, q-p_h)}{\|\boldsymbol{v}\|_{1}}\lesssim \|p-q\|_0+\sup _{\boldsymbol{v} \in  V_h}
\frac{(\div\boldsymbol{v}, p-p_h)}{\|\boldsymbol{v}\|_{1}}.
\end{equation*}
利用误差方程\eqref{StokesErrorEqn1},
\begin{equation*}
\|q-p_h\|_{0}\lesssim \|p-q\|_0+\sup _{\boldsymbol{v} \in  V_h}
\frac{(\nabla(\boldsymbol{u}_h-\boldsymbol{u}),\nabla\boldsymbol{v})}{\|\boldsymbol{v}\|_{1}}\lesssim \|p-q\|_0+|\boldsymbol{u}-\boldsymbol{u}_h|_1.
\end{equation*}
然后,借助三角不等式和\eqref{stokeserroruh}可得\eqref{stokeserrorph}.

进一步,若有$\div V_h=P_h$,可取$q=Q_hp$, 则\eqref{stokeserroruh}式右端的第二项为零,故\eqref{stokeserroruhu}成立.
\end{prf}

如果$\boldsymbol{u}-\boldsymbol{u}_h$的误差估计只依赖于速度$\boldsymbol{u}$,不依赖于压力$p$,则称混合元方法(\ref{CO1})-(\ref{CO2})是\textbf{压力鲁棒}的. 显然,divergence-free的混合元方法(\ref{CO1})-(\ref{CO2})是压力鲁棒的. 关于Stokes方程数值格式的压力鲁棒性详见\cite{JohnLinkeMerdonNeilanEtAl2017}.

\subsection{压力间断的协调元方法}
 
令$\mathcal{T}_h$是$\Omega$的单形网格剖分,假设$\mathcal{T}_h$是形状正则的.

% 在这之前设 $Q_{K}\mathbf{:L^{2}}(K)\rightarrow P_{0}(K)$ 是 $K$ 上的 $L^{2}$ 正交投影,
% 即对 $\forall\,\boldsymbol{v}\in \mathbf{L^{2}}(K)$, 有:
% $$Q_{K}\boldsymbol{v}=\frac{1}{|K|}\int_{K}\boldsymbol{v}d\boldsymbol{x}.$$
% 且满足:
% $$ (Q_K \boldsymbol{v},q_h)_K=(\boldsymbol{v},q_h)_K,\quad\forall\, q_h\in P_0(K).$$

\subsubsection{两维$P_2$-$P_0$元}
用向量值二次Lagrange元和分片常数分别离散速度和压力,即令
\begin{align*}
 V_{h}&:=\{\boldsymbol{v}\in H_{0}^{1}(\Omega,\mathbb{R}^{2}): \boldsymbol{v}|_T\in \mathbb P_2(T;\mathbb{R}^{2}), T\in \mathcal{T}_h\},
\\
P_{h}&:=\{q\in L_{0}^2(\Omega): q|_T\in \mathbb P_0(T), T\in \mathcal{T}_h\}.
\end{align*}
二次Lagrange元的自由度为
\begin{subequations}\label{quadLagrangeDoF}
\begin{align}
\label{quadLagrangeDoF1}
\boldsymbol{v}(\texttt{v}), &\quad \texttt{v}\in\Delta_0(T), \\
\label{quadLagrangeDoF2}
\int_{e}\boldsymbol{v}\dd s, &\quad e\in\Delta_1(T).
\end{align}
\end{subequations}

\begin{lemma}
$P_2$-$P_0$元满足离散inf-sup条件\eqref{u4}.
\end{lemma}
\begin{prf}
记$I_h^{\rm SZ}: H_{0}^{1}(\Omega,\mathbb{R}^{2})\to V_{h}$为Scott-Zhang插值算子\cite{ScottZhang1990},$I_h^{\rm SZ}$显然满足\eqref{fortin:pi1}.
% \begin{equation*}
% \|I_h^{\rm SZ}\boldsymbol{v}\|_1\lesssim \|\boldsymbol{v}\|_1,\quad \boldsymbol{v}\in H_{0}^{1}(\Omega,\mathbb{R}^{2}).
% \end{equation*} 
引入插值算子$\Pi_2: H_{0}^{1}(\Omega,\mathbb{R}^{2})\to V_{h}$, 定义如下
\begin{align*}
(\Pi_2\boldsymbol{v})(\texttt{v})&=0, \quad\quad\quad\;\texttt{v}\in\Delta_0(\mathcal T_h), \\
\int_{e}\Pi_2\boldsymbol{v}\dd s&=\int_{e}\boldsymbol{v}\dd s, \quad e\in\Delta_1(\mathcal T_h).
\end{align*}
由仿射等价性和迹不等式可得,
\begin{equation*}
\|\Pi_{2}\boldsymbol{v}\|_{0,T}^{2}
\eqsim  h_{T}\sum_{e\in\partial T}\|Q_{0,e}\boldsymbol{v}\|_{0,e}^{2}\lesssim h_{T}\|\boldsymbol{v}\|_{0,\partial T}^{2}\lesssim \|\boldsymbol{v}\|_{0,T}^{2}+h_T^2|\boldsymbol{v}|_{1,T}^{2}.
\end{equation*}
故$\Pi_2$满足\eqref{fortin:pi21}.
% 结合逆不等式得
% \begin{equation*}
% \|\Pi_{2}\boldsymbol{v}\|_{0,T}+h_T|\Pi_{2}\boldsymbol{v}|_{1,T}\lesssim \|\boldsymbol{v}\|_{0,T}+h_T|\boldsymbol{v}|_{1,T}.
% \end{equation*}
利用分部积分可得
\begin{equation*}
(\div(\boldsymbol{v}-\Pi_2\boldsymbol{v}),q)=-(\boldsymbol{v}-\Pi_2\boldsymbol{v},\nabla q)=0,
\quad\;\; q\in P_{h}, \boldsymbol{v}\in H_0^{1}(\Omega; \mathbb{R}^{d}).
\end{equation*}
即$\Pi_2$满足\eqref{fortin:pi22}.

根据\eqref{PihPi12}定义插值算子$\Pi_{h}: H_0^{1}(\Omega; \mathbb{R}^{d})\rightarrow  V_{h}$,即
$\Pi_{h}\boldsymbol{v}=I_h^{\rm SZ}\boldsymbol{v}
+\Pi_{2}(\boldsymbol{v}-I_h^{\rm SZ}\boldsymbol{v})$. 由引理\ref{lem:fortinoperator}可知,$\Pi_{h}$是Fortin算子,故由Fortin 准则可得离散inf-sup条件\eqref{u4}成立.
\end{prf}

\begin{remark}
这里之所以选择二次Lagrange元不是线性Lagrange元来离散速度,是因为线性Lagrange元只有顶点处函数值的自由度, 没有边上函数值积分平均的自由度,这样就无法得到$\int_{T} \div(\boldsymbol{v}-\Pi_2\boldsymbol{v})\dx
= \int_{\partial T}(\boldsymbol{v}-\Pi_2\boldsymbol{v})\cdot\boldsymbol{n}\dd s
=0$. 边上函数值积分平均的自由度保证了$\div V_h$能映满分片常数.
\end{remark}

当$\boldsymbol{u}\in H_0^1(\Omega;\mathbb R^2)\cap H^3(\Omega;\mathbb R^2)$和$p\in L_0^2(\Omega)\cap H^1(\Omega)$时, 由定理~\ref{thm:stokeserrorestimate}可知$P_2$-$P_0$元方法的误差估计
\begin{equation*}
\|\boldsymbol{u}-\boldsymbol{u}_h\|_{1} + \|p-p_h\|_{0}
\lesssim h^2\|\boldsymbol{u}\|_3+h\|p\|_1\lesssim h(\|\boldsymbol{u}\|_3+\|p\|_1).
\end{equation*}
该误差估计的收敛阶对于压力$p$是最优的,但对于速度$\boldsymbol{u}$不是最优的.

将$P_2$-$P_0$元推广到任意$d$维,即$P_d$-$P_0$元,此时离散速度的收敛阶的丢阶现象会更加严重.
下面考虑改进$P_2$-$P_0$元方法.

\subsubsection{SMALL元}

为了证明离散inf-sup条件,关键是要用到自由度\eqref{quadLagrangeDoF2}的法向部分,切向部分并不需要. 为此,只要在向量值一次多项式空间的基础上增加边上的法向泡函数,即可避免丢阶现象. 由此得到SMALL元,参见\cite[Remark 8.4.2]{BoffiBrezziFortin2013}和\cite{BernardiRaugel1981,Fortin1981}. 在SMALL元方法中,压力仍然用分片常数逼近.


速度的形函数空间取为$\mathbb P_1(T;\mathbb R^2)\oplus\mathrm{span}\{\lambda_i\lambda_j\nabla\lambda_k: i\neq j\neq k, 0\leq i,j,k\leq2\}$. %, 其中边泡函数$b_e=\lambda_i\lambda_j$.
自由度为
\begin{subequations}\label{SMALLDoF}
\begin{align}
\label{SMALLDoF1}
\boldsymbol{v}(\texttt{v}), &\quad \texttt{v}\in\Delta_0(T), \\
\label{SMALLDoF2}
\int_{e}\boldsymbol{v}\cdot\boldsymbol{n}\dd s, &\quad e\in\Delta_1(T).
\end{align}
\end{subequations}

\begin{lemma}
形函数空间$\mathbb P_1(T;\mathbb R^2)\oplus\mathrm{span}\{\lambda_i\lambda_j\nabla\lambda_k: i\neq j\neq k, 0\leq i,j,k\leq2\}$由自由度\eqref{SMALLDoF}所唯一确定.
\end{lemma}
\begin{prf}
形函数空间的维数和自由度的个数均为$9$. 设$\boldsymbol{v}=\boldsymbol{q}+c_0\lambda_1\lambda_2\nabla\lambda_0+c_1\lambda_2\lambda_0\nabla\lambda_1+c_2\lambda_0\lambda_1\nabla\lambda_2$满足\eqref{SMALLDoF}中所有自由度为零,其中$\boldsymbol{q}\in\mathbb P_1(T;\mathbb R^2)$, $c_0,c_1,c_2\in\mathbb R$. 由自由度\eqref{SMALLDoF1}可知$\boldsymbol{q}$在所有顶点处取值为零,故$\boldsymbol{q}=0$. 从而$\boldsymbol{v}=c_0\lambda_1\lambda_2\nabla\lambda_0+c_1\lambda_2\lambda_0\nabla\lambda_1+c_2\lambda_0\lambda_1\nabla\lambda_2$. 在边$e_0$上, $\boldsymbol{v}|_{e_0}=(c_0\lambda_1\lambda_2\nabla\lambda_0)|_{e_0}$, 故由自由度\eqref{SMALLDoF2}可知$c_0=0$. 类似可得$c_1=c_2=0$. 于是$\boldsymbol{v}=0$.
\end{prf}

类似$P_2$-$P_0$元,我们可以定义SMALL元整体有限元空间,并证明离散inf-sup条件. Stokes方程SMALL元方法的误差估计为
\begin{equation*}
\|\boldsymbol{u}-\boldsymbol{u}_h\|_{1} + \|p-p_h\|_{0}
\lesssim h(\|\boldsymbol{u}\|_2+\|p\|_1).
\end{equation*}


任意$d$维SMALL元的速度形函数空间为$\mathbb P_1(T;\mathbb R^d)\oplus\mathrm{span}\{b_F\boldsymbol{n}_F, F\in\partial T\}$, 其中$b_F$和$\boldsymbol{n}_F$分别为面$F$的泡函数和法向量. %, 其中边泡函数$b_e=\lambda_i\lambda_j$.
自由度为
% \begin{subequations}%\label{SMALLDoF}
\begin{align*}
% \label{SMALLDoF1}
\boldsymbol{v}(\texttt{v}), &\quad \texttt{v}\in\Delta_0(T), \\
% \label{SMALLDoF2}
\int_{F}\boldsymbol{v}\cdot\boldsymbol{n}\dd s, &\quad F\in\partial T.
\end{align*}
% \end{subequations}


\subsubsection{Crouzeix-Raviart元}

同样为了克服$P_2$-$P_0$元的丢阶现象,Crouzeix-Raviart元\cite{CrouzeixRaviart1973}考虑将压力空间的多项式次数提高,同时对速度空间增补泡函数.

设$k\geq2$, Crouzeix-Raviart元的速度形函数空间为$\mathbb P_k(T;\mathbb R^2)+b_T\mathbb P_{k-2}(T;\mathbb R^2)$, 自由度为
\begin{subequations}\label{CrouzeixRaviart2dDoF}
\begin{align}
\label{CrouzeixRaviart2dDoF1}
\boldsymbol{v}(\texttt{v}), &\quad \texttt{v}\in\Delta_0(T), \\
\label{CrouzeixRaviart2dDoF2}
(\boldsymbol{v}, \boldsymbol{q})_e, &\quad \boldsymbol{q}\in\mathbb P_{k-2}(e;\mathbb R^2), e\in\Delta_1(T), \\
\label{CrouzeixRaviart2dDoF3}
(\boldsymbol{v}, \boldsymbol{q})_T, &\quad \boldsymbol{q}\in\mathbb P_{k-2}(T;\mathbb R^2).
\end{align}
\end{subequations}

\begin{lemma}
设$k\geq2$, 形函数空间$\mathbb P_k(T;\mathbb R^2)+b_T\mathbb P_{k-2}(T;\mathbb R^2)$由自由度\eqref{CrouzeixRaviart2dDoF}所唯一确定.
\end{lemma}
\begin{prf}
形函数空间的维数为$2(\dim\mathbb P_k(T)+\dim\mathbb P_{k-2}(T)-\dim\mathbb P_{k-3}(T))$和自由度的个数为
\begin{equation*}
6+6(k-1)+2\dim\mathbb P_{k-2}(T)=2(\dim\mathbb P_k(T)+\dim\mathbb P_{k-2}(T)-\dim\mathbb P_{k-3}(T)).
\end{equation*} 

设$\boldsymbol{v}\in\mathbb P_k(T;\mathbb R^2)+b_T\mathbb P_{k-2}(T;\mathbb R^2)$满足\eqref{CrouzeixRaviart2dDoF}中所有自由度为零. 由自由度\eqref{CrouzeixRaviart2dDoF1}-\eqref{CrouzeixRaviart2dDoF2}可知$\boldsymbol{v}|_{\partial T}=0$,故$\boldsymbol{v}\in b_T\mathbb P_{k-2}(T;\mathbb R^2)$. 再由自由度\eqref{CrouzeixRaviart2dDoF3}可得$\boldsymbol{v}=0$.
\end{prf}

分别定义离散速度和压力的整体有限元空间
\begin{align*}
 V_{h}&:=\{\boldsymbol{v}\in H_{0}^{1}(\Omega,\mathbb{R}^{2}): \boldsymbol{v}|_T\in \mathbb P_k(T;\mathbb R^2)+b_T\mathbb P_{k-2}(T;\mathbb R^2), T\in \mathcal{T}_h\},
\\
P_{h}&:=\{q\in L_{0}^2(\Omega): q|_T\in \mathbb P_{k-1}(T), T\in \mathcal{T}_h\}.
\end{align*}

\begin{lemma}
设$k\geq2$, Crouzeix-Raviart元满足离散inf-sup条件\eqref{u4}.
\end{lemma}
\begin{prf}
令 $V_{h}^L$为$k$次Lagrange元空间$\{\boldsymbol{v}\in H_{0}^{1}(\Omega,\mathbb{R}^{2}): \boldsymbol{v}|_T\in \mathbb P_k(T;\mathbb R^2), T\in \mathcal{T}_h\}$.
记$I_h^{\rm SZ}: H_{0}^{1}(\Omega,\mathbb{R}^{2})\to V_{h}^L$为Scott-Zhang插值算子\cite{ScottZhang1990},满足
\begin{equation*}
(I_h^{\rm SZ}\boldsymbol{v}, \boldsymbol{q})_e=(\boldsymbol{v}, \boldsymbol{q})_e\quad\forall~\boldsymbol{q}\in\mathbb P_{k-2}(e;\mathbb R^2), e\in\Delta_1(\mathcal T_h).
\end{equation*}
引入插值算子$\Pi_2: H_{0}^{1}(\Omega;\mathbb{R}^{2})\to V_{h}$, 定义如下: 对任意的$T\in\mathcal{T}_h$, $(\Pi_2\boldsymbol{v})|_T\in b_T\nabla\mathbb P_{k-1}(T)$满足
\begin{align*}
(\div(\Pi_2\boldsymbol{v}), q)_T&=(\div\boldsymbol{v}, q)_T, \quad q\in\mathbb P_{k-1}(T)\cap L_0^2(T).
\end{align*}
由仿射等价性可得,
\begin{equation*}
\|\Pi_{2}\boldsymbol{v}\|_{0,T}\lesssim h_T\|Q_{k-1,T}\div\boldsymbol{v}\|_{0,T}\leq h_T\|\div\boldsymbol{v}\|_{0,T}.
\end{equation*}
故$\Pi_2$满足\eqref{fortin:pi21}.
% 结合逆不等式得
% \begin{equation*}
% \|\Pi_{2}\boldsymbol{v}\|_{0,T}+h_T|\Pi_{2}\boldsymbol{v}|_{1,T}\lesssim \|\boldsymbol{v}\|_{0,T}+h_T|\boldsymbol{v}|_{1,T}.
% \end{equation*}
由 $\Pi_{2}\boldsymbol{v}$ 的定义显然有
\begin{equation*}
(\div(\boldsymbol{v}-\Pi_2\boldsymbol{v}),q)_T=0,
\quad\;\; q\in\mathbb P_{k-1}(T)\cap L_0^2(T), T\in\mathcal T_h, \boldsymbol{v}\in H_0^{1}(\Omega; \mathbb{R}^{d}).
\end{equation*}
% 利用分部积分可得
% \begin{equation*}
% (\div(\boldsymbol{v}-\Pi_2\boldsymbol{v}),q)=-(\boldsymbol{v}-\Pi_2\boldsymbol{v},\nabla q)=0,
% \quad\;\; q\in P_{h}, \boldsymbol{v}\in H_0^{1}(\Omega; \mathbb{R}^{d}).
% \end{equation*}
% 即$\Pi_2$满足\eqref{fortin:pi22}.

根据\eqref{PihPi12}定义插值算子$\Pi_{h}: H_0^{1}(\Omega; \mathbb{R}^{d})\rightarrow  V_{h}$,即
$\Pi_{h}\boldsymbol{v}=I_h^{\rm SZ}\boldsymbol{v}
+\Pi_{2}(\boldsymbol{v}-I_h^{\rm SZ}\boldsymbol{v})$. 类似引理~\ref{lem:fortinoperator}的证明可得
\begin{equation*}
\|\Pi_{h}\boldsymbol{v}\|_1\leq\|I_h^{\rm SZ}\boldsymbol{v}\|_1+\|\Pi_{2}(\boldsymbol{v}-I_h^{\rm SZ}\boldsymbol{v})\|_1\lesssim \|\boldsymbol{v}\|_1, \quad\forall~\boldsymbol{v}\in H_0^{1}(\Omega; \mathbb{R}^{d}),
\end{equation*}
以及对任意的 $q\in P_h$成立
\begin{align*}
(\div(\boldsymbol{v}-\Pi_{h}\boldsymbol{v}), q) &= \sum_{T\in\mathcal T_h}(\div(\boldsymbol{v}-I_h^{\rm SZ}\boldsymbol{v}
-\Pi_{2}(\boldsymbol{v}-I_h^{\rm SZ}\boldsymbol{v})), q)_T \\
&= \sum_{T\in\mathcal T_h}(\div(\boldsymbol{v}-I_h^{\rm SZ}\boldsymbol{v}
-\Pi_{2}(\boldsymbol{v}-I_h^{\rm SZ}\boldsymbol{v})), Q_T^0q)_T \\
&= \sum_{T\in\mathcal T_h}(\div(\boldsymbol{v}-I_h^{\rm SZ}\boldsymbol{v}), Q_T^0q)_T =0.
\end{align*}
这表明 $\Pi_{h}$ 是 Fortin 算子,故由Fortin 准则可得离散inf-sup条件\eqref{u4}成立.
\end{prf}

当$\boldsymbol{u}\in H_0^1(\Omega;\mathbb R^2)\cap H^{k+1}(\Omega;\mathbb R^2)$和$p\in L_0^2(\Omega)\cap H^k(\Omega)$时, 由定理~\ref{thm:stokeserrorestimate}可知Crouzeix-Raviart元方法的误差估计
\begin{equation*}
\|\boldsymbol{u}-\boldsymbol{u}_h\|_{1} + \|p-p_h\|_{0}
\lesssim h^k(\|\boldsymbol{u}\|_{k+1}+\|p\|_k).
\end{equation*}

设$k\geq2$, 三维情形Crouzeix-Raviart元\cite[Example 8.7.2]{BoffiBrezziFortin2013}的压力空间仍为分片$k-1$次多项式空间,
速度形函数空间为
\begin{equation*}
\begin{cases}
\mathbb P_2(T;\mathbb R^3)+b_T\mathbb P_{0}(T;\mathbb R^3)+\mathrm{span}\{b_F\boldsymbol{n}_F, F\in\partial T\}, & k=2,\\
\mathbb P_k(T;\mathbb R^3)+b_T\mathbb P_{k-2}(T;\mathbb R^3), & k\geq3.	
\end{cases}
\end{equation*}
自由度为
\begin{align*}
\boldsymbol{v}(\texttt{v}), &\quad \texttt{v}\in\Delta_0(T), \\
(\boldsymbol{v}, \boldsymbol{q})_e, &\quad \boldsymbol{q}\in\mathbb P_{k-2}(e;\mathbb R^3), e\in\Delta_1(T), \\
(\boldsymbol{v}\cdot\boldsymbol{n}, q)_F, &\quad q\in\mathbb P_{0}(F), F\in\Delta_2(T), \text{ 当$k=2$时},\\
(\boldsymbol{v}, \boldsymbol{q})_F, &\quad \boldsymbol{q}\in\mathbb P_{k-3}(F;\mathbb R^3), F\in\Delta_2(T), \text{ 当$k\geq3$时}, \\
(\boldsymbol{v}, \boldsymbol{q})_T, &\quad \boldsymbol{q}\in\mathbb P_{k-2}(T;\mathbb R^3).
\end{align*}
当$k\geq3$时, 自由度的个数为
\begin{align*}
12+18(k-1)+12{k-1\choose 2}+3\dim\mathbb P_{k-2}(T)&=6k^2+6+3\dim\mathbb P_{k-2}(T)\\
&=3\dim\mathbb P_k(T)+3\dim\mathbb P_{k-2}(T)-3\dim\mathbb P_{k-4}(T).
\end{align*}

\subsubsection{Scott-Vogelius元}

两维Scott-Vogelius元\cite{ScottVogelius1985}离散速度和压力的有限元空间为
\begin{align*}
V_{h}&:=\{\boldsymbol{v}\in H_{0}^{1}(\Omega,\mathbb{R}^{2}): \boldsymbol{v}|_T\in \mathbb P_k(T;\mathbb{R}^{2}), T\in \mathcal{T}_h\},
\\
P_{h}&:=\{q\in L_{0}^2(\Omega): q|_T\in \mathbb P_{k-1}(T), T\in \mathcal{T}_h\}.
\end{align*}

对于三角剖分$\mathcal T_h$中的顶点$V$,记$\theta_1, \ldots, \theta_n$为以$V$顶点的角的角度,假设这些角按逆时针排列. 若$V$为内部顶点,令$\theta_{n+1}:=\theta_1$. 定义 
\begin{equation*}
S(V)=\begin{cases}
0, & n=1,\\
\max\{\pi-\theta_1-\theta_{2},\pi-\theta_1-\theta_{n}\}, & n>1, V\in\partial\Omega, \\
\max\{\pi-\theta_1-\theta_{2},\pi-\theta_{n}-\theta_{n+1}\}, & V\not\in\partial\Omega.
\end{cases}
\end{equation*}
显然,$S(V)=0$当且仅当$\mathcal T_h$中所有以$V$为顶点的边落在两条直线上. 此时,称$V$是奇异的. 若$S(V)$是一个很小的正数,则$V$接近奇异的. 因此,$S(V)$度量了$V$的奇异程度.

\begin{lemma}[\cite{ScottVogelius1985}]
设$\mathcal T_h$是拟一致的三角剖分. 假设存在常数$\delta>0$使得
\begin{equation*}
S(V)\geq\delta,\quad v\in\Delta_0(\mathcal T_h).
\end{equation*}
则当$k\geq4$时,Scott-Vogelius元满足离散inf-sup条件,其中常数依赖于$\delta$.
\end{lemma}

Scott-Vogelius元是divergence-free的.

\begin{remark}\rm
在任意维单形剖分的Alfeld加密下,	Scott-Vogelius元对任意的$k\geq d$均满足离散inf-sup条件\cite{GuzmanNeilan2018,Zhang2005,ArnoldQin1992}. 当$1\leq k<d$时,速度形函数空间需要增加修正的Bernardi-Raugel面泡函数.
\end{remark}

\begin{remark}\rm
文献\cite{ChenHuang2024}详细研究了Stokes方程divergence-free的协调元,包括压力间断和压力连续的divergence-free协调元. Divergence-free协调元一般含有超光滑自由度.
\end{remark}




\subsection{压力连续的协调元方法}
这一节考虑压力连续的协调元方法,此时由分部积分公式知$(\div\boldsymbol{v},q)=-(\boldsymbol{v}, \nabla q)$,因此选取的空间$V_h$的维数相对于$\nabla P_h$的维数足够大即可.

\subsubsection{MINI元}

MINI元\cite{ArnoldBrezziFortin1984}用Lagrange元离散压力,在Lagrange元的基础上增补泡函数离散速度,即
\begin{align*}
V_{h}&:=\{\boldsymbol{v}\in H_{0}^{1}(\Omega,\mathbb{R}^{d}): \boldsymbol{v}|_T\in \mathbb P_k(T;\mathbb{R}^{d})+b_T\nabla\mathbb P_k(T), T\in \mathcal{T}_h\},
\\
P_{h}&:=\{q\in H^{1}(\Omega)\cap L_{0}^2(\Omega): q|_T\in \mathbb P_{k}(T), T\in \mathcal{T}_h\},
\end{align*}
其中$k\geq1$. 当$k=1$时,
\begin{equation*}
V_{h}=\{\boldsymbol{v}\in H_{0}^{1}(\Omega,\mathbb{R}^{d}): \boldsymbol{v}|_T\in \mathbb P_1(T;\mathbb{R}^{d})\oplus b_T\mathbb P_0(T;\mathbb{R}^{d}), T\in \mathcal{T}_h\}.
\end{equation*}

MINI元速度空间$V_h$在边界上的自由度同Lagrange元在边界上的自由度是一样的,其内部自由度为
\begin{equation*}
% \label{MINIDoF0}
(\boldsymbol{v}, \boldsymbol{q})_T, \quad \boldsymbol{q}\in\mathbb P_{k-d-1}(T;\mathbb{R}^{d})+\nabla\mathbb P_k(T).
\end{equation*}
特别地,当$k=1$时,其
内部自由度为
\begin{equation*}
% \label{MINIDoF0}
(\boldsymbol{v}, \boldsymbol{q})_T, \quad \boldsymbol{q}\in\mathbb P_{0}(T;\mathbb{R}^{d}).
\end{equation*}

\begin{lemma}
设$k\geq1$, MINI元满足离散inf-sup条件\eqref{u4}.
\end{lemma}
\begin{prf}
证明过程类似于Crouzeix-Raviart元.
记$I_h^{\rm SZ}: H_{0}^{1}(\Omega,\mathbb{R}^{d})\to V_{h}^L$为Scott-Zhang插值算子\cite{ScottZhang1990},其中$V_{h}^L$为$k$次Lagrange元空间$\{\boldsymbol{v}\in H_{0}^{1}(\Omega,\mathbb{R}^{d}): \boldsymbol{v}|_T\in \mathbb P_k(T;\mathbb R^d), T\in \mathcal{T}_h\}$.
引入插值算子$\Pi_2: H_{0}^{1}(\Omega,\mathbb{R}^{d})\to V_{h}$, 定义如下: 对任意的$T\in\mathcal{T}_h$, $(\Pi_2\boldsymbol{v})|_T\in b_T\nabla\mathbb P_{k}(T)$满足
\begin{align*}
(\Pi_2\boldsymbol{v}, \boldsymbol{q})_T&=(\boldsymbol{v}, \boldsymbol{q})_T, \quad \boldsymbol{q}\in\nabla\mathbb P_{k}(T).
\end{align*}
由仿射等价性可得,
\begin{equation*}
\|\Pi_{2}\boldsymbol{v}\|_{0,T}\lesssim  \|\boldsymbol{v}\|_{0,T}.
\end{equation*}
故$\Pi_2$满足\eqref{fortin:pi21}.
% 结合逆不等式得
% \begin{equation*}
% \|\Pi_{2}\boldsymbol{v}\|_{0,T}+h_T|\Pi_{2}\boldsymbol{v}|_{1,T}\lesssim \|\boldsymbol{v}\|_{0,T}+h_T|\boldsymbol{v}|_{1,T}.
% \end{equation*}
利用分部积分可得
\begin{equation*}
(\div(\boldsymbol{v}-\Pi_2\boldsymbol{v}),q)=-(\boldsymbol{v}-\Pi_2\boldsymbol{v},\nabla q)=0,
\quad\;\; q\in P_{h}, \boldsymbol{v}\in H_0^{1}(\Omega; \mathbb{R}^{d}).
\end{equation*}
即$\Pi_2$满足\eqref{fortin:pi22}.

根据\eqref{PihPi12}定义插值算子$\Pi_{h}: H_0^{1}(\Omega; \mathbb{R}^{d})\rightarrow  V_{h}$,即
$\Pi_{h}\boldsymbol{v}=I_h^{\rm SZ}\boldsymbol{v}
+\Pi_{2}(\boldsymbol{v}-I_h^{\rm SZ}\boldsymbol{v})$. 由引理\ref{lem:fortinoperator}可知,$\Pi_{h}$是Fortin算子,故由Fortin 准则可得离散inf-sup条件\eqref{u4}成立.
\end{prf}

当$\boldsymbol{u}\in H_0^1(\Omega;\mathbb R^d)\cap H^{k+1}(\Omega;\mathbb R^d)$和$p\in L_0^2(\Omega)\cap H^{k+1}(\Omega)$时, 由定理~\ref{thm:stokeserrorestimate}可知Crouzeix-Raviart元方法的误差估计
\begin{equation*}
\|\boldsymbol{u}-\boldsymbol{u}_h\|_{1} + \|p-p_h\|_{0}
\lesssim h^k(\|\boldsymbol{u}\|_{k+1}+h\|p\|_{k+1}).
\end{equation*}
该误差估计的收敛阶对于速度$\boldsymbol{u}$是最优的,但对于压力$p$不是最优的.

最低次MINI元因为简单常被用于离散Stokes方程.



\subsubsection{Taylor-Hood元}

Taylor-Hood元\cite{TaylorHood1973,Boffi1994,Boffi1997}对速度$\boldsymbol{u}$和压力$p$均采用Lagrange元离散,其误差估计对速度$\boldsymbol{u}$和压力$p$均是最优的. 具体来说,Taylor-Hood元离散速度$\boldsymbol{u}$和压力$p$的有限元空间分别为
\begin{align*}
V_{h}&:=\{\boldsymbol{v}\in H_{0}^{1}(\Omega,\mathbb{R}^{d}): \boldsymbol{v}|_T\in \mathbb P_k(T;\mathbb{R}^{d}), T\in \mathcal{T}_h\},
\\
P_{h}&:=\{q\in H^{1}(\Omega)\cap L_{0}^2(\Omega): q|_T\in \mathbb P_{k-1}(T), T\in \mathcal{T}_h\},
\end{align*}
其中$k\geq2$.

对于Taylor-Hood元,对单形剖分$\mathcal T_h$作如下假设:

\begin{center}
每一个$d$-维单形$T\in\mathcal T_h$至少有一个顶点位于开区域$\Omega$内部. 
\end{center}

在此网格假设下,Taylor-Hood元满足离散inf-sup条件\eqref{u4},其证明相对复杂,参考\cite{BoffiBrezziFortin2013,Boffi1994,Boffi1997,GiraultScott2003,Chen2014,DieningStornTscherpel2022}. 这里给出文献\cite{DieningStornTscherpel2022}中任意维最低次Taylor-Hood元离散inf-sup条件\eqref{u4}的证明, 关键之处是$\nabla P_{h}$为最低次棱元的子空间.
\begin{lemma}
设$k=2$, Taylor-Hood元满足离散inf-sup条件\eqref{u4}.
\end{lemma}
\begin{prf}
记$I_h^{\rm SZ}: H_{0}^{1}(\Omega,\mathbb{R}^{d})\to V_{h}$为Scott-Zhang插值算子\cite{ScottZhang1990}.
下面考虑插值算子$\Pi_2: H_{0}^{1}(\Omega,\mathbb{R}^{d})\to V_{h}$的构造.

记$\mathcal E_h$, $\mathring{\mathcal E}_h$和$\mathcal E^{\partial}_h$分别为单形剖分$\mathcal T_h$所有一维边、内边和边界边的集合. 对端点为$\texttt{v}_i$和$\texttt{v}_j$的边$e_{ij}\in\mathcal E_h$,令$\boldsymbol{b}_{ij}=\lambda_i\lambda_j\boldsymbol{t}_{ij}/\int_{\Omega}\lambda_i\lambda_j\dx$, 则有
\begin{equation*}
(\div\boldsymbol{b}_{ij}, \lambda_k)=-(\boldsymbol{b}_{ij}, \nabla \lambda_{k})=-\frac{1}{\int_{\Omega}\lambda_{i}\lambda_{j}\dx}\int_{\Omega}\lambda_{i}\lambda_{j}\boldsymbol{t}_{ij}\cdot\nabla\lambda_{k}\dx=-\boldsymbol{t}_{ij}\cdot\nabla\lambda_{k}=\delta_{ik}-\delta_{jk}.
\end{equation*}
易知,当$e_{ij}\in\mathring{\mathcal E}_h$时$\boldsymbol{b}_{ij}\in V_{h}$,当$e_{ij}\in\mathcal E^{\partial}_h$时$\boldsymbol{b}_{ij}\not\in V_{h}$. 为此,对$\boldsymbol{b}_{ij}$进行修正.
对于边界边$e_{ij}\in\mathcal E^{\partial}_h$,由网格假设知,存在一个$\Omega$内的顶点$\texttt{v}_m$使得$e_{im}, e_{jm}\in\mathring{\mathcal E}_h$. 令
\begin{equation*}
\boldsymbol{\psi}_{ij}=\begin{cases}
\boldsymbol{b}_{ij}, & e_{ij}\in\mathring{\mathcal E}_h,\\
\boldsymbol{b}_{im}+\boldsymbol{b}_{mj}, & e_{ij}\in\mathcal E^{\partial}_h,
\end{cases}
\end{equation*}
则所有的$\boldsymbol{\psi}_{ij}$均属于$V_h$, 且成立
\begin{equation*}
(\div\boldsymbol{\psi}_{ij}, \lambda_k)=\delta_{ik}-\delta_{jk}.
\end{equation*}
现在如下定义插值算子$\Pi_2: H_{0}^{1}(\Omega,\mathbb{R}^{d})\to V_{h}$, 
\begin{equation*}
\Pi_2\boldsymbol{v}:=\sum_{e_{ij}\in\mathcal E_h, i<j}(\boldsymbol{v}, \lambda_i\nabla\lambda_j-\lambda_j\nabla\lambda_i)\boldsymbol{\psi}_{ij}.
\end{equation*}
可知
\begin{equation*}
\|\Pi_{2}\boldsymbol{v}\|_{0,T}\lesssim  \|\boldsymbol{v}\|_{0,\omega_T}.
\end{equation*}
故$\Pi_2$满足\eqref{fortin:pi21}.
由$\Pi_{2}\boldsymbol{v}$的定义可得,
\begin{align*}
(\div\Pi_{2}\boldsymbol{v},\lambda_{k})
&=\sum_{e_{ij}\in\mathcal{E}_{h},\,i<j}(\boldsymbol{v},\lambda_{i}\nabla\lambda_{j}-\lambda_{j}\nabla\lambda_{i})(\div\boldsymbol{\psi}_{ij},\lambda_{k}) \\
&=\sum_{e_{ij}\in \mathcal{E}_{h},\,i<j}(\boldsymbol{v},\lambda_{i}\nabla\lambda_{j}-\lambda_{j}\nabla\lambda_{i})(\delta_{ik}-\delta_{jk}) \\
&=\sum_{e_{kj}\in \mathcal{E}_{h},\,k<j}(\boldsymbol{v},\lambda_{k}\nabla\lambda_{j}-\lambda_{j}\nabla\lambda_{k}) -\sum_{e_{ik}\in \mathcal{E}_{h},\,i<k}(\boldsymbol{v},\lambda_{i}\nabla\lambda_{k}-\lambda_{k}\nabla\lambda_{i}) \\
&=\sum_{e_{ki}\in \mathcal{E}_{h},\,i>k}(\boldsymbol{v},\lambda_{k}\nabla\lambda_{i}-\lambda_{i}\nabla\lambda_{k}) + \sum_{e_{ik}\in \mathcal{E}_{h},\,i<k}(\boldsymbol{v}, \lambda_{k}\nabla\lambda_{i}-\lambda_{i}\nabla\lambda_{k}) \\
&=\sum_{e_{ki}\in \mathcal{E}_{h}}(\boldsymbol{v},\lambda_{k}\nabla\lambda_{i}-\lambda_{i}\nabla\lambda_{k}) \\
&=(\boldsymbol{v},\lambda_{k}\nabla 1-\nabla\lambda_{k})=-(\boldsymbol{v},\nabla\lambda_{k})=(\div\boldsymbol{v},\lambda_{k}).
\end{align*}
即可推得
$(\div(\Pi_{2}\boldsymbol{v}-\boldsymbol{v}),\lambda_{k})=0$. 故$\Pi_2$满足\eqref{fortin:pi22}.

根据\eqref{PihPi12}定义插值算子$\Pi_{h}: H_0^{1}(\Omega; \mathbb{R}^{d})\rightarrow  V_{h}$,即
$\Pi_{h}\boldsymbol{v}=I_h^{\rm SZ}\boldsymbol{v}
+\Pi_{2}(\boldsymbol{v}-I_h^{\rm SZ}\boldsymbol{v})$. 由引理\ref{lem:fortinoperator}可知,$\Pi_{h}$是Fortin算子,故由Fortin 准则可得离散inf-sup条件\eqref{u4}成立.
\end{prf}


当$\boldsymbol{u}\in H_0^1(\Omega;\mathbb R^d)\cap H^{k+1}(\Omega;\mathbb R^d)$和$p\in L_0^2(\Omega)\cap H^{k}(\Omega)$时, 由定理~\ref{thm:stokeserrorestimate}可知Taylor-Hood元方法的误差估计
\begin{equation*}
\|\boldsymbol{u}-\boldsymbol{u}_h\|_{1} + \|p-p_h\|_{0}
\lesssim h^k(\|\boldsymbol{u}\|_{k+1}+h\|p\|_{k}).
\end{equation*}
该误差估计的收敛阶对于速度$\boldsymbol{u}$和压力$p$均是最优的.



\section{Stokes方程的非协调混合元方法}

这一节考虑Stokes方程的非协调混合元方法,即$V_h \not\subset H_0^{1}(\Omega; \mathbb{R}^{d})$. 此时,
混合变分问题(\ref{BI2})-(\ref{BI02})的有限元离散为 : 找 $\boldsymbol{u}_h \in  V_h$ 和 $p_h \in P_h$满足
\begin{align}
(\nabla_h\boldsymbol{u}_h,\nabla_h\boldsymbol{v}) + (\div_h\boldsymbol{v},p_h) &=(\boldsymbol{f},\boldsymbol{v}),\quad \boldsymbol{v}\in   V_h,\label{StokesNcfm1}\\
(\div_h\boldsymbol{u}_h,q) &=0, \quad\quad\quad\, q\in P_h,\label{StokesNcfm2}
\end{align}
其中$\nabla_h$和$\div_h$关于网格剖分$\mathcal T_h$分片定义的梯度算子和散度算子.
为了得到非协调元方法(\ref{StokesNcfm1})-(\ref{StokesNcfm2})解的适定性, 我们需要验证Brezzi 定理的两个条件.
\begin{enumerate}[label=(\alph*)]
\item 强制性: 即离散Poincar\'e不等式
\begin{align}\label{StokesNcfmu3}
\|\boldsymbol{v}\|^{2}_{0}
\lesssim (\nabla_h\boldsymbol{v},\nabla_h\boldsymbol{v}),\quad \boldsymbol{v}\in  V_h.\end{align}
\item 离散inf-sup 条件
\begin{equation}\label{StokesNcfmu4}
\|q\|_{0}\lesssim\sup _{\boldsymbol{v} \in  V_h}
\frac{(\div_h\boldsymbol{v}, q)}{\|\nabla_h\boldsymbol{v}\|_{0}},\quad\, q\in P_h.
\end{equation}
\end{enumerate}

类似于引理~\ref{lem01},
离散inf-sup 条件(\ref{StokesNcfmu4})等价于存在Fortin算子,即
存在一个有界线性算子 $\Pi_h :H_0^{1}(\Omega; \mathbb{R}^{d})\rightarrow V_h$满足
\begin{align}
 (\div_h(\boldsymbol{v}-\Pi_h\boldsymbol{v}), q) &=0,\qquad\qquad  \boldsymbol{v}\in H_0^{1}(\Omega; \mathbb{R}^{d}),q\in P_{h} ,\label{ncfmk1}\\
 \|\nabla_h(\Pi_h\boldsymbol{v})\|_{0} &\lesssim\|\nabla_h\boldsymbol{v}\|_{0}, \quad\;  \boldsymbol{v}\in H_0^{1}(\Omega; \mathbb{R}^{d}).\label{ncfmk2}
\end{align}

\begin{theorem}\label{thm:stokesncfmerrorestimate}
设$(\boldsymbol{u}, p)\in H_0^{1}(\Omega; \mathbb{R}^{d})\times L_{0}^{2}(\Omega)$是Stokes方程\eqref{eq-Stokes2}的解, $(\boldsymbol{u}_h, p_h)\in V_h\times P_h$是混合元方法(\ref{StokesNcfm1})-(\ref{StokesNcfm2})的解,则有误差估计
\begin{equation}\label{stokesncfmerroruh}	
\|\nabla_h(\boldsymbol{u}-\boldsymbol{u}_h)\|_0\lesssim \|\nabla_h(\boldsymbol{u}-\Pi_{h}\boldsymbol{u})\|_0 + \sup_{\boldsymbol{v}\in V_h}\frac{E_h(\boldsymbol{u},p;\boldsymbol{v})}{\|\nabla_h\boldsymbol{v}\|_0} + \inf_{q\in P_h}\sup_{\boldsymbol{v}\in V_h}\frac{(\div_h\boldsymbol{v},p-q)}{\|\nabla_h\boldsymbol{v}\|_0},
\end{equation}
\begin{equation}\label{stokesncfmerrorph}	
\|p-p_h\|_0\lesssim \|\nabla_h(\boldsymbol{u}-\Pi_{h}\boldsymbol{u})\|_0 + \sup_{\boldsymbol{v}\in V_h}\frac{E_h(\boldsymbol{u},p;\boldsymbol{v})}{\|\nabla_h\boldsymbol{v}\|_0} + \inf_{q\in P_h}\|p-q\|_0,
\end{equation}
其中$E_h(\boldsymbol{u},p;\boldsymbol{v}):=(\nabla\boldsymbol{u}, \nabla_h\boldsymbol{v}) + (\div_h\boldsymbol{v}, p) -(\boldsymbol{f},\boldsymbol{v})$.
进一步,若有$\div_h V_h=P_h$,则
\begin{equation}\label{stokesncfmerroruhu}	
\|\nabla_h(\boldsymbol{u}-\boldsymbol{u}_h)\|_0\lesssim \|\nabla_h(\boldsymbol{u}-\Pi_{h}\boldsymbol{u})\|_0+ \sup_{\boldsymbol{v}\in V_h}\frac{E_h(\boldsymbol{u},p;\boldsymbol{v})}{\|\nabla_h\boldsymbol{v}\|_0}.
\end{equation}
这里$\displaystyle\sup_{\boldsymbol{v}\in V_h}\frac{E_h(\boldsymbol{u},p;\boldsymbol{v})}{\|\nabla_h\boldsymbol{v}\|_0}$称为非协调元的相容性误差.
\end{theorem}
\begin{prf}
任取$q\in P_h$, 记$\boldsymbol{v}=\Pi_h\boldsymbol{u}-\boldsymbol{u}_h$, 则由\eqref{ncfmk1}和\eqref{StokesNcfm2}可得
\begin{equation*}
(\div_h\boldsymbol{v}, q-p_h)=(\div_h(\Pi_h\boldsymbol{u}-\boldsymbol{u}_h), q-p_h)=0.
\end{equation*}
接着利用\eqref{StokesNcfm1}可得
\begin{align*}
\|\nabla_h(\Pi_{h}\boldsymbol{u}-\boldsymbol{u}_h)\|_0^2&=(\nabla_h(\Pi_{h}\boldsymbol{u}-\boldsymbol{u}_h), \nabla_h\boldsymbol{v}) = (\nabla_h(\Pi_{h}\boldsymbol{u}-\boldsymbol{u}_h), \nabla_h\boldsymbol{v}) + (\div_h\boldsymbol{v}, q-p_h) \\
&= (\nabla_h(\Pi_{h}\boldsymbol{u}), \nabla_h\boldsymbol{v}) + (\div_h\boldsymbol{v}, q)-(\boldsymbol{f},\boldsymbol{v}) \\
&= (\nabla_h(\Pi_{h}\boldsymbol{u}-\boldsymbol{u}), \nabla_h\boldsymbol{v}) + (\div_h\boldsymbol{v}, q-p) + (\nabla\boldsymbol{u}, \nabla_h\boldsymbol{v}) + (\div_h\boldsymbol{v}, p) -(\boldsymbol{f},\boldsymbol{v}).
\end{align*}
故由Cauchy-Schwarz可得
\begin{equation*}
\|\nabla_h(\Pi_{h}\boldsymbol{u}-\boldsymbol{u}_h)\|_0\lesssim \|\nabla_h(\boldsymbol{u}-\Pi_{h}\boldsymbol{u})\|_0 + \sup_{\boldsymbol{v}\in V_h}\frac{E_h(\boldsymbol{u},p;\boldsymbol{v})}{\|\nabla_h\boldsymbol{v}\|_0} + \inf_{q\in P_h}\sup_{\boldsymbol{v}\in V_h}\frac{(\div_h\boldsymbol{v},p-q)}{\|\nabla_h\boldsymbol{v}\|_0}.
\end{equation*}
再结合三角不等式即有\eqref{stokesncfmerroruh}.

对任意的$\boldsymbol{v} \in  V_h$,利用\eqref{StokesNcfm1}可推得
\begin{equation*}	
(\div_h\boldsymbol{v}, p-p_h)=(\div_h\boldsymbol{v}, p)+(\nabla_h\boldsymbol{u}_h,\nabla_h\boldsymbol{v})-(\boldsymbol{f},\boldsymbol{v}) =(\nabla_h(\boldsymbol{u}_h-\boldsymbol{u}),\nabla_h\boldsymbol{v}) + E_h(\boldsymbol{u},p;\boldsymbol{v}).
\end{equation*}
由离散inf-sup 条件\eqref{u4},
\begin{align*}
\|q-p_h\|_{0}&\lesssim\sup _{\boldsymbol{v} \in  V_h}
\frac{(\div_h\boldsymbol{v}, q-p_h)}{\|\nabla_h\boldsymbol{v}\|_0}\lesssim \|p-q\|_0+\sup _{\boldsymbol{v} \in  V_h}
\frac{(\div_h\boldsymbol{v}, p-p_h)}{\|\nabla_h\boldsymbol{v}\|_0} \\
&\lesssim \|p-q\|_0 + \|\nabla_h(\boldsymbol{u}-\boldsymbol{u}_h)\|_0 + \sup_{\boldsymbol{v}\in V_h}\frac{E_h(\boldsymbol{u},p;\boldsymbol{v})}{\|\nabla_h\boldsymbol{v}\|_0}.
\end{align*}
进而借助三角不等式和\eqref{stokesncfmerroruh}可推得\eqref{stokesncfmerrorph}.

进一步,若有$\div_h V_h=P_h$,可取$q=Q_hp$, 则\eqref{stokesncfmerroruh}式右端的第三项为零,故\eqref{stokesncfmerroruhu}成立.
\end{prf}

\subsection{非协调$P_1$-$P_0$元}

向量值非协调线性元的形函数空间为$\mathbb P_1(T;\mathbb R^d)$,自由度为
\begin{equation*}
% \label{MINIDoF0}
\int_F\boldsymbol{v}\dd s, \quad F\in\Delta_{d-1}(T).
\end{equation*}

在文献\cite{CrouzeixRaviart1973}中,分别用非协调线性元和分片常数离散速度和压力,即令
\begin{align*}
 V_{h}
&:=\big\{\boldsymbol{v}\in L^{2}(\Omega,\mathbb{R}^{d}):
\boldsymbol{v}|_T\in \mathbb P_1(T,\mathbb{R}^{d}), T\in \mathcal{T}_h,\,\int_{F}[\boldsymbol{v}]\dd s=0,\,F\in \triangle_{d-1}(\mathcal{T}_h)\big\},
\\
P_{h}&:=\{q\in L_0^2(\Omega): q|_T\in \mathbb P_0(T), T\in \mathcal{T}_h\},
\end{align*}
其中$[\boldsymbol{v}]|_F$为$\boldsymbol{v}$跨过面$F$的跳量; 当$F\subset\partial\Omega$时,$[\boldsymbol{v}]|_F=\boldsymbol{v}$. 显然$ V_h  \not\subset H_0^{1}(\Omega; \mathbb{R}^{d})$. 对于非协调元空间$V_h$,成立离散Poincar\'e不等式 \cite{Brenner2003}
\begin{equation*}
\|\boldsymbol{v}\|_0\lesssim \|\nabla_h\boldsymbol{v}\|_0,\quad\boldsymbol{v}\in V_h.
\end{equation*}


\begin{lemma}
非协调$P_1$-$P_0$元满足$\div_hV_h=P_h$和离散inf-sup 条件\eqref{StokesNcfmu4}.
\end{lemma}
\begin{prf}
如下定义插值算子
$\Pi_{h}:H_0^{1}(\Omega; \mathbb{R}^{d})\to  V_{h}$:
\begin{align}\label{qi1}
\int_{F}\Pi_{h}\boldsymbol{v}\dd s=\int_{F} \boldsymbol{v}\dd s,
\quad \boldsymbol{v}\in H_0^{1}(\Omega; \mathbb{R}^{d}),\,F\in\triangle_{d-1}(\mathcal{T}_h).
\end{align}
由分部积分可得
\begin{equation*}
(\div_h(\boldsymbol{v}-\Pi_h\boldsymbol{v}), q) =0,\quad  \boldsymbol{v}\in H_0^{1}(\Omega; \mathbb{R}^{d}), \, q\in P_{h}.	
\end{equation*}
故$\div_h(\Pi_h\boldsymbol{v})=Q_h(\div\boldsymbol{v})$对$\boldsymbol{v}\in H_0^{1}(\Omega; \mathbb{R}^{d})$成立. 于是, $\div_hV_h=P_h$.
由尺度论证技巧可证
\begin{equation*}
\|\nabla_h(\Pi_h\boldsymbol{v})\|_{0} \lesssim\|\nabla_h\boldsymbol{v}\|_{0}, \quad\;  \boldsymbol{v}\in H_0^{1}(\Omega; \mathbb{R}^{d}).
\end{equation*}
因此,$\Pi_{h}$是Fortin算子,从而由Fortin准则知离散inf-sup 条件\eqref{StokesNcfmu4}成立.
\end{prf}

非协调$P_1$-$P_0$元方法是分片divergence-free的,但不是divergence-free的.

\begin{lemma}
设$(\boldsymbol{u}, p)\in H_0^{1}(\Omega; \mathbb{R}^{d})\times L_{0}^{2}(\Omega)$是Stokes方程\eqref{eq-Stokes2}的解, $(\boldsymbol{u}_h, p_h)\in V_h\times P_h$是非协调$P_1$-$P_0$元方法(\ref{StokesNcfm1})-(\ref{StokesNcfm2})的解. 假设$\boldsymbol{u}\in H^{2}(\Omega; \mathbb{R}^{d})$,$p\in H^{1}(\Omega)$,则有误差估计
\begin{equation}\label{eq:stokesncfmp1p0error}	
\|\nabla_h(\boldsymbol{u}-\boldsymbol{u}_h)\|_0+\|p-p_h\|_0\lesssim h(\|\boldsymbol{u}\|_2+\|p\|_1).
\end{equation}
\end{lemma}
\begin{prf}
由定理~\ref{thm:stokesncfmerrorestimate}和$\Pi_h$的插值误差估计,我们只要估计相容性误差即可.

由分部积分和弱连续性可推得,
\begin{align*}
E_h(\boldsymbol{u},p;\boldsymbol{v})&=(\nabla\boldsymbol{u}, \nabla_h\boldsymbol{v}) + (\div_h\boldsymbol{v}, p) -(\boldsymbol{f},\boldsymbol{v})=(\nabla\boldsymbol{u}, \nabla_h\boldsymbol{v}) + (\div_h\boldsymbol{v}, p) + (\Delta\boldsymbol{u}+\nabla p,\boldsymbol{v}) \\
&=\sum_{T\in\mathcal T_h}\big((\partial_n\boldsymbol{u}, \boldsymbol{v})_{\partial T}+(p, \boldsymbol{v}\cdot\boldsymbol{n})_{\partial T}\big)=\sum_{F\in\mathcal F_h}\big((\partial_n\boldsymbol{u}, [\boldsymbol{v}])_F+(p, [\boldsymbol{v}\cdot\boldsymbol{n}])_F\big) \\
&=\sum_{F\in\mathcal F_h}\big((\partial_n\boldsymbol{u}-Q_{0,F}(\partial_n\boldsymbol{u}), [\boldsymbol{v}-Q_{0,F}\boldsymbol{v}])_F+(p-Q_{0,F}p, [(\boldsymbol{v}-Q_{0,F}\boldsymbol{v})\cdot\boldsymbol{n}])_F\big).
\end{align*}
再利用投影算子$Q_{0,F}$的误差估计可得
\begin{equation*}
E_h(\boldsymbol{u},p;\boldsymbol{v})\lesssim h(\|\boldsymbol{u}\|_2+\|p\|_1).
\end{equation*}
得证.
\end{prf}

由误差估计\eqref{eq:stokesncfmp1p0error}知,非协调$P_1$-$P_0$元方法不是压力鲁棒的.
通过对\eqref{StokesNcfm1}右端项中的$\boldsymbol{v}$进行最低次Raviart–Thomas重构\cite{Linke2014,BrenneckeLinkeMerdonSchoeberl2015},得到的非协调$P_1$-$P_0$元方法是压力鲁棒的. 更多Stokes方程数值方法压力鲁棒性的介绍参见\cite{JohnLinkeMerdonNeilanEtAl2017}.




\subsection{一个新的非协调元}

记二次非协调泡函数
\begin{equation*}
b_{T}^{\rm NC}=2-(d+1)(\lambda_0^2+\lambda_1^2+\cdots+\lambda_d^2).
\end{equation*}
泡函数$b_{T}^{\rm NC}$在单形重心处取值为$1$.

\begin{lemma}
对任意的$F\in\Delta_{d-1}(T)$,
成立
\begin{equation*}
(b_{T}^{\rm NC}, q)_F=0,\quad q\in \mathbb P_1(F).
\end{equation*}
\end{lemma}
\begin{prf}
不妨设面$F$对应顶点$\texttt{v}_0$,故等价于证明 %和$i=0,\ldots, d$
\begin{equation*}
(b_{T}^{\rm NC}, \lambda_i)_F=0,\quad i=1,2,\ldots, d.
\end{equation*}
通过直接计算
\begin{equation*}
\frac{1}{|F|}(b_{T}^{\rm NC}, \lambda_i)_F=\frac{2}{d}-(d+1)\sum_{j=1}^d\frac{1}{|F|}\int_F\lambda_j^2\lambda_i\dd s=\frac{2}{d}-(d+1)\frac{(d-1)!}{(d+2)!}(2d+4)=0.
\end{equation*}
得证.
\end{prf}

速度的形函数空间取为$\mathbb P_1(T;\mathbb R^d)+b_{T}^{\rm NC}\mathbb P_0(T;\mathbb R^d)$,自由度为
\begin{align*}
\int_F\boldsymbol{v}\dd s&, \quad F\in\Delta_{d-1}(T),\\
\int_T\boldsymbol{v}\dx.&
\end{align*}
压力的形函数空间取为$\mathbb P_1(T)$

\begin{itemize}
\item 压力用分片线性元离散,由此给出Stokes方程的一种非协调元方法,进行误差分析和数值试验
\item 压力用非协调线性元离散,由此给出Stokes方程的一种非协调元方法,进行误差分析和数值试验
\item 压力用线性元Lagrange元离散,由此给出Stokes方程的一种非协调元方法,进行误差分析和数值试验
\end{itemize}



\subsection{Divergence-free非协调元}

文献\cite{XieXuXue2008}在$H(\div)$协调元的基础上,通过泡函数增加切向连续性来构造divergence-free非协调元.
文献\cite{XieXuXue2008}给出了两维、三维低阶divergence-free非协调元,这里统一给出任意维divergence-free非协调元.

记$\mathbb K$为所有$d$阶反对称矩阵所组成的线性空间. 设$d$维单形$T$的顶点为$\texttt{v}_0, \texttt{v}_1, \ldots, \texttt{v}_d$,相对应的$(d-1)$维面为$F_i$ ($i=0,1,\ldots, d$), 面$F_i$的法向量记为$\boldsymbol{n}_{F_i}$, 不在引起混淆的情况下简记为$\boldsymbol{n}_{i}$.

\begin{lemma}\label{lem:divfreencfmK}
设$\boldsymbol{\tau}_0, \boldsymbol{\tau}_1, \ldots, \boldsymbol{\tau}_d\in\mathbb K$满足
\begin{equation*}
\boldsymbol{n}_i^{\intercal}\boldsymbol{\tau}_j\boldsymbol{n}_k=\boldsymbol{n}_j^{\intercal}\boldsymbol{\tau}_i\boldsymbol{n}_k,\quad 0\leq i,j,k\leq d, i\neq j, i\neq k, j\neq k, 
\end{equation*}
则$\boldsymbol{\tau}_i=0$, 其中$i=0,1,\ldots, d$.
\end{lemma}
\begin{prf}
由$\boldsymbol{\tau}_i$的反对称性知$\boldsymbol{n}_k^{\intercal}\boldsymbol{\tau}_j\boldsymbol{n}_i=-\boldsymbol{n}_i^{\intercal}\boldsymbol{\tau}_j\boldsymbol{n}_k$, $\boldsymbol{n}_k^{\intercal}\boldsymbol{\tau}_i\boldsymbol{n}_j=-\boldsymbol{n}_j^{\intercal}\boldsymbol{\tau}_i\boldsymbol{n}_k$, 因此
\begin{equation*}
\boldsymbol{n}_k^{\intercal}\boldsymbol{\tau}_j\boldsymbol{n}_i=\boldsymbol{n}_k^{\intercal}\boldsymbol{\tau}_i\boldsymbol{n}_j,\quad 0\leq i,j,k\leq d, i\neq j, i\neq k, j\neq k. 
\end{equation*}
从而,$\boldsymbol{n}_j^{\intercal}\boldsymbol{\tau}_i\boldsymbol{n}_k$关于前两个下标和后两个下标均是对称的. 由对称性可得
\begin{equation*}
\boldsymbol{n}_k^{\intercal}\boldsymbol{\tau}_i\boldsymbol{n}_j=\boldsymbol{n}_i^{\intercal}\boldsymbol{\tau}_k\boldsymbol{n}_j=\boldsymbol{n}_i^{\intercal}\boldsymbol{\tau}_j\boldsymbol{n}_k=\boldsymbol{n}_j^{\intercal}\boldsymbol{\tau}_i\boldsymbol{n}_k.
\end{equation*}
又由$\boldsymbol{\tau}_i$的反对称性知 $\boldsymbol{n}_k^{\intercal}\boldsymbol{\tau}_i\boldsymbol{n}_j=-\boldsymbol{n}_j^{\intercal}\boldsymbol{\tau}_i\boldsymbol{n}_k$. 故
\begin{equation*}
\boldsymbol{n}_j^{\intercal}\boldsymbol{\tau}_i\boldsymbol{n}_k=0,\quad 0\leq i,j,k\leq d, j\neq i, k\neq i. 
\end{equation*}
注意到$\{\boldsymbol{n}_j\boldsymbol{n}_k^{\intercal}: 0\leq j,k\leq d, j\neq i, k\neq i\}$构成了$d$阶矩阵的一组基. 于是$\boldsymbol{\tau}_i=0$,得证.
\end{prf}

引入低次$H(\div)$有限元形函数空间$V^{\div}(T)=\mathbb P_1(T;\mathbb R^d)+\boldsymbol{x}\mathbb P_k(T)$, $k=0,1$. 当$k=0$时,$V^{\div}(T)=\mathbb P_1(T;\mathbb R^d)$为最低次Brezzi-Douglas-Marini (BDM)元\cite{BrezziDouglasMarini1985,BrezziDouglasDuranFortin1987,Nedelec1986}的形函数空间;当$k=1$时,$V^{\div}(T)$为一次Raviart-Thomas (RT)元\cite{RaviartThomas1977,Nedelec1980,ChenHuang2022}的形函数空间. 定义刚体运动空间${\rm RM}(T):=\mathbb P_0(T;\mathbb R^d)+\mathbb K\boldsymbol{x}$. ${\rm RM}(T)$也是第一类最低次Nedelec元的形函数空间\cite{Nedelec1980},且有
\begin{equation*}
{\rm RM}(T)={\rm span}\{\lambda_i\nabla \lambda_j-\lambda_j\nabla \lambda_i: 0\leq i<j\leq d\}, \quad\dim{\rm RM}(T)=\frac{1}{2}d(d+1).
\end{equation*}
当$d=1$时,${\rm RM}(T)=\mathbb P_0(T)$.


现在在低次$H(\div)$有限元的基础上定义divergence-free非协调元. 形函数空间取为$V(T):=V^{\div}(T)\oplus\div(b_T\mathbb P_1(T;\mathbb K))$, 其中$b_T:=\lambda_0\lambda_1\ldots\lambda_d$为泡函数.
自由度为
\begin{subequations}\label{divfreencfmDoFs}
\begin{align}
\label{divfreencfmDoFs1}
(\boldsymbol{v}\cdot\boldsymbol{n}, q)_F, &\quad q\in\mathbb P_{1}(F), F\in\Delta_{d-1}(T),\\
\label{divfreencfmDoFs2}
(\boldsymbol{n}\times\boldsymbol{v}\times\boldsymbol{n}, \boldsymbol{q})_F, &\quad \boldsymbol{q}\in{\rm RM}(F), F\in\Delta_{d-1}(T),\\
\label{divfreencfmDoFs3}
(\boldsymbol{v}, \boldsymbol{q})_T, &\quad \boldsymbol{q}\in\mathbb P_{0}(T;\mathbb R^d)\quad\textrm{ if } k=1.
\end{align}
\end{subequations}

\begin{lemma}
设 $T$ 是一个 $d$ 维单形,$\boldsymbol{\tau}\in\mathbb P_1(T;\mathbb K)$,则$\boldsymbol{v}=\div(b_T\boldsymbol{\tau})$满足$\div\boldsymbol{v}=0$,以及$(\boldsymbol{v}\cdot\boldsymbol{n})|_{\partial T}=0$.
\end{lemma}
\begin{prf}
由$\boldsymbol{\tau}$的反对称性可得$\div\boldsymbol{v}=0$. 对于$F\in\Delta_{d-1}(T)$,由$\boldsymbol{\tau}$的反对称性和$b_T|_F=0$可得
\begin{equation*}
(\boldsymbol{v}\cdot\boldsymbol{n})|_{F}=(\div(b_T\boldsymbol{n}^{\intercal}\boldsymbol{\tau}))|_{F}=(\div(b_T\boldsymbol{n}^{\intercal}\boldsymbol{\tau}\Pi_F))|_{F}=(\div_F(b_T\boldsymbol{n}^{\intercal}\boldsymbol{\tau}\Pi_F))|_{F}=0.
\end{equation*}
\end{prf}

\begin{lemma}\label{lem:divfreencfmunisol}
形函数空间$V(T)=V^{\div}(T)\oplus\div(b_T\mathbb P_1(T;\mathbb K))$由自由度\eqref{divfreencfmDoFs}所唯一确定.
\end{lemma}
\begin{prf}
先证明$V^{\div}(T)\cap\div(b_T\mathbb P_1(T;\mathbb K))=0$. 对于$\boldsymbol{v}\in V^{\div}(T)\cap\div(b_T\mathbb P_1(T;\mathbb K))$, 由$\div\boldsymbol{v}=0$得$\boldsymbol{v}\in\mathbb P_1(T;\mathbb R^d)$. 注意到$\boldsymbol{v}\in\div(b_T\mathbb P_1(T;\mathbb K))$意味着$(\boldsymbol{v}\cdot\boldsymbol{n})|_{\partial T}=0$,故由BDM元的唯一可解性知$\boldsymbol{v}=0$.
于是,形函数空间$V(T)$的维数为
\begin{equation*}
\dim V^{\div}(T)+\dim\mathbb P_1(T;\mathbb K)=d(d+1)+dk +\frac{1}{2}d(d^2-1),
\end{equation*}
恰好等于自由度\eqref{divfreencfmDoFs}的个数.

设$\boldsymbol{v}=\boldsymbol{v}_1+\boldsymbol{v}_2\in V(T)$满足\eqref{divfreencfmDoFs}中所有自由度为零,其中$\boldsymbol{v}_1\in V^{\div}(T)$,$\boldsymbol{v}_2\in \div(b_T\mathbb P_1(T;\mathbb K))$. 易知$\div\boldsymbol{v}_2=0$, 以及$(\boldsymbol{v}_2\cdot\boldsymbol{n})|_{\partial T}=0$.
因此$\div\boldsymbol{v}=\div\boldsymbol{v}_1$, 以及$(\boldsymbol{v}\cdot\boldsymbol{n})|_{\partial T}=(\boldsymbol{v}_1\cdot\boldsymbol{n})|_{\partial T}$. 借助分部积分,由自由度\eqref{divfreencfmDoFs1}和\eqref{divfreencfmDoFs3}可推
\begin{equation*}
(\div\boldsymbol{v}, q)_T=(\boldsymbol{v}\cdot\boldsymbol{n}, q)_{\partial T}-(\boldsymbol{v}, \nabla q)_T=0,\quad q\in\mathbb P_1(T).
\end{equation*}
于是$\boldsymbol{v}_1\in\mathbb P_1(T;\mathbb R^d)$且满足$(\boldsymbol{v}_1\cdot\boldsymbol{n})|_{\partial T}=0$. 由BDM元的唯一可解性知$\boldsymbol{v}_1=0$. 故$\boldsymbol{v}\in \div(b_T\mathbb P_1(T;\mathbb K))$.

设$\boldsymbol{v}=\sum_{i=0}^d\div(b_T\lambda_i\boldsymbol{\tau}_i)$, 其中$\boldsymbol{\tau}_0, \boldsymbol{\tau}_1,\ldots, \boldsymbol{\tau}_d\in\mathbb K$. 由自由度\eqref{divfreencfmDoFs2}可得
\begin{equation*}
\sum_{m=0}^d\big(\div(b_T\lambda_{m}\boldsymbol{\tau}_{m}), \lambda_i\nabla_F \lambda_j-\lambda_j\nabla_F\lambda_i\big)_{F_{\ell}}=\big(\boldsymbol{v}, \lambda_i\nabla_F \lambda_j-\lambda_j\nabla_F\lambda_i\big)_{F_{\ell}}=0,
\end{equation*}
其中$0\leq i,j,\ell\leq d, i\neq\ell, j\neq\ell$. 注意到$(\div b_T)|_{F_{\ell}}=b_{F_{\ell}}\nabla\lambda_{\ell}$,
\begin{equation*}
\sum_{m=0}^d\big(b_{F_{\ell}}\lambda_{m}\boldsymbol{\tau}_{m}\boldsymbol{n}_{\ell}, \lambda_i\nabla_F \lambda_j-\lambda_j\nabla_F\lambda_i\big)_{F_{\ell}}=0.
\end{equation*}
直接计算可得
\begin{equation*}
2\sum_{m\neq\ell}(\nabla \lambda_j)^{\intercal}\boldsymbol{\tau}_{m}\boldsymbol{n}_{\ell} + (\nabla \lambda_j)^{\intercal}\boldsymbol{\tau}_{i}\boldsymbol{n}_{\ell} - 2\sum_{m\neq\ell}(\nabla \lambda_i)^{\intercal}\boldsymbol{\tau}_{m}\boldsymbol{n}_{\ell} - (\nabla \lambda_i)^{\intercal}\boldsymbol{\tau}_{j}\boldsymbol{n}_{\ell}=0,
\end{equation*}
其中$0\leq i,j,\ell\leq d, i\neq\ell, j\neq\ell$. 将上式关于下标$i$从$0$到$d$除了$j$和$\ell$进行求和可得
\begin{equation*}
(2d+1)\sum_{m\neq\ell}(\nabla \lambda_j)^{\intercal}\boldsymbol{\tau}_{m}\boldsymbol{n}_{\ell}=0, \quad 0\leq j,\ell\leq d, j\neq\ell.
\end{equation*}
因此$(\nabla\lambda_j)^{\intercal}\boldsymbol{\tau}_{i}\boldsymbol{n}_{\ell} = (\nabla\lambda_i)^{\intercal}\boldsymbol{\tau}_{j}\boldsymbol{n}_{\ell}$, 也即
\begin{equation*}
\boldsymbol{n}_j^{\intercal}\boldsymbol{\tau}_{i}\boldsymbol{n}_{\ell} = \boldsymbol{n}_i^{\intercal}\boldsymbol{\tau}_{j}\boldsymbol{n}_{\ell}, \quad 0\leq i,j,\ell\leq d, i\neq\ell, j\neq\ell.
\end{equation*}
由引理~\ref{lem:divfreencfmK}知, $\boldsymbol{\tau}_{i}=0$, $i=0,1,\ldots, d$. 故$\boldsymbol{v}=0$.
\end{prf}

分别定义离散速度和压力的整体有限元空间
\begin{align*}
 V_{h}&:=\{\boldsymbol{v}\in L^{2}(\Omega;\mathbb{R}^{d}): \boldsymbol{v}|_T\in V^{\div}(T)\oplus\div(b_T\mathbb P_1(T;\mathbb K)), T\in \mathcal{T}_h,\\
&\qquad\qquad\qquad\qquad\quad \textrm{所有自由度\eqref{divfreencfmDoFs}跨过内部边界是连续的},\\
&\qquad\qquad\qquad\qquad\qquad\quad \textrm{自由度\eqref{divfreencfmDoFs1}-\eqref{divfreencfmDoFs2}在边界上为零}\},
\\
P_{h}&:=\{q\in L_{0}^2(\Omega): q|_T\in \mathbb P_{k}(T), T\in \mathcal{T}_h\}.
\end{align*}
显然有$V_{h}\subset H_0(\div,\Omega)$但$V_{h}\not\subseteq H^1(\Omega;\mathbb R^d)$, 且有弱连续性
\begin{equation}\label{stokesdivfreencfmweakcontinuity}
\int_F[\boldsymbol{v}]\dd s=0,\quad F\in\mathcal F_h.
\end{equation}

\begin{lemma}
Divergence-free非协调元满足$\div V_h=P_h$和离散inf-sup 条件\eqref{StokesNcfmu4}.
\end{lemma}
\begin{prf}
令$\Pi_{h}:H_0^{1}(\Omega; \mathbb{R}^{d})\to  V_{h}$为基于自由度\eqref{divfreencfmDoFs}的节点插值算子,则利用分部积分和尺度论证技巧可推得$\Pi_{h}$是Fortin算子,且有$\div V_h=P_h$.
\end{prf}

Divergence-free非协调元方法是divergence-free的.

\begin{lemma}
设$(\boldsymbol{u}, p)\in H_0^{1}(\Omega; \mathbb{R}^{d})\times L_{0}^{2}(\Omega)$是Stokes方程\eqref{eq-Stokes2}的解, $(\boldsymbol{u}_h, p_h)\in V_h\times P_h$是divergence-free非协调元方法(\ref{StokesNcfm1})-(\ref{StokesNcfm2})的解. 假设$\boldsymbol{u}\in H^{2}(\Omega; \mathbb{R}^{d})$,$p\in H^{1}(\Omega)$,则有误差估计
\begin{equation}\label{stokesdivfreencfmerroruh}
\|\nabla_h(\boldsymbol{u}-\boldsymbol{u}_h)\|_0\lesssim h\|\boldsymbol{u}\|_2,
\end{equation}
\begin{equation}\label{stokesdivfreencfmerrorph}
\|p-p_h\|_0\lesssim h(\|\boldsymbol{u}\|_2+h^{k}\|p\|_1).
\end{equation}
\end{lemma}
\begin{prf}
由定理~\ref{thm:stokesncfmerrorestimate}和$\Pi_h$的插值误差估计,我们只要估计相容性误差即可.

由分部积分和弱连续性可推得,
\begin{align*}
E_h(\boldsymbol{u},p;\boldsymbol{v})&=(\nabla\boldsymbol{u}, \nabla_h\boldsymbol{v}) + (\div\boldsymbol{v}, p) -(\boldsymbol{f},\boldsymbol{v})=(\nabla\boldsymbol{u}, \nabla_h\boldsymbol{v}) + (\div\boldsymbol{v}, p) + (\Delta\boldsymbol{u}+\nabla p,\boldsymbol{v}) \\
&=\sum_{T\in\mathcal T_h}(\partial_n\boldsymbol{u}, \boldsymbol{v})_{\partial T}=\sum_{F\in\mathcal F_h}(\partial_n\boldsymbol{u}, [\boldsymbol{v}])_F \\
&=\sum_{F\in\mathcal F_h}(\partial_n\boldsymbol{u}-Q_{0,F}(\partial_n\boldsymbol{u}), [\boldsymbol{v}-Q_{0,F}\boldsymbol{v}])_F.
\end{align*}
再利用投影算子$Q_{0,F}$的误差估计可得
\begin{equation*}
E_h(\boldsymbol{u},p;\boldsymbol{v})\lesssim h\|\boldsymbol{u}\|_2.
\end{equation*}
得证.
\end{prf}

当$d=2,3$时,我们得到了文献\cite{XieXuXue2008}中的两维、三维低阶divergence-free非协调元.

只要有弱连续性\eqref{stokesdivfreencfmweakcontinuity}, 即可得到误差估计\eqref{stokesdivfreencfmerroruh}和\eqref{stokesdivfreencfmerrorph}. 因此,可以减少自由度\eqref{divfreencfmDoFs2}中的矩量.
令反对称矩阵线性函数约化空间
\begin{equation*}
\mathbb P_1^r(T;\mathbb K)=\mathbb P_1(T;\mathbb K)/\oplus_{i=0}^d(d\lambda_i-1)\mathbb K_i,
\end{equation*}
其中$\mathbb K_i$是面$F_i$上全部切向反对称矩阵所形成的线性空间.

\begin{lemma}
$\dim\mathbb P_1^r(T;\mathbb K)=d^2-1$.
\end{lemma}
\begin{prf}
由于
\begin{equation*}
\sum_{i=0}^d\dim\mathbb K_i=\frac{1}{2}(d-2)(d^2-1),
\end{equation*}
故只需证明$\{(d\lambda_i-1)\mathbb K_i, i=0,1,\ldots, d\}$线性无关. 设
\begin{equation*}
\sum_{i=0}^d(d\lambda_i-1)\boldsymbol{\tau}_i=0,\quad \boldsymbol{\tau}_i\in\mathbb K_i.
\end{equation*}
将重心的坐标代入,即$\lambda_0=\lambda_1=\ldots=\lambda_d=\frac{1}{d+1}$,可推得 
\begin{equation*}
\sum_{i=0}^d\boldsymbol{\tau}_i=0,\quad \sum_{i=0}^d\lambda_i\boldsymbol{\tau}_i=0.
\end{equation*}
由$\lambda_0,\lambda_1,\ldots,\lambda_d$的线性无关性可得$\boldsymbol{\tau}_i=0$, $i=0,1,\ldots, d$.
\end{prf}

% \begin{equation*}
% \mathbb P_1^r(T;\mathbb K)=\{\boldsymbol{\tau}\in \mathbb P_1(T;\mathbb K): (\div(b_T\boldsymbol{\tau}), \boldsymbol{q})_F=0, \boldsymbol{q}\in{\rm RM}(F)/\mathbb P_0(F;\mathbb R^{d-1}), F\in\Delta_{d-1}(T)\},
% \end{equation*}
% 则有
% \begin{equation*}
% \dim\mathbb P_1^r(T;\mathbb K)\geq\frac{1}{2}d(d^2-1)-\frac{1}{2}(d-2)(d^2-1)=d^2-1.
% \end{equation*}
约化后的形函数空间取为$V^r(T):=V^{\div}(T)\oplus\div(b_T\mathbb P_1^r(T;\mathbb K))$.
自由度为
\begin{subequations}\label{reducedivfreencfmDoFs}
\begin{align}
\label{reducedivfreencfmDoFs1}
(\boldsymbol{v}\cdot\boldsymbol{n}, q)_F, &\quad q\in\mathbb P_{1}(F), F\in\Delta_{d-1}(T),\\
\label{reducedivfreencfmDoFs2}
(\boldsymbol{n}\times\boldsymbol{v}\times\boldsymbol{n}, \boldsymbol{q})_F, &\quad \boldsymbol{q}\in\mathbb P_0(F;\mathbb R^{d-1}), F\in\Delta_{d-1}(T),\\
\label{reducedivfreencfmDoFs3}
(\boldsymbol{v}, \boldsymbol{q})_T, &\quad \boldsymbol{q}\in\mathbb P_{0}(T;\mathbb R^d)\quad\textrm{ if } k=1.
\end{align}
\end{subequations}

\begin{lemma}
形函数空间$V^r(T)=V^{\div}(T)\oplus\div(b_T\mathbb P_1^r(T;\mathbb K))$由自由度\eqref{reducedivfreencfmDoFs}所唯一确定.
\end{lemma}
\begin{prf}
形函数空间$V^r(T)$的维数为
\begin{equation*}
\dim V^r(T)= d(d+1)+dk +d^2-1,
\end{equation*}
等于自由度\eqref{divfreencfmDoFs}的个数为$d(d+1) +d^2-1+dk$.

设$\boldsymbol{v}\in V^r(T)$满足\eqref{reducedivfreencfmDoFs}中所有自由度为零. 
类似于引理~\ref{lem:divfreencfmunisol}的证明可得$\boldsymbol{v}\in \div(b_T\mathbb P_1^r(T;\mathbb K))$. 


设$\boldsymbol{v}=\sum_{i=0}^d\div(b_T\lambda_i\boldsymbol{\tau}_i)$, 其中$\boldsymbol{\tau}_0, \boldsymbol{\tau}_1,\ldots, \boldsymbol{\tau}_d\in\mathbb K$. 由自由度\eqref{reducedivfreencfmDoFs2}可得
\begin{equation*}
\sum_{i=0}^d\int_{F_j}b_{F_j}\lambda_{i}\boldsymbol{\tau}_{i}\nabla\lambda_j\dd s=\int_{F_j}\boldsymbol{n}\times \boldsymbol{v} \times\boldsymbol{n}\dd s=0,\quad j=0,1,\ldots,d.
\end{equation*}
由此可得
% \begin{equation*}
% \sum_{i\neq j}\boldsymbol{\tau}_{i}\boldsymbol{n}_{j}=0,\quad j=0,1,\ldots,d.
% \end{equation*}
% 所以
\begin{equation*}
\sum_{i=0}^d\boldsymbol{\tau}_{i}\boldsymbol{n}_{j}=\boldsymbol{\tau}_{j}\boldsymbol{n}_{j},\quad j=0,1,\ldots,d.
\end{equation*}
由$\boldsymbol{\tau}_{i}$的反对称性可得
\begin{equation*}
\boldsymbol{\tau}_{j}=\boldsymbol{\sigma}_{j}+\sum_{i=0}^d\boldsymbol{\tau}_{i}, \quad j=0,1,\ldots,d,
\end{equation*}
其中$\boldsymbol{\sigma}_{j}\in\mathbb K_j$.
对上式关于$j$进行求和有$d\sum_{i=0}^d\boldsymbol{\tau}_{i}=-\sum_{i=0}^d\boldsymbol{\sigma}_{i}$, 故
\begin{equation*}
\boldsymbol{\tau}_{j}=\boldsymbol{\sigma}_{j}-\frac{1}{d}\sum_{i=0}^d\boldsymbol{\sigma}_{i}, \quad j=0,1,\ldots,d.
\end{equation*}
从而
\begin{equation*}
\sum_{i=0}^d\lambda_i\boldsymbol{\tau}_i=\sum_{i=0}^d\lambda_i\boldsymbol{\sigma}_i-\frac{1}{d}\sum_{i=0}^d\boldsymbol{\sigma}_{i}=\frac{1}{d}\sum_{i=0}^d(d\lambda_i-1)\boldsymbol{\sigma}_i\in \oplus_{i=0}^d(d\lambda_i-1)\mathbb K_i.
\end{equation*}
另一方面,$\sum_{i=0}^d\lambda_i\boldsymbol{\tau}_i\in\mathbb P_1^r(T;\mathbb K)=\mathbb P_1(T;\mathbb K)/\oplus_{i=0}^d(d\lambda_i-1)\mathbb K_i$. 所以,$\sum_{i=0}^d\lambda_i\boldsymbol{\tau}_i=0$, 从而$\boldsymbol{v}=0$.
\end{prf}


约化后的divergence-free非协调元方法具有与\eqref{stokesdivfreencfmerroruh}和\eqref{stokesdivfreencfmerrorph}一样的误差估计.

三维情形$d=3$,约化向量有限元形函数空间为$V^{\div}(T)\oplus\curl(b_T\mathbb P_1^r(T;\mathbb R^3))$, 其中$\mathbb P_1^r(T;\mathbb R^3):=\mathbb P_1(T;\mathbb R^3)/{\rm span}\{(3\lambda_i-1)\nabla\lambda_i, i=0,1,2,3\}$, 与文献\cite{XieXuXue2008}中的相同.

% 在三维情形$d=3$,文献\cite{XieXuXue2008}中的约化向量有限元形函数空间为$V^{\div}(T)\oplus\curl(b_T\mathbb P_1^r(T;\mathbb R^3))$, 其中$\mathbb P_1^r(T;\mathbb R^3):=\mathbb P_1(T;\mathbb R^3)/{\rm span}\{(3\lambda_i-1)\nabla\lambda_i, i=0,1,2,3\}$.

\section{质量守恒的超收敛混合有限元方法}

本节提出一类超收敛且无散度(质量守恒)的有限元方法. 速度和压力分别采用 $H(\mathrm{div})$ 协调向量元和分片间断多项式进行离散化. 离散形式引入了一种弱偏梯度算子,该算子由切向-法向连续的无迹张量有限元构建. 
该格式具有内蕴稳定性,无需额外稳定化处理. 



\subsection{有限元空间}

\subsubsection{速度场的有限元空间}

我们采用 $H(\div)$ 协调有限元对速度场 $\boldsymbol{u}$ 进行离散化. 
对于单形单元 $T$ 和整数 $k \ge 0$、$\ell \ge -1$,局部形函数空间定义为
\begin{equation*}
\mathbb V_{k,\ell}^{\div}(T)
:= \mathbb P_{k}(T;\mathbb R^d) + \mathbb H_{\ell}(T)\boldsymbol{x}, 
\quad \ell = k \text{ 或 }\, k - 1.
\end{equation*}
当 $\ell \ge 0$ 时,自由度(DoFs)定义为(参见~\cite[Section~3]{ChenHuang2022})
\begin{subequations}\label{eq:HdivDoF}
\begin{align}
(\boldsymbol v\cdot\boldsymbol n, q)_F, & \quad q\in\mathbb P_{k}(F), \quad F\in\mathcal F(T), \label{Hdivfemdof1}\\
(\boldsymbol v, \boldsymbol q)_T, & \quad \boldsymbol q\in \grad\mathbb P_{\ell}(T)\oplus
\{\boldsymbol q\in \mathbb P_{k-1}(T;\mathbb R^d): \boldsymbol q\cdot\boldsymbol x=0\}. \label{Hdivfemdof2}
\end{align}
\end{subequations}
当 $\ell = k$ 时,该空间对应 Raviart--Thomas (RT) 元~\cite{RaviartThomas1977,Nedelec1980},其内部矩 \eqref{Hdivfemdof2} 张成 $\mathbb{P}_{k-1}(T;\mathbb{R}^d)$;
当 $\ell = k-1$ 时,则对应 Brezzi--Douglas--Marini (BDM) 元~\cite{BrezziDouglasMarini1985,Nedelec1986},其内部矩形成严格子空间. 


对应的全局有限元空间定义为
\begin{align*}
\mathbb V_{k,\ell}^{\div}(\mathcal T_h)
&:= \{\boldsymbol v_h\in H(\div,\Omega): 
\boldsymbol v_h|_T \in \mathbb V_{k,\ell}^{\div}(T)
\text{ 对所有 }\, T\in\mathcal T_h\},\\
\mathring{\mathbb V}_{k,\ell}^{\div}(\mathcal T_h)
&:= \mathbb V_{k,\ell}^{\div}(\mathcal T_h)\cap H_0(\div,\Omega).
\end{align*}
为简便起见,后文中将省略 $\mathcal T_h$. 

有限元空间 $\mathring{\mathbb V}_{k,\ell}^{\div}$ 可用于离散化 Stokes 方程中的散度算子,但由于缺少切向连续性,不适合用于拉普拉斯项的离散化. 为了施加切向连续性,我们面集合 $\mathcal{F}_h$ 上的有限元空间用于离散速度场在面集合 $\mathcal{F}_h$ 上的切向分量:
\begin{equation*}
\Lambda_k = \mathbb P_k(\mathcal F_h; \mathbb R^{d-1}), \quad  
\mathring{\Lambda}_k = \mathbb P_k(\mathring{\mathcal F}_h; \mathbb R^{d-1}) 
\end{equation*}

对于每个单元 $T \in \mathcal{T}_h$, 记 $I_{(k,\ell),T}^{\rm div}: H^1(T;\mathbb{R}^d) \to \mathbb V_{k,\ell}^{\div}(T)$  
为由~\eqref{eq:HdivDoF} 中给定自由度定义的局部插值算子. 
相应的全局插值算子
 $I_{k,\ell}^{\div}: H^1(\mathcal{T}_h;\mathbb{R}^d)\to L^2(\Omega;\mathbb R^d)$ 定义为
\[
(I_{k,\ell}^{\div}\boldsymbol{v})|_T := I_{(k,\ell),T}^{\rm div}(\boldsymbol{v}|_T), 
\quad \forall\, T \in \mathcal{T}_h, \ \boldsymbol{v} \in H^1(\mathcal{T}_h; \mathbb{R}^d).
\] 
由此,对于任意 $\boldsymbol{v}\in H^1(\Omega;\mathbb R^d)$,有
$I_{k,\ell}^{\div}\boldsymbol{v}\in \mathbb V_{k,\ell}^{\div}$. 

对于 $T\in\mathcal{T}_h$ 以及 $\boldsymbol{v}\in H^m(T;\mathbb{R}^d)$、$1\le m\le k+1$,
该插值算子满足以下误差估计:
\begin{equation}\label{eq:Ihdivprop2}
\|\boldsymbol{v}-I_{(k,\ell),T}^{\rm div}\boldsymbol{v}\|_{T}
+ h_T|\boldsymbol{v}-I_{(k,\ell),T}^{\rm div}\boldsymbol{v}|_{1,T}
\lesssim h_T^m|\boldsymbol{v}|_{m,T}.
\end{equation}

我们有如下交换性质(参见~\cite{ArnoldFalkWinther2006}):
\begin{equation}\label{eq:Ihdivprop1}
\div\!\big(I_{k,\ell}^{\div}\boldsymbol{v}\big)
= Q_{\ell}\big(\div\boldsymbol{v}\big),
\qquad \forall\,\boldsymbol{v}\in H^1(\Omega;\mathbb{R}^d),
\end{equation}
这意味着对于任意 $\boldsymbol{v}\in H^1(\Omega;\mathbb{R}^d)\cap\ker(\div)$,有 $\div\!\big(I_{k,\ell}^{\div}\boldsymbol{v}\big)=0$.  

与 ${\rm BDM}_k$ 空间相比,${\rm RT}_k$ 空间中额外的内部矩仅用于丰富散度算子的值域. 
因此,对于任意 $\boldsymbol{v}\in H^1(\Omega;\mathbb{R}^d)\cap\ker(\div)$ 且 $k\ge 1$,有
\begin{equation*}
I_{k,k}^{\div}\boldsymbol{v}
= I_{k,k-1}^{\div}\boldsymbol{v}
\in \mathbb V_{k,k-1}^{\div}\cap \ker(\div).
\end{equation*}

\begin{lemma}
成立如下范数等价关系:
\begin{equation}
\label{eq:discretdevgradnormequiv11}
\begin{aligned}
|\boldsymbol{v}_h|_{1,h}^2&\eqsim \|\grad_h\boldsymbol{v}_h\|^2 + \sum_{F \in \mathcal{F}_h} h_F^{-1}\|[\![\boldsymbol{v}_h]\!]\|_{F}^2 \\
&\eqsim \|\dev\grad_h\boldsymbol{v}_h\|^2  + \sum_{F \in \mathcal{F}_h} h_F^{-1}\|Q_{\mathrm{RT},F}[\![\Pi_F\boldsymbol{v}_h]\!]\|_{F}^2
\end{aligned}
\end{equation}
对所有 $\boldsymbol{v}_h\in \mathring{\mathbb{V}}_{k,\ell}^{\div}$ 成立,
其中 $Q_{\mathrm{RT},F}: L^2(F;\mathbb R^{d-1})\to \mathbb{P}_0(F;\mathbb{R}^{d-1})\oplus (\Pi_F\boldsymbol{x})\mathbb{P}_0(F)$ 为 $L^2$ 正交投影算子,以及
\begin{equation*}
|\boldsymbol{v}|_{1,h}^2 :=  \sum_{T\in\mathcal{T}_h}\|\dev\grad\boldsymbol{v}\|^2_T+\sum_{F \in \mathcal{F}_h}  h_F^{-1}\|[\![\boldsymbol{v}]\!]\|_{F}^2.
\end{equation*}
这意味着 $|\cdot|_{1,h}$ 是空间 $\mathring{\mathbb{V}}_{k,\ell}^{\div}$ 上的范数. 
\end{lemma}
\begin{proof}
范数等价关系 \eqref{eq:discretdevgradnormequiv11} 可直接由 \eqref{eq:discretdevgradnormequiv} 推得. 
\end{proof}


\subsubsection{压力场的有限元空间}
在离散格式中,我们采用分片多项式空间 $\mathbb{P}_{\ell}(\mathcal{T}_h)$ 来近似压力 $p$. 
由交换性质 \eqref{eq:Ihdivprop1} 可得
\begin{equation*}
\div\mathring{\mathbb{V}}_{k,\ell}^{\div} = \mathbb{P}_{\ell}(\mathcal{T}_h)/\mathbb R.
\end{equation*}



\subsubsection{伪应力场的有限元空间}
对于整数 $k \ge 0$,局部形函数空间为 $\mathbb P_{k}(T;\mathbb T)$,其自由度定义为
\begin{subequations}\label{eq:curldivdof}
\begin{align}\label{eq:curdivfemdof1}
\int_{F} \boldsymbol{t}_i^{\intercal}\boldsymbol{\tau}\boldsymbol{n}\, q \, \mathrm{d}S, 
& \quad q \in \mathbb P_{k}(F), \ i = 1,2,\ldots, d-1,\ F\in\mathcal F(T), \\
\label{eq:curdivfemdof2}
\int_{T} \boldsymbol{\tau}: \boldsymbol{q} \, \mathrm{d}x, 
& \quad \boldsymbol{q} \in \mathbb{P}_{k-1}(T; \mathbb{T}).
\end{align}
\end{subequations}
此无迹张量有限元的唯一可解性可参见文献~\cite{ChenHuangZhang2025,GopalakrishnanLedererSchoeberl2020}. 
对应的全局有限元空间为
\begin{align*}
\Sigma_{k}^{\rm tn}(\mathcal T_h)
:= \{\boldsymbol{\tau}_h \in \mathbb P_{k}(\mathcal{T}_h;\mathbb T) :
\text{ 所有 \eqref{eq:curldivdof} 中的自由度都是单值的}\}.
\end{align*}
空间 $\Sigma_{k}^{\rm tn}$ 中的函数在切向-法向方向上连续,即对所有 $F \in \mathring{\mathcal F}_h$,有
\[
[\Pi_F\boldsymbol{\sigma}\boldsymbol{n}]_F = 0.
\]


对于每个单元 $T \in \mathcal{T}_h$,记
$I_T^{\rm tn}: H^1(T;\mathbb{T}) \to \mathbb{P}_{k}(T; \mathbb{T})$
为由自由度~\eqref{eq:curldivdof} 定义的局部插值算子. 
对应的全局插值算子
$I_k^{\rm tn}: H^1(\mathcal{T}_h;\mathbb{T}) \to \mathbb P_{k}(\mathcal{T}_h;\mathbb T)$
定义为
\[
(I_k^{\rm tn}\boldsymbol{\tau})|_T := I_T^{\rm tn}(\boldsymbol{\tau}|_T), 
\quad \forall\, T \in \mathcal{T}_h, \ \boldsymbol{\tau} \in H^1(\mathcal{T}_h; \mathbb{T}).
\] 
于是,对于任意 $\boldsymbol{\tau}\in H^1(\Omega;\mathbb T)$,有 $I_k^{\rm tn}\boldsymbol{\tau}\in \Sigma_{k}^{\rm tn}$. 

对于任意 $T\in\mathcal{T}_h$ 以及 $\boldsymbol{\tau}\in H^m(T;\mathbb{T})$,$1\le m\le k+1$,有标准插值误差估计:
\begin{equation}\label{eq:Ihtnprop2}
\|\boldsymbol{\tau}-I_T^{\rm tn}\boldsymbol{\tau}\|_{T}
+ h_T|\boldsymbol{\tau}-I_T^{\rm tn}\boldsymbol{\tau}|_{1,T}
\lesssim h_T^m|\boldsymbol{\tau}|_{m,T}.
\end{equation}

% We now prove the following important commutative property.
% \begin{lemma}\label{lm:interpolant}
% For $\bs \tau \in H^1(\Omega;\mathbb T)$, we have 
% \begin{equation}\label{eq:divinterpolant}
% (\div \bs \tau, \bs v_h) = (\div_w I_k^{\rm tn} \bs \tau, \bs v_h)_{0,h}, \quad \forall\,\bs v_h\in \mathring{\mathbb V}_{k,k-1}^{\div}.
% \end{equation}
% \end{lemma}
% \begin{proof}
% Notice that $\dev\grad \bs v_h\in \Sigma_{k-1}^{-1}(\mathbb T)$ and $[\Pi_F\bs v_h]\in \mathbb P_{k}(F;\mathbb R^{d-1})$. So
% \begin{align*}
% (\div_w I_k^{\rm tn} \bs \tau, \bs v_h)_{0,h} &= - \sum_{T\in\mathcal{T}_h}(I_k^{\rm tn} \bs \tau, \dev\grad\boldsymbol{v}_h)_T + \sum_{F\in\mathcal{F}_h}(\Pi_FI_k^{\rm tn} \bs \tau \boldsymbol{n}, [\Pi_F\boldsymbol{v}_h])_F\\
% &= - \sum_{T\in\mathcal{T}_h}(\bs \tau, \dev\grad\boldsymbol{v}_h)_T + \sum_{F\in\mathcal{F}_h}(\Pi_F\bs \tau \boldsymbol{n}, [\Pi_F\boldsymbol{v}_h])_F\\
% & = (\div \bs \tau, \bs v_h).
% \end{align*}
% \end{proof}

% 弱散度算子 $\div_{w}$ 定义了 $\Sigma_k^{\rm tn} \times \mathring{\mathbb V}_{k,\ell}^{\div}$ 上的双线性形式:
% \begin{equation*}
% \begin{aligned}
% (\div_{w}\boldsymbol{\sigma}, \boldsymbol{v})_{0,h}
% := (\div_{w}\boldsymbol{\sigma}, E\boldsymbol{v})_{0,h}
% &= \sum_{T\in\mathcal{T}_h}(\div\boldsymbol{\sigma}, \boldsymbol{v})_T
% - \sum_{F\in\mathring{\mathcal{F}}_h}([\boldsymbol{n}^{\intercal}\boldsymbol{\sigma}\boldsymbol{n}], \boldsymbol{n}\cdot\boldsymbol{v})_F\\
% &= - \sum_{T\in\mathcal{T}_h}(\boldsymbol{\sigma}, \dev\grad\boldsymbol{v})_T
% + \sum_{F\in\mathcal{F}_h}(\Pi_F\boldsymbol{\sigma}\boldsymbol{n}, [\Pi_F\boldsymbol{v}])_F.
% \end{aligned}
% \end{equation*}

% \begin{remark}\rm
% The bilinear form $(\div_{w}\cdot, \cdot)_{0,h}$ defines a mapping, still denoted by $\div_w: \Sigma_k^{\rm tn} \to (\mathring{\mathbb V}_{k,\ell}^{\div})'$. 
% With the inner product $(\cdot, \cdot)_{0,h}$, we identify $\mathring{M}_{k-1,k}^{-1}(\mathbb R^d)$ with its dual.  
% Through the embedding $E$, we further identify $\mathring{\mathbb V}_{k,\ell}^{\div} \cong E(\mathring{\mathbb V}_{k,\ell}^{\div})$ with its dual,  
% and hence can write the mapping as
% \[
% \div_{w}: \Sigma_k^{\rm tn} \to \mathring{\mathbb V}_{k,\ell}^{\div}.
% \]
% \end{remark}

\subsection{弱偏梯度算子}

为给出混合有限元方法,引入弱偏梯度算子
$$
\dev \grad_w: H^1(\mathcal{T}_h;\mathbb{R}^d)\times L^2(\mathcal{F}_h;\mathbb R^{d-1}) \to \mathbb{P}_{k}(\mathcal{T}_h; \mathbb{T}),
$$ 
定义如下: 对于任意 $(\boldsymbol{v}, \boldsymbol{\mu})\in H^1(\mathcal{T}_h;\mathbb{R}^d)\times L^2(\mathcal{F}_h;\mathbb R^{d-1})$, $\dev \grad_w(\boldsymbol{v}, \boldsymbol{\mu})\in \mathbb{P}_{k}(\mathcal{T}_h; \mathbb{T})$ 在每个单元 $T\in\mathcal{T}_h$ 上由下式确定:
\begin{equation*}
(\dev \grad_w(\boldsymbol{v}, \boldsymbol{\mu}), \boldsymbol{\tau})_T=-(\boldsymbol{v},\div\boldsymbol{\tau})_T+(\boldsymbol{n}\cdot\boldsymbol{v},\boldsymbol{n}^{\intercal}\boldsymbol{\tau}\boldsymbol{n})_{\partial T}+(\boldsymbol{\mu}, \Pi_F\boldsymbol{\tau}\boldsymbol{n})_{\partial T},\quad \forall\,\boldsymbol{\tau}\in \mathbb{P}_{k}(T;\mathbb{T}).
\end{equation*}
应用分部积分可得
\begin{equation*}
(\dev \grad_w(\boldsymbol{v}, \boldsymbol{\mu}), \boldsymbol{\tau})_T=(\dev \grad\boldsymbol{v},\boldsymbol{\tau})_T - (\Pi_F\boldsymbol{v}-\boldsymbol{\mu}, \Pi_F\boldsymbol{\tau}\boldsymbol{n})_{\partial T},\quad \forall\,\boldsymbol{\tau}\in \mathbb{P}_{k}(T;\mathbb{T}).
\end{equation*}

% 我们将证明如下交换性质:
% \begin{equation}\label{eq:Qkgrad}
% Q_{k}^{\rm tn}\dev \grad = \dev\grad_w I_{k,k}^{\div},
% \end{equation}
% 其中 $Q_k^{\rm tn}: L^2(\Omega; \mathbb T)\to \Sigma_{k}^{\rm tn}$ 表示 $L^2$ 投影算子. 


% \begin{lemma}\label{lm:devinterpolant}
% 对于任意 $\boldsymbol{u}\in H_0^1(\Omega; \mathbb R^d)$,有
% \begin{equation}\label{eq:devinterpolant}
% (\dev \grad \boldsymbol{u}, \boldsymbol{\tau}_h) 
% =  (\dev\grad_w I_{k,k}^{\div} \boldsymbol{u}, \boldsymbol{\tau}_h), 
% \quad \forall\,\boldsymbol{\tau}_h\in \Sigma_k^{\rm tn}.
% \end{equation}
% \end{lemma}
% \begin{proof}
% 由于 $\div(\bs \tau_h|_T)\in \mathbb P_{k-1}(T;\mathbb R^d)$,且 $[\bs n^{\intercal}\bs \tau_h\bs n]|_F\in \mathbb P_k(F)$,我们有
% \begin{align*}
% (\dev \grad \bs u, \bs \tau_h) &= -\sum_{T\in\mathcal{T}_h} (\bs u, \div \bs \tau_h)_T + \sum_{F\in\mathring{\mathcal{F}}_h}(\bs u\cdot \bs n, [\bs n^{\intercal}\bs \tau_h\bs n])_F,\\
%  &= -\sum_{T\in\mathcal{T}_h} (I_{k,k}^{\div} \bs u, \div \bs \tau_h)_T + \sum_{F\in\mathring{\mathcal{F}}_h}(I_{k,k}^{\div} \bs u\cdot \bs n, [\bs n^{\intercal}\bs \tau_h\bs n])_F \\
%  &= (\dev\grad_w I_{k,k}^{\div} \bs u, \bs \tau_h).
% \end{align*}
% \end{proof}

\begin{lemma}\label{lm:weakdevgrad}
对于 $\boldsymbol{v}\in H^1(\Omega;\mathbb R^d)$ 和 $\boldsymbol{\mu}\in L^2(\mathcal{F}_h;\mathbb R^{d-1})$,有
\begin{equation}\label{eq:weakdevgradcommu}
\dev \grad_w(I_{k,k}^{\div}\boldsymbol{v},\, Q_{k,\mathcal{F}_h}\boldsymbol{\mu}) 
= \dev \grad_w(\boldsymbol{v},\, \boldsymbol{\mu}),
\end{equation}
其中 $Q_{k,\mathcal{F}_h}$ 是从 $L^2(\mathcal{F}_h;\mathbb R^{d-1})$ 到 $\Lambda_k$ 的 $L^2$ 正交投影算子. 
\end{lemma}
\begin{proof}
根据算子 $\dev \grad_w$ 的定义,对于任意 $\boldsymbol{\tau}\in \mathbb{P}_{k}(T;\mathbb{T})$ 和 $T\in\mathcal{T}_h$,有
\begin{align*}
&\quad\; (\dev \grad_w(I_{k,k}^{\div}\boldsymbol{v}, Q_{k,\mathcal{F}_h}\boldsymbol{\mu}), \boldsymbol{\tau})_T \\
&=-(I_{k,k}^{\div}\boldsymbol{v},\div\boldsymbol{\tau})_T+(\boldsymbol{n}\cdot(I_{k,k}^{\div}\boldsymbol{v}),\boldsymbol{n}^{\intercal}\boldsymbol{\tau}\boldsymbol{n})_{\partial T}+(\boldsymbol{\mu}, \Pi_F\boldsymbol{\tau}\boldsymbol{n})_{\partial T}.
\end{align*}
由插值算子 $I_{k,k}^{\div}$ 的定义,即可得到 \eqref{eq:weakdevgradcommu}. 
\end{proof}


为保证上述等式成立,$\mathrm{RT}_k$ 空间是必要的,因为其内部矩(参见~\eqref{Hdivfemdof2})能够完全覆盖 $\div \boldsymbol{\tau}_h$ 的多项式空间 $\mathbb P_{k-1}(T;\mathbb R^d)$. 
然而,对于 $\mathrm{BDM}_k$ 空间 $\mathbb V_{k,k-1}^{\div}$,其内部矩条件不足以满足该要求,
因此等式~\eqref{eq:weakdevgradcommu} 对 $I_{k,k-1}^{\div}$ 不再成立. 


\subsection{混合有限元方法}

% 为此,我们定义**弱偏应变梯度算子**
% $$
% \dev \grad_w: H^1(\mathcal{T}_h;\mathbb{R}^d)\times L^2(\mathcal{F}*h;\mathbb R^{d-1}) \to \mathbb{P}*{k}(\mathcal{T}_h; \mathbb{T})
% $$
% 其定义如下:对于任意 $(\boldsymbol{v}, \boldsymbol{\mu})\in H^1(\mathcal{T}_h;\mathbb{R}^d)\times L^2(\mathcal{F}_h;\mathbb R^{d-1})$,令 $(\dev \grad)*w(\boldsymbol{v}, \boldsymbol{\mu})\in \mathbb{P}*{k}(\mathcal{T}_h; \mathbb{T})$ 在每个单元 $T\in\mathcal{T}_h$ 上由下式确定:
% \begin{equation*}
% (\dev \grad_w(\boldsymbol{v}, \boldsymbol{\mu}), \boldsymbol{\tau})_T
% = -(\boldsymbol{v},\div\boldsymbol{\tau})_T

% * (\boldsymbol{n}\cdot\boldsymbol{v}, \boldsymbol{n}^{\intercal}\boldsymbol{\tau}\boldsymbol{n})_{\partial T}
% * (\boldsymbol{\mu}, \Pi_F\boldsymbol{\tau}\boldsymbol{n})*{\partial T},
%   \end{equation*}
%   其中 $\boldsymbol{\tau}\in \mathbb{P}*{k}(T;\mathbb{T})$. 



借助算子 $\dev \grad_w$, 我们为 Stokes 方程~\eqref{eq-Stokes2} 建立如下混合有限元格式:  
找 $\boldsymbol{u}_h\in \mathring{\mathbb{V}}_{k,\ell}^{\div}$, 
$\boldsymbol{\lambda}_h\in\mathring{\Lambda}_k$, 和 
$p_h\in\mathbb{P}_{\ell}(\mathcal{T}_h)/\mathbb R$, 
其中 $k \ge 0$ and $\ell\in\{k,\,k-1\}$, 使得
\begin{subequations}\label{distribustokesfemWG}
\begin{align}
\label{distribustokesfemWG1}
(\dev \grad_w(\boldsymbol{u}_h,\boldsymbol{\lambda}_h),\, 
\dev \grad_w(\boldsymbol{v}_h,\boldsymbol{\mu}_h))
+ (\div\boldsymbol{v}_h,\, p_h)
&= (\boldsymbol{f},\, \boldsymbol{v}_h), \\
\label{distribustokesfemWG2}
(\div\boldsymbol{u}_h,\, q_h) &= 0,
\end{align}
\end{subequations}
对所有
$\boldsymbol{v}_h\in \mathring{\mathbb{V}}_{k,\ell}^{\div}$,
$\boldsymbol{\mu}_h\in\mathring{\Lambda}_k$,以及
$q_h\in\mathbb{P}_{\ell}(\mathcal{T}_h)/\mathbb R$ 均成立.

在最低阶情况下,即 $k=0$ 且 $\ell=-1$,速度使用 $H(\div)$ 协调的分片常数向量空间 $\mathbb V_{(0,-1), h}^{\div}=\mathbb P_0(\mathcal{T}_h;\mathbb R^d)\cap H_0(\div,\Omega)$ 近似,压力则使用零空间. 这意味着离散速度 $\boldsymbol{u}_h$ 自动满足无散度条件. 
因此,混合方法退化为:寻找
$\boldsymbol{u}_h\in \mathring{\mathbb{V}}_{(0,-1),h}^{\div}$ 和 $\boldsymbol{\lambda}_h\in\mathbb P_{0}(\mathring{\mathcal{F}}_h;\mathbb R^{d-1})$,使得
$$
(\dev \grad_w(\boldsymbol{u}_h,\boldsymbol{\lambda}_h), \dev \grad_w(\boldsymbol{v}_h,\boldsymbol{\mu}_h)) = (\boldsymbol{f}, \boldsymbol{v}_h)
$$
对任意 $\boldsymbol{v}_h\in \mathring{\mathbb{V}}_{(0,-1),h}^{\div}$ 和 $\boldsymbol{\mu}_h\in\mathbb P_{0}(\mathring{\mathcal{F}}_h;\mathbb R^{d-1})$ 成立. 


% When $k\geq2$, equation \eqref{distribustokesfemWG2} is equivalent to $\div\boldsymbol{\sigma}_h-\nabla_wu_h\in H(\div,\Omega)$, and
% \begin{equation}\label{distribustokesfemWG20}
% (\div(\div\boldsymbol{\sigma}_h-\nabla_wu_h), v_0)=(f, v_0)\quad \forall~v_0\in\mathbb P_{m-2}(\mathcal{T}_h).
% \end{equation}

根据范数等价 \eqref{eq:discretdevgradnormequiv11},
$$
\interleave(\boldsymbol{v}_h,\boldsymbol{\mu}_h)\interleave_{1,h}:=(\|\dev \grad_h\boldsymbol{v}_h\|^2+\sum_{T\in\mathcal{T}_h}h_T^{-1}\|\Pi_F\boldsymbol{v}_h-\boldsymbol{\mu}_h\|_{\partial T}^2)^{1/2}
$$ 
定义了空间 $\mathring{\mathbb{V}}_{(k,\ell),h}^{\div}\times \mathbb P_{k}(\mathring{\mathcal{F}}_h;\mathbb R^{d-1})$ 的范数. 


\begin{lemma}
对于 $(\boldsymbol{v}_h, \boldsymbol{\mu}_h)\in \mathring{\mathbb{V}}_{k,k-1}^{\div}\times \mathbb P_{k}(\mathring{\mathcal{F}}_h;\mathbb R^{d-1})$,成立范数等价
\begin{equation}\label{eq:weakdevgradnormequiv}
\|\dev \grad_w(\boldsymbol{v}_h,\boldsymbol{\mu}_h)\|\eqsim \interleave(\boldsymbol{v}_h,\boldsymbol{\mu}_h)\interleave_{1,h}.
\end{equation}
\end{lemma}
\begin{proof}
对每个单元应用分部积分可得
\begin{equation}\label{eq:20250107}
(\dev \grad_w(\boldsymbol{v}_h,\boldsymbol{\mu}_h), \boldsymbol{\tau})_T=(\dev\grad\boldsymbol{v}_h, \boldsymbol{\tau})_T+(\boldsymbol{\mu}_h-\Pi_F\boldsymbol{v}_h, \Pi_F\boldsymbol{\tau}\boldsymbol{n})_{\partial T}
\end{equation}
对任意 $\boldsymbol{\tau}\in \mathbb{P}_{k}(T;\mathbb{T})$ 和 $T\in\mathcal{T}_h$ 成立. 
在 \eqref{eq:20250107} 中取 $\boldsymbol{\tau} = \dev \grad_w(\boldsymbol{v}_h,\boldsymbol{\mu}_h)|_T$,利用逆不等式可得
\begin{equation}\label{eq:202501071}
\|\dev \grad_w(\boldsymbol{v}_h,\boldsymbol{\mu}_h)\|_T^2\lesssim \|\dev \grad\boldsymbol{v}_h\|_T^2+h_T^{-1}\|\Pi_F\boldsymbol{v}_h-\boldsymbol{\mu}_h\|_{\partial T}^2.
\end{equation}

另一方面,取 $\boldsymbol{\tau}\in \mathbb{P}_{k}(T;\mathbb{T})$,使其满足
\begin{equation*}
(\Pi_F\boldsymbol{\tau}\boldsymbol{n})|_F = h_T^{-1}(\boldsymbol{\mu}_h-\Pi_F\boldsymbol{v}_h)|_F\quad\forall~F\in\mathcal{F}(T); \quad
Q_{k-1,T}\boldsymbol{\tau}_h  = \dev\grad(\boldsymbol{v}_h|_T).
\end{equation*}
则有
\begin{equation*}
\|\boldsymbol{\tau}\|_{T}^2 \eqsim \|\dev \grad\boldsymbol{v}_h\|_T^2+h_T^{-1}\|\Pi_F\boldsymbol{v}_h-\boldsymbol{\mu}_h\|_{\partial T}^2=(\dev \grad_w(\boldsymbol{v}_h,\boldsymbol{\mu}_h), \boldsymbol{\tau})_T.  
\end{equation*}
由此得到
\begin{equation*}
\|\dev \grad\boldsymbol{v}_h\|_T^2+h_T^{-1}\|\Pi_F\boldsymbol{v}_h-\boldsymbol{\mu}_h\|_{\partial T}^2\lesssim \|\dev \grad_w(\boldsymbol{v}_h,\boldsymbol{\mu}_h)\|_{T}^2.
\end{equation*}
结合 \eqref{eq:202501071},即得范数等价 \eqref{eq:weakdevgradnormequiv}. 
\end{proof}

\begin{theorem}\label{thm:unisolmfem}
混合有限元方法 \eqref{distribustokesfemWG} 是适定的. 
设其数值解为 $(\boldsymbol{u}_h, \boldsymbol{\lambda}_h, p_h)\in \mathring{\mathbb{V}}_{(k,\ell),h}^{\div} \times \mathbb P_{k}(\mathring{\mathcal{F}}_h;\mathbb R^{d-1}) \times \mathbb{P}_{\ell}(\mathcal{T}_h)/\mathbb R$,则有 $\boldsymbol{u}_h\in\mathring{\mathbb V}_{k,k-1}^{\div}\cap\ker(\div)$, 并且 $\boldsymbol{\sigma}_h:= \dev \grad_w(\boldsymbol{u}_h, \boldsymbol{\lambda}_h)$ 是切向-法向连续的,即 $\boldsymbol{\sigma}_h\in\Sigma_{k}^{\rm tn}$.
此外,有限元解 $\boldsymbol{u}_h$、$\boldsymbol{\lambda}_h$ 和 $\boldsymbol{\sigma}_h$ 与整数 $\ell$ 无关. 
\end{theorem}
\begin{proof}
假设 $\boldsymbol{f}=0$,利用齐次问题仅有零解来验证方法的适定性.
由式 \eqref{distribustokesfemWG2} 可得 $\div \boldsymbol{u}_h = 0$,从而 $\boldsymbol{u}_h \in \mathring{\mathbb{V}}_{k,k-1}^{\div} \cap \ker(\div)$.
在式 \eqref{distribustokesfemWG1} 中取 $\boldsymbol{v}_h = \boldsymbol{u}_h$ 和 $\boldsymbol{\mu}_h = \boldsymbol{\lambda}_h$,并在式 \eqref{distribustokesfemWG2} 中取 $q_h = p_h$,两式相减可得
\[
\|\dev \grad_w(\boldsymbol{u}_h,\boldsymbol{\lambda}_h)\| = 0.
\]
由范数等价关系 \eqref{eq:weakdevgradnormequiv} 可知 $\boldsymbol{u}_h = 0$ 且 $\boldsymbol{\lambda}_h = 0$.
此时,由式 \eqref{distribustokesfemWG1} 及等式 $\div \mathring{\mathbb{V}}_{k,\ell}^{\div} = \mathbb{P}_{\ell}(\mathcal{T}_h)/\mathbb{R}$,可进一步推出 $p_h = 0$.


% 当 $\ell = -1$ 时,$\mathbb{P}_{\ell}(\mathcal{T}_h)/\mathbb R = {0}$,此时离散方法 \eqref{distribustokesfemWG} 的适定性可由范数等价 \eqref{eq:weakdevgradnormequiv} 和 Lax-Milgram 引理得到. 
% 现在假设 $\ell \ge 0$. 
% 对于任意 $q_h\in \mathbb{P}_{\ell}(\mathcal{T}_h)/\mathbb R$,由 $\div H_0^1(\Omega;\mathbb R^d)=L_0^2(\Omega)$ 和 \eqref{eq:Ihdivprop1},有
% \begin{equation}\label{eq:qhinfsup}
% \|q_h\| \lesssim \sup_{\boldsymbol{v}\in H_0^1(\Omega;\mathbb{R}^d)} \frac{(\div\boldsymbol{v},q_h)}{\|\boldsymbol{v}\|_1}=\sup_{\boldsymbol{v}\in H_0^1(\Omega;\mathbb{R}^d)} \frac{(\div(I_{k, \ell}^{\div}\boldsymbol{v}),q_h)}{\|\boldsymbol{v}\|_1}.
% \end{equation}
% 选取 $\boldsymbol{\mu}_h\in\mathbb P_{k}(\mathring{\mathcal{F}}_h;\mathbb R^{d-1})$,使得 $\boldsymbol{\mu}_h|_F\in \Pi_F(Q_{k,F}\boldsymbol{v})$ 对所有 $F\in \mathring{\mathcal{F}}_h$ 成立. 由 \eqref{eq:Ihdivprop2} 可得
% \begin{align*}
% \interleave(I_{k, \ell}^{\div}\boldsymbol{v},\boldsymbol{\mu}_h)\interleave_{1,h}^2& =\|\dev \grad_h(I_{k, \ell}^{\div}\boldsymbol{v})\|^2+\sum_{T\in\mathcal{T}_h}h_T^{-1}\|\Pi_F(I_{k, \ell}^{\div}\boldsymbol{v})-\boldsymbol{\mu}_h\|_{\partial T}^2 \\
% &\lesssim \|\boldsymbol{v}\|_1^2+\sum_{T\in\mathcal{T}_h}h_T^{-1}\|I_{k, \ell}^{\div}\boldsymbol{v}-Q_{k,F}\boldsymbol{v}\|_{\partial T}^2\lesssim \|\boldsymbol{v}\|_1^2.
% \end{align*}
% 将上两式结合,可得
% \begin{equation}\label{eq:20250926}
% \|q_h\| \lesssim \sup_{\boldsymbol{v}\in H_0^1(\Omega;\mathbb{R}^d)} \frac{(\div(I_{k, \ell}^{\div}\boldsymbol{v}),q_h)}{\interleave(I_{k, \ell}^{\div}\boldsymbol{v},\boldsymbol{\mu}_h)\interleave_{1,h}} \leq \sup_{\boldsymbol{v}_h\in \mathring{\mathbb{V}}_{(k,\ell),h}^{\div}\atop \boldsymbol{\mu}_h\in\mathbb P_{k}(\mathring{\mathcal{F}}_h;\mathbb R^{d-1})} \frac{(\div\boldsymbol{v}_h,q_h)}{\interleave(\boldsymbol{v}_h,\boldsymbol{\mu}_h)\interleave_{1,h}}.
% \end{equation}
% 由 Babu{\v{s}}ka-Brezzi 理论,结合上述 inf-sup 条件和离散强制性 \eqref{eq:weakdevgradnormequiv},可知离散方法 \eqref{distribustokesfemWG} 是适定的. 

方程 \eqref{distribustokesfemWG2} 表明 $\div\boldsymbol{u}_h = 0$,因此 $\boldsymbol{u}_h \in \mathring{\mathbb V}_{k,k-1}^{\div}\cap \ker(\div)$. 
在 \eqref{distribustokesfemWG1} 中取 $\boldsymbol{v}_h = 0$,得到
$$
(\boldsymbol{\sigma}_h, \dev \grad_w(0,\boldsymbol{\mu}_h)) = 0\qquad\forall~\boldsymbol{\mu}_h
\in\mathbb P_{k}(\mathring{\mathcal{F}}_h;\mathbb R^{d-1}),
$$
即
\begin{equation*}
\sum_{T\in\mathcal{T}_h}(\Pi_F\boldsymbol{\sigma}_h\boldsymbol{n}, \boldsymbol{\mu}_h)_{\partial T} = 0\qquad\forall~\boldsymbol{\mu}_h
\in\mathbb P_{k}(\mathring{\mathcal{F}}_h;\mathbb R^{d-1}),
\end{equation*}
这表明 $\boldsymbol{\sigma}_h$ 是切向-法向连续的. 

假设 $(\boldsymbol{u}_h, \boldsymbol{\lambda}_h, p_h)\in \mathring{\mathbb{V}}_{k,k}^{\div} \times \mathbb P_{k}(\mathring{\mathcal{F}}_h;\mathbb R^{d-1}) \times \mathbb{P}_{\ell}(\mathcal{T}_h)/\mathbb R$ 是 $\ell=k$ 时离散方法 \eqref{distribustokesfemWG} 的解,则
$(\boldsymbol{u}_h, \boldsymbol{\lambda}_h,Q_{k-1,h}p_h)\in \mathring{\mathbb{V}}_{k,k-1}^{\div} \times \mathbb P_{k}(\mathring{\mathcal{F}}_h;\mathbb R^{d-1}) \times \mathbb{P}_{\ell}(\mathcal{T}_h)/\mathbb R$ 是 $\ell=k-1$ 时离散方法 \eqref{distribustokesfemWG} 的解.
\end{proof}


\begin{remark}\rm
文献 \cite{CockburnSayas2014} 中的 HDG 方法对所有变量都使用了 $k$ 阶多项式空间. 实际上,当速度 $\boldsymbol{u}_h$ 使用 $k$ 阶 BDM 元离散时,只使用 $p_h \in \mathbb{P}_{k-1}(\mathcal{T}_h)$ 就足够了. 此外,我们对 $\boldsymbol{\sigma}_h$ 使用了无迹张量空间 $\mathbb{P}_{k}(\mathcal{T}_h; \mathbb{T})$,而 \cite{CockburnSayas2014} 使用的是全张量空间 $\mathbb{P}_{k}(\mathcal{T}_h; \mathbb{M})$. 
\end{remark}

\subsection{误差分析}
下文中,设 $\boldsymbol{u} \in H_0^1(\Omega;\mathbb{R}^d)$ 为 Stokes 方程 \eqref{eq-Stokes2} 的解,并设
$\boldsymbol{\sigma} = \grad\boldsymbol{u} = \dev\grad\boldsymbol{u}$. 
则 \eqref{eq-Stokes2} 中的第一条方程可写为
\begin{equation}\label{stokes1}
-\div\boldsymbol{\sigma}- \nabla p=\boldsymbol{f}  \quad \text { 在 $\Omega$ 中} .
\end{equation}
此外,设 $\boldsymbol{\lambda} \in L^2(\mathcal{F}_h; \mathbb{R}^{d-1})$ 定义为
$$
\boldsymbol{\lambda}|_F = \Pi_F \boldsymbol{u}, \quad \forall\,F \in \mathcal{F}_h.
$$
此记号将在后续分析中使用. 


为分析混合有限元方法 \eqref{distribustokesfemWG},我们首先在空间 $H^{1}(\mathcal{T}_h;\mathbb{T})$ 上引入如下网格依赖范数:
\begin{align*}
\|\boldsymbol{\tau}_h\|_{0,h}^2 &:= \|\boldsymbol{\tau}_h\|^2+ \sum_{F \in \mathcal{F}_h}h_F\|\Pi_F\boldsymbol{\tau}_h \boldsymbol{n}\|_{F}^2.
\end{align*}


\begin{lemma}
设 $\boldsymbol{u} \in H_0^1(\Omega;\mathbb{R}^d)$ 为 Stokes 方程 \eqref{eq-Stokes2} 的解,则有
\begin{equation}\label{eq:weakdevgradulambda}
(\dev \grad)_w(\boldsymbol{u}, \boldsymbol{\lambda}) = Q_{k,h}(\dev \grad\boldsymbol{u}) = Q_{k,h}\boldsymbol{\sigma}.
\end{equation}
\end{lemma}
\begin{proof}
由算子 $(\dev \grad)_w$ 的定义,对于任意 $\boldsymbol{\tau}\in \mathbb{P}_{k}(T;\mathbb{T})$ 及 $T\in\mathcal{T}_h$,有
\begin{equation*}
((\dev \grad)_w(\boldsymbol{u}, \boldsymbol{\lambda}), \boldsymbol{\tau})_T=-(\boldsymbol{u},\div\boldsymbol{\tau})_T+(\boldsymbol{n}\cdot\boldsymbol{u},\boldsymbol{n}^{\intercal}\boldsymbol{\tau}\boldsymbol{n})_{\partial T}+(\Pi_F\boldsymbol{u}, \Pi_F\boldsymbol{\tau}\boldsymbol{n})_{\partial T}.
\end{equation*}
对上式进行分部积分后即可得到式 \eqref{eq:weakdevgradulambda}. 
\end{proof}

\begin{lemma}
对于 $\boldsymbol{v}_h\in \mathbb{V}_{k,k-1}^{\div}\cap\ker(\div)$ 及
$\boldsymbol{\mu}_h\in\mathbb P_{k}(\mathring{\mathcal{F}}_h;\mathbb R^{d-1})$,成立正交性
\begin{equation}\label{eq:Ihtnorth}
(I_h^{\rm tn}\boldsymbol{\sigma}-\boldsymbol{\sigma}_h, (\dev \grad)_w(\boldsymbol{v}_h,\boldsymbol{\mu}_h)) =0.
\end{equation}
\end{lemma}
\begin{proof}
由于对每个内部面 $F\in\mathring{\mathcal{F}}_h$ 都有 $[\![\Pi_F (I_h^{\rm tn} \boldsymbol{\sigma}) \boldsymbol{n}]\!] = 0$,因此
$$
(I_h^{\rm tn}\boldsymbol{\sigma}, (\dev \grad)_w(\boldsymbol{v}_h,\boldsymbol{\mu}_h)) = \sum_{T\in\mathcal{T}_h}\big((I_h^{\rm tn}\boldsymbol{\sigma}, \dev \grad \boldsymbol{v}_h)_T-(\Pi_F(I_h^{\rm tn}\boldsymbol{\sigma})\boldsymbol{n}, \Pi_F\boldsymbol{v}_h)_{\partial T}\big).
$$
由分部积分及算子 $I_h^{\rm tn}$ 的定义可得
\begin{equation*}
(I_h^{\rm tn}\boldsymbol{\sigma}, \dev \grad \boldsymbol{v}_h)_T-(\Pi_F(I_h^{\rm tn}\boldsymbol{\sigma})\boldsymbol{n}, \Pi_F\boldsymbol{v}_h)_{\partial T} = -(\div\boldsymbol{\sigma}, \boldsymbol{v}_h)_T+(\boldsymbol{n}^{\intercal}\boldsymbol{\sigma}\boldsymbol{n}, \boldsymbol{n}\cdot\boldsymbol{v}_h)_{\partial T}.
\end{equation*}
结合以上两式与 \eqref{stokes1},得到
\begin{equation*}
(I_h^{\rm tn}\boldsymbol{\sigma}, (\dev \grad)_w(\boldsymbol{v}_h,\boldsymbol{\mu}_h)) = -(\div\boldsymbol{\sigma}, \boldsymbol{v}_h)=(\boldsymbol{f}+\nabla p, \boldsymbol{v}_h).
\end{equation*}
由 $\div\boldsymbol{v}_h=0$,可推得
\begin{equation}\label{eq:Ihtnprop3}
(I_h^{\rm tn}\boldsymbol{\sigma}, (\dev \grad)_w(\boldsymbol{v}_h,\boldsymbol{\mu}_h)) =(\boldsymbol{f}, \boldsymbol{v}_h).
\end{equation}
另一方面,由 \eqref{distribustokesfemWG1} 知
\begin{equation*}
(\boldsymbol{\sigma}_h, (\dev \grad)_w(\boldsymbol{v}_h,\boldsymbol{\mu}_h)) =(\boldsymbol{f}, \boldsymbol{v}_h).
\end{equation*}
故正交性 \eqref{eq:Ihtnorth} 成立.
\end{proof}

% By the definitions of interpolation operators $I_{k,k}^{\div}$, $I_h^{\rm tn}$ and $Q_{\ell,h}$, we have
% \begin{align}\label{eq:Ihdivprop3}
% b_h(\boldsymbol{\tau}_h,q_h;\boldsymbol{v}-I_{k,k}^{\div}\boldsymbol{v})&=0\quad \forall~\boldsymbol{\tau}_h \in \Sigma_{k}^{\rm tn}, q_h\in\mathbb{P}_{\ell}(\mathcal{T}_h)/\mathbb R,\boldsymbol{v}\in H^1(\Omega;\mathbb{R}^d), \\
% \label{eq:Ihtnprop1}
% b_h(\boldsymbol{\tau}-I_h^{\rm tn}\boldsymbol{\tau},q-Q_{\ell,h}q;\boldsymbol{v}_h)&=0\quad \forall~\boldsymbol{\tau}\in \Sigma^{\rm tn}, q\in L^2(\Omega), \boldsymbol{v}_h\in \mathring{\mathbb V}_{k,k-1}^{\div}.
% \end{align}

\begin{lemma}
有如下估计式成立:
\begin{equation}\label{eq:preestimate4p}
(\div(I_{k, \ell}^{\div}\boldsymbol{v}), p-p_h)\lesssim \|\boldsymbol{\sigma}-\boldsymbol{\sigma}_h\|_{0,h}|\boldsymbol{v}|_{1}\qquad\forall\,\boldsymbol{v}\in H_0^1(\Omega; \mathbb{R}^d).
\end{equation}
\end{lemma}
\begin{proof}
由 \eqref{distribustokesfemWG1}(取 $\boldsymbol{\mu}_h=0$)与 \eqref{stokes1} 可得
\begin{align*}
(\div(I_{k, \ell}^{\div}\boldsymbol{v}), p-p_h) &= (\div(I_{k, \ell}^{\div}\boldsymbol{v}), p) - (\boldsymbol{f}, I_{k, \ell}^{\div}\boldsymbol{v}) + (\boldsymbol{\sigma}_h, (\dev \grad)_w(I_{k, \ell}^{\div}\boldsymbol{v},0)) \\
&= (\div\boldsymbol{\sigma}, I_{k, \ell}^{\div}\boldsymbol{v}) + (\boldsymbol{\sigma}_h, (\dev \grad)_w(I_{k, \ell}^{\div}\boldsymbol{v},0)) \\
&= -(\boldsymbol{\sigma}-\boldsymbol{\sigma}_h,\dev \grad_h(I_{k, \ell}^{\div}\boldsymbol{v})) + \sum_{T\in\mathcal{T}_h}(\Pi_F(I_{k, \ell}^{\div}\boldsymbol{v}), \Pi_F(\boldsymbol{\sigma}-\boldsymbol{\sigma}_h)\boldsymbol{n})_{\partial T}.
\end{align*}
由于 $[\![\Pi_F(\boldsymbol{\sigma}-\boldsymbol{\sigma}_h)\boldsymbol{n}]\!]=0$ 对所有 $F \in \mathring{\mathcal{F}}_{h}$ 成立,
$$
\sum_{T\in\mathcal{T}_h}(\Pi_F(I_{k, \ell}^{\div}\boldsymbol{v}), \Pi_F(\boldsymbol{\sigma}-\boldsymbol{\sigma}_h)\boldsymbol{n})_{\partial T}=\sum_{F\in\mathcal{F}_h}([\![\Pi_F(I_{k, \ell}^{\div}\boldsymbol{v})]\!], \Pi_F(\boldsymbol{\sigma}-\boldsymbol{\sigma}_h)\boldsymbol{n})_{F}.
$$
综合上述两个等式可得
\begin{equation*}
(\div(I_{k, \ell}^{\div}\boldsymbol{v}), p-p_h)\lesssim \|\boldsymbol{\sigma}-\boldsymbol{\sigma}_h\|_{0,h}|I_{k, \ell}^{\div}\boldsymbol{v}|_{1,h}.
\end{equation*}
最后由插值误差估计 \eqref{eq:Ihdivprop2} 即得所需结果. 
\end{proof}

\begin{theorem}
设 $(\boldsymbol{u}, p)$ 为 Stokes 方程 \eqref{eq-Stokes2} 的解,$(\boldsymbol{u}_h, \boldsymbol{\lambda}_h, p_h)$ 为混合有限元方法 \eqref{distribustokesfemWG} 的解. 假设 $\boldsymbol{u}\in H^{k+2}(\Omega;\mathbb R^d)$ 且 $p\in H^{\ell+1}(\Omega)$,则有误差估计:
\begin{align}
\label{eq:errorestimate1}
\|\boldsymbol{\sigma} - \boldsymbol{\sigma}_h\|_{0,h} + \interleave(I_{k,k}^{\div}\boldsymbol{u} - \boldsymbol{u}_h,Q_{k,\mathcal{F}_h}\boldsymbol{\lambda}-\boldsymbol{\lambda}_h)\interleave_{1,h} \hskip-1cm & \\
\notag
 + \|Q_{\ell,h} p - p_h\| &\lesssim h^{k+1} |\boldsymbol{u}|_{k+2},\\
\label{eq:errorestimate2}
\|\boldsymbol{u} - \boldsymbol{u}_h\|+h|\boldsymbol{u} - \boldsymbol{u}_h|_{1,h} &\lesssim h^{k+1} ( |\boldsymbol{u}|_{k+2} + |\boldsymbol{u}|_{k+1}),\\
\label{eq:errorestimate3}
\|p - p_h\| &\lesssim h^{\ell+1} (|\boldsymbol{u}|_{k+2} + |p|_{\ell+1}).
\end{align}
\end{theorem}
\begin{proof}
为了简化表示,设 $\boldsymbol{v}_h=I_{k,k}^{\div}\boldsymbol{u} - \boldsymbol{u}_h$, 以及 $\boldsymbol{\mu}_h=Q_{k,\mathcal{F}_h}\boldsymbol{\lambda}-\boldsymbol{\lambda}_h$.
根据 \eqref{eq:Ihdivprop1} 和定理~\ref{thm:unisolmfem},可知 $\boldsymbol{v}_h \in \mathbb{V}_{k,k-1}^{\div} \cap \ker(\div)$. 
利用 \eqref{eq:weakdevgradcommu} 和 \eqref{eq:weakdevgradulambda},有
$$
(\dev \grad)_w(\boldsymbol{v}_h,\boldsymbol{\mu}_h)= (\dev \grad)_w(\boldsymbol{u},\boldsymbol{\lambda})-(\dev \grad)_w(\boldsymbol{u}_h,\boldsymbol{\lambda}_h)=Q_{k,h}\boldsymbol{\sigma}-\boldsymbol{\sigma}_h.
$$
根据 \eqref{eq:Ihtnorth},得到
\begin{equation*}
(I_h^{\rm tn}\boldsymbol{\sigma}-\boldsymbol{\sigma}_h, \boldsymbol{\sigma}-\boldsymbol{\sigma}_h) =(I_h^{\rm tn}\boldsymbol{\sigma}-\boldsymbol{\sigma}_h, Q_{k,h}\boldsymbol{\sigma}-\boldsymbol{\sigma}_h) =0.
\end{equation*}
由此可得
$$
\|I_h^{\rm tn}\boldsymbol{\sigma}-\boldsymbol{\sigma}_h\|^2=(I_h^{\rm tn}\boldsymbol{\sigma}-\boldsymbol{\sigma}_h, I_h^{\rm tn}\boldsymbol{\sigma}-\boldsymbol{\sigma}) \quad\Rightarrow\quad \|I_h^{\rm tn}\boldsymbol{\sigma}-\boldsymbol{\sigma}_h\|\leq \|\boldsymbol{\sigma}-I_h^{\rm tn}\boldsymbol{\sigma}\|,
$$
\begin{equation}\label{eq:20251118}
\|Q_{k,h}\boldsymbol{\sigma}-\boldsymbol{\sigma}_h\|^2=(Q_{k,h}\boldsymbol{\sigma}-I_h^{\rm tn}\boldsymbol{\sigma}, Q_{k,h}\boldsymbol{\sigma}-\boldsymbol{\sigma}_h) \quad\Rightarrow\quad \|Q_{k,h}\boldsymbol{\sigma}-\boldsymbol{\sigma}_h\|\leq \|Q_{k,h}\boldsymbol{\sigma}-I_h^{\rm tn}\boldsymbol{\sigma}\|.
\end{equation}
因此,利用插值误差估计 \eqref{eq:Ihtnprop2} 可得
$$
\|\boldsymbol{\sigma} - \boldsymbol{\sigma}_h\|_{0,h}\leq \|\boldsymbol{\sigma} - I_h^{\rm tn}\boldsymbol{\sigma}\|_{0,h} + \|I_h^{\rm tn}\boldsymbol{\sigma} - \boldsymbol{\sigma}_h\|_{0,h} \lesssim \|\boldsymbol{\sigma} - I_h^{\rm tn}\boldsymbol{\sigma}\|_{0,h}\lesssim h^{k+1} |\boldsymbol{u}|_{k+2}.
$$
结合 \eqref{eq:weakdevgradnormequiv} 和 \eqref{eq:20251118},得到
\begin{equation*}
\interleave(\boldsymbol{v}_h,\boldsymbol{\mu}_h)\interleave_{1,h}\lesssim\|(\dev \grad)_w(\boldsymbol{v}_h,\boldsymbol{\mu}_h)\|=\|Q_{k,h}\boldsymbol{\sigma}-\boldsymbol{\sigma}_h\|\leq\|Q_{k,h}\boldsymbol{\sigma}-I_h^{\rm tn}\boldsymbol{\sigma}\|\lesssim h^{k+1} |\boldsymbol{u}|_{k+2}.
\end{equation*}
另一方面,利用 \eqref{eq:qhinfsup} 和 \eqref{eq:preestimate4p},可得
\begin{equation*}
\|Q_{\ell,h} p - p_h\| \lesssim \sup_{\boldsymbol{v}\in H_0^1(\Omega;\mathbb{R}^d)} \frac{(\div(I_{k, \ell}^{\div}\boldsymbol{v}),Q_{\ell,h} p - p_h)}{\|\boldsymbol{v}\|_1}\lesssim \|\boldsymbol{\sigma}-\boldsymbol{\sigma}_h\|_{0,h}\lesssim h^{k+1} |\boldsymbol{\sigma}|_{k+1}.
\end{equation*}
因此,误差估计 \eqref{eq:errorestimate1} 成立. 最后,估计 \eqref{eq:errorestimate2} 与 \eqref{eq:errorestimate3} 可由 \eqref{eq:errorestimate1} 以及插值算子 $I_{k,k}^{\div}\boldsymbol{u}$ 和 $Q_{\ell,h} p$ 的估计直接得到. 
\end{proof}

估计 \eqref{eq:errorestimate1}-\eqref{eq:errorestimate2} 表明,混合有限元方法 \eqref{distribustokesfemWG} 是压力稳健的(pressure-robust). 

% \begin{remark}\rm
% 文献 \cite{ChenWangZhong2015} 中的 Stokes 方程三角形 MAC 格点方法,使用 $\textrm{RT}_0$-$\textrm{P}_0$ 元对(即 $k=\ell=0$)对速度和压力进行离散,其误差量 $|I_{(0,0),h}^{\div}\boldsymbol{u} - \boldsymbol{u}_h|_{1,h} + \|Q_{0,h} p - p_h\|$ 的收敛阶为 $O(h^{\min\{1,\sigma\}}|\ln h|^{1/2})$,其中 $\sigma \geq \frac{1}{2}$ 描述了网格的对称性.  当使用 $\textrm{BDM}_1$-$\textrm{P}_0$ 元对(即 $k=1$, $\ell=0$)时,MAC 格点方法对应误差量的收敛阶为 $O(h^{1+\min\{1/2,\sigma\}})$. 
% 可见,文献 \cite{ChenWangZhong2015} 中的两种三角形 MAC 格点方法的收敛速度均低于混合有限元方法 \eqref{distribustokesfemWG} 所给出的估计 \eqref{eq:errorestimate1}. 
% \end{remark}





\subsection{$L^2$ 误差估计}

通过对偶论证可以得到
$\|I_{k,k}^{\div}\boldsymbol{u}-\boldsymbol{u}_h\|$ 的超收敛性. 
引入对偶问题:求 $\tilde{\bs u}\in H_0^1(\Omega;\mathbb R^d)$ 和 $\tilde p\in L^2(\Omega)/\mathbb R$,使其满足
\begin{equation}\label{dualstokes}
\begin{cases}
%\tilde{\boldsymbol{\sigma}}-\grad\tilde{\boldsymbol{u}}=0 & \text { in } \Omega,\\
%\div \tilde{\boldsymbol{\sigma}} 
- \Delta \tilde{\bs u} + \nabla \tilde{p}= I_{k,k}^{\div}\boldsymbol{u}-\boldsymbol{u}_h& \text{在 }\, \Omega \text{ 内}, \\
\qquad\;\;\div\tilde{\boldsymbol{u}} =0 & \text{在 }\, \Omega \text{ 内}.
\end{cases}
\end{equation}
假设对偶问题 \eqref{dualstokes} 具有 $H^2$ 正则性:
\begin{equation}\label{eq:regularity}
\|\tilde{\boldsymbol{u}}\|_2 \lesssim \|I_{k,k}^{\div}\boldsymbol{u}-\boldsymbol{u}_h\|.
\end{equation}
关于凸域上 $H^2$ 正则性的参考文献见 \cite[Remark I.5.6]{GiraultRaviart1986} 和 \cite[Section 11.5]{MazyaRossmann2010}. 

\begin{lemma}
设 $(\boldsymbol{u}, p)$ 为 Stokes 方程 \eqref{eq-Stokes2} 的解,$(\boldsymbol{u}_h$, $\boldsymbol{\lambda}_h$, $p_h)$ 为混合有限元方法 \eqref{distribustokesfemWG} 的解. 设 $\tilde{\bs u}$ 为对偶问题 \eqref{dualstokes} 的解,并假设 $\skw\grad\boldsymbol{f}\in L^2(\Omega;\mathbb K)$. 则有
\begin{equation}
\label{eq:errorestimate40}
(\boldsymbol{\sigma}-\boldsymbol{\sigma}_h,\grad\tilde{\boldsymbol{u}}) \lesssim \begin{cases}
h\|\boldsymbol{\sigma}-\boldsymbol{\sigma}_h\|_{0,h} |\tilde{\boldsymbol{u}}|_{2}, & k\geq1,   \\
h^2\|\skw\grad\boldsymbol{f}\|\|\tilde{\boldsymbol{u}}\|_1, & k=0.
\end{cases}
\end{equation}
\end{lemma}
\begin{proof}
在 \eqref{distribustokesfemWG1} 中取 $\boldsymbol{v}_h = I_{k,k}^{\div}\tilde{\boldsymbol{u}}$ 和 $\boldsymbol{\mu}_h = 0$,结合 $\div(I_{k,k}^{\div}\tilde{\boldsymbol{u}})=0$ 和 \eqref{stokes1} 得
\begin{equation*}
(\boldsymbol{\sigma}_h, \dev \grad_w(I_{k,k}^{\div}\tilde{\boldsymbol{u}},0)) = (\boldsymbol{f}, I_{k,k}^{\div}\tilde{\boldsymbol{u}}) = -(\div\boldsymbol{\sigma}, I_{k,k}^{\div}\tilde{\boldsymbol{u}}),
\end{equation*}
即 
\[
\sum_{T\in\mathcal{T}_h}(\boldsymbol{\sigma}-\boldsymbol{\sigma}_h,\dev\grad(I_{k,k}^{\div}\tilde{\boldsymbol{u}}))_T - \sum_{F\in\mathcal{F}_h}(\Pi_F(\boldsymbol{\sigma}-\boldsymbol{\sigma}_h)\boldsymbol{n}, [\![\Pi_F(I_{k,k}^{\div}\tilde{\boldsymbol{u}})]\!])_{F}=0.
\]
于是
\begin{equation}\label{eq:20250923}
\begin{aligned}
(\boldsymbol{\sigma}-\boldsymbol{\sigma}_h,\grad\tilde{\boldsymbol{u}}) &= \sum_{T\in\mathcal{T}_h}(\boldsymbol{\sigma}-\boldsymbol{\sigma}_h,\dev\grad(\tilde{\boldsymbol{u}}-I_{k,k}^{\div}\tilde{\boldsymbol{u}}))_T \\
&\quad - \sum_{F\in\mathcal{F}_h}(\Pi_F(\boldsymbol{\sigma}-\boldsymbol{\sigma}_h)\boldsymbol{n}, [\![\Pi_F(\tilde{\boldsymbol{u}}-I_{k,k}^{\div}\tilde{\boldsymbol{u}})]\!])_{F}.
\end{aligned}
\end{equation}
当 $k\geq1$ 时,由 Cauchy-Schwarz 不等式和 $I_{k,k}^{\div}$ 的插值估计 \eqref{eq:Ihdivprop2} 可得
\begin{equation*}
(\boldsymbol{\sigma}-\boldsymbol{\sigma}_h,\grad\tilde{\boldsymbol{u}}) \lesssim h\|\boldsymbol{\sigma}-\boldsymbol{\sigma}_h\|_{0,h} |\tilde{\boldsymbol{u}}|_{2}.
\end{equation*}

当 $k=0$ 时,对 \eqref{eq:20250923} 右端进行分部积分,得到
\begin{equation*}
(\boldsymbol{\sigma}-\boldsymbol{\sigma}_h,\grad\tilde{\boldsymbol{u}}) =-(\div\boldsymbol{\sigma},\tilde{\boldsymbol{u}}-I_{0,0}^{\div}\tilde{\boldsymbol{u}}).
\end{equation*}
由于 $\div\tilde{\boldsymbol{u}}=0$ 且 $\div H_0^2(\Omega;\mathbb K)=H_0^1(\Omega;\mathbb R^d)\cap \ker(\div)$(参见 \cite[Theorem 1.1]{CostabelMcIntosh2010}),存在 $\tilde{\boldsymbol{\tau}}\in H_0^2(\Omega;\mathbb K)$ 使得
\begin{equation*}
\div\tilde{\boldsymbol{\tau}}=\tilde{\boldsymbol{u}},\quad \|\tilde{\boldsymbol{\tau}}\|_2\lesssim \|\tilde{\boldsymbol{u}}\|_1.
\end{equation*}
回顾以反对称矩阵形式表示的微分 $(d-2)$-形式线性有限元空间 \cite{Arnold2018,ArnoldFalkWinther2006}:
% Recall the linear finite element space of the differential $(d-2)$-form in skew-symmetric form \cite{Arnold2018,ArnoldFalkWinther2006}
\begin{equation*}
\mathbb V_{h}^{d-2}:=\{\boldsymbol{\tau}_h\in H(\div,\Omega;\mathbb K): \boldsymbol{\tau}_h|_T \in \mathbb P_{1}(T;\mathbb K) \;\;\textrm{ for } T \in \mathcal{T}_{h}\}.
\end{equation*}
其自由度定义参见文献 \cite[(3.10)]{ChenHuangWei2024}. 
记 $I_h^{d-2}: H^2(\Omega;\mathbb K)\to \mathbb V_{h}^{d-2}$ 为相应的节点插值算子,则它满足如下的交换性质:
\begin{equation*}
\div(I_h^{d-2}\boldsymbol{\tau})=I_{0,0}^{\div}(\div\boldsymbol{\tau})\quad\forall~\boldsymbol{\tau}\in H^2(\Omega;\mathbb K).
\end{equation*}
于是,利用分部积分和 \eqref{stokes1} 可得
\begin{align*}
(\boldsymbol{\sigma}-\boldsymbol{\sigma}_h,\grad\tilde{\boldsymbol{u}}) &= -(\div\boldsymbol{\sigma},\div(\tilde{\boldsymbol{\tau}}-I_{h}^{d-2}\tilde{\boldsymbol{\tau}})) = (\skw\grad(\div\boldsymbol{\sigma}), \tilde{\boldsymbol{\tau}}-I_{h}^{d-2}\tilde{\boldsymbol{\tau}}) \\
& = -(\skw\grad\boldsymbol{f}, \tilde{\boldsymbol{\tau}}-I_{h}^{d-2}\tilde{\boldsymbol{\tau}}) .
\end{align*}
结合 $I_{h}^{d-2}$ 的插值误差估计,得到
\begin{equation*}
(\boldsymbol{\sigma}_h-\boldsymbol{\sigma},\grad\tilde{\boldsymbol{u}})\lesssim h^2\|\skw\grad\boldsymbol{f}\||\tilde{\boldsymbol{\tau}}|_2\lesssim h^2\|\skw\grad\boldsymbol{f}\|\|\tilde{\boldsymbol{u}}\|_1.
\end{equation*}
证毕. 
\end{proof}

\begin{theorem}
设 $(\boldsymbol{u}, p)$ 为 Stokes 方程 \eqref{eq-Stokes2} 的解,$(\boldsymbol{u}_h, \boldsymbol{\lambda}_h, p_h)$ 为混合有限元方法 \eqref{distribustokesfemWG} 的解. 若 $\boldsymbol{u}\in H^{k+2}(\Omega;\mathbb R^d)$、$\skw\grad\boldsymbol{f}\in L^2(\Omega;\mathbb K)$,且 $H^2$ 正则性 \eqref{eq:regularity} 成立,则有
\begin{equation}
\label{eq:errorestimate4}
\|I_{k,k}^{\div}\boldsymbol{u} - \boldsymbol{u}_h\| \lesssim h^{k+2}(|\boldsymbol{u}|_{k+2}  + \delta_{k0}\|\skw\grad\boldsymbol{f}\|),
\end{equation}
其中 $\delta_{00}=1$,而当 $k\geq1$ 时 $\delta_{k0}=0$. 
\end{theorem}
\begin{proof}
令 $\boldsymbol{v}_h=I_{k,k}^{\div}\boldsymbol{u}-\boldsymbol{u}_h\in\mathring{\mathbb V}_{k,k-1}^{\div}\cap\ker(\div)$ 以简化表述.
设 $\tilde{\boldsymbol{\sigma}} = \grad \tilde{\boldsymbol{u}}$. 由 $\div\boldsymbol{v}_h=0$ 以及算子 $I_k^{\rm tn}$ 的定义,有
\begin{align*}
\|\boldsymbol{v}_h\|^2 &= -(\div \tilde{\boldsymbol{\sigma}}, \boldsymbol{v}_h) = \sum_{T\in\mathcal{T}_h}(\tilde{\boldsymbol{\sigma}},\dev \grad\boldsymbol{v}_h)_T - \sum_{T\in\mathcal{T}_h}(\Pi_F\boldsymbol{v}_h, \Pi_F\tilde{\boldsymbol{\sigma}}\boldsymbol{n})_{\partial T} \\
& = \sum_{T\in\mathcal{T}_h}(I_k^{\rm tn}\tilde{\boldsymbol{\sigma}},\dev \grad\boldsymbol{v}_h)_T -\sum_{T\in\mathcal{T}_h}(\Pi_F\boldsymbol{v}_h, \Pi_F(I_k^{\rm tn}\tilde{\boldsymbol{\sigma}})\boldsymbol{n})_{\partial T} \\
&=\big(I_k^{\rm tn}\tilde{\boldsymbol{\sigma}},\dev \grad_w(I_{k,k}^{\div}\boldsymbol{u}-\boldsymbol{u}_h,Q_{k,\mathcal{F}_h}\boldsymbol{\lambda}-\boldsymbol{\lambda}_h)\big).
\end{align*}
结合式 \eqref{eq:weakdevgradcommu} 可得
\[
\begin{aligned}
\|I_{k,k}^{\div}\boldsymbol{u}-\boldsymbol{u}_h\|^2& = (\boldsymbol{\sigma}-\boldsymbol{\sigma}_h,I_k^{\rm tn}\tilde{\boldsymbol{\sigma}})\\
&=(\boldsymbol{\sigma}-\boldsymbol{\sigma}_h,I_k^{\rm tn}\tilde{\boldsymbol{\sigma}}-\tilde{\boldsymbol{\sigma}}) + (\boldsymbol{\sigma}-\boldsymbol{\sigma}_h,\grad\tilde{\boldsymbol{u}}).
\end{aligned}
\]
由 Cauchy–Schwarz 不等式、算子 $I_k^{\rm tn}$ 的误差估计 \eqref{eq:Ihtnprop2}、\eqref{eq:errorestimate40} 以及 $H^2$ 正则性 \eqref{eq:regularity},可得
% \begin{equation*}
% \|I_{k,k}^{\div}\boldsymbol{u}-\boldsymbol{u}_h\|^2\lesssim h\|\boldsymbol{\sigma}-\boldsymbol{\sigma}_h\|_{0,h}(|\tilde{\boldsymbol{\sigma}}|_1 + |\tilde{\boldsymbol{u}}|_{2}).
% \end{equation*}
% This together with the $H^2$-regularity \eqref{eq:regularity} gives
\begin{equation*}
\|I_{k,k}^{\div}\boldsymbol{u}-\boldsymbol{u}_h\|\lesssim h\|\boldsymbol{\sigma}-\boldsymbol{\sigma}_h\|_{0,h} + \delta_{k0}h^2\|\skw\grad\boldsymbol{f}\|.
\end{equation*}
最后结合误差估计 \eqref{eq:errorestimate1} 与上述不等式,即得结论 \eqref{eq:errorestimate4} 成立. 
\end{proof}

\subsection{后处理}
利用误差估计式 \eqref{eq:errorestimate1} 与 \eqref{eq:errorestimate4},我们可以构造出 $\boldsymbol{u}$ 的一种超收敛近似. 
对于每个单元 $T \in \mathcal{T}_h$,寻找 $\boldsymbol{u}_h^{*}|_T\in \mathbb{P}_{k+1}(T;\mathbb{R}^d)$ 与 $p_h^{*}\in\mathbb{P}_k(T)\cap L_0^2(T)$,使其满足
\begin{subequations}\label{distribustokespostfem}
\begin{align}
(\boldsymbol{u}_h^{*}\cdot\boldsymbol{n},q)_F &=
(\boldsymbol{u}_h\cdot \boldsymbol{n},q)_F 
\quad\;\;\; \forall\,q\in\mathbb{P}_0(F),\ F\in\mathcal{F}(T), \label{eq:postprocessingu1}\\
(\grad\boldsymbol{u}_h^{*},\grad\boldsymbol{v})_T
+(\div\boldsymbol{v},p_h^{*})_T &=
(\boldsymbol{\sigma}_h,\grad\boldsymbol{v})_T 
\quad \forall\,\boldsymbol{v}\in\widetilde{\mathbb{P}}_{k+1}(T;\mathbb{R}^d), \label{eq:postprocessingu2}\\
(\div \boldsymbol{u}_h^{*},q)_T &= 0 
\qquad\qquad\qquad\; \forall\,q\in{\mathbb{P}}_{k}(T)\cap L_0^2(T), \label{eq:postprocessingu3}
\end{align}
\end{subequations}
其中 
$\widetilde{\mathbb{P}}_{k+1}(T;\mathbb{R}^d)
:= \{\boldsymbol{v}_h\in\mathbb{P}_{k+1}(T;\mathbb{R}^d): 
Q_{0,F}(\boldsymbol{v}_h\cdot\boldsymbol{n})=0\ \ \forall\,F\subset\partial T\}$.

局部问题 \eqref{distribustokespostfem} 是适定的. 
由式 \eqref{eq:postprocessingu1} 可得
\begin{equation*}
Q_{0,T}(\div\boldsymbol{u}_h^{*})
=Q_{0,T}(\div\boldsymbol{u}_h)=0,
\end{equation*} 
结合式 \eqref{eq:postprocessingu3},可知在每个单元 $T\in\mathcal{T}_h$ 上都有 $\div\boldsymbol{u}_h^{*}=0$. 
因此,$\boldsymbol{u}_h^{*}\in  \mathbb P_{k+1}(\mathcal T_h; \mathbb R^d)$ 在单元内部逐点满足散度为零. 

\begin{theorem}
设 $(\boldsymbol{u}, p)$ 为 Stokes 方程~\eqref{eq-Stokes2} 的精确解,$(\boldsymbol{u}_h$, $\boldsymbol{\lambda}_h$, $p_h)$ 为混合有限元方法~\eqref{distribustokesfemWG} 的数值解. 若 $\boldsymbol{u}\in H^{k+2}(\Omega;\mathbb R^d)$,则有
\begin{equation}\label{eq:postH1error}
\|\grad_h(\boldsymbol{u}-\boldsymbol{u}_h^{*})\|\lesssim h^{k+1}|\boldsymbol{u}|_{k+2}.
\end{equation}
若进一步假设 $\skw\grad\boldsymbol{f} \in L^2(\Omega; \mathbb{K})$, 且 $H^2$ 正则性估计~\eqref{eq:regularity} 成立,则有
\begin{equation}\label{eq:postL2error}
\|\boldsymbol{u}-\boldsymbol{u}_h^{*}\|\lesssim h^{k+2}\big(|\boldsymbol{u}|_{k+2}  + \delta_{k0}\|\skw\grad\boldsymbol{f}\|\big).
\end{equation}
\end{theorem}

\begin{proof}
令
\[
\boldsymbol{w}=(I-I_{(0,0),T}^{\div})(I_{(k+1,k),T}^{\div}\boldsymbol{u}-\boldsymbol{u}_h^{*})\in\widetilde{\mathbb{P}}_{k+1}(T;\mathbb{R}^d).
\]
由性质 \eqref{eq:Ihdivprop1} 可得
\[
\div\big(I_{(0,0),T}^{\div}(I_{(k+1,k),T}^{\div}\boldsymbol{u}-\boldsymbol{u}_h^{*})\big)
=Q_{0,T}\div\boldsymbol{u}-Q_{0,T}\div\boldsymbol{u}_h^{*}=0.
\]
因此,$I_{(0,0),T}^{\div}(I_{(k+1,k),T}^{\div}\boldsymbol{u}-\boldsymbol{u}_h^{*})\in\mathbb{P}_{0}(T;\mathbb{R}^d)$. 再次应用 \eqref{eq:Ihdivprop1} 可得
\[
\div\boldsymbol{w}
=\div(I_{(k+1,k),T}^{\div}\boldsymbol{u}-\boldsymbol{u}_h^{*})
=Q_{k,T}\div\boldsymbol{u}-\div\boldsymbol{u}_h^{*}=0.
\]
在式 \eqref{eq:postprocessingu2} 中取 $\boldsymbol{v}=\boldsymbol{w}$,得
\begin{align*}
|\boldsymbol{w}|_{1,T}^2
&=(\grad(I_{(k+1,k),T}^{\div}\boldsymbol{u}-\boldsymbol{u}_h^{*}),\grad\boldsymbol{w})_T \\
&=(\grad(I_{(k+1,k),T}^{\div}\boldsymbol{u}-\boldsymbol{u}),\grad\boldsymbol{w})_T
+(\boldsymbol{\sigma}-\boldsymbol{\sigma}_h,\grad\boldsymbol{w})_T.
\end{align*}
由插值误差估计 \eqref{eq:Ihdivprop2} 及逆不等式可得
\begin{equation}\label{eq:20250703}
\|\boldsymbol{w}\|_T\eqsim h_T|I_{(k+1,k),T}^{\div}\boldsymbol{u}-\boldsymbol{u}_h^{*}|_{1,T}
\lesssim h_T\big(|\boldsymbol{u}-I_{(k+1,k),T}^{\div}\boldsymbol{u}|_{1,T}+\|\boldsymbol{\sigma}-\boldsymbol{\sigma}_h\|_T\big).
\end{equation}
由三角不等式、插值估计式 \eqref{eq:Ihdivprop2} 以及应力误差估计 \eqref{eq:errorestimate1},可得式 \eqref{eq:postH1error} 成立. 

再由式 \eqref{eq:postprocessingu1} 有
\[
I_{(0,0),T}^{\div}(I_{(k+1,k),T}^{\div}\boldsymbol{u}-\boldsymbol{u}_h^{*})
=I_{(0,0),T}^{\div}(I_{(k,k),T}^{\div}\boldsymbol{u}-\boldsymbol{u}_h).
\]
利用三角不等式、插值估计 \eqref{eq:Ihdivprop2}、逆不等式以及式 \eqref{eq:20250703},可得
\begin{align*}
\|\boldsymbol{u}-\boldsymbol{u}_h^{*}\|_T
&\le \|\boldsymbol{u}-I_{(k+1,k),T}^{\div}\boldsymbol{u}\|_T
+\|I_{(0,0),T}^{\div}(I_{(k,k),T}^{\div}\boldsymbol{u}-\boldsymbol{u}_h)\|_T
+\|\boldsymbol{w}\|_T\\
&\lesssim \|\boldsymbol{u}-I_{(k+1,k),T}^{\div}\boldsymbol{u}\|_T
+\|I_{(k,k),T}^{\div}\boldsymbol{u}-\boldsymbol{u}_h\|_T\\
&\quad +h_T\big(|\boldsymbol{u}-I_{(k+1,k),T}^{\div}\boldsymbol{u}|_{1,T}
+\|\boldsymbol{\sigma}-\boldsymbol{\sigma}_h\|_T\big).
\end{align*}
结合插值估计式 \eqref{eq:Ihdivprop2}、误差估计 \eqref{eq:errorestimate4} 与 \eqref{eq:errorestimate1},即得结论 \eqref{eq:postL2error}. 
\end{proof}



\subsection{等价的离散方法}\label{sec:hybrid}

本节探讨与混合有限元格式 \eqref{distribustokesfemWG} 等价的几种离散方法. 

\subsubsection{速度场的杂交化}

回顾文献 \cite{AyusodeDiosLipnikovManzini2016,ChenHuang2020} 中定义的 $H^1$ 非协调虚拟元. 
对整数 $k\geq 0$ 和 $\ell=k,k-1$,其单元形函数空间定义为:
\begin{equation*}
%\label{ncve}
V_{k+1,\ell+2}^{\rm VE}(T)=\{v\in H^1(T):\Delta v\in\mathbb{P}_{\ell}(T), \partial_n v|_F\in\mathbb{P}_{k}(F), \ \forall\, F\in\mathcal{F}(T)\}.
\end{equation*}
%Obviously $\mathbb{P}_{k+1}(T)\subseteq V_{k+1,\ell+2}^{\rm VE}(T)$, $V^{\rm VE}_{1,1}(T)=\mathbb{P}_{1}(T)$, and $V^{\rm VE}_{1,2}(T)=\mathbb{P}_{1}(T)\oplus\textrm{span}\{\boldsymbol{x}\cdot\boldsymbol{x}\}$.
该形函数空间的自由度如下给出:
\begin{subequations}\label{ve-dof}
	\begin{align}
		\label{ve-dof1}
		(v,q)_F, & \quad q\in \mathbb{P}_{k}(F), \, F\in\mathcal{F}(T),\\
		\label{ve-dof2}
		(v,q)_T, & \quad q\in \mathbb{P}_{\ell}(T).
	\end{align}
\end{subequations}
全局虚拟元空间 $V_{k+1,\ell+2}^{\rm VE} = V_{k+1,\ell+2}^{\rm VE}(\mathcal T_h)$ 定义为
\begin{align*}
V_{k+1,\ell+2}^{\rm VE}(\mathcal T_h) &= \{v\in L^2(\Omega):v|_T\in V_{k+1,\ell+2}^{\rm VE}(T),\ \forall\, T\in\mathcal{T}_h; \textrm{ 自由度 \eqref{ve-dof1}} \textrm{ 在内部面 $\mathring{\mathcal{F}}_h$ 上取单值}\}.
\end{align*}
该空间满足弱连续性条件:
\begin{equation*}%\label{weak-c}
([\![v]\!], q)_F = 0 
\quad \forall\, v\in V_{k+1,\ell+2}^{\rm VE}(\mathcal T_h), \ 
q\in\mathbb{P}_{k}(F), \ 
F\in\mathring{\mathcal{F}}_h.
\end{equation*}
当 $(k,\ell) = (0,-1)$ 时,空间 $V_{1,1}^{\rm VE}$ 退化为 Crouzeix--Raviart (CR) 元~\cite{CrouzeixRaviart1973};
当 $(k,\ell) = (0,0)$ 时,空间 $V_{1,2}^{\rm VE}$ 对应于增强性 Crouzeix--Raviart 元~\cite{HuMa2015,HuHuangLin2014}. 


通过杂交化空间 $\mathring{\mathbb{V}}_{k,\ell}^{\div}$ 的法向连续性,并将拉格朗日乘子与空间 $\mathbb{P}_{\ell}(\mathcal{T}_h)$ 结合以形成用于离散压力的 $H^1$-非协调空间 $V_{k+1,\ell+2}^{\rm VE}(\mathcal T_h)$,我们得到如下混合格式:
寻找 $\boldsymbol{u}_h\in \prod_{T\in \mathcal T_h}\mathbb V_{k,\ell}^{\div}(T)$, $\boldsymbol{\lambda}_h\in\mathbb P_{k}(\mathring{\mathcal{F}}_h;\mathbb R^{d-1})$ 和 $p_h\in V_{k+1,\ell+2}^{\rm VE}(\mathcal T_h)/\mathbb R$ 使得
\begin{subequations}\label{distribustokesH1pfemWG}
\begin{align}
\label{distribustokesH1pfemWG1}
(\dev \grad_w(\boldsymbol{u}_h,\boldsymbol{\lambda}_h), \dev \grad_w(\boldsymbol{v}_h,\boldsymbol{\mu}_h)) - (\boldsymbol{v}_h, \nabla_hp_h) &= (\boldsymbol{f}, \boldsymbol{v}_h), \\
\label{distribustokesH1pfemWG2}
(\boldsymbol{u}_h, \nabla_hq_h)&=0
\end{align}
\end{subequations}
对任意 $\boldsymbol{v}_h\in \prod_{T\in \mathcal T_h}\mathbb V_{k,\ell}^{\div}(T)$, $\boldsymbol{\mu}_h\in\mathbb P_{k}(\mathring{\mathcal{F}}_h;\mathbb R^{d-1})$, $q_h\in V_{k+1,\ell+2}^{\rm VE}(\mathcal T_h)/\mathbb R$ 成立. 


\begin{theorem}\label{thm:equivalence0}
混合方法 \eqref{distribustokesH1pfemWG} 是适定的. 
设 $(\boldsymbol{u}_h, \boldsymbol{\lambda}_h, p_h)\in \prod_{T\in \mathcal T_h}$ $\mathbb V_{k,\ell}^{\div}(T)$ $\times \mathring{\Lambda}_k\times (V_{k+1,\ell+2}^{\rm VE}(\mathcal T_h)/\mathbb R)$ 是其解,则 $(\boldsymbol{u}_h,\boldsymbol{\lambda}_h,Q_{\ell}p_h)\in \mathring{\mathbb{V}}_{k,\ell}^{\div}\times \mathring{\Lambda}_k\times(\mathbb{P}_{\ell}(\mathcal{T}_h)/\mathbb R)$ 即为混合有限元方法 \eqref{distribustokesfemWG} 的解.
\end{theorem}
\begin{proof}
首先证明当 $\boldsymbol{f}=0$ 时离散格式 \eqref{distribustokesH1pfemWG} 只有零解. 
对 \eqref{distribustokesH1pfemWG2} 应用分部积分得
\begin{equation*}
\sum_{T\in\mathcal{T}_h}(\div\boldsymbol{u}_h, Q_{\ell,T}q_h)_T - \sum_{F\in\mathcal{F}_h}([\![\boldsymbol{u}_h\cdot\boldsymbol{n}]\!], Q_{k,F}q_h)_F=0 \quad \forall~q_h\in V_{k+1,\ell+2}^{\rm VE}(\mathcal T_h).
\end{equation*}
由自由度 \eqref{ve-dof} 可知上式蕴含 $\boldsymbol{u}_h\in\mathring{\mathbb{V}}_{k,\ell}^{\div}$,以及 $\div\boldsymbol{u}_h=0$. 

记 $Q_T^{\rm div}: L^2(T; \mathbb{R}^d) \to \mathbb V_{k,\ell}^{\div}(T)$ 为 $L^2$ 正交投影算子. 
回顾文献 ~\cite{HuangTang2025,ChenHuang2020,ChenHuangWei2024} 中建立的单元 $T\in\mathcal{T}_h$ 上的范数等价关系:
\begin{align}
%\label{ve-normeq}
%\|v\|^2_{T} &\eqsim \|Q_{\ell,T}v\|^2_{T}+\sum_{F\in\mathcal{F}(T)}h_F\|Q_{k,F}v\|^2_{F} \quad \forall \ v\in V_{k+1,\ell+2}^{\rm VE}(T), \\
\label{ve-infsup2}
\|Q_{T}^{\rm div}\nabla v\|_{0,T}&\eqsim\|\nabla v\|_{0,T} \quad \forall \ v\in V_{k+1,\ell+2}^{\rm VE}(T).
\end{align}

在 \eqref{distribustokesH1pfemWG1} 中取 $(\boldsymbol{v}_h,\boldsymbol{\mu}_h)=(\boldsymbol{u}_h,\boldsymbol{\lambda}_h)$, \eqref{distribustokesH1pfemWG2} 中取 $q_h=p_h$,并相加得 $\dev \grad_w(\boldsymbol{u}_h,\boldsymbol{\lambda}_h)=0$,从而 $\boldsymbol{u}_h=0$,$\boldsymbol{\lambda}_h=0$. 再结合 \eqref{distribustokesH1pfemWG1} 与范数等价关系 \eqref{ve-infsup2} 可得 $p_h=0$. 
因此离散格式 \eqref{distribustokesH1pfemWG} 是适定的. 

对于第二部分,由 \eqref{distribustokesH1pfemWG2} 可知 $\boldsymbol{u}_h\in\mathring{\mathbb{V}}_{k,\ell}^{\div}$,以及 $\div\boldsymbol{u}_h=0$. 
取 $\boldsymbol{v}_h\in\mathring{\mathbb{V}}_{k,\ell}^{\div}$,由分部积分得
\[
 - (\boldsymbol{v}_h, \nabla_hp_h) = (\div\boldsymbol{v}_h, Q_{\ell}p_h).
\]
因此,$(\boldsymbol{u}_h,\boldsymbol{\lambda}_h,Q_{\ell}p_h)$ 满足方程 \eqref{distribustokesfemWG1},证毕. 
\end{proof}

%Theorem~\ref{thm:equivalence0} establishes the equivalence between the discrete method \eqref{distribustokesH1pfemWG} and the mixed finite element method \eqref{distribustokesfemWG}.  

在 \eqref{distribustokesH1pfemWG} 中,速度空间的法向连续性被松弛,这在 $\ell = -1$ 时尤为有用. 由于空间 $\mathring{\mathbb{V}}_{0,-1}^{\div}$ 缺乏显式的局部基函数,我们可以通过程序实现等价的混合格式 \eqref{distribustokesH1pfemWG} 来替代混合方法 \eqref{distribustokesfemWG}. 

即使在 $(k,\ell)=(0,-1)$ 的最低阶情形下,该方法仍具有如下误差估计:
\begin{equation*}
\begin{aligned}
&\|\boldsymbol{\sigma} - \boldsymbol{\sigma}_h\|_{0,h} + \|\grad_h(\boldsymbol{u}-\boldsymbol{u}_h^{*})\| + \|\dev\grad_w(I_{0,0}^{\div}\boldsymbol{u} - \boldsymbol{u}_h,Q_{0,\mathcal{F}_h}\boldsymbol{\lambda}-\boldsymbol{\lambda}_h)\| \lesssim h |\boldsymbol{u}|_{2},\\
&\|\boldsymbol{u} - \boldsymbol{u}_h\|_0+h\|\grad_h(\boldsymbol{u} - \boldsymbol{u}_h)\| \lesssim h( |\boldsymbol{u}|_{2} + |\boldsymbol{u}|_{1}),\\	
&\|\boldsymbol{u}-\boldsymbol{u}_h^{*}\|+\|I_{0,0}^{\div}\boldsymbol{u} - \boldsymbol{u}_h\| \lesssim h^{2}(|\boldsymbol{u}|_{2}  + \|\skw\grad\boldsymbol{f}\|).
\end{aligned}
\end{equation*}
%\end{remark}

\subsubsection{无稳定项虚拟元方法}
对整数 $k\geq0$,定义 $H^1$-非协调虚拟元的局部形函数空间为:
\begin{align*}
\mathbb{V}_{k+1}^{\rm VE}(T)&:=\{\bs v\in H^1(T;\mathbb R^d): \div\bs v\in\mathbb P_{k}(T), \textrm{ 存在 }\, s\in L^2(T) \textrm{ 使得 } \\
&  \quad\quad \Delta\bs v+\nabla s\in \big(\mathbb P_{k-1}(T;\mathbb R^d)\cap\ker(\cdot\boldsymbol x)\big),\;  (\partial_n\bs v+s\bs n)|_F\in\mathbb P_{k}(F;\mathbb R^d) \;\forall~F\in\mathcal F(T)\}.
\end{align*}
参照 \cite[Section 3.1]{WeiHuangLi2021} 的论证,局部虚拟元空间 $\mathbb{V}_{k+1}^{\rm VE}(T)$ 由以下自由度唯一确定:
\begin{subequations}\label{vev-dof}
\begin{align}
(\bs v, \bs q)_F, & \quad \bs q\in\mathbb P_{k}(F; \mathbb R^d),  F\in\mathcal F(T), \label{vev-dof1}\\
(\bs v, \bs q)_T, & \quad \bs q\in\mathbb P_{k-1}(T;\mathbb R^d). \label{vev-dof2}
\end{align}
\end{subequations}
显然有 $\mathbb P_{k+1}(T;\mathbb R^d)\subseteq\mathbb{V}_{k+1}^{\rm VE}(T)$,且 $\mathbb{V}_{1}^{\rm VE}(T)=\mathbb P_1(T;\mathbb R^d)$. 
全局虚拟元空间定义为:
\begin{align*}
\mathring{\mathbb{V}}_{h}^{\rm VE} &:= \{\boldsymbol{v}\in L^2(\Omega;\mathbb R^d):\boldsymbol{v}|_T\in\mathbb{V}_{k+1}^{\rm VE}(T) \textrm{ 对所有 }\, T\in\mathcal{T}_h; \\
&\qquad\qquad\qquad\quad\quad\;\;\textrm{ 自由度 \eqref{vev-dof1} 在 $\mathring{\mathcal{F}}_h$ 中每个面上取单值,且在 $\partial\Omega$ 上为零 }\}.
\end{align*}

Stokes 方程 \eqref{eq-Stokes2} 的一种混合虚拟元方法是:对整数 $k\geq 0$,求 $\boldsymbol{u}_h\in \mathring{\mathbb{V}}_{h}^{\rm VE}$ 和 $p_h\in\mathbb{P}_{k}(\mathcal{T}_h)/\mathbb R$,使得
\begin{subequations}\label{distribustokesvem}
\begin{align}
\label{distribustokesvem1}
(Q_{k}(\dev \grad_h\boldsymbol{u}_h), Q_{k}(\dev \grad_h\boldsymbol{v}_h)) + (\div_h\boldsymbol{v}_h, p_h) &= (\boldsymbol{f}, I_{k,k}^{\div}\boldsymbol{v}_h), \\
\label{distribustokesvem2}
(\div_h\boldsymbol{u}_h, q_h)&=0
\end{align}
\end{subequations}
对所有 $\boldsymbol{v}_h\in \mathring{\mathbb{V}}_{h}^{\rm VE}$ 和 $q_h\in\mathbb{P}_{k}(\mathcal{T}_h)/\mathbb R$ 成立.
当 $k=0$ 时,混合格式 \eqref{distribustokesvem} 即为文献 \cite[(23)]{Linke2014} 中的修正 Crouzeix-Raviart 元方法. 


% To show the well-posedness of the mixed virtual element method \eqref{distribustokesvem}, we equip space $\mathring{\mathbb{V}}_{h}^{\rm VE}$ with norm $\interleave(I_{k,k}^{\div}\boldsymbol{v}_h, Q_{k,\mathcal{F}_h}(\Pi_F\boldsymbol{v}_h))\interleave_{1,h}$. By the commuting property \eqref{eq:Ihdivprop1} and the norm equivalence \eqref{eq:discretdevgradnormequiv11}, we have
% \begin{equation*}
% (\div_h\boldsymbol{v}_h, q_h)=(\div(I_{k,k}^{\div}\boldsymbol{v}_h), q_h)\lesssim \interleave(I_{k,k}^{\div}\boldsymbol{v}_h, Q_{k,\mathcal{F}_h}(\Pi_F\boldsymbol{v}_h))\interleave_{1,h}\|q_h\|
% \end{equation*}
% for all $\boldsymbol{v}_h\in \mathring{\mathbb{V}}_{h}^{\rm VE}$ and $q_h\in\mathbb{P}_{k}(\mathcal{T}_h)/\mathbb R$.

\begin{lemma}
成立以下交换性:
\begin{equation}\label{eq:weakdevgradvemcommu}
\dev \grad_w(I_{k,k}^{\div}\boldsymbol{v}_h, Q_{k,\mathcal{F}_h}(\Pi_F\boldsymbol{v}_h))	= Q_{k}(\dev \grad_h\boldsymbol{v}_h)\quad\forall\,\boldsymbol{v}_h\in \mathring{\mathbb{V}}_{h}^{\rm VE}.
\end{equation}
\end{lemma}
\begin{proof}
由算子 $\dev \grad_w$ 与 $I_{k,k}^{\div}$ 的定义,对任意 $\boldsymbol{\tau}\in \mathbb{P}_{k}(T;\mathbb{T})$ 及 $T\in\mathcal{T}_h$ 有
\begin{align*}
&\quad\; (\dev \grad_w(I_{k,k}^{\div}\boldsymbol{v}_h, Q_{k,\mathcal{F}_h}(\Pi_F\boldsymbol{v}_h)), \boldsymbol{\tau})_T \\
&=-(I_{k,k}^{\div}\boldsymbol{v}_h,\div\boldsymbol{\tau})_T+(\boldsymbol{n}\cdot(I_{k,k}^{\div}\boldsymbol{v}_h),\boldsymbol{n}^{\intercal}\boldsymbol{\tau}\boldsymbol{n})_{\partial T}+(\Pi_F\boldsymbol{v}_h, \Pi_F\boldsymbol{\tau}\boldsymbol{n})_{\partial T}\\
&=-(\boldsymbol{v}_h,\div\boldsymbol{\tau})_T+(\boldsymbol{n}\cdot\boldsymbol{v}_h,\boldsymbol{n}^{\intercal}\boldsymbol{\tau}\boldsymbol{n})_{\partial T}+(\Pi_F\boldsymbol{v}_h, \Pi_F\boldsymbol{\tau}\boldsymbol{n})_{\partial T}.
\end{align*}
再结合分部积分即得 \eqref{eq:weakdevgradvemcommu}. 
\end{proof}

由 \eqref{eq:weakdevgradvemcommu} 可知,$\|Q_{k}(\dev \grad_h\boldsymbol{v}_h)\|$ 定义了空间 $\mathring{\mathbb{V}}_{h}^{\rm VE}\cap\ker(\div_h)$ 上的一个范数. 

% \begin{equation*}
% \|Q_{k,h}(\dev \grad_h\boldsymbol{v}_h)\|\lesssim \interleave(I_{k,k}^{\div}\boldsymbol{v}_h, Q_{k,\mathcal{F}_h}(\Pi_F\boldsymbol{v}_h))\interleave_{1,h} \quad \forall\,\boldsymbol{v}_h\in \mathring{\mathbb{V}}_{h}^{\rm VE}.
% \end{equation*}

\begin{theorem}
混合虚拟元方法 \eqref{distribustokesvem} 是适定的. 
设 $(\boldsymbol{u}_h, p_h)\in \mathring{\mathbb{V}}_{h}^{\rm VE} \times \mathbb{P}_{k}(\mathcal{T}_h)/\mathbb R$ 为其解,则 $$(I_{k,k}^{\div}\boldsymbol{u}_h, Q_{k,\mathcal{F}_h}(\Pi_F\boldsymbol{u}_h), Q_{\ell}p_h)\in \mathring{\mathbb{V}}_{k,\ell}^{\div} \times\mathbb P_{k}(\mathring{\mathcal{F}}_h; \mathbb R^{d-1})\times \mathbb{P}_{\ell}(\mathcal{T}_h)/\mathbb R$$  是混合有限元方法 \eqref{distribustokesfemWG} 的解. 
\end{theorem}
\begin{proof}
混合有限元方法 \eqref{distribustokesvem} 的适定性可采用与定理~\ref{thm:equivalence0} 类似的思路证明. 

下面证明离散方法 \eqref{distribustokesvem} 与 \eqref{distribustokesfemWG} 的等价性. 
由交换性 \eqref{eq:weakdevgradvemcommu},可知方程 \eqref{distribustokesvem1} 意味着 $(I_{k,k}^{\div}\boldsymbol{u}_h$, $Q_{k,\mathcal{F}_h}(\Pi_F\boldsymbol{u}_h)$, $Q_{\ell}p_h)$ 满足方程 \eqref{distribustokesfemWG1}. 
由交换性质 \eqref{eq:Ihdivprop1},方程 \eqref{distribustokesvem1} 说明 $I_{k,k}^{\div}\boldsymbol{u}_h$ 满足方程 \eqref{distribustokesfemWG2}. 
\end{proof}

\begin{remark}\rm
我们也可采用如下虚拟元空间离散速度:
% \begin{align*}
% &\Big{\boldsymbol{v}\in L^2(\Omega;\mathbb R^d):\boldsymbol{v}|T\in V{k+1,k+1}^{\rm VE}(T)\otimes\mathbb R^d \text{ 对所有 }T\in\mathcal{T}_h; \
% &\qquad\qquad \text{自由度 \eqref{ve-dof1} 在 $\mathring{\mathcal{F}}_h$ 中每个面上取单值,且在 $\partial\Omega$ 上为零}\Big}.
% \end{align*}
\begin{align*}
% \mathring{V}_{h}^{\rm VE}:= 
&\{\boldsymbol{v}\in L^2(\Omega;\mathbb R^d):\boldsymbol{v}|_T\in V_{k+1,k+1}^{\rm VE}(T)\otimes\mathbb R^d \textrm{ 对所有 }\, T\in\mathcal{T}_h;  \\
&\qquad\qquad\qquad\;\;\;\textrm{ 自由度 \eqref{ve-dof1} 在 $\mathring{\mathcal{F}}_h$ 中每个面上取单值,且在 $\partial\Omega$ 上为零}\}.
\end{align*}
% Then the equivalent mixed virtual element method is to find $\boldsymbol{u}_h\in \mathring{V}_{h}^{\rm VE}$ and $p_h\in\mathbb{P}_{k}(\mathcal{T}_h)/\mathbb R$ with integer $k\geq 0$, such that
% \begin{align*}
% (Q_{k,h}(\dev \grad_h\boldsymbol{u}_h), Q_{k,h}(\dev \grad_h\boldsymbol{v}_h)) + (\div(I_{k,k}^{\div}\boldsymbol{v}_h), p_h) &= (\boldsymbol{f}, I_{k,k}^{\div}\boldsymbol{v}_h), \\
% (\div(I_{k,k}^{\div}\boldsymbol{u}_h), q_h)&=0
% \end{align*}
% for all $\boldsymbol{v}_h\in \mathring{V}_{h}^{\rm VE}$ and $q_h\in\mathbb{P}_{k}(\mathcal{T}_h)/\mathbb R$.
\end{remark}

\subsubsection{伪应力–速度–压力离散格式}
通过引入伪应力变量
$\boldsymbol{\sigma}_h:= \dev \grad_w\boldsymbol{u}_h\in\Sigma_{k}^{\rm tn}$,
可将混合有限元方法 \eqref{distribustokesfemWG}
改写为 MCS 方法~\cite{GopalakrishnanLedererSchoeberl2020a}:
求 $\boldsymbol{\sigma}_h\in \Sigma_{k}^{\rm tn}$, $\boldsymbol{u}_h\in\mathring{\mathbb{V}}_{k,\ell}^{\div}$ 和 $p_h\in\mathbb{P}_{\ell}(\mathcal{T}_h)/\mathbb R$,使得
% \begin{subequations}\label{distribustokesfem}
\begin{align*}
% \label{distribustokesfem1}
(\boldsymbol{\sigma}_h,\boldsymbol{\tau}_h)+b_h(\boldsymbol{\tau}_h,q_h;\boldsymbol{u}_h)&=0  & & \forall~\boldsymbol{\tau}_h \in \Sigma_{k}^{\rm tn}, q_h\in\mathbb{P}_{\ell}(\mathcal{T}_h)/\mathbb R, \\
% \label{distribustokesfem2}
b_h(\boldsymbol{\sigma}_h,p_h;\boldsymbol{v}_h) &=-(\boldsymbol{f}, \boldsymbol{v}_h) & &\forall~\boldsymbol{v}_h \in \mathring{\mathbb{V}}_{k,\ell}^{\div},
\end{align*}
% \end{subequations}
其中双线性形式
\begin{align*}
b_h(\boldsymbol{\tau}_h,q_h;\boldsymbol{v}_h)&:=\sum_{T\in \mathcal{T}_{h}}(\div\boldsymbol{\tau}_h, \boldsymbol{v}_h)_T-\sum_{T\in \mathcal{T}_{h}}(\boldsymbol{n}^\intercal\boldsymbol{\tau}_h\boldsymbol{n}, \boldsymbol{n}\cdot\boldsymbol{v}_h)_{\partial T} - (\div\boldsymbol{v}_h, q_h) \\
&\;=-\sum_{T\in \mathcal{T}_{h}}(\boldsymbol{\tau}_h, \grad\boldsymbol{v}_h)_T+\sum_{T\in \mathcal{T}_{h}}(\Pi_F\boldsymbol{\tau}_h\boldsymbol{n}, \Pi_F\boldsymbol{v}_h)_{\partial T} - (\div\boldsymbol{v}_h, q_h).
\end{align*}

在文献 \cite{GopalakrishnanLedererSchoeberl2020a} 的 MCS 方法中,伪应力空间选取如下子空间:
\begin{equation*}
\left\{
\boldsymbol{\tau}_h \in \Sigma_{k}^{\rm tn} :
\Pi_F \boldsymbol{\tau}_h \boldsymbol{n} \in \mathbb{P}_{k-1}(F; \mathbb{R}^{d-1})
\ \text{对所有 }\, F \in \mathcal{F}_h
\right\}
\end{equation*}
来离散无迹应力张量 $\boldsymbol{\sigma}$. 
然而,约束条件 $\Pi_F \boldsymbol{\tau}_h \boldsymbol{n} \in \mathbb{P}_{k-1}(F; \mathbb{R}^{d-1})$ 使得~\cite{GopalakrishnanLedererSchoeberl2020a} 中的离散格式无法获得超收敛性质,例如估计式 \eqref{eq:errorestimate1} 中的 $\|\bs \sigma - \bs \sigma_h\|$ 与 $\|Q_{\ell} p - p_h\|$,以及 \eqref{eq:errorestimate4} 中的 $\|I_{k,k}^{\div}\boldsymbol{u} - \boldsymbol{u}_h\|$. 

在后续工作 \cite{GopalakrishnanLedererSchoeberl2020} 中,通过丰富应力空间获得了超收敛结果,但其方法要求 $k=\ell\ge1$. 
相比之下,本节的离散格式不仅涵盖高阶情形,还自然适用于低阶情形 $(k,\ell)=(0,-1)$、$(0,0)$、$(1,0)$,并且在理论分析上更加简洁、透明. 




\section{ Stokes 复形}
\subsection{有限元Stokes复形}
回顾二维de Rham复形
\begin{equation*}
0\xrightarrow{} H_0^1(\Omega)\xrightarrow{\curl} H_0(\div,\Omega)\xrightarrow{\div} L_0^2(\Omega)\xrightarrow{}0,
\end{equation*}
\begin{equation*}
\mathbb R\xrightarrow{} H^1(\Omega)\xrightarrow{\curl} H(\div,\Omega)\xrightarrow{\div} L^2(\Omega)\xrightarrow{}0,
\end{equation*}
\begin{equation*}
0\xrightarrow{} H_0^{s+2}(\Omega)\xrightarrow{\curl} H_0^{s+1}(\Omega;\mathbb R^2)\xrightarrow{\div} H_0^s(\Omega)\cap L_0^2(\Omega)\xrightarrow{}0,
\end{equation*}
\begin{equation*}
\mathbb R\xrightarrow{} H^{s+2}(\Omega)\xrightarrow{\curl} H^{s+1}(\Omega;\mathbb R^2)\xrightarrow{\div} H^s(\Omega)\xrightarrow{}0.
\end{equation*}


回顾三维de Rham复形
\begin{equation*}
0\xrightarrow{} H_0^1(\Omega)\xrightarrow{\grad} H_0(\curl,\Omega)\xrightarrow{\curl} H_0(\div,\Omega)\xrightarrow{\div} L_0^2(\Omega)\xrightarrow{}0,
\end{equation*}
\begin{equation*}
\mathbb R\xrightarrow{} H^1(\Omega)\xrightarrow{\grad} H(\curl,\Omega)\xrightarrow{\curl} H(\div,\Omega)\xrightarrow{\div} L^2(\Omega)\xrightarrow{}0,
\end{equation*}
\begin{equation*}
0\xrightarrow{} H_0^{s+3}(\Omega)\xrightarrow{\grad} H_0^{s+2}(\Omega;\mathbb R^3)\xrightarrow{\curl} H_0^{s+1}(\Omega;\mathbb R^3)\xrightarrow{\div} H_0^{s}(\Omega)\cap L_0^2(\Omega)\xrightarrow{}0,
\end{equation*}
\begin{equation*}
\mathbb R\xrightarrow{} H^{s+3}(\Omega)\xrightarrow{\grad} H^{s+2}(\Omega;\mathbb R^3)\xrightarrow{\curl} H^{s+1}(\Omega;\mathbb R^3)\xrightarrow{\div} H^{s}(\Omega)\xrightarrow{}0.
\end{equation*}

