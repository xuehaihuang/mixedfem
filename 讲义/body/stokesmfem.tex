% !TEX root = lecture.tex
\chapter{Stokes方程的混合有限元方法}

\section{Stokes方程}
设$\Omega\subset\mathbb{R}^{d}$ ($d\geq2$)是有界区域.
考虑具有齐次 Dirichlet边界条件下的  Stokes 方程
\begin{equation}\label{eq-Stokes2}
\left\{
\begin{array}{l}
-\Delta \boldsymbol{u} - \nabla p=\boldsymbol{f}(x),\qquad \,\boldsymbol{x}\in\Omega,\\
\div \boldsymbol{u}=0, \qquad \,\qquad \qquad \,\boldsymbol{x}\in\Omega,\\
\boldsymbol{u}=0,
\quad \qquad \qquad \qquad\;\;\; \boldsymbol{x}\in\partial\Omega,
\end{array}
\right.
\end{equation}
其中$\boldsymbol{u}$ 表示流体速度, $p$ 表示压力.
问题(\ref{eq-Stokes2})的混合变分问题为 : 找 $\boldsymbol{u}\in H_0^{1}(\Omega;\mathbb{R}^{d}), p\in L_{0}^{2}(\Omega)$,使得
\begin{align}
(\nabla\boldsymbol{u},\nabla\boldsymbol{v}) + (\div\boldsymbol{v},p) &=(\boldsymbol{f},\boldsymbol{v}),\quad \boldsymbol{v}\in  H_0^{1}(\Omega;\mathbb{R}^{d}),\label{BI2}\\
(\div\boldsymbol{u},q) &=0, \quad\quad\quad\, q\in L_{0}^{2}(\Omega).\label{BI02}
\end{align}

对于上述混合变分问题(\ref{BI2})-(\ref{BI02}), 根据 Brezzi 定理可以得到下面的适定性结果.
\begin{theorem}\label{Tem01}
% 若 (\ref{BI5})和 (\ref{BT2}) 成立, 则
混合变分问题(\ref{BI2})-(\ref{BI02})存在唯一的解
 $\boldsymbol{u}\in H_0^{1}(\Omega; \mathbb{R}^{d}), p\in L_{0}^{2}(\Omega)$, 且
\begin{equation*}
\|\boldsymbol{u}\|_{1}+\|p\|_{0}\lesssim \|\boldsymbol{f}\|_{-1}.
\end{equation*}
\end{theorem}
\begin{prf}
为了得到混合变分问题(\ref{BI2})-(\ref{BI02})解的适定性, 我们来验证  Brezzi 定理的两个条件.
\begin{enumerate}
\item 强制性:
由Poinc\'{a}re 不等式可得强制性
\begin{align}\label{BI5}
\|\boldsymbol{v}\|^{2}_{1}
\lesssim (\nabla\boldsymbol{v},\nabla\boldsymbol{v}),\quad \boldsymbol{v}\in H_0^{1}(\Omega; \mathbb{R}^{d}).
\end{align}
\item Inf-sup 条件
\begin{align}\label{BT2}
\|q\|_{0}\lesssim\sup _{\boldsymbol{v} \in H_0^{1}(\Omega; \mathbb{R}^{d})}
\frac{(\div\boldsymbol{v}, q)}{\|\boldsymbol{v}\|_{1}},\quad\, q\in L_{0}^{2}(\Omega).
\end{align}
由$\div H_0^{1}(\Omega; \mathbb{R}^{d})=L_{0}^{2}(\Omega)$可知, 存在$\boldsymbol{u}\in H_0^{1}(\Omega; \mathbb{R}^{d})$满足
\begin{align*}
\div{\boldsymbol{u}}=q\in L_{0}^{2}(\Omega),\quad \text{且}\quad\|\boldsymbol{u}\|_1\lesssim \|q\|_0.
\end{align*}
于是$\|q\|_0\|\boldsymbol{u}\|_1\lesssim \|q\|_0^{2}=(\div{\boldsymbol{u}},q)$.
进而有
\begin{align}\label{BW2}
\|q\|_{0}\lesssim \frac{(\div{\boldsymbol{u}},q)}{\|\boldsymbol{u}\|_1}\lesssim\sup _{\boldsymbol{v} \in H_0^{1}(\Omega; \mathbb{R}^{d})}
\frac{(\div\boldsymbol{v}, q)}{\|\boldsymbol{v}\|_{1}},\quad\, q\in L_{0}^{2}(\Omega).
\end{align}
即inf-sup 条件成立.
\end{enumerate}
\end{prf}
事实上,inf-sup条件\eqref{BT2}等价于$\div H_0^{1}(\Omega; \mathbb{R}^{d})=L_{0}^{2}(\Omega)$.


\section{Stokes方程的协调混合有限元方法}
设$ V_h \subset H_0^{1}(\Omega; \mathbb{R}^{d})$ 和 $P_h \subset  L_{0}^{2}(\Omega)$为两个有限元空间,则混合变分问题(\ref{BI2})-(\ref{BI02})的有限元离散为 : 找 $\boldsymbol{u}_h \in  V_h$ 和 $p_h \in P_h$满足
\begin{align}
(\nabla\boldsymbol{u}_h,\nabla\boldsymbol{v}) + (\div\boldsymbol{v},p_h) &=(\boldsymbol{f},\boldsymbol{v}),\quad \boldsymbol{v}\in   V_h,\label{CO1}\\
(\div\boldsymbol{u}_h,q) &=0, \quad\quad\quad\, q\in P_h.\label{CO2}
\end{align}
为了得到混合元方法(\ref{CO1})-(\ref{CO2})解的适定性, 我们需要验证Brezzi 定理的两个条件.
\begin{enumerate}[label=(\alph*)]
\item 强制性:
\begin{align}\label{u3}
\|\boldsymbol{v}\|^{2}_{1}
\lesssim (\nabla\boldsymbol{v},\nabla\boldsymbol{v}),\quad \boldsymbol{v}\in  V_h.\end{align}
\item 离散inf-sup 条件
\begin{equation}\label{u4}
\|q\|_{0}\lesssim\sup _{\boldsymbol{v} \in  V_h}
\frac{(\div\boldsymbol{v}, q)}{\|\boldsymbol{v}\|_{1}},\quad\, q\in P_h.
\end{equation}
\end{enumerate}

由于$ V_h \subset H_0^{1}(\Omega; \mathbb{R}^{d})$,连续情形强制性条件\eqref{BI5}意味着离散强制性条件\eqref{u3}.

下面考虑离散inf-sup条件\eqref{u4}. 连续情形,inf-sup条件\eqref{BT2}等价于$\div H_0^{1}(\Omega; \mathbb{R}^{d})=L_{0}^{2}(\Omega)$. 但是,离散inf-sup条件\eqref{u4}并不等价于$\div V_h=P_h$.

\begin{lemma}
记$Q_h :L_0^{2}(\Omega)\rightarrow  P_h$是$L^2$正交投影算子.
假设$\|\boldsymbol{v}\|_1\eqsim \|Q_h\div\boldsymbol{v}\|_0$对所有的$\boldsymbol{v}\in V_h/\ker(\div)$成立, 则离散inf-sup条件\eqref{u4}等价于$Q_h\div V_h=P_h$.
\end{lemma}
\begin{prf}
先来证明离散inf-sup条件\eqref{u4}意味着$Q_h\div V_h=P_h$. 显然$Q_h\div V_h\subseteq P_h$. 若存在$q\in P_h$满足$q$与$Q_h\div V_h$正交,则与离散inf-sup条件\eqref{u4}矛盾,故有$Q_h\div V_h=P_h$.

再证$Q_h\div V_h=P_h$意味着离散inf-sup条件\eqref{u4}. 对任意的$q\in P_h$, 存在$\boldsymbol{v}\in V_h/\ker(\div)$使得$Q_h\div\boldsymbol{v}=q$. 由假设条件可知$\|\boldsymbol{v}\|_1\eqsim \|Q_h\div\boldsymbol{v}\|_0=\|q\|_0$. 于是
\begin{equation*}
\|q\|_0\|\boldsymbol{v}\|_1\lesssim \|q\|_0^2=(Q_h\div\boldsymbol{v}, q)=(\div\boldsymbol{v}, q).
\end{equation*}
故离散inf-sup条件\eqref{u4}成立.
\end{prf}

% 回顾连续性的 Inf-sup条件, 满足:
% \begin{align}
% \|q\|_{0}\lesssim\sup _{\boldsymbol{v} \in H_0^{1}(\Omega; \mathbb{R}^{d})}
% \frac{(\div\boldsymbol{v}, q)}{\|\boldsymbol{v}\|_{1}},\quad \forall q\in L_0^{2}(\Omega).\,\Leftrightarrow\,
% \div H_0^{1}(\Omega,\mathbb{R}^{2})=L_0^{2}(\Omega).
% \end{align}
% 但对任一空间 $Q_{h}$,\,$ V_h$ 而非上面的 $L_0^{2}(\Omega), \,H_0^{1}(\Omega; \mathbb{R}^{d})$ 来说,
% \begin{align*}
% \|q\|_{0}\lesssim\sup _{\boldsymbol{v} \in  V_h} \frac{(\div\boldsymbol{v}, q)}{\|\boldsymbol{v}\|_{1}},\quad \forall q\in Q_h.
% &\,\nRightarrow\,\div  V_h=Q_h.\\
% \|q\|_{0}\lesssim\sup _{\boldsymbol{v} \in  V_h} \frac{(\div\boldsymbol{v}, q)}{\|\boldsymbol{v}\|_{1}},\quad \forall q\in Q.&\,\Leftarrow\,\div  V_h=Q_h.
% \end{align*}
% 所以作 $\tilde{Q}_h :L_0^{2}(\Omega)\rightarrow  Q_h$ 的 $L^2$ 正交投影, 满足:
% $$ (\tilde{Q}_h \boldsymbol{v},q)=(\boldsymbol{v},q),\quad\forall\, q\in L_0^{2}(\Omega).$$
% 则有
% \begin{align*}
% \|q\|_{0}
% \lesssim\sup _{\boldsymbol{v} \in  V_h} \frac{(\tilde{Q}_h\div\boldsymbol{v}, q)}{\|\boldsymbol{v}\|_{1}},
% \quad \forall q\in Q_h.\,\Leftrightarrow\,\tilde{Q}_h\div  V_h=Q_h.\\
% \Rightarrow
% \|q\|_{0}
% \lesssim\sup _{\boldsymbol{v} \in  V_h} \frac{(\tilde{Q}_h \div\boldsymbol{v}, q)}{\|\boldsymbol{v}\|_{1}}
% =\sup _{\boldsymbol{v} \in  V_h}
% \frac{(\div\boldsymbol{v}, q)}{\|\boldsymbol{v}\|_{1}}
% \quad \forall q\in Q_h.
% \end{align*}

因此,$\div V_h=P_h$意味着离散inf-sup条件\eqref{u4}成立,但是
离散inf-sup条件\eqref{u4}成立并不推出$\div V_h=P_h$. 如果有限元空间$ V_h$和$P_h$满足$\div V_h=P_h$,由方程\eqref{CO2}可得$\div\boldsymbol{u}_h=0$,此时称混合元方法(\ref{CO1})-(\ref{CO2})是\textbf{divergence-free}的,或\textbf{质量守恒}的. 

% 综上构造的有限元空间要想满足 离散的 Inf-sup 条件, 当不满足 $\div  V_h=Q_h $ 时(一般的 Stokes 方程的有限元空间都不满足), 则必须满足:
% \begin{align*}
% \|q\|_{0}
% \lesssim\sup _{\boldsymbol{v} \in  V_h}
% \frac{(\div\boldsymbol{v}, q)}{\|\boldsymbol{v}\|_{1}},
% \quad \forall q\in Q_h.\,\Leftrightarrow\,\tilde{Q}_h\div  V_h=Q_h.
% \end{align*}
构建的有限元空间 $ V_h$ 和 $P_h$ 必须满足离散inf-sup 条件\eqref{u4}. Fortin\cite{Fortin1977}在1977 年提出了一个简单而实用的判别准则,称之为 Fortin 准则.
\begin{lemma}[Fortin 准则]\label{lem01}
离散inf-sup 条件(\ref{u4})等价于
存在一个有界线性算子 $\Pi_h :H_0^{1}(\Omega; \mathbb{R}^{d})\rightarrow V_h$满足
\begin{align}
 (\div(\boldsymbol{v}-\Pi_h\boldsymbol{v}), q) &=0,\qquad\quad  \boldsymbol{v}\in H_0^{1}(\Omega; \mathbb{R}^{d}),q\in P_{h} ,\label{k1}\\
 \|\Pi_h \boldsymbol{v}\|_{1} &\lesssim\|\boldsymbol{v}\|_{1}, \quad\;  \boldsymbol{v}\in H_0^{1}(\Omega; \mathbb{R}^{d}).\label{k2}
\end{align}
算子$\Pi_h$称为Fortin算子.
\end{lemma}
\begin{prf}
先证明存在Fortin算子意味着离散inf-sup 条件(\ref{u4})成立. 对于$q\in P_h$, 由inf-sup 条件(\ref{BT2})和\eqref{k1}-\eqref{k2} 可得
\begin{align*}
\|q\|_{0}\lesssim\sup _{\boldsymbol{v} \in H_0^{1}(\Omega; \mathbb{R}^{d})}
\frac{(\div\boldsymbol{v}, q)}{\|\boldsymbol{v}\|_{1}}=\sup _{\boldsymbol{v} \in H_0^{1}(\Omega; \mathbb{R}^{d})}
\frac{(\div(\Pi_h\boldsymbol{v}), q)}{\|\boldsymbol{v}\|_{1}}\lesssim\sup _{\boldsymbol{v} \in H_0^{1}(\Omega; \mathbb{R}^{d})}
\frac{(\div(\Pi_h\boldsymbol{v}), q)}{\|\Pi_h\boldsymbol{v}\|_{1}}\leq \sup _{\boldsymbol{v} \in V_h}
\frac{(\div\boldsymbol{v}, q)}{\|\boldsymbol{v}\|_{1}}.
\end{align*}
% 结合inf-sup 条件(\ref{BT2})和 Fortin可得
% \begin{align*}
% \|q\|_{0}
% &\lesssim\sup _{\boldsymbol{v} \in H_0^{1}(\Omega; \mathbb{R}^{d})}
% \frac{(\div\boldsymbol{v}, q)}{\|\boldsymbol{v}\|_{1}}\|q\|_{0}\\
% &\lesssim\sup _{\boldsymbol{v} \in H_0^{1}(\Omega; \mathbb{R}^{d})}
% \frac{(\div\Pi_h\boldsymbol{v}, q)}{\|\boldsymbol{v}\|_{1}}\\
% &\lesssim\sup _{\boldsymbol{v} \in H_0^{1}(\Omega; \mathbb{R}^{d})}
% \frac{(\div\Pi_h\boldsymbol{v}, q)}{\|\Pi_h\boldsymbol{v}\|_{1}}\\
% &=\sup _{\boldsymbol{v}_h \in \Pi_h H_0^{1}(\Omega; \mathbb{R}^{d})}
% \frac{(\div\boldsymbol{v}_h, q)}{\|\boldsymbol{v}_h\|_{1}}\\
% &\lesssim\sup _{\boldsymbol{v} \in  V_h}
% \frac{(\div\boldsymbol{v}, q)}{\|\boldsymbol{v}\|_{1}},\quad \forall q\in Q_{h}.
% \end{align*}
故离散inf-sup 条件(\ref{u4})成立.

再证明离散inf-sup 条件(\ref{u4})成立意味着存在Fortin算子. 
% 当$\boldsymbol{v}\in V_h$时,取$\Pi_{h,1}\boldsymbol{v}=\boldsymbol{v}$显然满足\eqref{k1}-\eqref{k2}.
% 记$H_0^{1}(\Omega; \mathbb{R}^{d})/V_h$为$V_h$在$H_0^{1}(\Omega; \mathbb{R}^{d})$中关于$H^1$内积的正交补.
对于$\boldsymbol{v}\in H_0^{1}(\Omega; \mathbb{R}^{d})$, $Q_h(\div\boldsymbol{v})\in P_h$. 由离散inf-sup条件(\ref{u4})知,存在$\boldsymbol{v}_h\in V_h$满足
\begin{equation*}
Q_h(\div\boldsymbol{v}_h)=Q_h(\div\boldsymbol{v}),\quad \|\boldsymbol{v}_h\|_1\lesssim \|Q_h(\div\boldsymbol{v})\|_0\lesssim \|\boldsymbol{v}\|_1.
\end{equation*}
记$\Pi_{h}\boldsymbol{v}=\boldsymbol{v}_h$, 则$\Pi_{h}$满足\eqref{k1}-\eqref{k2}.
% 最后将两部分合并,对$\boldsymbol{v}\in H_0^{1}(\Omega; \mathbb{R}^{d})$, 存在$H^1$正交分解$\boldsymbol{v}=\boldsymbol{v}_1+\boldsymbol{v}_2$,其中$\boldsymbol{v}_1\in V_h$, $\boldsymbol{v}_2\in H_0^{1}(\Omega; \mathbb{R}^{d})/V_h$,令$\Pi_h\boldsymbol{v}=\Pi_{h,1}\boldsymbol{v}_1+\Pi_{h,2}\boldsymbol{v}_2$. 可知$\Pi_h$是Fortin算子. 
\end{prf}

%  当我们利用 Fortin 准则来验证离散的 inf-sup 条件时,首先需要建立一个有界线性算子
%  $\Pi_h :H_0^{1}(\Omega; \mathbb{R}^{d})\rightarrow  V_h$, 使其满足:
%  \begin{align}
% (\div(\boldsymbol{v}-\Pi_h\boldsymbol{v}),q) &=0,
% \qquad \forall q\in Q_{h} ,\,\forall \boldsymbol{v}\in H_0^{1}(\Omega; \mathbb{R}^{d}),\label{o1}\\
%  \|\Pi_h\boldsymbol{v} \|_{1}  &\lesssim\|\boldsymbol{v}\|_{1},
%  \qquad\, \forall \boldsymbol{v}\in H_0^{1}(\Omega; \mathbb{R}^{d}).\label{o2}
% \end{align}
一般情况下分两步来构造Fortin算子$\Pi_h$.
先构造两个有界线性算子 $\Pi_1,\, \Pi_2:H_0^{1}(\Omega; \mathbb{R}^{d})\rightarrow  V_h$, 使其满足:
 \begin{align}
|\Pi_{1}\boldsymbol{v}|_{1}^2+\sum_{T\in\mathcal T_h}h_T^{-2}\|\boldsymbol{v}-\Pi_{1}\boldsymbol{v}\|_{0,T}^2 &\lesssim |\boldsymbol{v}|_1^2, \qquad\quad\qquad\qquad\; \boldsymbol{v}\in H_0^{1}(\Omega; \mathbb{R}^{d}), \label{fortin:pi1}\\
 \|\Pi_2\boldsymbol{v} \|_{0,T}  &\lesssim\|\boldsymbol{v} \|_{0,\omega_T}+h_T|\boldsymbol{v} |_{1,\omega_T},
 \quad\, \boldsymbol{v}\in H_0^{1}(\Omega; \mathbb{R}^{d}), \label{fortin:pi21}\\
(\div(\boldsymbol{v}-\Pi_2\boldsymbol{v}),q)&=0,
\quad\;\; q\in P_{h}, \boldsymbol{v}\in H_0^{1}(\Omega; \mathbb{R}^{d}),\label{fortin:pi22}
\end{align}
其中$\omega_T$是$\mathcal T_h$中所有与$T$相交非空的单形的并集.
再如下定义有界线性算子$\Pi_{h}: H_0^{1}(\Omega; \mathbb{R}^{d})\rightarrow  V_{h}$:
\begin{equation}\label{PihPi12}
\Pi_{h}\boldsymbol{v}=\Pi_{1}\boldsymbol{v}
+\Pi_{2}(\boldsymbol{v}-\Pi_{1}\boldsymbol{v}),\quad \, \boldsymbol{v}\in H_0^{1}(\Omega; \mathbb{R}^{d}).
\end{equation}
这里有界线性算子$\Pi_{1}$通常用于保证收敛阶, 有界线性算子 $\Pi_{2}$用于保证离散inf-sup 条件.

\begin{lemma}\label{lem:fortinoperator}
假设\eqref{fortin:pi1}-\eqref{fortin:pi22}成立,则由式\eqref{PihPi12}定义的$\Pi_{h}$是Fortin算子,即满足 (\ref{k1}) 和 (\ref{k2}).
\end{lemma}
\begin{prf}	
利用逆不等式、\eqref{fortin:pi21}和\eqref{fortin:pi1},
\begin{equation*}
|\Pi_{2}(\boldsymbol{v}-\Pi_{1}\boldsymbol{v})|_1^2\lesssim \sum_{T\in\mathcal T_h}(h_T^{-2}\|\boldsymbol{v}-\Pi_{1}\boldsymbol{v}\|_{0,T}^2 + |\boldsymbol{v}-\Pi_{1}\boldsymbol{v}|_{1,T}^2)\lesssim |\boldsymbol{v}|_1^2, \quad\boldsymbol{v}\in H_0^{1}(\Omega; \mathbb{R}^{d}).
\end{equation*}
再由\eqref{fortin:pi1}可得
\begin{equation*}
\|\Pi_{h}\boldsymbol{v}\|_{1}\leq\|\Pi_{1}\boldsymbol{v}\|_1
+\|\Pi_{2}(\boldsymbol{v}-\Pi_{1}\boldsymbol{v})\|_1\lesssim \|\Pi_{1}\boldsymbol{v}\|_1+|\boldsymbol{v}|_1\lesssim \|\boldsymbol{v}\|_1, \quad \boldsymbol{v}\in H_0^{1}(\Omega; \mathbb{R}^{d}).
\end{equation*}
故(\ref{k2})成立.
对于$\boldsymbol{v}\in H_0^{1}(\Omega; \mathbb{R}^{d})$和$q\in P_h$, 利用\eqref{fortin:pi22}可得
\begin{equation*}
(\div(\boldsymbol{v}-\Pi_h\boldsymbol{v}),q)
=(\div(\boldsymbol{v}-\Pi_{1}\boldsymbol{v}
-\Pi_{2}(\boldsymbol{v}-\Pi_{1}\boldsymbol{v})),q)=0.
\end{equation*}
从而(\ref{k1})成立.
\end{prf}

\begin{theorem}\label{thm:stokeserrorestimate}
设$(\boldsymbol{u}, p)\in H_0^{1}(\Omega; \mathbb{R}^{d})\times L_{0}^{2}(\Omega)$是Stokes方程\eqref{eq-Stokes2}的解, $(\boldsymbol{u}_h, p_h)\in V_h\times P_h$是混合元方法(\ref{CO1})-(\ref{CO2})的解,则有误差估计
\begin{equation}\label{stokeserroruh}	
|\boldsymbol{u}-\boldsymbol{u}_h|_1\lesssim |\boldsymbol{u}-\Pi_{h}\boldsymbol{u}|_1 + \inf_{q\in P_h}\sup_{\boldsymbol{v}\in V_h}\frac{(\div\boldsymbol{v},p-q)}{|\boldsymbol{v}|_1},
\end{equation}
\begin{equation}\label{stokeserrorph}	
\|p-p_h\|_0\lesssim |\boldsymbol{u}-\Pi_{h}\boldsymbol{u}|_1 + \inf_{q\in P_h}\|p-q\|_0.
\end{equation}
进一步,若有$\div V_h=P_h$,则
\begin{equation}\label{stokeserroruhu}	
|\boldsymbol{u}-\boldsymbol{u}_h|_1\lesssim |\boldsymbol{u}-\Pi_{h}\boldsymbol{u}|_1.
\end{equation}
\end{theorem}
\begin{prf}
将(\ref{CO1})-(\ref{CO2})减去(\ref{BI2})-(\ref{BI02})可得
误差方程
\begin{align}
(\nabla(\boldsymbol{u}-\boldsymbol{u}_h),\nabla\boldsymbol{v}) + (\div\boldsymbol{v},p-p_h) &=0,\quad \boldsymbol{v}\in   V_h,\label{StokesErrorEqn1}\\
(\div(\boldsymbol{u}-\boldsymbol{u}_h),q) &=0, \quad q\in P_h.\label{StokesErrorEqn2}
\end{align}
由\eqref{k1}和误差方程\eqref{StokesErrorEqn2}可知,
\begin{equation}
(\div(\Pi_h\boldsymbol{u}-\boldsymbol{u}_h),q) =0, \quad q\in P_h.\label{StokesErrorEqn3}
\end{equation}
将误差方程\eqref{StokesErrorEqn1}中的$\boldsymbol{v}$用$\Pi_{h}\boldsymbol{u}-\boldsymbol{u}_h$代入,有
\begin{equation*}
|\Pi_{h}\boldsymbol{u}-\boldsymbol{u}_h|_1^2=-(\nabla(\boldsymbol{u}-\Pi_{h}\boldsymbol{u}),\nabla(\Pi_{h}\boldsymbol{u}-\boldsymbol{u}_h))-(\div(\Pi_{h}\boldsymbol{u}-\boldsymbol{u}_h),p-p_h).
\end{equation*}
利用\eqref{StokesErrorEqn3}可得
\begin{equation*}
|\Pi_{h}\boldsymbol{u}-\boldsymbol{u}_h|_1^2=-(\nabla(\boldsymbol{u}-\Pi_{h}\boldsymbol{u}),\nabla(\Pi_{h}\boldsymbol{u}-\boldsymbol{u}_h))-(\div(\Pi_{h}\boldsymbol{u}-\boldsymbol{u}_h),p-q).
\end{equation*}
从而
\begin{equation*}
|\Pi_{h}\boldsymbol{u}-\boldsymbol{u}_h|_1\lesssim |\boldsymbol{u}-\Pi_{h}\boldsymbol{u}|_1 + \inf_{q\in P_h}\sup_{\boldsymbol{v}\in V_h}\frac{(\div\boldsymbol{v},p-q)}{|\boldsymbol{v}|_1}
\end{equation*}
进一步利用三角不等式可得\eqref{stokeserroruh}.

由离散inf-sup 条件\eqref{u4},
\begin{equation*}
\|q-p_h\|_{0}\lesssim\sup _{\boldsymbol{v} \in  V_h}
\frac{(\div\boldsymbol{v}, q-p_h)}{\|\boldsymbol{v}\|_{1}}\lesssim \|p-q\|_0+\sup _{\boldsymbol{v} \in  V_h}
\frac{(\div\boldsymbol{v}, p-p_h)}{\|\boldsymbol{v}\|_{1}}.
\end{equation*}
利用误差方程\eqref{StokesErrorEqn1},
\begin{equation*}
\|q-p_h\|_{0}\lesssim \|p-q\|_0+\sup _{\boldsymbol{v} \in  V_h}
\frac{(\nabla(\boldsymbol{u}_h-\boldsymbol{u}),\nabla\boldsymbol{v})}{\|\boldsymbol{v}\|_{1}}\lesssim \|p-q\|_0+|\boldsymbol{u}-\boldsymbol{u}_h|_1.
\end{equation*}
然后,借助三角不等式和\eqref{stokeserroruh}可得\eqref{stokeserrorph}.

进一步,若有$\div V_h=P_h$,可取$q=Q_hp$, 则\eqref{stokeserroruh}式右端的第二项为零,故\eqref{stokeserroruhu}成立.
\end{prf}

如果$\boldsymbol{u}-\boldsymbol{u}_h$的误差估计只依赖于速度$\boldsymbol{u}$,不依赖于压力$p$,则称混合元方法(\ref{CO1})-(\ref{CO2})是\textbf{压力鲁棒}的. 显然,divergence-free的混合元方法(\ref{CO1})-(\ref{CO2})是压力鲁棒的. 关于Stokes方程数值格式的压力鲁棒性详见\cite{JohnLinkeMerdonNeilanEtAl2017}.

\subsection{压力间断的协调元方法}
 
令$\mathcal{T}_h$是$\Omega$的单形网格剖分,假设$\mathcal{T}_h$是形状正则的.

% 在这之前设 $Q_{K}\mathbf{:L^{2}}(K)\rightarrow P_{0}(K)$ 是 $K$ 上的 $L^{2}$ 正交投影,
% 即对 $\forall\,\boldsymbol{v}\in \mathbf{L^{2}}(K)$, 有:
% $$Q_{K}\boldsymbol{v}=\frac{1}{|K|}\int_{K}\boldsymbol{v}d\boldsymbol{x}.$$
% 且满足:
% $$ (Q_K \boldsymbol{v},q_h)_K=(\boldsymbol{v},q_h)_K,\quad\forall\, q_h\in P_0(K).$$

\subsubsection{两维$P_2$-$P_0$元}
用向量值二次Lagrange元和分片常数分别离散速度和压力,即令
\begin{align*}
 V_{h}&:=\{\boldsymbol{v}\in H_{0}^{1}(\Omega,\mathbb{R}^{2}): \boldsymbol{v}|_T\in \mathbb P_2(T;\mathbb{R}^{2}), T\in \mathcal{T}_h\},
\\
P_{h}&:=\{q\in L_{0}^2(\Omega): q|_T\in \mathbb P_0(T), T\in \mathcal{T}_h\}.
\end{align*}
二次Lagrange元的自由度为
\begin{subequations}\label{quadLagrangeDoF}
\begin{align}
\label{quadLagrangeDoF1}
\boldsymbol{v}(\texttt{v}), &\quad \texttt{v}\in\Delta_0(T), \\
\label{quadLagrangeDoF2}
\int_{e}\boldsymbol{v}\dd s, &\quad e\in\Delta_1(T).
\end{align}
\end{subequations}

\begin{lemma}
$P_2$-$P_0$元满足离散inf-sup条件\eqref{u4}.
\end{lemma}
\begin{prf}
记$I_h^{\rm SZ}: H_{0}^{1}(\Omega,\mathbb{R}^{2})\to V_{h}$为Scott-Zhang插值算子\cite{ScottZhang1990},$I_h^{\rm SZ}$显然满足\eqref{fortin:pi1}.
% \begin{equation*}
% \|I_h^{\rm SZ}\boldsymbol{v}\|_1\lesssim \|\boldsymbol{v}\|_1,\quad \boldsymbol{v}\in H_{0}^{1}(\Omega,\mathbb{R}^{2}).
% \end{equation*} 
引入插值算子$\Pi_2: H_{0}^{1}(\Omega,\mathbb{R}^{2})\to V_{h}$, 定义如下
\begin{align*}
(\Pi_2\boldsymbol{v})(\texttt{v})&=0, \quad\quad\quad\;\texttt{v}\in\Delta_0(\mathcal T_h), \\
\int_{e}\Pi_2\boldsymbol{v}\dd s&=\int_{e}\boldsymbol{v}\dd s, \quad e\in\Delta_1(\mathcal T_h).
\end{align*}
由仿射等价性和迹不等式可得,
\begin{equation*}
\|\Pi_{2}\boldsymbol{v}\|_{0,T}^{2}
\eqsim  h_{T}\sum_{e\in\partial T}\|Q_{0,e}\boldsymbol{v}\|_{0,e}^{2}\lesssim h_{T}\|\boldsymbol{v}\|_{0,\partial T}^{2}\lesssim \|\boldsymbol{v}\|_{0,T}^{2}+h_T^2|\boldsymbol{v}|_{1,T}^{2}.
\end{equation*}
故$\Pi_2$满足\eqref{fortin:pi21}.
% 结合逆不等式得
% \begin{equation*}
% \|\Pi_{2}\boldsymbol{v}\|_{0,T}+h_T|\Pi_{2}\boldsymbol{v}|_{1,T}\lesssim \|\boldsymbol{v}\|_{0,T}+h_T|\boldsymbol{v}|_{1,T}.
% \end{equation*}
利用分部积分可得
\begin{equation*}
(\div(\boldsymbol{v}-\Pi_2\boldsymbol{v}),q)=-(\boldsymbol{v}-\Pi_2\boldsymbol{v},\nabla q)=0,
\quad\;\; q\in P_{h}, \boldsymbol{v}\in H_0^{1}(\Omega; \mathbb{R}^{d}).
\end{equation*}
即$\Pi_2$满足\eqref{fortin:pi22}.

根据\eqref{PihPi12}定义插值算子$\Pi_{h}: H_0^{1}(\Omega; \mathbb{R}^{d})\rightarrow  V_{h}$,即
$\Pi_{h}\boldsymbol{v}=I_h^{\rm SZ}\boldsymbol{v}
+\Pi_{2}(\boldsymbol{v}-I_h^{\rm SZ}\boldsymbol{v})$. 由引理\ref{lem:fortinoperator}可知,$\Pi_{h}$是Fortin算子,故由Fortin 准则可得离散inf-sup条件\eqref{u4}成立.
\end{prf}

\begin{remark}
这里之所以选择二次Lagrange元不是线性Lagrange元来离散速度,是因为线性Lagrange元只有顶点处函数值的自由度, 没有边上函数值积分平均的自由度,这样就无法得到$\int_{T} \div(\boldsymbol{v}-\Pi_2\boldsymbol{v})\dx
= \int_{\partial T}(\boldsymbol{v}-\Pi_2\boldsymbol{v})\cdot\boldsymbol{n}\dd s
=0$. 边上函数值积分平均的自由度保证了$\div V_h$能映满分片常数.
\end{remark}

当$\boldsymbol{u}\in H_0^1(\Omega;\mathbb R^2)\cap H^3(\Omega;\mathbb R^2)$和$p\in L_0^2(\Omega)\cap H^1(\Omega)$时, 由定理~\ref{thm:stokeserrorestimate}可知$P_2$-$P_0$元方法的误差估计
\begin{equation*}
\|\boldsymbol{u}-\boldsymbol{u}_h\|_{1} + \|p-p_h\|_{0}
\lesssim h^2\|\boldsymbol{u}\|_3+h\|p\|_1\lesssim h(\|\boldsymbol{u}\|_3+\|p\|_1).
\end{equation*}
该误差估计的收敛阶对于压力$p$是最优的,但对于速度$\boldsymbol{u}$不是最优的.

将$P_2$-$P_0$元推广到任意$d$维,即$P_d$-$P_0$元,此时离散速度的收敛阶的丢阶现象会更加严重.
下面考虑改进$P_2$-$P_0$元方法.

\subsubsection{SMALL元}

为了证明离散inf-sup条件,关键是要用到自由度\eqref{quadLagrangeDoF2}的法向部分,切向部分并不需要. 为此,只要在向量值一次多项式空间的基础上增加边上的法向泡函数,即可避免丢阶现象. 由此得到SMALL元,参见\cite[Remark 8.4.2]{BoffiBrezziFortin2013}和\cite{BernardiRaugel1981,Fortin1981}. 在SMALL元方法中,压力仍然用分片常数逼近.


速度的形函数空间取为$\mathbb P_1(T;\mathbb R^2)\oplus\mathrm{span}\{\lambda_i\lambda_j\nabla\lambda_k: i\neq j\neq k, 0\leq i,j,k\leq2\}$. %, 其中边泡函数$b_e=\lambda_i\lambda_j$.
自由度为
\begin{subequations}\label{SMALLDoF}
\begin{align}
\label{SMALLDoF1}
\boldsymbol{v}(\texttt{v}), &\quad \texttt{v}\in\Delta_0(T), \\
\label{SMALLDoF2}
\int_{e}\boldsymbol{v}\cdot\boldsymbol{n}\dd s, &\quad e\in\Delta_1(T).
\end{align}
\end{subequations}

\begin{lemma}
形函数空间$\mathbb P_1(T;\mathbb R^2)\oplus\mathrm{span}\{\lambda_i\lambda_j\nabla\lambda_k: i\neq j\neq k, 0\leq i,j,k\leq2\}$由自由度\eqref{SMALLDoF}所唯一确定.
\end{lemma}
\begin{prf}
形函数空间的维数和自由度的个数均为$9$. 设$\boldsymbol{v}=\boldsymbol{q}+c_0\lambda_1\lambda_2\nabla\lambda_0+c_1\lambda_2\lambda_0\nabla\lambda_1+c_2\lambda_0\lambda_1\nabla\lambda_2$满足\eqref{SMALLDoF}中所有自由度为零,其中$\boldsymbol{q}\in\mathbb P_1(T;\mathbb R^2)$, $c_0,c_1,c_2\in\mathbb R$. 由自由度\eqref{SMALLDoF1}可知$\boldsymbol{q}$在所有顶点处取值为零,故$\boldsymbol{q}=0$. 从而$\boldsymbol{v}=c_0\lambda_1\lambda_2\nabla\lambda_0+c_1\lambda_2\lambda_0\nabla\lambda_1+c_2\lambda_0\lambda_1\nabla\lambda_2$. 在边$e_0$上, $\boldsymbol{v}|_{e_0}=(c_0\lambda_1\lambda_2\nabla\lambda_0)|_{e_0}$, 故由自由度\eqref{SMALLDoF2}可知$c_0=0$. 类似可得$c_1=c_2=0$. 于是$\boldsymbol{v}=0$.
\end{prf}

类似$P_2$-$P_0$元,我们可以定义SMALL元整体有限元空间,并证明离散inf-sup条件. Stokes方程SMALL元方法的误差估计为
\begin{equation*}
\|\boldsymbol{u}-\boldsymbol{u}_h\|_{1} + \|p-p_h\|_{0}
\lesssim h(\|\boldsymbol{u}\|_2+\|p\|_1).
\end{equation*}


任意$d$维SMALL元的速度形函数空间为$\mathbb P_1(T;\mathbb R^d)\oplus\mathrm{span}\{b_F\boldsymbol{n}_F, F\in\partial T\}$, 其中$b_F$和$\boldsymbol{n}_F$分别为面$F$的泡函数和法向量. %, 其中边泡函数$b_e=\lambda_i\lambda_j$.
自由度为
% \begin{subequations}%\label{SMALLDoF}
\begin{align*}
% \label{SMALLDoF1}
\boldsymbol{v}(\texttt{v}), &\quad \texttt{v}\in\Delta_0(T), \\
% \label{SMALLDoF2}
\int_{F}\boldsymbol{v}\cdot\boldsymbol{n}\dd s, &\quad F\in\partial T.
\end{align*}
% \end{subequations}


\subsubsection{Crouzeix-Raviart元}

同样为了克服$P_2$-$P_0$元的丢阶现象,Crouzeix-Raviart元\cite{CrouzeixRaviart1973}考虑将压力空间的多项式次数提高,同时对速度空间增补泡函数.

设$k\geq2$, Crouzeix-Raviart元的速度形函数空间为$\mathbb P_k(T;\mathbb R^2)+b_T\mathbb P_{k-2}(T;\mathbb R^2)$, 自由度为
\begin{subequations}\label{CrouzeixRaviart2dDoF}
\begin{align}
\label{CrouzeixRaviart2dDoF1}
\boldsymbol{v}(\texttt{v}), &\quad \texttt{v}\in\Delta_0(T), \\
\label{CrouzeixRaviart2dDoF2}
(\boldsymbol{v}, \boldsymbol{q})_e, &\quad \boldsymbol{q}\in\mathbb P_{k-2}(e;\mathbb R^2), e\in\Delta_1(T), \\
\label{CrouzeixRaviart2dDoF3}
(\boldsymbol{v}, \boldsymbol{q})_T, &\quad \boldsymbol{q}\in\mathbb P_{k-2}(T;\mathbb R^2).
\end{align}
\end{subequations}

\begin{lemma}
设$k\geq2$, 形函数空间$\mathbb P_k(T;\mathbb R^2)+b_T\mathbb P_{k-2}(T;\mathbb R^2)$由自由度\eqref{CrouzeixRaviart2dDoF}所唯一确定.
\end{lemma}
\begin{prf}
形函数空间的维数为$2(\dim\mathbb P_k(T)+\dim\mathbb P_{k-2}(T)-\dim\mathbb P_{k-3}(T))$和自由度的个数为
\begin{equation*}
6+6(k-1)+2\dim\mathbb P_{k-2}(T)=2(\dim\mathbb P_k(T)+\dim\mathbb P_{k-2}(T)-\dim\mathbb P_{k-3}(T)).
\end{equation*} 

设$\boldsymbol{v}\in\mathbb P_k(T;\mathbb R^2)+b_T\mathbb P_{k-2}(T;\mathbb R^2)$满足\eqref{CrouzeixRaviart2dDoF}中所有自由度为零. 由自由度\eqref{CrouzeixRaviart2dDoF1}-\eqref{CrouzeixRaviart2dDoF2}可知$\boldsymbol{v}|_{\partial T}=0$,故$\boldsymbol{v}\in b_T\mathbb P_{k-2}(T;\mathbb R^2)$. 再由自由度\eqref{CrouzeixRaviart2dDoF3}可得$\boldsymbol{v}=0$.
\end{prf}

分别定义离散速度和压力的整体有限元空间
\begin{align*}
 V_{h}&:=\{\boldsymbol{v}\in H_{0}^{1}(\Omega,\mathbb{R}^{2}): \boldsymbol{v}|_T\in \mathbb P_k(T;\mathbb R^2)+b_T\mathbb P_{k-2}(T;\mathbb R^2), T\in \mathcal{T}_h\},
\\
P_{h}&:=\{q\in L_{0}^2(\Omega): q|_T\in \mathbb P_{k-1}(T), T\in \mathcal{T}_h\}.
\end{align*}

\begin{lemma}
设$k\geq2$, Crouzeix-Raviart元满足离散inf-sup条件\eqref{u4}.
\end{lemma}
\begin{prf}
令 $V_{h}^L$为$k$次Lagrange元空间$\{\boldsymbol{v}\in H_{0}^{1}(\Omega,\mathbb{R}^{2}): \boldsymbol{v}|_T\in \mathbb P_k(T;\mathbb R^2), T\in \mathcal{T}_h\}$.
记$I_h^{\rm SZ}: H_{0}^{1}(\Omega,\mathbb{R}^{2})\to V_{h}^L$为Scott-Zhang插值算子\cite{ScottZhang1990},满足
\begin{equation*}
(I_h^{\rm SZ}\boldsymbol{v}, \boldsymbol{q})_e=(\boldsymbol{v}, \boldsymbol{q})_e\quad\forall~\boldsymbol{q}\in\mathbb P_{k-2}(e;\mathbb R^2), e\in\Delta_1(\mathcal T_h).
\end{equation*}
引入插值算子$\Pi_2: H_{0}^{1}(\Omega;\mathbb{R}^{2})\to V_{h}$, 定义如下: 对任意的$T\in\mathcal{T}_h$, $(\Pi_2\boldsymbol{v})|_T\in b_T\nabla\mathbb P_{k-1}(T)$满足
\begin{align*}
(\div(\Pi_2\boldsymbol{v}), q)_T&=(\div\boldsymbol{v}, q)_T, \quad q\in\mathbb P_{k-1}(T)\cap L_0^2(T).
\end{align*}
由仿射等价性可得,
\begin{equation*}
\|\Pi_{2}\boldsymbol{v}\|_{0,T}\lesssim h_T\|Q_{k-1,T}\div\boldsymbol{v}\|_{0,T}\leq h_T\|\div\boldsymbol{v}\|_{0,T}.
\end{equation*}
故$\Pi_2$满足\eqref{fortin:pi21}.
% 结合逆不等式得
% \begin{equation*}
% \|\Pi_{2}\boldsymbol{v}\|_{0,T}+h_T|\Pi_{2}\boldsymbol{v}|_{1,T}\lesssim \|\boldsymbol{v}\|_{0,T}+h_T|\boldsymbol{v}|_{1,T}.
% \end{equation*}
由 $\Pi_{2}\boldsymbol{v}$ 的定义显然有
\begin{equation*}
(\div(\boldsymbol{v}-\Pi_2\boldsymbol{v}),q)_T=0,
\quad\;\; q\in\mathbb P_{k-1}(T)\cap L_0^2(T), T\in\mathcal T_h, \boldsymbol{v}\in H_0^{1}(\Omega; \mathbb{R}^{d}).
\end{equation*}
% 利用分部积分可得
% \begin{equation*}
% (\div(\boldsymbol{v}-\Pi_2\boldsymbol{v}),q)=-(\boldsymbol{v}-\Pi_2\boldsymbol{v},\nabla q)=0,
% \quad\;\; q\in P_{h}, \boldsymbol{v}\in H_0^{1}(\Omega; \mathbb{R}^{d}).
% \end{equation*}
% 即$\Pi_2$满足\eqref{fortin:pi22}.

根据\eqref{PihPi12}定义插值算子$\Pi_{h}: H_0^{1}(\Omega; \mathbb{R}^{d})\rightarrow  V_{h}$,即
$\Pi_{h}\boldsymbol{v}=I_h^{\rm SZ}\boldsymbol{v}
+\Pi_{2}(\boldsymbol{v}-I_h^{\rm SZ}\boldsymbol{v})$. 类似引理~\ref{lem:fortinoperator}的证明可得
\begin{equation*}
\|\Pi_{h}\boldsymbol{v}\|_1\leq\|I_h^{\rm SZ}\boldsymbol{v}\|_1+\|\Pi_{2}(\boldsymbol{v}-I_h^{\rm SZ}\boldsymbol{v})\|_1\lesssim \|\boldsymbol{v}\|_1, \quad\forall~\boldsymbol{v}\in H_0^{1}(\Omega; \mathbb{R}^{d}),
\end{equation*}
以及对任意的 $q\in P_h$成立
\begin{align*}
(\div(\boldsymbol{v}-\Pi_{h}\boldsymbol{v}), q) &= \sum_{T\in\mathcal T_h}(\div(\boldsymbol{v}-I_h^{\rm SZ}\boldsymbol{v}
-\Pi_{2}(\boldsymbol{v}-I_h^{\rm SZ}\boldsymbol{v})), q)_T \\
&= \sum_{T\in\mathcal T_h}(\div(\boldsymbol{v}-I_h^{\rm SZ}\boldsymbol{v}
-\Pi_{2}(\boldsymbol{v}-I_h^{\rm SZ}\boldsymbol{v})), Q_T^0q)_T \\
&= \sum_{T\in\mathcal T_h}(\div(\boldsymbol{v}-I_h^{\rm SZ}\boldsymbol{v}), Q_T^0q)_T =0.
\end{align*}
这表明 $\Pi_{h}$ 是 Fortin 算子,故由Fortin 准则可得离散inf-sup条件\eqref{u4}成立.
\end{prf}

当$\boldsymbol{u}\in H_0^1(\Omega;\mathbb R^2)\cap H^{k+1}(\Omega;\mathbb R^2)$和$p\in L_0^2(\Omega)\cap H^k(\Omega)$时, 由定理~\ref{thm:stokeserrorestimate}可知Crouzeix-Raviart元方法的误差估计
\begin{equation*}
\|\boldsymbol{u}-\boldsymbol{u}_h\|_{1} + \|p-p_h\|_{0}
\lesssim h^k(\|\boldsymbol{u}\|_{k+1}+\|p\|_k).
\end{equation*}

设$k\geq2$, 三维情形Crouzeix-Raviart元\cite[Example 8.7.2]{BoffiBrezziFortin2013}的压力空间仍为分片$k-1$次多项式空间,
速度形函数空间为
\begin{equation*}
\begin{cases}
\mathbb P_2(T;\mathbb R^3)+b_T\mathbb P_{0}(T;\mathbb R^3)+\mathrm{span}\{b_F\boldsymbol{n}_F, F\in\partial T\}, & k=2,\\
\mathbb P_k(T;\mathbb R^3)+b_T\mathbb P_{k-2}(T;\mathbb R^3), & k\geq2.	
\end{cases}
\end{equation*}
自由度为
\begin{align*}
\boldsymbol{v}(\texttt{v}), &\quad \texttt{v}\in\Delta_0(T), \\
(\boldsymbol{v}, \boldsymbol{q})_e, &\quad \boldsymbol{q}\in\mathbb P_{k-2}(e;\mathbb R^3), e\in\Delta_1(T), \\
(\boldsymbol{v}\cdot\boldsymbol{n}, q)_F, &\quad q\in\mathbb P_{0}(F), F\in\Delta_2(T), \text{ 当$k=2$时},\\
(\boldsymbol{v}, \boldsymbol{q})_F, &\quad \boldsymbol{q}\in\mathbb P_{k-3}(F;\mathbb R^3), F\in\Delta_2(T), \text{ 当$k\geq3$时}, \\
(\boldsymbol{v}, \boldsymbol{q})_T, &\quad \boldsymbol{q}\in\mathbb P_{k-2}(T;\mathbb R^3).
\end{align*}
当$k\geq3$时, 自由度的个数为
\begin{align*}
12+18(k-1)+12{k-1\choose 2}+3\dim\mathbb P_{k-2}(T)&=6k^2+6+3\dim\mathbb P_{k-2}(T)\\
&=3\dim\mathbb P_k(T)+3\dim\mathbb P_{k-2}(T)-3\dim\mathbb P_{k-4}(T).
\end{align*}

\subsubsection{Scott-Vogelius元}

两维Scott-Vogelius元\cite{ScottVogelius1985}离散速度和压力的有限元空间为
\begin{align*}
V_{h}&:=\{\boldsymbol{v}\in H_{0}^{1}(\Omega,\mathbb{R}^{2}): \boldsymbol{v}|_T\in \mathbb P_k(T;\mathbb{R}^{2}), T\in \mathcal{T}_h\},
\\
P_{h}&:=\{q\in L_{0}^2(\Omega): q|_T\in \mathbb P_{k-1}(T), T\in \mathcal{T}_h\}.
\end{align*}

对于三角剖分$\mathcal T_h$中的顶点$V$,记$\theta_1, \ldots, \theta_n$为以$V$顶点的角的角度,假设这些角按逆时针排列. 若$V$为内部顶点,令$\theta_{n+1}:=\theta_1$. 定义 
\begin{equation*}
S(V)=\begin{cases}
0, & n=1,\\
\max\{\pi-\theta_1-\theta_{2},\pi-\theta_1-\theta_{n}\}, & n>1, V\in\partial\Omega, \\
\max\{\pi-\theta_1-\theta_{2},\pi-\theta_{n}-\theta_{n+1}\}, & V\not\in\partial\Omega.
\end{cases}
\end{equation*}
显然,$S(V)=0$当且仅当$\mathcal T_h$中所有以$V$为顶点的边落在两条直线上. 此时,称$V$是奇异的. 若$S(V)$是一个很小的正数,则$V$接近奇异的. 因此,$S(V)$度量了$V$的奇异程度.

\begin{lemma}[\cite{ScottVogelius1985}]
设$\mathcal T_h$是拟一致的三角剖分. 假设存在常数$\delta>0$使得
\begin{equation*}
S(V)\geq\delta,\quad v\in\Delta_0(\mathcal T_h).
\end{equation*}
则当$k\geq4$时,Scott-Vogelius元满足离散inf-sup条件,其中常数依赖于$\delta$.
\end{lemma}

Scott-Vogelius元是divergence-free的.

\begin{remark}\rm
在任意维单形剖分的Alfeld加密下,	Scott-Vogelius元对任意的$k\geq d$均满足离散inf-sup条件\cite{GuzmanNeilan2018,Zhang2005,ArnoldQin1992}. 当$1\leq k<d$时,速度形函数空间需要增加修正的Bernardi-Raugel面泡函数.
\end{remark}

\begin{remark}\rm
文献\cite{ChenHuang2024}详细研究了Stokes方程divergence-free的协调元,包括压力间断和压力连续的divergence-free协调元. Divergence-free协调元一般含有超光滑自由度.
\end{remark}




\subsection{压力连续的协调元方法}
这一节考虑压力连续的协调元方法,此时由分部积分公式知$(\div\boldsymbol{v},q)=-(\boldsymbol{v}, \nabla q)$,因此选取的空间$V_h$的维数相对于$\nabla P_h$的维数足够大即可.

\subsubsection{MINI元}

MINI元\cite{ArnoldBrezziFortin1984}用Lagrange元离散压力,在Lagrange元的基础上增补泡函数离散速度,即
\begin{align*}
V_{h}&:=\{\boldsymbol{v}\in H_{0}^{1}(\Omega,\mathbb{R}^{d}): \boldsymbol{v}|_T\in \mathbb P_k(T;\mathbb{R}^{d})+b_T\nabla\mathbb P_k(T), T\in \mathcal{T}_h\},
\\
P_{h}&:=\{q\in H^{1}(\Omega)\cap L_{0}^2(\Omega): q|_T\in \mathbb P_{k}(T), T\in \mathcal{T}_h\},
\end{align*}
其中$k\geq1$. 当$k=1$时,
\begin{equation*}
V_{h}=\{\boldsymbol{v}\in H_{0}^{1}(\Omega,\mathbb{R}^{d}): \boldsymbol{v}|_T\in \mathbb P_1(T;\mathbb{R}^{d})\oplus b_T\mathbb P_0(T;\mathbb{R}^{d}), T\in \mathcal{T}_h\}.
\end{equation*}

MINI元速度空间$V_h$在边界上的自由度同Lagrange元在边界上的自由度是一样的,其内部自由度为
\begin{equation*}
% \label{MINIDoF0}
(\boldsymbol{v}, \boldsymbol{q})_T, \quad \boldsymbol{q}\in\mathbb P_{k-d-1}(T;\mathbb{R}^{d})+\nabla\mathbb P_k(T).
\end{equation*}
特别地,当$k=1$时,其
内部自由度为
\begin{equation*}
% \label{MINIDoF0}
(\boldsymbol{v}, \boldsymbol{q})_T, \quad \boldsymbol{q}\in\mathbb P_{0}(T;\mathbb{R}^{d}).
\end{equation*}

\begin{lemma}
设$k\geq1$, MINI元满足离散inf-sup条件\eqref{u4}.
\end{lemma}
\begin{prf}
证明过程类似于Crouzeix-Raviart元.
记$I_h^{\rm SZ}: H_{0}^{1}(\Omega,\mathbb{R}^{d})\to V_{h}^L$为Scott-Zhang插值算子\cite{ScottZhang1990},其中$V_{h}^L$为$k$次Lagrange元空间$\{\boldsymbol{v}\in H_{0}^{1}(\Omega,\mathbb{R}^{d}): \boldsymbol{v}|_T\in \mathbb P_k(T;\mathbb R^d), T\in \mathcal{T}_h\}$.
引入插值算子$\Pi_2: H_{0}^{1}(\Omega,\mathbb{R}^{d})\to V_{h}$, 定义如下: 对任意的$T\in\mathcal{T}_h$, $(\Pi_2\boldsymbol{v})|_T\in b_T\nabla\mathbb P_{k}(T)$满足
\begin{align*}
(\Pi_2\boldsymbol{v}, \boldsymbol{q})_T&=(\boldsymbol{v}, \boldsymbol{q})_T, \quad \boldsymbol{q}\in\nabla\mathbb P_{k}(T).
\end{align*}
由仿射等价性可得,
\begin{equation*}
\|\Pi_{2}\boldsymbol{v}\|_{0,T}\lesssim  \|\boldsymbol{v}\|_{0,T}.
\end{equation*}
故$\Pi_2$满足\eqref{fortin:pi21}.
% 结合逆不等式得
% \begin{equation*}
% \|\Pi_{2}\boldsymbol{v}\|_{0,T}+h_T|\Pi_{2}\boldsymbol{v}|_{1,T}\lesssim \|\boldsymbol{v}\|_{0,T}+h_T|\boldsymbol{v}|_{1,T}.
% \end{equation*}
利用分部积分可得
\begin{equation*}
(\div(\boldsymbol{v}-\Pi_2\boldsymbol{v}),q)=-(\boldsymbol{v}-\Pi_2\boldsymbol{v},\nabla q)=0,
\quad\;\; q\in P_{h}, \boldsymbol{v}\in H_0^{1}(\Omega; \mathbb{R}^{d}).
\end{equation*}
即$\Pi_2$满足\eqref{fortin:pi22}.

根据\eqref{PihPi12}定义插值算子$\Pi_{h}: H_0^{1}(\Omega; \mathbb{R}^{d})\rightarrow  V_{h}$,即
$\Pi_{h}\boldsymbol{v}=I_h^{\rm SZ}\boldsymbol{v}
+\Pi_{2}(\boldsymbol{v}-I_h^{\rm SZ}\boldsymbol{v})$. 由引理\ref{lem:fortinoperator}可知,$\Pi_{h}$是Fortin算子,故由Fortin 准则可得离散inf-sup条件\eqref{u4}成立.
\end{prf}

当$\boldsymbol{u}\in H_0^1(\Omega;\mathbb R^d)\cap H^{k+1}(\Omega;\mathbb R^d)$和$p\in L_0^2(\Omega)\cap H^{k+1}(\Omega)$时, 由定理~\ref{thm:stokeserrorestimate}可知Crouzeix-Raviart元方法的误差估计
\begin{equation*}
\|\boldsymbol{u}-\boldsymbol{u}_h\|_{1} + \|p-p_h\|_{0}
\lesssim h^k(\|\boldsymbol{u}\|_{k+1}+h\|p\|_{k+1}).
\end{equation*}
该误差估计的收敛阶对于速度$\boldsymbol{u}$是最优的,但对于压力$p$不是最优的.

最低次MINI元因为简单常被用于离散Stokes方程.



\subsubsection{Taylor-Hood元}

Taylor-Hood元\cite{TaylorHood1973,Boffi1994,Boffi1997}对速度$\boldsymbol{u}$和压力$p$均采用Lagrange元离散,其误差估计对速度$\boldsymbol{u}$和压力$p$均是最优的. 具体来说,Taylor-Hood元离散速度$\boldsymbol{u}$和压力$p$的有限元空间分别为
\begin{align*}
V_{h}&:=\{\boldsymbol{v}\in H_{0}^{1}(\Omega,\mathbb{R}^{d}): \boldsymbol{v}|_T\in \mathbb P_k(T;\mathbb{R}^{d}), T\in \mathcal{T}_h\},
\\
P_{h}&:=\{q\in H^{1}(\Omega)\cap L_{0}^2(\Omega): q|_T\in \mathbb P_{k-1}(T), T\in \mathcal{T}_h\},
\end{align*}
其中$k\geq2$.

对于Taylor-Hood元,对单形剖分$\mathcal T_h$作如下假设:

\begin{center}
每一个$d$-维单形$T\in\mathcal T_h$至少有一个顶点位于开区域$\Omega$内部. 
\end{center}

在此网格假设下,Taylor-Hood元满足离散inf-sup条件\eqref{u4},其证明相对复杂,参考\cite{BoffiBrezziFortin2013,Boffi1994,Boffi1997,GiraultScott2003,Chen2014,DieningStornTscherpel2022}. 这里给出文献\cite{DieningStornTscherpel2022}中任意维最低次Taylor-Hood元离散inf-sup条件\eqref{u4}的证明, 关键之处是$\nabla V_{h}$为最低次棱元的子空间.
\begin{lemma}
设$k=2$, Taylor-Hood元满足离散inf-sup条件\eqref{u4}.
\end{lemma}
\begin{prf}
记$I_h^{\rm SZ}: H_{0}^{1}(\Omega,\mathbb{R}^{d})\to V_{h}$为Scott-Zhang插值算子\cite{ScottZhang1990}.
下面考虑插值算子$\Pi_2: H_{0}^{1}(\Omega,\mathbb{R}^{d})\to V_{h}$的构造.

记$\mathcal E_h$, $\mathring{\mathcal E}_h$和$\mathcal E^{\partial}_h$分别为单形剖分$\mathcal T_h$所有一维边、内边和边界边的集合. 对端点为$\texttt{v}_i$和$\texttt{v}_j$的边$e_{ij}\in\mathcal E_h$,令$\boldsymbol{b}_{ij}=\lambda_i\lambda_j\boldsymbol{t}_{ij}/\int_{\Omega}\lambda_i\lambda_j\dx$, 则有
\begin{equation*}
(\div\boldsymbol{b}_{ij}, \lambda_k)=-(\boldsymbol{b}_{ij}, \nabla \lambda_{k})=-\frac{1}{\int_{\Omega}\lambda_{i}\lambda_{j}\dx}\int_{\Omega}\lambda_{i}\lambda_{j}\boldsymbol{t}_{ij}\cdot\nabla\lambda_{k}\dx=-\boldsymbol{t}_{ij}\cdot\nabla\lambda_{k}=\delta_{ik}-\delta_{jk}.
\end{equation*}
易知,当$e_{ij}\in\mathring{\mathcal E}_h$时$\boldsymbol{b}_{ij}\in V_{h}$,当$e_{ij}\in\mathcal E^{\partial}_h$时$\boldsymbol{b}_{ij}\not\in V_{h}$. 为此,对$\boldsymbol{b}_{ij}$进行修正.
对于边界边$e_{ij}\in\mathcal E^{\partial}_h$,由网格假设知,存在一个$\Omega$内的顶点$\texttt{v}_m$使得$e_{im}, e_{jm}\in\mathring{\mathcal E}_h$. 令
\begin{equation*}
\boldsymbol{\psi}_{ij}=\begin{cases}
\boldsymbol{b}_{ij}, & e_{ij}\in\mathring{\mathcal E}_h,\\
\boldsymbol{b}_{im}+\boldsymbol{b}_{mj}, & e_{ij}\in\mathcal E^{\partial}_h,
\end{cases}
\end{equation*}
则所有的$\boldsymbol{\psi}_{ij}$均属于$V_h$, 且成立
\begin{equation*}
(\div\boldsymbol{\psi}_{ij}, \lambda_k)=\delta_{ik}-\delta_{jk}.
\end{equation*}
现在如下定义插值算子$\Pi_2: H_{0}^{1}(\Omega,\mathbb{R}^{d})\to V_{h}$, 
\begin{equation*}
\Pi_2\boldsymbol{v}:=\sum_{e_{ij}\in\mathcal E_h, i<j}(\boldsymbol{v}, \lambda_i\nabla\lambda_j-\lambda_j\nabla\lambda_i)\boldsymbol{\psi}_{ij}.
\end{equation*}
可知
\begin{equation*}
\|\Pi_{2}\boldsymbol{v}\|_{0,T}\lesssim  \|\boldsymbol{v}\|_{0,\omega_T}.
\end{equation*}
故$\Pi_2$满足\eqref{fortin:pi21}.
由$\Pi_{2}\boldsymbol{v}$的定义可得,
\begin{align*}
(\div\Pi_{2}\boldsymbol{v},\lambda_{k})
&=\sum_{e_{ij}\in\mathcal{E}_{h},\,i<j}(\boldsymbol{v},\lambda_{i}\nabla\lambda_{j}-\lambda_{j}\nabla\lambda_{i})(\div\boldsymbol{\psi}_{ij},\lambda_{k}) \\
&=\sum_{e_{ij}\in \mathcal{E}_{h},\,i<j}(\boldsymbol{v},\lambda_{i}\nabla\lambda_{j}-\lambda_{j}\nabla\lambda_{i})(\delta_{ik}-\delta_{jk}) \\
&=\sum_{e_{kj}\in \mathcal{E}_{h},\,k<j}(\boldsymbol{v},\lambda_{k}\nabla\lambda_{j}-\lambda_{j}\nabla\lambda_{k}) -\sum_{e_{ik}\in \mathcal{E}_{h},\,i<k}(\boldsymbol{v},\lambda_{i}\nabla\lambda_{k}-\lambda_{k}\nabla\lambda_{i}) \\
&=\sum_{e_{ki}\in \mathcal{E}_{h},\,i>k}(\boldsymbol{v},\lambda_{k}\nabla\lambda_{i}-\lambda_{i}\nabla\lambda_{k}) + \sum_{e_{ik}\in \mathcal{E}_{h},\,i<k}(\boldsymbol{v}, \lambda_{k}\nabla\lambda_{i}-\lambda_{i}\nabla\lambda_{k}) \\
&=\sum_{e_{ki}\in \mathcal{E}_{h}}(\boldsymbol{v},\lambda_{k}\nabla\lambda_{i}-\lambda_{i}\nabla\lambda_{k}) \\
&=(\boldsymbol{v},\lambda_{k}\nabla 1-\nabla\lambda_{k})=-(\boldsymbol{v},\nabla\lambda_{k})=(\div\boldsymbol{v},\lambda_{k}).
\end{align*}
即可推得
$(\div(\Pi_{2}\boldsymbol{v}-\boldsymbol{v}),\lambda_{k})=0$. 故$\Pi_2$满足\eqref{fortin:pi22}.

根据\eqref{PihPi12}定义插值算子$\Pi_{h}: H_0^{1}(\Omega; \mathbb{R}^{d})\rightarrow  V_{h}$,即
$\Pi_{h}\boldsymbol{v}=I_h^{\rm SZ}\boldsymbol{v}
+\Pi_{2}(\boldsymbol{v}-I_h^{\rm SZ}\boldsymbol{v})$. 由引理\ref{lem:fortinoperator}可知,$\Pi_{h}$是Fortin算子,故由Fortin 准则可得离散inf-sup条件\eqref{u4}成立.
\end{prf}


当$\boldsymbol{u}\in H_0^1(\Omega;\mathbb R^d)\cap H^{k+1}(\Omega;\mathbb R^d)$和$p\in L_0^2(\Omega)\cap H^{k}(\Omega)$时, 由定理~\ref{thm:stokeserrorestimate}可知Taylor-Hood元方法的误差估计
\begin{equation*}
\|\boldsymbol{u}-\boldsymbol{u}_h\|_{1} + \|p-p_h\|_{0}
\lesssim h^k(\|\boldsymbol{u}\|_{k+1}+h\|p\|_{k}).
\end{equation*}
该误差估计的收敛阶对于速度$\boldsymbol{u}$和压力$p$均是最优的.



\section{Stokes方程的非协调混合元方法}

这一节考虑Stokes方程的非协调混合元方法,即$V_h \not\subset H_0^{1}(\Omega; \mathbb{R}^{d})$. 此时,
混合变分问题(\ref{BI2})-(\ref{BI02})的有限元离散为 : 找 $\boldsymbol{u}_h \in  V_h$ 和 $p_h \in P_h$满足
\begin{align}
(\nabla_h\boldsymbol{u}_h,\nabla_h\boldsymbol{v}) + (\div_h\boldsymbol{v},p_h) &=(\boldsymbol{f},\boldsymbol{v}),\quad \boldsymbol{v}\in   V_h,\label{StokesNcfm1}\\
(\div_h\boldsymbol{u}_h,q) &=0, \quad\quad\quad\, q\in P_h,\label{StokesNcfm2}
\end{align}
其中$\nabla_h$和$\div_h$关于网格剖分$\mathcal T_h$分片定义的梯度算子和散度算子.
为了得到非协调元方法(\ref{StokesNcfm1})-(\ref{StokesNcfm2})解的适定性, 我们需要验证Brezzi 定理的两个条件.
\begin{enumerate}[label=(\alph*)]
\item 强制性: 即离散Poincar\'e不等式
\begin{align}\label{StokesNcfmu3}
\|\boldsymbol{v}\|^{2}_{0}
\lesssim (\nabla_h\boldsymbol{v},\nabla_h\boldsymbol{v}),\quad \boldsymbol{v}\in  V_h.\end{align}
\item 离散inf-sup 条件
\begin{equation}\label{StokesNcfmu4}
\|q\|_{0}\lesssim\sup _{\boldsymbol{v} \in  V_h}
\frac{(\div_h\boldsymbol{v}, q)}{\|\nabla_h\boldsymbol{v}\|_{0}},\quad\, q\in P_h.
\end{equation}
\end{enumerate}

类似于引理~\ref{lem01},
离散inf-sup 条件(\ref{StokesNcfmu4})等价于存在Fortin算子,即
存在一个有界线性算子 $\Pi_h :H_0^{1}(\Omega; \mathbb{R}^{d})\rightarrow V_h$满足
\begin{align}
 (\div_h(\boldsymbol{v}-\Pi_h\boldsymbol{v}), q) &=0,\qquad\qquad  \boldsymbol{v}\in H_0^{1}(\Omega; \mathbb{R}^{d}),q\in P_{h} ,\label{ncfmk1}\\
 \|\nabla_h(\Pi_h\boldsymbol{v})\|_{0} &\lesssim\|\nabla_h\boldsymbol{v}\|_{0}, \quad\;  \boldsymbol{v}\in H_0^{1}(\Omega; \mathbb{R}^{d}).\label{ncfmk2}
\end{align}

\begin{theorem}\label{thm:stokesncfmerrorestimate}
设$(\boldsymbol{u}, p)\in H_0^{1}(\Omega; \mathbb{R}^{d})\times L_{0}^{2}(\Omega)$是Stokes方程\eqref{eq-Stokes2}的解, $(\boldsymbol{u}_h, p_h)\in V_h\times P_h$是混合元方法(\ref{StokesNcfm1})-(\ref{StokesNcfm2})的解,则有误差估计
\begin{equation}\label{stokesncfmerroruh}	
\|\nabla_h(\boldsymbol{u}-\boldsymbol{u}_h)\|_0\lesssim \|\nabla_h(\boldsymbol{u}-\Pi_{h}\boldsymbol{u})\|_0 + \sup_{\boldsymbol{v}\in V_h}\frac{E_h(\boldsymbol{u},p;\boldsymbol{v})}{\|\nabla_h\boldsymbol{v}\|_0} + \inf_{q\in P_h}\sup_{\boldsymbol{v}\in V_h}\frac{(\div_h\boldsymbol{v},p-q)}{\|\nabla_h\boldsymbol{v}\|_0},
\end{equation}
\begin{equation}\label{stokesncfmerrorph}	
\|p-p_h\|_0\lesssim \|\nabla_h(\boldsymbol{u}-\Pi_{h}\boldsymbol{u})\|_0 + \sup_{\boldsymbol{v}\in V_h}\frac{E_h(\boldsymbol{u},p;\boldsymbol{v})}{\|\nabla_h\boldsymbol{v}\|_0} + \inf_{q\in P_h}\|p-q\|_0,
\end{equation}
其中$E_h(\boldsymbol{u},p;\boldsymbol{v}):=(\nabla\boldsymbol{u}, \nabla_h\boldsymbol{v}) + (\div_h\boldsymbol{v}, p) -(\boldsymbol{f},\boldsymbol{v})$.
进一步,若有$\div_h V_h=P_h$,则
\begin{equation}\label{stokesncfmerroruhu}	
\|\nabla_h(\boldsymbol{u}-\boldsymbol{u}_h)\|_0\lesssim \|\nabla_h(\boldsymbol{u}-\Pi_{h}\boldsymbol{u})\|_0+ \sup_{\boldsymbol{v}\in V_h}\frac{E_h(\boldsymbol{u},p;\boldsymbol{v})}{\|\nabla_h\boldsymbol{v}\|_0}.
\end{equation}
这里$\displaystyle\sup_{\boldsymbol{v}\in V_h}\frac{E_h(\boldsymbol{u},p;\boldsymbol{v})}{\|\nabla_h\boldsymbol{v}\|_0}$称为非协调元的相容性误差.
\end{theorem}
\begin{prf}
任取$q\in P_h$, 记$\boldsymbol{v}=\Pi_h\boldsymbol{u}-\boldsymbol{u}_h$, 则由\eqref{ncfmk1}和\eqref{StokesNcfm2}可得
\begin{equation*}
(\div_h\boldsymbol{v}, q-p_h)=(\div_h(\Pi_h\boldsymbol{u}-\boldsymbol{u}_h), q-p_h)=0.
\end{equation*}
接着利用\eqref{StokesNcfm1}可得
\begin{align*}
\|\nabla_h(\Pi_{h}\boldsymbol{u}-\boldsymbol{u}_h)\|_0^2&=(\nabla_h(\Pi_{h}\boldsymbol{u}-\boldsymbol{u}_h), \nabla_h\boldsymbol{v}) = (\nabla_h(\Pi_{h}\boldsymbol{u}-\boldsymbol{u}_h), \nabla_h\boldsymbol{v}) + (\div_h\boldsymbol{v}, q-p_h) \\
&= (\nabla_h(\Pi_{h}\boldsymbol{u}), \nabla_h\boldsymbol{v}) + (\div_h\boldsymbol{v}, q)-(\boldsymbol{f},\boldsymbol{v}) \\
&= (\nabla_h(\Pi_{h}\boldsymbol{u}-\boldsymbol{u}), \nabla_h\boldsymbol{v}) + (\div_h\boldsymbol{v}, q-p) + (\nabla\boldsymbol{u}, \nabla_h\boldsymbol{v}) + (\div_h\boldsymbol{v}, p) -(\boldsymbol{f},\boldsymbol{v}).
\end{align*}
故由Cauchy-Schwarz可得
\begin{equation*}
\|\nabla_h(\Pi_{h}\boldsymbol{u}-\boldsymbol{u}_h)\|_0\lesssim \|\nabla_h(\boldsymbol{u}-\Pi_{h}\boldsymbol{u})\|_0 + \sup_{\boldsymbol{v}\in V_h}\frac{E_h(\boldsymbol{u},p;\boldsymbol{v})}{\|\nabla_h\boldsymbol{v}\|_0} + \inf_{q\in P_h}\sup_{\boldsymbol{v}\in V_h}\frac{(\div_h\boldsymbol{v},p-q)}{\|\nabla_h\boldsymbol{v}\|_0}.
\end{equation*}
再结合三角不等式即有\eqref{stokesncfmerroruh}.

对任意的$\boldsymbol{v} \in  V_h$,利用\eqref{StokesNcfm1}可推得
\begin{equation*}	
(\div_h\boldsymbol{v}, p-p_h)=(\div_h\boldsymbol{v}, p)+(\nabla_h\boldsymbol{u}_h,\nabla_h\boldsymbol{v})-(\boldsymbol{f},\boldsymbol{v}) =(\nabla_h(\boldsymbol{u}_h-\boldsymbol{u}),\nabla_h\boldsymbol{v}) + E_h(\boldsymbol{u},p;\boldsymbol{v}).
\end{equation*}
由离散inf-sup 条件\eqref{u4},
\begin{align*}
\|q-p_h\|_{0}&\lesssim\sup _{\boldsymbol{v} \in  V_h}
\frac{(\div_h\boldsymbol{v}, q-p_h)}{\|\nabla_h\boldsymbol{v}\|_0}\lesssim \|p-q\|_0+\sup _{\boldsymbol{v} \in  V_h}
\frac{(\div_h\boldsymbol{v}, p-p_h)}{\|\nabla_h\boldsymbol{v}\|_0} \\
&\lesssim \|p-q\|_0 + \|\nabla_h(\boldsymbol{u}-\boldsymbol{u}_h)\|_0 + \sup_{\boldsymbol{v}\in V_h}\frac{E_h(\boldsymbol{u},p;\boldsymbol{v})}{\|\nabla_h\boldsymbol{v}\|_0}.
\end{align*}
进而借助三角不等式和\eqref{stokesncfmerroruh}可推得\eqref{stokesncfmerrorph}.

进一步,若有$\div_h V_h=P_h$,可取$q=Q_hp$, 则\eqref{stokesncfmerroruh}式右端的第三项为零,故\eqref{stokesncfmerroruhu}成立.
\end{prf}

\subsection{非协调$P_1$-$P_0$元}

向量值非协调线性元的形函数空间为$\mathbb P_1(T;\mathbb R^d)$,自由度为
\begin{equation*}
% \label{MINIDoF0}
\int_F\boldsymbol{v}\dd s, \quad F\in\Delta_{d-1}(T).
\end{equation*}

在文献\cite{CrouzeixRaviart1973}中,分别用非协调线性元和分片常数离散速度和压力,即令
\begin{align*}
 V_{h}
&:=\big\{\boldsymbol{v}\in L^{2}(\Omega,\mathbb{R}^{d}):
\boldsymbol{v}|_T\in \mathbb P_1(T,\mathbb{R}^{d}), T\in \mathcal{T}_h,\,\int_{F}[\boldsymbol{v}]\dd s=0,\,F\in \triangle_{d-1}(\mathcal{T}_h)\big\},
\\
P_{h}&:=\{q\in L_0^2(\Omega): q|_T\in \mathbb P_0(T), T\in \mathcal{T}_h\},
\end{align*}
其中$[\boldsymbol{v}]|_F$为$\boldsymbol{v}$跨过面$F$的跳量; 当$F\subset\partial\Omega$时,$[\boldsymbol{v}]|_F=\boldsymbol{v}$. 显然$ V_h  \not\subset H_0^{1}(\Omega; \mathbb{R}^{d})$. 对于非协调元空间$V_h$,成立离散Poincar\'e不等式 \cite{Brenner2003}
\begin{equation*}
\|\boldsymbol{v}\|_0\lesssim \|\nabla_h\boldsymbol{v}\|_0,\quad\boldsymbol{v}\in V_h.
\end{equation*}


\begin{lemma}
非协调$P_1$-$P_0$元满足$\div_hV_h=P_h$和离散inf-sup 条件\eqref{StokesNcfmu4}.
\end{lemma}
\begin{prf}
如下定义插值算子
$\Pi_{h}:H_0^{1}(\Omega; \mathbb{R}^{d})\to  V_{h}$:
\begin{align}\label{qi1}
\int_{F}\Pi_{h}\boldsymbol{v}\dd s=\int_{F} \boldsymbol{v}\dd s,
\quad \boldsymbol{v}\in H_0^{1}(\Omega; \mathbb{R}^{d}),\,F\in\triangle_{d-1}(\mathcal{T}_h).
\end{align}
由分部积分可得
\begin{equation*}
(\div_h(\boldsymbol{v}-\Pi_h\boldsymbol{v}), q) =0,\quad  \boldsymbol{v}\in H_0^{1}(\Omega; \mathbb{R}^{d}), \, q\in P_{h}.	
\end{equation*}
故$\div_h(\Pi_h\boldsymbol{v})=Q_h(\div\boldsymbol{v})$对$\boldsymbol{v}\in H_0^{1}(\Omega; \mathbb{R}^{d})$成立. 于是, $\div_hV_h=P_h$.
由尺度论证技巧可证
\begin{equation*}
\|\nabla_h(\Pi_h\boldsymbol{v})\|_{0} \lesssim\|\nabla_h\boldsymbol{v}\|_{0}, \quad\;  \boldsymbol{v}\in H_0^{1}(\Omega; \mathbb{R}^{d}).
\end{equation*}
因此,$\Pi_{h}$是Fortin算子,从而由Fortin准则知离散inf-sup 条件\eqref{StokesNcfmu4}成立.
\end{prf}

非协调$P_1$-$P_0$元方法是分片divergence-free的,但不是divergence-free的.

\begin{lemma}
设$(\boldsymbol{u}, p)\in H_0^{1}(\Omega; \mathbb{R}^{d})\times L_{0}^{2}(\Omega)$是Stokes方程\eqref{eq-Stokes2}的解, $(\boldsymbol{u}_h, p_h)\in V_h\times P_h$是非协调$P_1$-$P_0$元方法(\ref{StokesNcfm1})-(\ref{StokesNcfm2})的解. 假设$\boldsymbol{u}\in H^{2}(\Omega; \mathbb{R}^{d})$,$p\in H^{1}(\Omega)$,则有误差估计
\begin{equation}\label{eq:stokesncfmp1p0error}	
\|\nabla_h(\boldsymbol{u}-\boldsymbol{u}_h)\|_0+\|p-p_h\|_0\lesssim h(\|\boldsymbol{u}\|_2+\|p\|_1).
\end{equation}
\end{lemma}
\begin{prf}
由定理~\ref{thm:stokesncfmerrorestimate}和$\Pi_h$的插值误差估计,我们只要估计相容性误差即可.

由分部积分和弱连续性可推得,
\begin{align*}
E_h(\boldsymbol{u},p;\boldsymbol{v})&=(\nabla\boldsymbol{u}, \nabla_h\boldsymbol{v}) + (\div_h\boldsymbol{v}, p) -(\boldsymbol{f},\boldsymbol{v})=(\nabla\boldsymbol{u}, \nabla_h\boldsymbol{v}) + (\div_h\boldsymbol{v}, p) + (\Delta\boldsymbol{u}+\nabla p,\boldsymbol{v}) \\
&=\sum_{T\in\mathcal T_h}\big((\partial_n\boldsymbol{u}, \boldsymbol{v})_{\partial T}+(p, \boldsymbol{v}\cdot\boldsymbol{n})_{\partial T}\big)=\sum_{F\in\mathcal F_h}\big((\partial_n\boldsymbol{u}, [\boldsymbol{v}])_F+(p, [\boldsymbol{v}\cdot\boldsymbol{n}])_F\big) \\
&=\sum_{F\in\mathcal F_h}\big((\partial_n\boldsymbol{u}-Q_{0,F}(\partial_n\boldsymbol{u}), [\boldsymbol{v}-Q_{0,F}\boldsymbol{v}])_F+(p-Q_{0,F}p, [(\boldsymbol{v}-Q_{0,F}\boldsymbol{v})\cdot\boldsymbol{n}])_F\big).
\end{align*}
再利用投影算子$Q_{0,F}$的误差估计可得
\begin{equation*}
E_h(\boldsymbol{u},p;\boldsymbol{v})\lesssim h(\|\boldsymbol{u}\|_2+\|p\|_1).
\end{equation*}
得证.
\end{prf}

由误差估计\eqref{eq:stokesncfmp1p0error}知,非协调$P_1$-$P_0$元方法不是压力鲁棒的.
通过对\eqref{StokesNcfm1}右端项中的$\boldsymbol{v}$进行最低次Raviart–Thomas重构\cite{Linke2014,BrenneckeLinkeMerdonSchoeberl2015},得到的非协调$P_1$-$P_0$元方法是压力鲁棒的. 更多Stokes方程数值方法压力鲁棒性的介绍参见\cite{JohnLinkeMerdonNeilanEtAl2017}.




\subsection{一个新的非协调元}

记二次非协调泡函数
\begin{equation*}
b_{T}^{\rm NC}=2-(d+1)(\lambda_0^2+\lambda_1^2+\cdots+\lambda_d^2).
\end{equation*}
泡函数$b_{T}^{\rm NC}$在单形重心处取值为$1$.

\begin{lemma}
对任意的$F\in\Delta_{d-1}(T)$,
成立
\begin{equation*}
(b_{T}^{\rm NC}, q)_F=0,\quad q\in \mathbb P_1(F).
\end{equation*}
\end{lemma}
\begin{prf}
不妨设面$F$对应顶点$\texttt{v}_0$,故等价于证明 %和$i=0,\ldots, d$
\begin{equation*}
(b_{T}^{\rm NC}, \lambda_i)_F=0,\quad i=1,2,\ldots, d.
\end{equation*}
通过直接计算
\begin{equation*}
\frac{1}{|F|}(b_{T}^{\rm NC}, \lambda_i)_F=\frac{2}{d}-(d+1)\sum_{j=1}^d\frac{1}{|F|}\int_F\lambda_j^2\lambda_i\dd s=\frac{2}{d}-(d+1)\frac{(d-1)!}{(d+2)!}(2d+4)=0.
\end{equation*}
得证.
\end{prf}

速度的形函数空间取为$\mathbb P_1(T;\mathbb R^d)+b_{T}^{\rm NC}\mathbb P_0(T;\mathbb R^d)$,自由度为
\begin{align*}
\int_F\boldsymbol{v}\dd s&, \quad F\in\Delta_{d-1}(T),\\
\int_T\boldsymbol{v}\dx.&
\end{align*}
压力的形函数空间取为$\mathbb P_1(T)$

\begin{itemize}
\item 压力用分片线性元离散,由此给出Stokes方程的一种非协调元方法,进行误差分析和数值试验
\item 压力用非协调线性元离散,由此给出Stokes方程的一种非协调元方法,进行误差分析和数值试验
\item 压力用线性元Lagrange元离散,由此给出Stokes方程的一种非协调元方法,进行误差分析和数值试验
\end{itemize}



\subsection{Divergence-free非协调元}

文献\cite{XieXuXue2008}在$H(\div)$协调元的基础上,通过泡函数增加切向连续性来构造divergence-free非协调元.
文献\cite{XieXuXue2008}给出了两维、三维低阶divergence-free非协调元,这里统一给出任意维divergence-free非协调元.

记$\mathbb K$为所有$d$阶反对称矩阵所组成的线性空间. 设$d$维单形$T$的顶点为$\texttt{v}_0, \texttt{v}_1, \ldots, \texttt{v}_d$,相对应的$(d-1)$维面为$F_i$ ($i=0,1,\ldots, d$), 面$F_i$的法向量记为$\boldsymbol{n}_{F_i}$, 不在引起混淆的情况下简记为$\boldsymbol{n}_{i}$.

\begin{lemma}\label{lem:divfreencfmK}
设$\boldsymbol{\tau}_0, \boldsymbol{\tau}_1, \ldots, \boldsymbol{\tau}_d\in\mathbb K$满足
\begin{equation*}
\boldsymbol{n}_i^{\intercal}\boldsymbol{\tau}_j\boldsymbol{n}_k=\boldsymbol{n}_j^{\intercal}\boldsymbol{\tau}_i\boldsymbol{n}_k,\quad 0\leq i,j,k\leq d, i\neq j, i\neq k, j\neq k, 
\end{equation*}
则$\boldsymbol{\tau}_i=0$, 其中$i=0,1,\ldots, d$.
\end{lemma}
\begin{prf}
由$\boldsymbol{\tau}_i$的反对称性知$\boldsymbol{n}_k^{\intercal}\boldsymbol{\tau}_j\boldsymbol{n}_i=-\boldsymbol{n}_i^{\intercal}\boldsymbol{\tau}_j\boldsymbol{n}_k$, $\boldsymbol{n}_k^{\intercal}\boldsymbol{\tau}_i\boldsymbol{n}_j=-\boldsymbol{n}_j^{\intercal}\boldsymbol{\tau}_i\boldsymbol{n}_k$, 因此
\begin{equation*}
\boldsymbol{n}_k^{\intercal}\boldsymbol{\tau}_j\boldsymbol{n}_i=\boldsymbol{n}_k^{\intercal}\boldsymbol{\tau}_i\boldsymbol{n}_j,\quad 0\leq i,j,k\leq d, i\neq j, i\neq k, j\neq k. 
\end{equation*}
从而,$\boldsymbol{n}_j^{\intercal}\boldsymbol{\tau}_i\boldsymbol{n}_k$关于前两个下标和后两个下标均是对称的. 由对称性可得
\begin{equation*}
\boldsymbol{n}_k^{\intercal}\boldsymbol{\tau}_i\boldsymbol{n}_j=\boldsymbol{n}_i^{\intercal}\boldsymbol{\tau}_k\boldsymbol{n}_j=\boldsymbol{n}_i^{\intercal}\boldsymbol{\tau}_j\boldsymbol{n}_k=\boldsymbol{n}_j^{\intercal}\boldsymbol{\tau}_i\boldsymbol{n}_k.
\end{equation*}
又由$\boldsymbol{\tau}_i$的反对称性知 $\boldsymbol{n}_k^{\intercal}\boldsymbol{\tau}_i\boldsymbol{n}_j=-\boldsymbol{n}_j^{\intercal}\boldsymbol{\tau}_i\boldsymbol{n}_k$. 故
\begin{equation*}
\boldsymbol{n}_j^{\intercal}\boldsymbol{\tau}_i\boldsymbol{n}_k=0,\quad 0\leq i,j,k\leq d, j\neq i, k\neq i. 
\end{equation*}
注意到$\{\boldsymbol{n}_j\boldsymbol{n}_k^{\intercal}: 0\leq j,k\leq d, j\neq i, k\neq i\}$构成了$d$阶矩阵的一组基. 于是$\boldsymbol{\tau}_i=0$,得证.
\end{prf}

引入低次$H(\div)$有限元形函数空间$V^{\div}(T)=\mathbb P_1(T;\mathbb R^d)+\boldsymbol{x}\mathbb P_k(T)$, $k=0,1$. 当$k=0$时,$V^{\div}(T)=\mathbb P_1(T;\mathbb R^d)$为最低次Brezzi-Douglas-Marini (BDM)元\cite{BrezziDouglasMarini1985,BrezziDouglasDuranFortin1987,Nedelec1986}的形函数空间;当$k=1$时,$V^{\div}(T)$为一次Raviart-Thomas (RT)元\cite{RaviartThomas1977,Nedelec1980,ChenHuang2022}的形函数空间. 定义刚体运动空间${\rm RM}(T):=\mathbb P_0(T;\mathbb R^d)+\mathbb K\boldsymbol{x}$. ${\rm RM}(T)$也是第一类最低次Nedelec元的形函数空间\cite{Nedelec1980},且有
\begin{equation*}
{\rm RM}(T)={\rm span}\{\lambda_i\nabla \lambda_j-\lambda_j\nabla \lambda_i: 0\leq i<j\leq d\}, \quad\dim{\rm RM}(T)=\frac{1}{2}d(d+1).
\end{equation*}
当$d=1$时,${\rm RM}(T)=\mathbb P_0(T)$.


现在在低次$H(\div)$有限元的基础上定义divergence-free非协调元. 形函数空间取为$V(T):=V^{\div}(T)\oplus\div(b_T\mathbb P_1(T;\mathbb K))$, 其中$b_T:=\lambda_0\lambda_1\ldots\lambda_d$为泡函数.
自由度为
\begin{subequations}\label{divfreencfmDoFs}
\begin{align}
\label{divfreencfmDoFs1}
(\boldsymbol{v}\cdot\boldsymbol{n}, q)_F, &\quad q\in\mathbb P_{1}(F), F\in\Delta_{d-1}(T),\\
\label{divfreencfmDoFs2}
(\boldsymbol{n}\times\boldsymbol{v}\times\boldsymbol{n}, \boldsymbol{q})_F, &\quad \boldsymbol{q}\in{\rm RM}(F), F\in\Delta_{d-1}(T),\\
\label{divfreencfmDoFs3}
(\boldsymbol{v}, \boldsymbol{q})_T, &\quad \boldsymbol{q}\in\mathbb P_{0}(T;\mathbb R^d)\quad\textrm{ if } k=1.
\end{align}
\end{subequations}

\begin{lemma}
设 $T$ 是一个 $d$ 维单形,$\boldsymbol{\tau}\in\mathbb P_1(T;\mathbb K)$,则$\boldsymbol{v}=\div(b_T\boldsymbol{\tau})$满足$\div\boldsymbol{v}=0$,以及$(\boldsymbol{v}\cdot\boldsymbol{n})|_{\partial T}=0$.
\end{lemma}
\begin{prf}
由$\boldsymbol{\tau}$的反对称性可得$\div\boldsymbol{v}=0$. 对于$F\in\Delta_{d-1}(T)$,由$\boldsymbol{\tau}$的反对称性和$b_T|_F=0$可得
\begin{equation*}
(\boldsymbol{v}\cdot\boldsymbol{n})|_{F}=(\div(b_T\boldsymbol{n}^{\intercal}\boldsymbol{\tau}))|_{F}=(\div(b_T\boldsymbol{n}^{\intercal}\boldsymbol{\tau}\Pi_F))|_{F}=(\div_F(b_T\boldsymbol{n}^{\intercal}\boldsymbol{\tau}\Pi_F))|_{F}=0.
\end{equation*}
\end{prf}

\begin{lemma}\label{lem:divfreencfmunisol}
形函数空间$V(T)=V^{\div}(T)\oplus\div(b_T\mathbb P_1(T;\mathbb K))$由自由度\eqref{divfreencfmDoFs}所唯一确定.
\end{lemma}
\begin{prf}
先证明$V^{\div}(T)\cap\div(b_T\mathbb P_1(T;\mathbb K))=0$. 对于$\boldsymbol{v}\in V^{\div}(T)\cap\div(b_T\mathbb P_1(T;\mathbb K))$, 由$\div\boldsymbol{v}=0$得$\boldsymbol{v}\in\mathbb P_1(T;\mathbb R^d)$. 注意到$\boldsymbol{v}\in\div(b_T\mathbb P_1(T;\mathbb K))$意味着$(\boldsymbol{v}\cdot\boldsymbol{n})|_{\partial T}=0$,故由BDM元的唯一可解性知$\boldsymbol{v}=0$.
于是,形函数空间$V(T)$的维数为
\begin{equation*}
\dim V^{\div}(T)+\dim\mathbb P_1(T;\mathbb K)=d(d+1)+dk +\frac{1}{2}d(d^2-1),
\end{equation*}
恰好等于自由度\eqref{divfreencfmDoFs}的个数.

设$\boldsymbol{v}=\boldsymbol{v}_1+\boldsymbol{v}_2\in V(T)$满足\eqref{divfreencfmDoFs}中所有自由度为零,其中$\boldsymbol{v}_1\in V^{\div}(T)$,$\boldsymbol{v}_2\in \div(b_T\mathbb P_1(T;\mathbb K))$. 易知$\div\boldsymbol{v}_2=0$, 以及$(\boldsymbol{v}_2\cdot\boldsymbol{n})|_{\partial T}=0$.
因此$\div\boldsymbol{v}=\div\boldsymbol{v}_1$, 以及$(\boldsymbol{v}\cdot\boldsymbol{n})|_{\partial T}=(\boldsymbol{v}_1\cdot\boldsymbol{n})|_{\partial T}$. 借助分部积分,由自由度\eqref{divfreencfmDoFs1}和\eqref{divfreencfmDoFs3}可推
\begin{equation*}
(\div\boldsymbol{v}, q)_T=(\boldsymbol{v}\cdot\boldsymbol{v}, q)_{\partial T}-(\boldsymbol{v}, \nabla q)_T=0,\quad q\in\mathbb P_1(T).
\end{equation*}
于是$\boldsymbol{v}_1\in\mathbb P_1(T;\mathbb R^d)$且满足$(\boldsymbol{v}_1\cdot\boldsymbol{n})|_{\partial T}=0$. 由BDM元的唯一可解性知$\boldsymbol{v}_1=0$. 故$\boldsymbol{v}\in \div(b_T\mathbb P_1(T;\mathbb K))$.

设$\boldsymbol{v}=\sum_{i=0}^d\div(b_T\lambda_i\boldsymbol{\tau}_i)$, 其中$\boldsymbol{\tau}_0, \boldsymbol{\tau}_1,\ldots, \boldsymbol{\tau}_d\in\mathbb K$. 由自由度\eqref{divfreencfmDoFs2}可得
\begin{equation*}
\sum_{m=0}^d\big(\div(b_T\lambda_{m}\boldsymbol{\tau}_{m}), \lambda_i\nabla_F \lambda_j-\lambda_j\nabla_F\lambda_i\big)_{F_{\ell}}=\big(\boldsymbol{v}, \lambda_i\nabla_F \lambda_j-\lambda_j\nabla_F\lambda_i\big)_{F_{\ell}}=0,
\end{equation*}
其中$0\leq i,j,\ell\leq d, i\neq\ell, j\neq\ell$. 注意到$(\div b_T)|_{F_{\ell}}=b_{F_{\ell}}\nabla\lambda_{\ell}$,
\begin{equation*}
\sum_{m=0}^d\big(b_{F_{\ell}}\lambda_{m}\boldsymbol{\tau}_{m}\boldsymbol{n}_{\ell}, \lambda_i\nabla_F \lambda_j-\lambda_j\nabla_F\lambda_i\big)_{F_{\ell}}=0.
\end{equation*}
直接计算可得
\begin{equation*}
2\sum_{m\neq\ell}(\nabla \lambda_j)^{\intercal}\boldsymbol{\tau}_{m}\boldsymbol{n}_{\ell} + (\nabla \lambda_j)^{\intercal}\boldsymbol{\tau}_{i}\boldsymbol{n}_{\ell} - 2\sum_{m\neq\ell}(\nabla \lambda_i)^{\intercal}\boldsymbol{\tau}_{m}\boldsymbol{n}_{\ell} - (\nabla \lambda_i)^{\intercal}\boldsymbol{\tau}_{j}\boldsymbol{n}_{\ell}=0,
\end{equation*}
其中$0\leq i,j,\ell\leq d, i\neq\ell, j\neq\ell$. 将上式关于下标$i$从$0$到$d$除了$j$和$\ell$进行求和可得
\begin{equation*}
(2d+1)\sum_{m\neq\ell}(\nabla \lambda_j)^{\intercal}\boldsymbol{\tau}_{m}\boldsymbol{n}_{\ell}=0, \quad 0\leq j,\ell\leq d, j\neq\ell.
\end{equation*}
因此$(\nabla\lambda_j)^{\intercal}\boldsymbol{\tau}_{i}\boldsymbol{n}_{\ell} = (\nabla\lambda_i)^{\intercal}\boldsymbol{\tau}_{j}\boldsymbol{n}_{\ell}$, 也即
\begin{equation*}
\boldsymbol{n}_j^{\intercal}\boldsymbol{\tau}_{i}\boldsymbol{n}_{\ell} = \boldsymbol{n}_i^{\intercal}\boldsymbol{\tau}_{j}\boldsymbol{n}_{\ell}, \quad 0\leq i,j,\ell\leq d, i\neq\ell, j\neq\ell.
\end{equation*}
由引理~\ref{lem:divfreencfmK}知, $\boldsymbol{\tau}_{i}=0$, $i=0,1,\ldots, d$. 故$\boldsymbol{v}=0$.
\end{prf}

分别定义离散速度和压力的整体有限元空间
\begin{align*}
 V_{h}&:=\{\boldsymbol{v}\in L^{2}(\Omega;\mathbb{R}^{d}): \boldsymbol{v}|_T\in V^{\div}(T)\oplus\div(b_T\mathbb P_1(T;\mathbb K)), T\in \mathcal{T}_h,\\
&\qquad\qquad\qquad\qquad\quad \textrm{所有自由度\eqref{divfreencfmDoFs}跨过内部边界是连续的},\\
&\qquad\qquad\qquad\qquad\qquad\quad \textrm{自由度\eqref{divfreencfmDoFs1}-\eqref{divfreencfmDoFs2}在边界上为零}\},
\\
P_{h}&:=\{q\in L_{0}^2(\Omega): q|_T\in \mathbb P_{k}(T), T\in \mathcal{T}_h\}.
\end{align*}
显然有$V_{h}\subset H_0(\div,\Omega)$但$V_{h}\not\subseteq H^1(\Omega;\mathbb R^d)$, 且有弱连续性
\begin{equation}\label{stokesdivfreencfmweakcontinuity}
\int_F[\boldsymbol{v}]\dd s=0,\quad F\in\mathcal F_h.
\end{equation}

\begin{lemma}
Divergence-free非协调元满足$\div V_h=P_h$和离散inf-sup 条件\eqref{StokesNcfmu4}.
\end{lemma}
\begin{prf}
令$\Pi_{h}:H_0^{1}(\Omega; \mathbb{R}^{d})\to  V_{h}$为基于自由度\eqref{divfreencfmDoFs}的节点插值算子,则利用分部积分和尺度论证技巧可推得$\Pi_{h}$是Fortin算子,且有$\div V_h=P_h$.
\end{prf}

Divergence-free非协调元方法是divergence-free的.

\begin{lemma}
设$(\boldsymbol{u}, p)\in H_0^{1}(\Omega; \mathbb{R}^{d})\times L_{0}^{2}(\Omega)$是Stokes方程\eqref{eq-Stokes2}的解, $(\boldsymbol{u}_h, p_h)\in V_h\times P_h$是divergence-free非协调元方法(\ref{StokesNcfm1})-(\ref{StokesNcfm2})的解. 假设$\boldsymbol{u}\in H^{2}(\Omega; \mathbb{R}^{d})$,$p\in H^{1}(\Omega)$,则有误差估计
\begin{equation}\label{stokesdivfreencfmerroruh}
\|\nabla_h(\boldsymbol{u}-\boldsymbol{u}_h)\|_0\lesssim h\|\boldsymbol{u}\|_2,
\end{equation}
\begin{equation}\label{stokesdivfreencfmerrorph}
\|p-p_h\|_0\lesssim h(\|\boldsymbol{u}\|_2+h^{k}\|p\|_1).
\end{equation}
\end{lemma}
\begin{prf}
由定理~\ref{thm:stokesncfmerrorestimate}和$\Pi_h$的插值误差估计,我们只要估计相容性误差即可.

由分部积分和弱连续性可推得,
\begin{align*}
E_h(\boldsymbol{u},p;\boldsymbol{v})&=(\nabla\boldsymbol{u}, \nabla_h\boldsymbol{v}) + (\div\boldsymbol{v}, p) -(\boldsymbol{f},\boldsymbol{v})=(\nabla\boldsymbol{u}, \nabla_h\boldsymbol{v}) + (\div\boldsymbol{v}, p) + (\Delta\boldsymbol{u}+\nabla p,\boldsymbol{v}) \\
&=\sum_{T\in\mathcal T_h}(\partial_n\boldsymbol{u}, \boldsymbol{v})_{\partial T}=\sum_{F\in\mathcal F_h}(\partial_n\boldsymbol{u}, [\boldsymbol{v}])_F \\
&=\sum_{F\in\mathcal F_h}(\partial_n\boldsymbol{u}-Q_{0,F}(\partial_n\boldsymbol{u}), [\boldsymbol{v}-Q_{0,F}\boldsymbol{v}])_F.
\end{align*}
再利用投影算子$Q_{0,F}$的误差估计可得
\begin{equation*}
E_h(\boldsymbol{u},p;\boldsymbol{v})\lesssim h\|\boldsymbol{u}\|_2.
\end{equation*}
得证.
\end{prf}

当$d=2,3$时,我们得到了文献\cite{XieXuXue2008}中的两维、三维低阶divergence-free非协调元.

只要有弱连续性\eqref{stokesdivfreencfmweakcontinuity}, 即可得到误差估计\eqref{stokesdivfreencfmerroruh}和\eqref{stokesdivfreencfmerrorph}. 因此,可以减少自由度\eqref{divfreencfmDoFs2}中的矩量.
令反对称矩阵线性函数约化空间
\begin{equation*}
\mathbb P_1^r(T;\mathbb K)=\mathbb P_1(T;\mathbb K)/\oplus_{i=0}^d(d\lambda_i-1)\mathbb K_i,
\end{equation*}
其中$\mathbb K_i$是面$F_i$上全部切向反对称矩阵所形成的线性空间.

\begin{lemma}
$\dim\mathbb P_1^r(T;\mathbb K)=d^2-1$.
\end{lemma}
\begin{prf}
由于
\begin{equation*}
\sum_{i=0}^d\dim\mathbb K_i=\frac{1}{2}(d-2)(d^2-1),
\end{equation*}
故只需证明$\{(d\lambda_i-1)\mathbb K_i, i=0,1,\ldots, d\}$线性无关. 设
\begin{equation*}
\sum_{i=0}^d(d\lambda_i-1)\boldsymbol{\tau}_i=0,\quad \boldsymbol{\tau}_i\in\mathbb K_i.
\end{equation*}
将重心的坐标代入,即$\lambda_0=\lambda_1=\ldots=\lambda_d=\frac{1}{d+1}$,可推得 
\begin{equation*}
\sum_{i=0}^d\boldsymbol{\tau}_i=0,\quad \sum_{i=0}^d\lambda_i\boldsymbol{\tau}_i=0.
\end{equation*}
由$\lambda_0,\lambda_1,\ldots,\lambda_d$的线性无关性可得$\boldsymbol{\tau}_i=0$, $i=0,1,\ldots, d$.
\end{prf}

% \begin{equation*}
% \mathbb P_1^r(T;\mathbb K)=\{\boldsymbol{\tau}\in \mathbb P_1(T;\mathbb K): (\div(b_T\boldsymbol{\tau}), \boldsymbol{q})_F=0, \boldsymbol{q}\in{\rm RM}(F)/\mathbb P_0(F;\mathbb R^{d-1}), F\in\Delta_{d-1}(T)\},
% \end{equation*}
% 则有
% \begin{equation*}
% \dim\mathbb P_1^r(T;\mathbb K)\geq\frac{1}{2}d(d^2-1)-\frac{1}{2}(d-2)(d^2-1)=d^2-1.
% \end{equation*}
约化后的形函数空间取为$V^r(T):=V^{\div}(T)\oplus\div(b_T\mathbb P_1^r(T;\mathbb K))$.
自由度为
\begin{subequations}\label{reducedivfreencfmDoFs}
\begin{align}
\label{reducedivfreencfmDoFs1}
(\boldsymbol{v}\cdot\boldsymbol{n}, q)_F, &\quad q\in\mathbb P_{1}(F), F\in\Delta_{d-1}(T),\\
\label{reducedivfreencfmDoFs2}
(\boldsymbol{n}\times\boldsymbol{v}\times\boldsymbol{n}, \boldsymbol{q})_F, &\quad \boldsymbol{q}\in\mathbb P_0(F;\mathbb R^{d-1}), F\in\Delta_{d-1}(T),\\
\label{reducedivfreencfmDoFs3}
(\boldsymbol{v}, \boldsymbol{q})_T, &\quad \boldsymbol{q}\in\mathbb P_{0}(T;\mathbb R^d)\quad\textrm{ if } k=1.
\end{align}
\end{subequations}

\begin{lemma}
形函数空间$V^r(T)=V^{\div}(T)\oplus\div(b_T\mathbb P_1^r(T;\mathbb K))$由自由度\eqref{reducedivfreencfmDoFs}所唯一确定.
\end{lemma}
\begin{prf}
形函数空间$V^r(T)$的维数为
\begin{equation*}
\dim V^r(T)= d(d+1)+dk +d^2-1,
\end{equation*}
等于自由度\eqref{divfreencfmDoFs}的个数为$d(d+1) +d^2-1+dk$.

设$\boldsymbol{v}\in V^r(T)$满足\eqref{reducedivfreencfmDoFs}中所有自由度为零. 
类似于引理~\ref{lem:divfreencfmunisol}的证明可得$\boldsymbol{v}\in \div(b_T\mathbb P_1^r(T;\mathbb K))$. 


设$\boldsymbol{v}=\sum_{i=0}^d\div(b_T\lambda_i\boldsymbol{\tau}_i)$, 其中$\boldsymbol{\tau}_0, \boldsymbol{\tau}_1,\ldots, \boldsymbol{\tau}_d\in\mathbb K$. 由自由度\eqref{reducedivfreencfmDoFs2}可得
\begin{equation*}
\sum_{i=0}^d\int_{F_j}b_{F_j}\lambda_{i}\boldsymbol{\tau}_{i}\nabla\lambda_j\dd s=\int_{F_j}\boldsymbol{n}\times \boldsymbol{v} \times\boldsymbol{n}\dd s=0,\quad j=0,1,\ldots,d.
\end{equation*}
由此可得
% \begin{equation*}
% \sum_{i\neq j}\boldsymbol{\tau}_{i}\boldsymbol{n}_{j}=0,\quad j=0,1,\ldots,d.
% \end{equation*}
% 所以
\begin{equation*}
\sum_{i=0}^d\boldsymbol{\tau}_{i}\boldsymbol{n}_{j}=\boldsymbol{\tau}_{j}\boldsymbol{n}_{j},\quad j=0,1,\ldots,d.
\end{equation*}
由$\boldsymbol{\tau}_{i}$的反对称性可得
\begin{equation*}
\boldsymbol{\tau}_{j}=\boldsymbol{\sigma}_{j}+\sum_{i=0}^d\boldsymbol{\tau}_{i}, \quad j=0,1,\ldots,d,
\end{equation*}
其中$\boldsymbol{\sigma}_{j}\in\mathbb K_j$.
对上式关于$j$进行求和有$d\sum_{i=0}^d\boldsymbol{\tau}_{i}=-\sum_{i=0}^d\boldsymbol{\sigma}_{i}$, 故
\begin{equation*}
\boldsymbol{\tau}_{j}=\boldsymbol{\sigma}_{j}-\frac{1}{d}\sum_{i=0}^d\boldsymbol{\sigma}_{i}, \quad j=0,1,\ldots,d.
\end{equation*}
从而
\begin{equation*}
\sum_{i=0}^d\lambda_i\boldsymbol{\tau}_i=\sum_{i=0}^d\lambda_i\boldsymbol{\sigma}_i-\frac{1}{d}\sum_{i=0}^d\boldsymbol{\sigma}_{i}=\frac{1}{d}\sum_{i=0}^d(d\lambda_i-1)\boldsymbol{\sigma}_i\in \oplus_{i=0}^d(d\lambda_i-1)\mathbb K_i.
\end{equation*}
另一方面,$\sum_{i=0}^d\lambda_i\boldsymbol{\tau}_i\in\mathbb P_1^r(T;\mathbb K)=\mathbb P_1(T;\mathbb K)/\oplus_{i=0}^d(d\lambda_i-1)\mathbb K_i$. 所以,$\sum_{i=0}^d\lambda_i\boldsymbol{\tau}_i=0$, 从而$\boldsymbol{v}=0$.
\end{prf}


约化后的divergence-free非协调元方法具有与\eqref{stokesdivfreencfmerroruh}和\eqref{stokesdivfreencfmerrorph}一样的误差估计.

三维情形$d=3$,约化向量有限元形函数空间为$V^{\div}(T)\oplus\curl(b_T\mathbb P_1^r(T;\mathbb R^3))$, 其中$\mathbb P_1^r(T;\mathbb R^3):=\mathbb P_1(T;\mathbb R^3)/{\rm span}\{(3\lambda_i-1)\nabla\lambda_i, i=0,1,2,3\}$, 与文献\cite{XieXuXue2008}中的相同.

% 在三维情形$d=3$,文献\cite{XieXuXue2008}中的约化向量有限元形函数空间为$V^{\div}(T)\oplus\curl(b_T\mathbb P_1^r(T;\mathbb R^3))$, 其中$\mathbb P_1^r(T;\mathbb R^3):=\mathbb P_1(T;\mathbb R^3)/{\rm span}\{(3\lambda_i-1)\nabla\lambda_i, i=0,1,2,3\}$.



\section{ Stokes 复形}
\subsection{有限元Stokes复形}
回顾二维de Rham复形
\begin{equation*}
0\xrightarrow{} H_0^1(\Omega)\xrightarrow{\curl} H_0(\div,\Omega)\xrightarrow{\div} L_0^2(\Omega)\xrightarrow{}0,
\end{equation*}
\begin{equation*}
\mathbb R\xrightarrow{} H^1(\Omega)\xrightarrow{\curl} H(\div,\Omega)\xrightarrow{\div} L^2(\Omega)\xrightarrow{}0,
\end{equation*}
\begin{equation*}
0\xrightarrow{} H_0^{s+2}(\Omega)\xrightarrow{\curl} H_0^{s+1}(\Omega;\mathbb R^2)\xrightarrow{\div} H_0^s(\Omega)\cap L_0^2(\Omega)\xrightarrow{}0,
\end{equation*}
\begin{equation*}
\mathbb R\xrightarrow{} H^{s+2}(\Omega)\xrightarrow{\curl} H^{s+1}(\Omega;\mathbb R^2)\xrightarrow{\div} H^s(\Omega)\xrightarrow{}0.
\end{equation*}


回顾三维de Rham复形
\begin{equation*}
0\xrightarrow{} H_0^1(\Omega)\xrightarrow{\grad} H_0(\curl,\Omega)\xrightarrow{\curl} H_0(\div,\Omega)\xrightarrow{\div} L_0^2(\Omega)\xrightarrow{}0,
\end{equation*}
\begin{equation*}
\mathbb R\xrightarrow{} H^1(\Omega)\xrightarrow{\grad} H(\curl,\Omega)\xrightarrow{\curl} H(\div,\Omega)\xrightarrow{\div} L^2(\Omega)\xrightarrow{}0,
\end{equation*}
\begin{equation*}
0\xrightarrow{} H_0^{s+3}(\Omega)\xrightarrow{\grad} H_0^{s+2}(\Omega;\mathbb R^3)\xrightarrow{\curl} H_0^{s+1}(\Omega;\mathbb R^3)\xrightarrow{\div} H_0^{s}(\Omega)\cap L_0^2(\Omega)\xrightarrow{}0,
\end{equation*}
\begin{equation*}
\mathbb R\xrightarrow{} H^{s+3}(\Omega)\xrightarrow{\grad} H^{s+2}(\Omega;\mathbb R^3)\xrightarrow{\curl} H^{s+1}(\Omega;\mathbb R^3)\xrightarrow{\div} H^{s}(\Omega)\xrightarrow{}0.
\end{equation*}

