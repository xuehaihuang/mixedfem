% !TEX root = lecture.tex
\chapter{Poisson方程的混合有限元方法}

\section{Poisson~方程}
	%%文献的引用常用以下几种形式:1, 参见文献\cite{a,b,c}; 2, Zhang 等$^{\scriptsize\cite{d}}$ 研究了.... %
	%参考文献的标记与引用
	考虑具有齐次~Dirichlet~边界条件的~Poisson~方程
	\begin{align}\label{1.1}
		\left\{
		\begin{array}{ll}
			-\Delta u= f (x), \ \ x\in \Omega, \\
			u = 0, \ \ x\in \partial\Omega,
		\end{array}
		\right.
	\end{align}
	其中~$\Omega\subset\mathbb{R}^3$~是有界多边形区域. 令 $\boldsymbol\sigma=\nabla u$ , 则
	\begin{align}\label{1.2}
		\left\{
		\begin{array}{ll}
			\boldsymbol\sigma = \nabla u, \ \ x\in \Omega, \\
			-\div\boldsymbol\sigma = f (x), \ \ x\in \Omega, \\
			u = 0, \ \ x\in \partial\Omega.
		\end{array}
		\right.
	\end{align}
	引入函数空间
	\begin{align}\label{1.3}
		H(\div,\Omega)=\{\boldsymbol\tau: \boldsymbol\tau\in L^2(\Omega;\mathbb{R}^3), \div\boldsymbol\tau\in L^2(\Omega)\},
	\end{align}
	并赋予以下函数
	$$\|\boldsymbol\tau\|^2_{H(\div, \Omega)}=\|\boldsymbol\tau\|^2_0+\|\div\boldsymbol\tau\|^2_0.$$
	通过分部积分, 我们有以下混合变分问题: 求 $(\boldsymbol\sigma,u)\in H(\div,\Omega)\times L^2(\Omega)$, 使得
	\begin{align}\label{1.4}
		\left\{
		\begin{array}{ll}
			(\boldsymbol\sigma, \boldsymbol\tau)+&(\div\boldsymbol\tau, u) = 0, \quad \forall \ \boldsymbol\tau\in  H(\div,\Omega), \\
			&(\div\boldsymbol\sigma, v) = -(f, v), \quad \forall \ v\in L^2(\Omega).
		\end{array}
		\right.
	\end{align}
	下面说明混合变分问题 (\ref{1.4}) 是适定的. 首先, 有界性显然. 其次,
	对任意 $\boldsymbol\tau\in H(\div,\Omega)\cap\ker(\div)$ 成立
	$$\|\boldsymbol\tau\|^2_{H(\div, \Omega)}\leq(\boldsymbol\tau, \boldsymbol\tau).$$ 
	最后, 我们需要证明以下~Inf-sup~条件.
	\begin{lemma}
		对于任意 $v\in L^2(\Omega)$, 有以下结果成立
		\begin{align}
			\label{1.5}
			\|v\|_0\lesssim\sup_{\boldsymbol\tau\in H(\div,\Omega)}\dfrac{(\div\boldsymbol\tau, v)}{\|\boldsymbol\tau\|_{H(\div, \Omega)}}.
		\end{align}
	\end{lemma}
	\begin{proof}
		由~De~Rham~复形可知, 对于任意 $v\in L^2(\Omega)$, 存在 $\boldsymbol\tau=\mathcal P v\in H(\div,\Omega)$ 使得
		$$\div\boldsymbol\tau=\div\mathcal P v=v, \qquad \|\boldsymbol\tau\|_{H(\div,\Omega)}\lesssim\|v\|_0,$$
		其中 $\mathcal P$ 为~Poincar\'{e}~算子. 于是有
		$$(\div\boldsymbol\tau, v)=\|v\|^2_0\gtrsim \|v\|_0\|\boldsymbol\tau\|_{H(\div,\Omega)}.$$
		由以上不等式可得(\ref{1.5}).
	\end{proof}
	
	综上可得混合变分问题 (\ref{1.4}) 解的适定性, 于是有以下稳定性结果.
	\begin{theorem}
		混合变分问题 (\ref{1.4}) 存在唯一的解 $(\boldsymbol\sigma, u)\in H(\div,\Omega)\times L^2(\Omega)$, 且
		\begin{align}
			\label{1.6}
			\|\boldsymbol\sigma\|_{H(\div,\Omega)}+\|u\|_0\lesssim\sup_{\boldsymbol\tau\in H(\div,\Omega), v\in L^2(\Omega)}\dfrac{(\boldsymbol\sigma, \boldsymbol\tau)+(\div\boldsymbol\tau, u)+(\div\boldsymbol\sigma, v)}{\|\boldsymbol\tau\|_{H(\div,\Omega)}+\|v\|_0}.
		\end{align}
	\end{theorem}
	
	
	%%%%%%%%%%%%%%%%%%%%%%%%%%%%%%%%%%%%%%%%%%%%%%%%%%%%%%%%%%%%%
	\section{Poisson~方程的混合有限元格式}
	引入以下有限元空间:
	$$\Sigma_h=\{\boldsymbol\tau\in H(\div,\Omega): \boldsymbol\tau|_K\in\mathbb{P}_k(K;\mathbb{R}^3), K\in\mathcal{T}_h\},$$
	$$V_h=\{v\in L^2(\Omega): v|_K\in\mathbb{P}_{k-1}(K), K\in\mathcal{T}_h\}.$$
	Poisson~方程的混合有限元格式如下:
	求 $\boldsymbol\sigma_h\in\Sigma_h$ 和 $u_h\in V_h$, 使得
	\begin{align}
		\label{2.1}
		\left\{
		\begin{array}{ll}
			(\boldsymbol\sigma_h, \boldsymbol\tau_h)+&(\div\boldsymbol\tau_h, u_h) = 0, \quad \forall \ \boldsymbol\tau_h\in\Sigma_h, \\
			&(\div\boldsymbol\sigma_h, v_h) = -(f, v_h), \quad \forall \ v_h\in V_h.
		\end{array}
		\right.
	\end{align}
	我们需要说明离散格式 (\ref{2.1}) 是适定的. 首先, 有界性显然. 其次, 对于任意 $\boldsymbol\tau_h\in\Sigma_h\cap\ker(\div)$, 强制性结果也显然. 下面证明离散的~Inf-sup~条件.
	\begin{lemma}
		对于任意 $v_h\in V_h$, 有以下结果成立
		\begin{align}
			\label{2.2}
			\|v_h\|_0\lesssim\sup_{\boldsymbol\tau_h\in \Sigma_h}\dfrac{(\div\boldsymbol\tau_h, v_h)}{\|\boldsymbol\tau_h\|_{H(\div, \Omega)}}.
		\end{align}
	\end{lemma}
	\begin{proof}
		由~Stokes~复形可知, 对于任意 $v_h\in L^2(\Omega)$, 存在 $\boldsymbol\tau\in H^1(\Omega;\mathbb{R}^3)$ 使得
		$\div\boldsymbol\tau=v_h$, 且有
		\begin{align}\label{2.3}
			\|\boldsymbol\tau\|_1\lesssim\|v_h\|_0.
		\end{align}
		定义插值算子 $I_h:H^1(\Omega;\mathbb{R}^3)\rightarrow\Sigma_h$ 如下: 对于任意 $\boldsymbol\tau\in H^1(\Omega;\mathbb{R}^3)$, 
		\begin{align*}
			(I_h\boldsymbol\tau\cdot\textbf{n}, q)_F&=(\boldsymbol\tau\cdot\textbf{n}, q)_F \quad \forall \ q\in\mathbb{P}_{k}(F), \ F\in\mathcal{F}(K),\\
			(I_h\boldsymbol\tau, q)_K&=(\boldsymbol\tau, q)_K \quad \forall \ q\in\nabla\mathbb{P}_{k-1}(K)\oplus \boldsymbol{x}\times\mathbb{P}_{k-2}(K;\mathbb{R}^3).
		\end{align*}
		且有以下估计式:
		\begin{align}
			\label{2.4}
			\|\boldsymbol\tau-I_h\boldsymbol\tau\|_{0,K}&\lesssim h_K|\boldsymbol\tau|_{1,K},\\
			\label{2.5}
			|\boldsymbol\tau-I_h\boldsymbol\tau|_{1,K}&\lesssim |\boldsymbol\tau|_{1,K}.
		\end{align}
		由 $I_h$ 的定义, 对于任意 $q\in V_h$, 我们有
		\begin{align*}
			&(\div(\boldsymbol\tau-I_h\boldsymbol\tau), q)=\sum_{K\in\mathcal{T}_h}(\div(\boldsymbol\tau-I_h\boldsymbol\tau), q)_K\\
			=&\sum_{K\in\mathcal{T}_h}((\boldsymbol\tau-I_h\boldsymbol\tau)\cdot\textbf{n}, q)_{\partial K}-\sum_{K\in\mathcal{T}_h}(\boldsymbol\tau-I_h\boldsymbol\tau, \nabla q)_K=0.
		\end{align*}
		由此可得 $\div I_h\boldsymbol\tau=\div\boldsymbol\tau=v_h$, 同时记 $I_h\boldsymbol\tau=\boldsymbol\tau_h$.
		根据 (\ref{2.3})-(\ref{2.5}) 可知
		\begin{align*}
			\|\boldsymbol\tau_h\|^2_{H(\div,\Omega)}&=\|\boldsymbol\tau_h\|^2_0+\|\div\boldsymbol\tau_h\|^2\\
			&\lesssim\|\boldsymbol\tau\|^2_0+\|\boldsymbol\tau-\boldsymbol\tau_h\|^2_0+\|\div\boldsymbol\tau_h\|^2_0\\
			&\lesssim\|\boldsymbol\tau\|^2_1+\|v_h\|^2_0\lesssim\|v_h\|^2_0.
		\end{align*}
		进一步, 我们有
		$$(\div\boldsymbol\tau_h, v_h)=\|v_h\|^2_0\gtrsim\|v_h\|_0\|\boldsymbol\tau_h\|_{H(\div,\Omega)}.$$
		(\ref{2.2}) 由上式得证.
	\end{proof}
	
	综上可得混合有限元格式 (\ref{2.1}) 解的适定性, 于是有以下稳定性结果.
	\begin{theorem}
		混合有限元格式 (\ref{2.1}) 存在唯一的解 $(\boldsymbol\sigma_h, u_h)\in \Sigma_h\times V_h$, 且
		\begin{align}
			\label{2.6}
			\|\boldsymbol\sigma_h\|_{H(\div,\Omega)}+\|u_h\|_0\lesssim\sup_{\boldsymbol\tau_h\in \Sigma_h, v_h\in V_h}\dfrac{(\boldsymbol\sigma_h, \boldsymbol\tau_h)+(\div\boldsymbol\tau_h, u_h)+(\div\boldsymbol\sigma_h, v_h)}{\|\boldsymbol\tau_h\|_{H(\div,\Omega)}+\|v_h\|_0}.
		\end{align}
	\end{theorem}
	
		
	%%%%%%%%%%%%%%%%%%%%%%%%%%%%%%%%%%%%%%%%%%%%%%%%%%%%%%%%%%%%%
	\section{误差分析}
	\begin{theorem}\label{th3.1}
		令 $(\boldsymbol\sigma, u)$ 为 (\ref{1.4}) 的解, $(\boldsymbol\sigma_h, u_h)$ 为 (\ref{2.1}) 的解, 假设 $\boldsymbol\sigma\in H^{k+1}(\Omega;\mathbb{R}^3)$ 且 $u\in H^{k}(\Omega)$, 则
		\begin{align}
			\label{3.1}
			\|\boldsymbol\sigma-\boldsymbol\sigma_h\|_{H(\div,\Omega)}+\|u-u_h\|_0\lesssim h^k(\|\boldsymbol\sigma\|_{k+1}+\|u\|_k).
		\end{align}
	\end{theorem}
	\begin{proof}
		通过将 (\ref{1.4}) 与 (\ref{2.1})作差, 得到误差方程:
		\begin{align}
			\label{3.2}
			\left\{
			\begin{array}{ll}
				(\boldsymbol\sigma-\boldsymbol\sigma_h, \boldsymbol\tau_h)+&(\div\boldsymbol\tau_h, u-u_h) = 0, \quad \forall \ \boldsymbol\tau_h\in\Sigma_h, \\
				&(\div(\boldsymbol\sigma-\boldsymbol\sigma_h), v_h) = 0, \quad \forall \ v_h\in V_h.
			\end{array}
			\right.
		\end{align}
		显然有
		\begin{align}\begin{split}
				\label{3.3}
				&\|\boldsymbol\sigma-\boldsymbol\sigma_h\|_{H(\div,\Omega)}+\|u-u_h\|_0\\
				\leq&\|\boldsymbol\sigma-I_h\boldsymbol\sigma\|_{H(\div,\Omega)}+\|u-Q_h u\|_0+\|I_h\boldsymbol\sigma-\boldsymbol\sigma_h\|_{H(\div,\Omega)}+\|Q_h u-u_h\|_0,
			\end{split}
		\end{align}
		其中 $I_h$ 为引理2.1的证明中提及的插值算子, $Q_h$ 为 $k-1$ 次 $L^2$ 正交投影算子. 由 (\ref{2.6}) 和 (\ref{3.2}), 可得
		\begin{align}\begin{split}
				\label{3.4}
				&\|I_h\boldsymbol\sigma-\boldsymbol\sigma_h\|_{H(\div,\Omega)}+\|Q_h u-u_h\|_0\\
				\lesssim&\sup_{\boldsymbol\tau_h\in \Sigma_h, v_h\in V_h}\dfrac{(I_h\boldsymbol\sigma-\boldsymbol\sigma_h,\boldsymbol\tau_h)+(\div\boldsymbol\tau_h, Q_h u-u_h)+(\div(I_h\boldsymbol\sigma-\boldsymbol\sigma_h),v_h)}{\|\boldsymbol\tau_h\|_{H(\div,\Omega)}+\|v_h\|_0}\\
				\lesssim&\sup_{\boldsymbol\tau_h\in \Sigma_h, v_h\in V_h}\dfrac{(I_h\boldsymbol\sigma-\boldsymbol\sigma,\boldsymbol\tau_h)+(\div\boldsymbol\tau_h, Q_h u-u)+(\div(I_h\boldsymbol\sigma-\boldsymbol\sigma),v_h)}{\|\boldsymbol\tau_h\|_{H(\div,\Omega)}+\|v_h\|_0}\lesssim\|I_h\boldsymbol\sigma-\boldsymbol\sigma\|_0.
			\end{split}
		\end{align}
		因此, 由 $I_h$ 和 $Q_h$ 的误差估计可得
		\begin{align}\begin{split}
				\label{3.5}
				&\|\boldsymbol\sigma-\boldsymbol\sigma_h\|_{H(\div,\Omega)}+\|u-u_h\|_0\\
				\leq&\|\boldsymbol\sigma-I_h\boldsymbol\sigma\|_{H(\div,\Omega)}+\|u-Q_h u\|_0+\|I_h\boldsymbol\sigma-\boldsymbol\sigma_h\|_{H(\div,\Omega)}+\|Q_h u-u_h\|_0\\
				\lesssim&\|\boldsymbol\sigma-I_h\boldsymbol\sigma\|_1+\|u-Q_h u\|_0\lesssim h^k(\|\boldsymbol\sigma\|_{k+1}+\|u\|_k).
			\end{split}
		\end{align}
	\end{proof}
	
	此外, 在上述定理的证明过程中, 还能得到以下超收敛结果:
	\begin{align}
		\label{3.6}\|Q_h u-u_h\|_0\lesssim h^{k+1}\|\boldsymbol\sigma\|_{k+1}.
	\end{align}
	下面我们说明通过对偶论证技巧, (\ref{3.6}) 可以达到超2阶收敛的结果.
	
	引入以下网格依赖范数:
	\begin{align*}
		\|\boldsymbol\tau\|^2_{0,h}:=&\|\boldsymbol\tau\|^2_0+\sum_{F\in\mathcal{F}_h}h_F\|\boldsymbol\tau\cdot\textbf{n}\|^2_{0,F},\\
		|v|^2_{1,h}:=&\sum_{K\in\mathcal{T}_h}\|\nabla v\|^2_{0,K}+\sum_{F\in\mathcal{F}_h}h_F^{-1}\|[v]\|^2_{0,F}.
	\end{align*}
	需要说明在以上的范数下, 离散格式 (\ref{2.1}) 同样是适定的. 首先, 有界性显然. 其次, 对于任意 $\boldsymbol\tau_h\in\Sigma_h\cap\ker(\div)$, 强制性结果也显然. 下面证明离散的~Inf-sup~条件.
	\begin{lemma}
		对于任意 $v_h\in V_h$, 有以下结果成立
		\begin{align}
			\label{3.7}
			|v_h|_{1,h}\lesssim\sup_{\boldsymbol\tau_h\in \Sigma_h}\dfrac{(\div\boldsymbol\tau_h, v_h)}{\|\boldsymbol\tau_h\|_{0,h}}.
		\end{align}
	\end{lemma}
	\begin{proof}
		令 $\boldsymbol\tau_h\in\Sigma_h$ 使得对于任意 $F\in\mathcal{F}_h$ 和 $K\in\mathcal{T}_h$ 有
		\begin{align*}
			(\boldsymbol\tau_h\cdot\textbf{n}, q)_F&=h_F^{-1}([v_h], q)_F \quad \forall \ q\in\mathbb{P}_k(F),\\
			(\boldsymbol\tau_h, q)_K&
			=-(\nabla v_h, q)_K \quad \forall \ q\in\nabla\mathbb{P}_{k-1}(K)\oplus\boldsymbol{x}\times\mathbb{P}_{k-2}(K).
		\end{align*}
		使用尺度论证技巧, 可得
		\begin{align}
			\label{3.8}
			\|\boldsymbol\tau_h\|^2_{0,h}\lesssim\sum_{K\in\mathcal{T}_h}\|\boldsymbol\tau_h\|^2_{0,K}\eqsim\sum_{K\in\mathcal{T}_h}\|\nabla v_h\|^2_{0,K}+\sum_{F\in\mathcal{F}_h}h_F^{-1}\|[v_h]\|^2_{0,F}\lesssim|v_h|^2_{1,h}.
		\end{align}
		同时有
		\begin{align*}
			(\div\boldsymbol\tau_h, v_h)&=\sum_{F\in\mathcal{F}_h}(\boldsymbol\tau_h\cdot\textbf{n}, [v_h])_F-\sum_{K\in\mathcal{T}_h}(\boldsymbol\tau_h, \nabla v_h)_K\\
			&=\sum_{F\in\mathcal{F}_h}h_F^{-1}\|[v_h]\|^2_{0,F}+\sum_{K\in\mathcal{T}_h}\|\nabla v_h\|^2_{0,K}=|v_h|^2_{1,h},
		\end{align*}
		将其与 (\ref{3.8}) 联立即可证明 (\ref{3.7}).
	\end{proof}
	
	同样对任意 $\boldsymbol\sigma_h\in\Sigma_h$ 和 $u_h\in V_h$ 有以下稳定性结果:
		\begin{align}
			\label{3.9}
			\|\boldsymbol\sigma_h\|_{0,h}+\|u_h\|_{1,h}\lesssim\sup_{\boldsymbol\tau_h\in \Sigma_h, v_h\in V_h}\dfrac{(\boldsymbol\sigma_h, \boldsymbol\tau_h)+(\div\boldsymbol\tau_h, u_h)+(\div\boldsymbol\sigma_h, v_h)}{\|\boldsymbol\tau_h\|_{0,h}+\|v_h\|_{1,h}}.
		\end{align}
	类似地, 在与定理 \ref{th3.1} 相同的假设下, 可得
	\begin{align}
		\label{3.10}
		\|\boldsymbol\sigma-\boldsymbol\sigma_h\|_{0,h}&\lesssim h^{k+1}\|\boldsymbol\sigma\|_{k+1},\\
		\label{3.11}
		|Q_h u-u_h|_{1,h}&\lesssim h^{k+1}\|\boldsymbol\sigma\|_{k+1}.
	\end{align}
	接下来使用对偶论证来得到 $u_h$ 与 $Q_h u$ 在 $L^2$ 范数下的超2阶收敛结果.
	令 $(\widetilde{\boldsymbol{\sigma}}, \widetilde{u})$ 为以下辅助问题的解:
	\begin{align}
		\label{3.12}
		\left\{
		\begin{array}{ll}
			\widetilde{\boldsymbol{\sigma}} = \nabla\widetilde{u}, \ \ x\in \Omega\\
			-\div\widetilde{\boldsymbol{\sigma}} = Q_h u-u_h \ \ x\in \Omega.
		\end{array}
		\right.
	\end{align}
	假设 $\Omega$ 是凸区域, 那么 $\widetilde{u}\in H^2(\Omega)$ 且
	\begin{align}
		\label{3.13}
		\|\widetilde{\boldsymbol{\sigma}}\|_1+\|\widetilde{u}\|_2\lesssim\|Q_h u-u_h\|_0.
	\end{align}
	\begin{theorem}
		令 $(\boldsymbol\sigma, u)$ 为 (\ref{1.4}) 的解, $(\boldsymbol\sigma_h, u_h)$ 为 (\ref{2.1}) 的解, 假设正则性结果 (\ref{3.13}) 成立, 同时 $\boldsymbol\sigma\in H^{k+1}(\Omega;\mathbb{R}^3)$ 且 $u\in H^{k}(\Omega)$, 则
		\begin{align}
			\label{3.14}
			\|Q_h u-u_h\|_0\lesssim h^{k+2}\|\boldsymbol\sigma\|_{k+1}.
		\end{align}
	\end{theorem}
	\begin{proof}
		根据 $I_h$ 和 $Q_h$ 的定义, 有以下交换图性质:
		\begin{align}
			\label{3.15}
			\div I_h\boldsymbol\sigma=Q_h\div\boldsymbol\sigma \quad \forall \ \boldsymbol\sigma\in H^1(\Omega;\mathbb{R}^3).
		\end{align}
		由 (\ref{3.2}), (\ref{3.12}) 和 (\ref{3.15}) 可得
		\begin{align*}
			\|Q_h u-u_h\|^2_0&=-(\div\widetilde{\boldsymbol{\sigma}}, Q_h u-u_h)\\
			&=-(\div(\widetilde{\boldsymbol{\sigma}}-I_h\widetilde{\boldsymbol{\sigma}}),Q_h u-u_h)-(\div(I_h\widetilde{\boldsymbol{\sigma}}),Q_h u-u_h)\\
			&=-(\div(I_h\widetilde{\boldsymbol{\sigma}}),Q_h u-u_h)=-(\div(I_h\widetilde{\boldsymbol{\sigma}}),u-u_h)\\
			&=(\boldsymbol\sigma-\boldsymbol\sigma_h,I_h\widetilde{\boldsymbol{\sigma}})=(\boldsymbol\sigma-\boldsymbol\sigma_h,I_h\widetilde{\boldsymbol{\sigma}}-\widetilde{\boldsymbol{\sigma}})+(\boldsymbol\sigma-\boldsymbol\sigma_h,\widetilde{\boldsymbol{\sigma}}).
		\end{align*}
		根据 $Q_h$ 的定义以及 (\ref{3.12}), 再使用分部积分, 我们有
		\begin{align*}
			(\boldsymbol\sigma-\boldsymbol\sigma_h,\widetilde{\boldsymbol{\sigma}})&=(\boldsymbol\sigma-\boldsymbol\sigma_h,\nabla\widetilde{u})=-(\div(\boldsymbol\sigma-\boldsymbol\sigma_h),\widetilde{u})\\
			&=(\div(\boldsymbol\sigma-\boldsymbol\sigma_h),Q_h\widetilde{u}-\widetilde{u})=(\div\boldsymbol\sigma,Q_h\widetilde{u}-\widetilde{u}).
		\end{align*}
		再由 $I_h$ 和 $Q_h$ 的误差估计, (\ref{3.10}), (\ref{3.13}) 以及 (\ref{3.15}) 可得
		\begin{align*}
			\|Q_h u-u_h\|^2_0&=(\boldsymbol\sigma-\boldsymbol\sigma_h,I_h\widetilde{\boldsymbol{\sigma}}-\widetilde{\boldsymbol{\sigma}})+(\div\boldsymbol\sigma,Q_h\widetilde{u}-\widetilde{u})\\
			&=(\boldsymbol\sigma-\boldsymbol\sigma_h,I_h\widetilde{\boldsymbol{\sigma}}-\widetilde{\boldsymbol{\sigma}})+(\div\boldsymbol\sigma-\div I_h\boldsymbol\sigma,Q_h\widetilde{u}-\widetilde{u})\\
			&\lesssim\|\boldsymbol\sigma-\boldsymbol\sigma_h\|_0\|I_h\widetilde{\boldsymbol{\sigma}}-\widetilde{\boldsymbol{\sigma}}\|_0+\|\div\boldsymbol\sigma-Q_h\div\boldsymbol\sigma\|_0\|Q_h\widetilde{u}-\widetilde{u}\|_0\\
			&\lesssim h\|\boldsymbol\sigma-\boldsymbol\sigma_h\|_0|\widetilde{\boldsymbol{\sigma}}|_1+h^2\|\div\boldsymbol\sigma-Q_h\div\boldsymbol\sigma\|_0|\widetilde{u}|_2\\
			&\lesssim h^{k+2}\|\boldsymbol\sigma\|_{k+1}\|Q_h u-u_h\|_0.
		\end{align*}
	\end{proof}
	
	%%%%%%%%%%%%%%%%%%%%%%%%%%%%%%%%%%%%%%%%%%%%%%%%%%%%%%%%%%%%%
	\section{后处理}
	在本节中, 我们将利用最优收敛阶结果 (\ref{3.1}) 和超收敛结果 (\ref{3.14}) 来构造关于 $u$ 的具有超收敛的逼近解.
	
	定义有限元空间:
	$$W_h=\{v\in L^2(\Omega): v|_K\in\mathbb{P}_{k+1}(K), K\in\mathcal{T}_h\}.$$
	定义 $u_h^\ast\in W_h$ 为以下问题的解: 对任意 $K\in\mathcal{T}_h$,
	\begin{align}
		\label{4.1}
		\left\{
		\begin{array}{ll}
			(\nabla u_h^\ast,\nabla v)_K = (\boldsymbol\sigma_h,\nabla v)_K, \quad \forall \ v\in\mathbb{P}_{k+1}(K)\\
			\int_K  u_h^\ast\dx = \int_K  u_h\dx.
		\end{array}
		\right.
	\end{align}
	\begin{theorem}
		假设正则性结果 (\ref{3.13}) 成立, 同时 $\boldsymbol\sigma\in H^{k+1}(\Omega;\mathbb{R}^3)$ 且 $u\in H^{k+2}(\Omega)$, 则
		\begin{align}
			\label{4.2}
			\|u-u^\ast_h\|_0\lesssim h^{k+2}(\|\boldsymbol\sigma\|_{k+1}+\|u\|_{k+2}).
		\end{align}
	\end{theorem}
	\begin{proof}
		对于任意 $K\in\mathcal{T}_h$, 有
		\begin{align*}
			\|\nabla(Q^{k+1}_K u-u^\ast_h)\|^2_{0,K}&=(\nabla(Q^{k+1}_K u-u^\ast_h),\nabla(Q^{k+1}_K u-u^\ast_h))_K\\
			&=(\nabla(Q^{k+1}_K u-u),\nabla(Q^{k+1}_K u-u^\ast_h))+(\boldsymbol\sigma-\boldsymbol\sigma_h,\nabla(Q^{k+1}_K u-u^\ast_h)),
		\end{align*}
		其中 $Q^{k+1}_K$ 为 $K$ 上的 $k+1$ 次 $L^2$ 正交投影. 由上述等式可得
		\begin{align*}
			\|\nabla(u-u^\ast_h)\|_{0,K}&\leq\|\nabla(u-Q^{k+1}_K u)\|_{0,K}+\|\nabla(Q^{k+1}_K u-u^\ast_h)\|_{0,K}\\
			&\leq 2\|\nabla(u-Q^{k+1}_K u)\|_{0,K}+\|\boldsymbol\sigma-\boldsymbol\sigma_h\|_{0,K}.
		\end{align*}
		记 $Q^{k+1}_h$ 为全局的$k+1$ 次 $L^2$ 正交投影, 通过 $Q^{k+1}_h$ 的误差估计以及 (\ref{3.10}), 我们有
		\begin{align}
			\label{4.3}
			|u-u^\ast_h|_{1,h}\lesssim|u-Q^{k+1}_h u|_{1,h}+\|\boldsymbol\sigma-\boldsymbol\sigma_h\|_0\lesssim h^{k+1}(\|\boldsymbol\sigma\|_{k+1}+\|u\|_{k+2}).
		\end{align}
		下面考虑 $L^2$ 误差. 容易看出
		\begin{align*}
			Q^0_K(Q^{k+1}_K u-u^\ast_h)&=Q^0_KQ^{k+1}_K u-Q^0_K u^\ast_h=Q^0_K u-Q^0_K u_h\\
			&=Q^0_K Q_h u-Q^0_K u_h=Q^0_K(Q_h u-u_h),
		\end{align*}
		其中 $Q^0_K$ 为 $K$ 上的 $0$ 次 $L^2$ 正交投影. 由 $Q^0_K$ 的误差估计可得
		\begin{align}\begin{split}
			\label{4.4}
			\|u-u^\ast_h\|_{0,K}&\leq \|u-Q^{k+1}_K u\|_{0,K}+\|Q^{k+1}_K u-u^\ast_h-Q^0_K(Q^{k+1}_K u-u^\ast_h)\|_{0,K}+\|Q^0_K(Q_h u-u_h)\|_{0,K}\\
			&\lesssim\|u-Q^{k+1}_K u\|_{0,K}+h_K|Q^{k+1}_K u-u^\ast_h|_{1,K}+\|Q_h u-u_h\|_{0,K}.
			\end{split}
		\end{align}
		通过 $Q^{k+1}_h$ 的误差估计, (\ref{3.14}) 和 (\ref{4.3}), 可以得出
		\begin{align*}
			\|u-u^\ast_h\|_0\lesssim\|u-Q^{k+1}_h u\|_0+h|Q^{k+1}_h u-u^\ast_h|_{1,h}+\|Q_h u-u_h\|_0\lesssim h^{k+2}(\|\boldsymbol\sigma\|_{k+1}+\|u\|_{k+2}).
		\end{align*}
	\end{proof}

\section{杂交化}
考虑 $k=0$的情形,没有自由度
