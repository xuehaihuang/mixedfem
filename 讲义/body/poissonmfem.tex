% !TEX root = lecture.tex
\chapter{Poisson方程的混合有限元方法}

\section{Poisson~方程}
%%文献的引用常用以下几种形式:1, 参见文献\cite{a,b,c}; 2, Zhang 等$^{\scriptsize\cite{d}}$ 研究了.... %
%参考文献的标记与引用
考虑具有齐次~Dirichlet~边界条件的~Poisson~方程
\begin{align}\label{1.1}
\left\{
\begin{array}{ll}
-\Delta u= f (x), \ \ x\in \Omega, \\
u = 0, \ \ x\in \partial\Omega,
\end{array}
\right.
\end{align}
其中~$\Omega\subset\mathbb{R}^d$~是有界多面体区域. 令 $\boldsymbol\sigma=\nabla u$, 则
\begin{align}\label{1.2}
\left\{
\begin{array}{ll}
\boldsymbol\sigma = \nabla u, \ \ x\in \Omega, \\
-\div\boldsymbol\sigma = f (x), \ \ x\in \Omega, \\
u = 0, \ \ x\in \partial\Omega.
\end{array}
\right.
\end{align}
引入函数空间
\begin{align}\label{1.3}
H(\div,\Omega)=\{\boldsymbol\tau: \boldsymbol\tau\in L^2(\Omega;\mathbb{R}^d), \div\boldsymbol\tau\in L^2(\Omega)\},
\end{align}
并赋予以下函数
$$\|\boldsymbol\tau\|^2_{H(\div, \Omega)}=\|\boldsymbol\tau\|^2+\|\div\boldsymbol\tau\|^2.$$
通过分部积分, 我们有以下混合变分问题: 求 $(\boldsymbol\sigma,u)\in H(\div,\Omega)\times L^2(\Omega)$, 使得
\begin{align}\label{1.4}
\left\{
\begin{array}{ll}
&(\boldsymbol\sigma, \boldsymbol\tau)+(\div\boldsymbol\tau, u) = 0, \qquad\quad\;\; \forall \ \boldsymbol\tau\in  H(\div,\Omega), \\
&\qquad\quad\;\;\;(\div\boldsymbol\sigma, v) = -(f, v), \quad \forall \ v\in L^2(\Omega).
\end{array}
\right.
\end{align}
下面说明混合变分问题 (\ref{1.4}) 是适定的. 首先, 有界性显然. 其次,
对任意 $\boldsymbol\tau\in H(\div,\Omega)\cap\ker(\div)$ 成立
$$\|\boldsymbol\tau\|^2_{H(\div, \Omega)}\leq(\boldsymbol\tau, \boldsymbol\tau).$$ 
最后, 我们需要证明以下~Inf-sup~条件.
\begin{lemma}
对于任意 $v\in L^2(\Omega)$, 有以下结果成立
\begin{align}
\label{1.5}
\|v\|\lesssim\sup_{\boldsymbol\tau\in H(\div,\Omega)}\dfrac{(\div\boldsymbol\tau, v)}{\|\boldsymbol\tau\|_{H(\div, \Omega)}}.
\end{align}
\end{lemma}
\begin{proof}
由~De~Rham~复形可知, 对于任意 $v\in L^2(\Omega)$, 存在 $\boldsymbol\tau=\mathcal P v\in H(\div,\Omega)$ 使得
$$\div\boldsymbol\tau=\div\mathcal P v=v, \qquad \|\boldsymbol\tau\|_{H(\div,\Omega)}\lesssim\|v\|,$$
其中 $\mathcal P$ 为~Poincar\'{e}~算子. 于是有
$$(\div\boldsymbol\tau, v)=\|v\|^2\gtrsim \|v\|\|\boldsymbol\tau\|_{H(\div,\Omega)}.$$
由以上不等式可得(\ref{1.5}).
\end{proof}

综上可得混合变分问题 (\ref{1.4}) 解的适定性, 于是有以下稳定性结果.
\begin{theorem}
混合变分问题 (\ref{1.4}) 存在唯一的解 $(\boldsymbol\sigma, u)\in H(\div,\Omega)\times L^2(\Omega)$, 且
\begin{align}
\label{1.6}
\|\boldsymbol\sigma\|_{H(\div,\Omega)}+\|u\|\lesssim\sup_{\boldsymbol\tau\in H(\div,\Omega), v\in L^2(\Omega)}\dfrac{(\boldsymbol\sigma, \boldsymbol\tau)+(\div\boldsymbol\tau, u)+(\div\boldsymbol\sigma, v)}{\|\boldsymbol\tau\|_{H(\div,\Omega)}+\|v\|}.
\end{align}
\end{theorem}


%%%%%%%%%%%%%%%%%%%%%%%%%%%%%%%%%%%%%%%%%%%%%%%%%%%%%%%%%%%%%
\section{Poisson~方程的混合有限元格式}
引入以下有限元空间:
$$\Sigma_h=\{\boldsymbol\tau\in H(\div,\Omega): \boldsymbol\tau|_T\in\mathbb{P}_k(T;\mathbb{R}^d), T\in\mathcal{T}_h\},$$
$$V_h=\{v\in L^2(\Omega): v|_T\in\mathbb{P}_{k-1}(T), T\in\mathcal{T}_h\}.$$
Poisson~方程的混合有限元格式如下:
求 $\boldsymbol\sigma_h\in\Sigma_h$ 和 $u_h\in V_h$, 使得
\begin{align}
\label{2.1}
\left\{
\begin{array}{ll}
(\boldsymbol\sigma_h, \boldsymbol\tau_h)+&(\div\boldsymbol\tau_h, u_h) = 0, \quad \forall \ \boldsymbol\tau_h\in\Sigma_h, \\
&(\div\boldsymbol\sigma_h, v_h) = -(f, v_h), \quad \forall \ v_h\in V_h.
\end{array}
\right.
\end{align}
我们需要说明离散格式 (\ref{2.1}) 是适定的. 首先, 有界性显然. 其次, 对于任意 $\boldsymbol\tau_h\in\Sigma_h\cap\ker(\div)$, 强制性结果也显然. 下面证明离散的~Inf-sup~条件.
\begin{lemma}
对于任意 $v_h\in V_h$, 有以下结果成立
\begin{align}
\label{2.2}
\|v_h\|\lesssim\sup_{\boldsymbol\tau_h\in \Sigma_h}\dfrac{(\div\boldsymbol\tau_h, v_h)}{\|\boldsymbol\tau_h\|_{H(\div, \Omega)}}.
\end{align}
\end{lemma}
\begin{proof}
由~Stokes~复形可知, 对于任意 $v_h\in L^2(\Omega)$, 存在 $\boldsymbol\tau\in H^1(\Omega;\mathbb{R}^d)$ 使得
$\div\boldsymbol\tau=v_h$, 且有
\begin{align}\label{2.3}
\|\boldsymbol\tau\|_1\lesssim\|v_h\|.
\end{align}
定义插值算子 $I_h:H^1(\Omega;\mathbb{R}^d)\rightarrow\Sigma_h$ 如下: 对于任意 $\boldsymbol\tau\in H^1(\Omega;\mathbb{R}^d)$, 
\begin{align*}
(I_h\boldsymbol\tau\cdot\boldsymbol{n}, q)_F&=(\boldsymbol\tau\cdot\boldsymbol{n}, q)_F \quad \forall \ q\in\mathbb{P}_{k}(F), \ F\in\mathcal{F}(T),\\
(I_h\boldsymbol\tau, q)_T&=(\boldsymbol\tau, q)_T \quad \forall \ q\in\nabla\mathbb{P}_{k-1}(T)\oplus \mathbb{P}_{k-2}(T;\mathbb{K})\boldsymbol{x}.
\end{align*}
且有以下估计式:
\begin{align}
\label{2.4}
\|\boldsymbol\tau-I_h\boldsymbol\tau\|_{T}&\lesssim h_T|\boldsymbol\tau|_{1,T},\\
\label{2.5}
|\boldsymbol\tau-I_h\boldsymbol\tau|_{1,T}&\lesssim |\boldsymbol\tau|_{1,T}.
\end{align}
由 $I_h$ 的定义, 对于任意 $q\in V_h$, 我们有
\begin{align*}
&(\div(\boldsymbol\tau-I_h\boldsymbol\tau), q)=\sum_{T\in\mathcal{T}_h}(\div(\boldsymbol\tau-I_h\boldsymbol\tau), q)_T\\
=&\sum_{T\in\mathcal{T}_h}((\boldsymbol\tau-I_h\boldsymbol\tau)\cdot\boldsymbol{n}, q)_{\partial T}-\sum_{T\in\mathcal{T}_h}(\boldsymbol\tau-I_h\boldsymbol\tau, \nabla q)_T=0.
\end{align*}
由此可得 $\div I_h\boldsymbol\tau=\div\boldsymbol\tau=v_h$, 同时记 $I_h\boldsymbol\tau=\boldsymbol\tau_h$.
根据 (\ref{2.3})-(\ref{2.5}) 可知
\begin{align*}
\|\boldsymbol\tau_h\|^2_{H(\div,\Omega)}&=\|\boldsymbol\tau_h\|^2+\|\div\boldsymbol\tau_h\|^2\\
&\lesssim\|\boldsymbol\tau\|^2+\|\boldsymbol\tau-\boldsymbol\tau_h\|^2+\|\div\boldsymbol\tau_h\|^2\\
&\lesssim\|\boldsymbol\tau\|^2_1+\|v_h\|^2\lesssim\|v_h\|^2.
\end{align*}
进一步, 我们有
$$(\div\boldsymbol\tau_h, v_h)=\|v_h\|^2\gtrsim\|v_h\|\|\boldsymbol\tau_h\|_{H(\div,\Omega)}.$$
(\ref{2.2}) 由上式得证.
\end{proof}

综上可得混合有限元格式 (\ref{2.1}) 解的适定性, 于是有以下稳定性结果.
\begin{theorem}
混合有限元格式 (\ref{2.1}) 存在唯一的解 $(\boldsymbol\sigma_h, u_h)\in \Sigma_h\times V_h$, 且
\begin{align}
\label{2.6}
\|\boldsymbol\sigma_h\|_{H(\div,\Omega)}+\|u_h\|\lesssim\sup_{\boldsymbol\tau_h\in \Sigma_h, v_h\in V_h}\dfrac{(\boldsymbol\sigma_h, \boldsymbol\tau_h)+(\div\boldsymbol\tau_h, u_h)+(\div\boldsymbol\sigma_h, v_h)}{\|\boldsymbol\tau_h\|_{H(\div,\Omega)}+\|v_h\|}.
\end{align}
\end{theorem}


%%%%%%%%%%%%%%%%%%%%%%%%%%%%%%%%%%%%%%%%%%%%%%%%%%%%%%%%%%%%%
\section{误差分析}
\begin{theorem}\label{th3.1}
令 $(\boldsymbol\sigma, u)$ 为 (\ref{1.4}) 的解, $(\boldsymbol\sigma_h, u_h)$ 为 (\ref{2.1}) 的解, 假设 $\boldsymbol\sigma\in H^{k+1}(\Omega;\mathbb{R}^d)$ 且 $u\in H^{k}(\Omega)$, 则
\begin{align}
\label{3.1}
\|\boldsymbol\sigma-\boldsymbol\sigma_h\|_{H(\div,\Omega)}+\|u-u_h\|\lesssim h^k(\|\boldsymbol\sigma\|_{k+1}+\|u\|_k).
\end{align}
\end{theorem}
\begin{proof}
通过将 (\ref{1.4}) 与 (\ref{2.1})作差, 得到误差方程:
\begin{align}
\label{3.2}
\left\{
\begin{array}{ll}
(\boldsymbol\sigma-\boldsymbol\sigma_h, \boldsymbol\tau_h)+&(\div\boldsymbol\tau_h, u-u_h) = 0, \quad \forall \ \boldsymbol\tau_h\in\Sigma_h, \\
&(\div(\boldsymbol\sigma-\boldsymbol\sigma_h), v_h) = 0, \quad \forall \ v_h\in V_h.
\end{array}
\right.
\end{align}
显然有
\begin{align}\begin{split}
\label{3.3}
&\|\boldsymbol\sigma-\boldsymbol\sigma_h\|_{H(\div,\Omega)}+\|u-u_h\|\\
\leq&\|\boldsymbol\sigma-I_h\boldsymbol\sigma\|_{H(\div,\Omega)}+\|u-Q_h u\|+\|I_h\boldsymbol\sigma-\boldsymbol\sigma_h\|_{H(\div,\Omega)}+\|Q_h u-u_h\|,
\end{split}
\end{align}
其中 $I_h$ 为引理2.1的证明中提及的插值算子, $Q_h$ 为 $k-1$ 次 $L^2$ 正交投影算子. 由 (\ref{2.6}) 和 (\ref{3.2}), 可得
\begin{align}\begin{split}
\label{3.4}
&\|I_h\boldsymbol\sigma-\boldsymbol\sigma_h\|_{H(\div,\Omega)}+\|Q_h u-u_h\|\\
\lesssim&\sup_{\boldsymbol\tau_h\in \Sigma_h, v_h\in V_h}\dfrac{(I_h\boldsymbol\sigma-\boldsymbol\sigma_h,\boldsymbol\tau_h)+(\div\boldsymbol\tau_h, Q_h u-u_h)+(\div(I_h\boldsymbol\sigma-\boldsymbol\sigma_h),v_h)}{\|\boldsymbol\tau_h\|_{H(\div,\Omega)}+\|v_h\|}\\
\lesssim&\sup_{\boldsymbol\tau_h\in \Sigma_h, v_h\in V_h}\dfrac{(I_h\boldsymbol\sigma-\boldsymbol\sigma,\boldsymbol\tau_h)+(\div\boldsymbol\tau_h, Q_h u-u)+(\div(I_h\boldsymbol\sigma-\boldsymbol\sigma),v_h)}{\|\boldsymbol\tau_h\|_{H(\div,\Omega)}+\|v_h\|}\lesssim\|I_h\boldsymbol\sigma-\boldsymbol\sigma\|.
\end{split}
\end{align}
因此, 由 $I_h$ 和 $Q_h$ 的误差估计可得
\begin{align}\begin{split}
\label{3.5}
&\|\boldsymbol\sigma-\boldsymbol\sigma_h\|_{H(\div,\Omega)}+\|u-u_h\|\\
\leq&\|\boldsymbol\sigma-I_h\boldsymbol\sigma\|_{H(\div,\Omega)}+\|u-Q_h u\|+\|I_h\boldsymbol\sigma-\boldsymbol\sigma_h\|_{H(\div,\Omega)}+\|Q_h u-u_h\|\\
\lesssim&\|\boldsymbol\sigma-I_h\boldsymbol\sigma\|_1+\|u-Q_h u\|\lesssim h^k(\|\boldsymbol\sigma\|_{k+1}+\|u\|_k).
\end{split}
\end{align}
\end{proof}

此外, 在上述定理的证明过程中, 还能得到以下超收敛结果:
\begin{align}
\label{3.6}\|Q_h u-u_h\|\lesssim h^{k+1}\|\boldsymbol\sigma\|_{k+1}.
\end{align}
下面我们说明通过对偶论证技巧, (\ref{3.6}) 可以达到超2阶收敛的结果.

引入以下网格依赖范数:
\begin{align*}
\|\boldsymbol\tau\|^2_{0,h}:=&\|\boldsymbol\tau\|^2+\sum_{F\in\mathcal{F}_h}h_F\|\boldsymbol\tau\cdot\boldsymbol{n}\|^2_{F},\\
|v|^2_{1,h}:=&\sum_{T\in\mathcal{T}_h}\|\nabla v\|^2_{T}+\sum_{F\in\mathcal{F}_h}h_F^{-1}\|[v]\|^2_{F}.
\end{align*}
需要说明在以上的范数下, 离散格式 (\ref{2.1}) 同样是适定的. 首先, 有界性显然. 其次, 对于任意 $\boldsymbol\tau_h\in\Sigma_h\cap\ker(\div)$, 强制性结果也显然. 下面证明离散的~Inf-sup~条件.
\begin{lemma}
对于任意 $v_h\in V_h$, 有以下结果成立
\begin{align}
\label{3.7}
|v_h|_{1,h}\lesssim\sup_{\boldsymbol\tau_h\in \Sigma_h}\dfrac{(\div\boldsymbol\tau_h, v_h)}{\|\boldsymbol\tau_h\|_{0,h}}.
\end{align}
\end{lemma}
\begin{proof}
令 $\boldsymbol\tau_h\in\Sigma_h$ 使得对于任意 $F\in\mathcal{F}_h$ 和 $T\in\mathcal{T}_h$ 有
\begin{align*}
(\boldsymbol\tau_h\cdot\boldsymbol{n}, q)_F&=h_F^{-1}([v_h], q)_F \quad \forall \ q\in\mathbb{P}_k(F),\\
(\boldsymbol\tau_h, \boldsymbol{q})_T&
=-(\nabla v_h, \boldsymbol{q})_T \quad\;\; \forall \, \boldsymbol{q}\in\nabla\mathbb{P}_{k-1}(T)\oplus\mathbb{P}_{k-2}(T;\mathbb K)\boldsymbol{x}.
\end{align*}
使用尺度论证技巧, 可得
\begin{align}
\label{3.8}
\|\boldsymbol\tau_h\|^2_{0,h}\lesssim\sum_{T\in\mathcal{T}_h}\|\boldsymbol\tau_h\|^2_{T}\eqsim\sum_{T\in\mathcal{T}_h}\|\nabla v_h\|^2_{T}+\sum_{F\in\mathcal{F}_h}h_F^{-1}\|[v_h]\|^2_{F}\lesssim|v_h|^2_{1,h}.
\end{align}
同时有
\begin{align*}
(\div\boldsymbol\tau_h, v_h)&=\sum_{F\in\mathcal{F}_h}(\boldsymbol\tau_h\cdot\boldsymbol{n}, [v_h])_F-\sum_{T\in\mathcal{T}_h}(\boldsymbol\tau_h, \nabla v_h)_T\\
&=\sum_{F\in\mathcal{F}_h}h_F^{-1}\|[v_h]\|^2_{F}+\sum_{T\in\mathcal{T}_h}\|\nabla v_h\|^2_{T}=|v_h|^2_{1,h},
\end{align*}
将其与 (\ref{3.8}) 联立即可证明 (\ref{3.7}).
\end{proof}

同样对任意 $\boldsymbol\sigma_h\in\Sigma_h$ 和 $u_h\in V_h$ 有以下稳定性结果:
\begin{align}
\label{3.9}
\|\boldsymbol\sigma_h\|_{0,h}+\|u_h\|_{1,h}\lesssim\sup_{\boldsymbol\tau_h\in \Sigma_h, v_h\in V_h}\dfrac{(\boldsymbol\sigma_h, \boldsymbol\tau_h)+(\div\boldsymbol\tau_h, u_h)+(\div\boldsymbol\sigma_h, v_h)}{\|\boldsymbol\tau_h\|_{0,h}+\|v_h\|_{1,h}}.
\end{align}
类似地, 在与定理 \ref{th3.1} 相同的假设下, 可得
\begin{align}
\label{3.10}
\|\boldsymbol\sigma-\boldsymbol\sigma_h\|_{0,h}&\lesssim h^{k+1}\|\boldsymbol\sigma\|_{k+1},\\
\label{3.11}
|Q_h u-u_h|_{1,h}&\lesssim h^{k+1}\|\boldsymbol\sigma\|_{k+1}.
\end{align}
接下来使用对偶论证来得到 $u_h$ 与 $Q_h u$ 在 $L^2$ 范数下的超2阶收敛结果.
令 $(\widetilde{\boldsymbol{\sigma}}, \widetilde{u})$ 为以下辅助问题的解:
\begin{align}
\label{3.12}
\left\{
\begin{array}{ll}
\widetilde{\boldsymbol{\sigma}} = \nabla\widetilde{u}, \qquad\qquad\quad\ \ x\in \Omega,\\
-\div\widetilde{\boldsymbol{\sigma}} = Q_h u-u_h, \ \ x\in \Omega.
\end{array}
\right.
\end{align}
假设 $\Omega$ 是凸区域, 那么 $\widetilde{u}\in H^2(\Omega)$ 且
\begin{align}
\label{3.13}
\|\widetilde{\boldsymbol{\sigma}}\|_1+\|\widetilde{u}\|_2\lesssim\|Q_h u-u_h\|.
\end{align}
\begin{theorem}
设 $k\geq2$ (\textcolor{red}{$k=1$?}).令 $(\boldsymbol\sigma, u)$ 为 (\ref{1.4}) 的解, $(\boldsymbol\sigma_h, u_h)$ 为 (\ref{2.1}) 的解, 假设正则性结果 (\ref{3.13}) 成立, 同时 $\boldsymbol\sigma\in H^{k+1}(\Omega;\mathbb{R}^d)$ 且 $u\in H^{k}(\Omega)$, 则
\begin{align}
\label{3.14}
\|Q_h u-u_h\|\lesssim h^{k+2}\|\boldsymbol\sigma\|_{k+1}.
\end{align}
\end{theorem}
\begin{proof}
根据 $I_h$ 和 $Q_h$ 的定义, 有以下交换图性质:
\begin{align}
\label{3.15}
\div I_h\boldsymbol\sigma=Q_h\div\boldsymbol\sigma \quad \forall \ \boldsymbol\sigma\in H^1(\Omega;\mathbb{R}^d).
\end{align}
由 (\ref{3.2}), (\ref{3.12}) 和 (\ref{3.15}) 可得
\begin{align*}
\|Q_h u-u_h\|^2&=-(\div\widetilde{\boldsymbol{\sigma}}, Q_h u-u_h)\\
% &=-(\div(\widetilde{\boldsymbol{\sigma}}-I_h\widetilde{\boldsymbol{\sigma}}),Q_h u-u_h)-(\div(I_h\widetilde{\boldsymbol{\sigma}}),Q_h u-u_h)\\
&=-(\div(I_h\widetilde{\boldsymbol{\sigma}}),Q_h u-u_h)=-(\div(I_h\widetilde{\boldsymbol{\sigma}}),u-u_h)\\
&=(\boldsymbol\sigma-\boldsymbol\sigma_h,I_h\widetilde{\boldsymbol{\sigma}})=(\boldsymbol\sigma-\boldsymbol\sigma_h,I_h\widetilde{\boldsymbol{\sigma}}-\widetilde{\boldsymbol{\sigma}})+(\boldsymbol\sigma-\boldsymbol\sigma_h,\widetilde{\boldsymbol{\sigma}}).
\end{align*}
根据 $Q_h$ 的定义以及 (\ref{3.12}), 再使用分部积分, 我们有
\begin{align*}
(\boldsymbol\sigma-\boldsymbol\sigma_h,\widetilde{\boldsymbol{\sigma}})&=(\boldsymbol\sigma-\boldsymbol\sigma_h,\nabla\widetilde{u})=-(\div(\boldsymbol\sigma-\boldsymbol\sigma_h),\widetilde{u})\\
&=(f-Q_hf,\widetilde{u}) =(f-Q_hf,\widetilde{u}-Q_h\widetilde{u}).
\end{align*}
再由 $I_h$ 和 $Q_h$ 的误差估计, (\ref{3.10}), (\ref{3.13}) 以及 (\ref{3.15}) 可得
\begin{align*}
\|Q_h u-u_h\|^2&=(\boldsymbol\sigma-\boldsymbol\sigma_h,I_h\widetilde{\boldsymbol{\sigma}}-\widetilde{\boldsymbol{\sigma}}) + (f-Q_hf,\widetilde{u}) =(f-Q_hf,\widetilde{u}-Q_h\widetilde{u}) \\
&\lesssim\|\boldsymbol\sigma-\boldsymbol\sigma_h\|\,\|I_h\widetilde{\boldsymbol{\sigma}}-\widetilde{\boldsymbol{\sigma}}\|+\|f-Q_hf\|\|\widetilde{u}-Q_h\widetilde{u}\|\\
&\lesssim h\|\boldsymbol\sigma-\boldsymbol\sigma_h\|\,|\widetilde{\boldsymbol{\sigma}}|_1+h^2\|f-Q_hf\|\,|\widetilde{u}|_2\\
&\lesssim h^{k+2}\|\boldsymbol\sigma\|_{k+1}\|Q_h u-u_h\|.
\end{align*}
\end{proof}

%%%%%%%%%%%%%%%%%%%%%%%%%%%%%%%%%%%%%%%%%%%%%%%%%%%%%%%%%%%%%
\section{后处理}
在本节中, 我们将利用最优收敛阶结果 (\ref{3.1}) 和超收敛结果 (\ref{3.14}) 来构造关于 $u$ 的具有超收敛的逼近解.

定义有限元空间:
$$W_h=\{v\in L^2(\Omega): v|_T\in\mathbb{P}_{k+1}(T), T\in\mathcal{T}_h\}.$$
定义 $u_h^\ast\in W_h$ 为以下问题的解: 对任意 $T\in\mathcal{T}_h$,
\begin{align}
\label{4.1}
\left\{
\begin{array}{ll}
(\nabla u_h^\ast,\nabla v)_T = (\boldsymbol\sigma_h,\nabla v)_T, \quad \forall \ v\in\mathbb{P}_{k+1}(T)\\
\int_T  u_h^\ast\dx = \int_T  u_h\dx.
\end{array}
\right.
\end{align}
\begin{theorem}
假设正则性结果 (\ref{3.13}) 成立, 且 $u\in H^{k+2}(\Omega)$.
当 $k\geq1$ 时, 成立
\begin{equation}
\label{4.3}
\|\nabla_h(u-u^\ast_h)\|\lesssim h^{k+1}\|u\|_{k+2}.
\end{equation}
当 $k\geq2$ 时, 成立
\begin{equation}
\label{4.2}
\|u-u^\ast_h\|\lesssim h^{k+2}\|u\|_{k+2}.
\end{equation}
\end{theorem}
\begin{proof}
对于任意 $T\in\mathcal{T}_h$, 有
\begin{align*}
\|\nabla(Q^{k+1}_T u-u^\ast_h)\|^2_{T}&=(\nabla(Q^{k+1}_T u-u^\ast_h),\nabla(Q^{k+1}_T u-u^\ast_h))_T\\
&=(\nabla(Q^{k+1}_T u-u),\nabla(Q^{k+1}_T u-u^\ast_h))+(\boldsymbol\sigma-\boldsymbol\sigma_h,\nabla(Q^{k+1}_T u-u^\ast_h)),
\end{align*}
其中 $Q^{k+1}_T$ 为 $T$ 上的 $k+1$ 次 $L^2$ 正交投影. 由上述等式可得
\begin{align*}
\|\nabla(u-u^\ast_h)\|_{T}&\leq\|\nabla(u-Q^{k+1}_T u)\|_{T}+\|\nabla(Q^{k+1}_T u-u^\ast_h)\|_{T}\\
&\leq 2\|\nabla(u-Q^{k+1}_T u)\|_{T}+\|\boldsymbol\sigma-\boldsymbol\sigma_h\|_{T}.
\end{align*}
记 $Q^{k+1}_h$ 为全局的$k+1$ 次 $L^2$ 正交投影, 通过 $Q^{k+1}_h$ 的误差估计以及 (\ref{3.10}), 我们有
\begin{align*}
\|\nabla_h(u-u^\ast_h)\| \lesssim|u-Q^{k+1}_h u|_{1,h}+\|\boldsymbol\sigma-\boldsymbol\sigma_h\|\lesssim h^{k+1}(\|\boldsymbol\sigma\|_{k+1}+\|u\|_{k+2}).
\end{align*}
故 \eqref{4.3} 成立.

下面考虑 $L^2$ 误差. 容易看出
\begin{align*}
Q^0_T(Q^{k+1}_T u-u^\ast_h)&=Q^0_TQ^{k+1}_T u-Q^0_T u^\ast_h=Q^0_T u-Q^0_T u_h\\
&=Q^0_T Q_h u-Q^0_T u_h=Q^0_T(Q_h u-u_h),
\end{align*}
其中 $Q^0_T$ 为 $T$ 上的 $0$ 次 $L^2$ 正交投影. 由 $Q^0_T$ 的误差估计可得
\begin{align}\begin{split}
\label{4.4}
\|u-u^\ast_h\|_{T}&\leq \|u-Q^{k+1}_T u\|_{T}+\|Q^{k+1}_T u-u^\ast_h-Q^0_T(Q^{k+1}_T u-u^\ast_h)\|_{T}+\|Q^0_T(Q_h u-u_h)\|_{T}\\
&\lesssim\|u-Q^{k+1}_T u\|_{T}+h_T|Q^{k+1}_T u-u^\ast_h|_{1,T}+\|Q_h u-u_h\|_{T}.
\end{split}
\end{align}
通过 $Q^{k+1}_h$ 的误差估计, (\ref{3.14}) 和 (\ref{4.3}), 可以得出
\begin{align*}
\|u-u^\ast_h\|\lesssim\|u-Q^{k+1}_h u\|+h\|\nabla_h(Q^{k+1}_h u-u^\ast_h)\| +\|Q_h u-u_h\|\lesssim h^{k+2} \|u\|_{k+2}.
\end{align*}
\end{proof}

\section{杂交化}
考虑 $k=0$的情形,没有自由度,VEM


对于Poisson方程的混合有限元格式,找$\boldsymbol\sigma_h\in\Sigma_h,\, u_h\in\mathbb{P}_{l}(\mathcal{T}_h)$,使得
\begin{subequations}
\begin{align}
(\boldsymbol\sigma_h, \boldsymbol\tau_h) + (\div \boldsymbol\tau_h, u_h) &= 0,\qquad\quad \;\;\; \forall \boldsymbol\tau_h \in \Sigma_h,\label{MixPoisson1}\\ 
(\div \boldsymbol\sigma_h, v_h) &= -(f, v_h), \quad \forall  v_h \in \mathbb{P}_{l}(\mathcal{T}_h)\label{MixPoisson2},
\end{align}
\end{subequations}
其中$l = k-1, k $,
\[
\Sigma_h  = \{\boldsymbol\tau_h\in H(\div,\Omega):\boldsymbol\tau_h|_T\in\mathbb{P}_{k}(T;\mathbb{R}^d)+\mathbb{H}_{k}(T)\boldsymbol x, \; T\in\mathcal{T}_h\}.
\]
对于速度变量,由于 $H(\div)$ 空间中协调元法要求法向分量在 $d-1$ 维面上连续,因此可以对其进行杂化,从而简化编程实现的复杂性。
引入$Lagrange$乘子:
\[
([\![\boldsymbol\sigma_h\cdot\boldsymbol n]\!],\mu_h)_{F} = 0,\quad \mu_h\in \mathbb{P}_{k}(F),\,F\in\mathring{\mathcal{F}}_h.
\]
因此可以放松对速度变量连续性的要求,引入离散的$L^2(\Omega,\mathbb{R}^d)$的有限元空间:
\[
\Sigma_h^{-1}  = \{\boldsymbol\tau_h\in L^2(\Omega):\boldsymbol\tau_h|_T\in\mathbb{P}_{k}(T;\mathbb{R}^d)+\mathbb{H}_{l}(T)\boldsymbol x, \; T\in\mathcal{T}_h\}.
\]
则混合变分问题可以改写为:找$\boldsymbol\sigma_h\in\Sigma_h^{-1},\, u_h\in\mathbb{P}_{l}(\mathcal{T}_h),\, \lambda_h\in\mathbb{P}_{k}(\mathring{\mathcal{F}}_h)$,使得
\begin{subequations}
\begin{align}
(\boldsymbol\sigma_h, \boldsymbol\tau_h) + (\div \boldsymbol\tau_h, u_h) -\sum_{T\in\mathcal{T}_h}(\boldsymbol\tau_h\cdot\boldsymbol n,\lambda_h)_{\partial T}&= 0,\qquad\;\;\; \forall \boldsymbol\tau_h \in \Sigma_h^{-1},\label{HMixPoisson1}\\ 
(\div \boldsymbol\sigma_h, v_h)-\sum_{T\in\mathcal{T}_h}(\boldsymbol\sigma_h\cdot\boldsymbol n,\mu_h)_{\partial T} &= -(f, v_h), \; \forall  v_h \in \mathbb{P}_{l}(\mathcal{T}_h),\,\mu_h\in \mathbb{P}_{k}(\mathring{\mathcal{F}}_h)\label{HMixPoisson2}.
\end{align}
\end{subequations}
\subsection{HDG方法}
我们考虑定义在 $\mathcal{T}_h \times \mathcal{F}_h$ 上的分片多项式的乘积空间$M_{l,k} : = \mathbb{P}_{l}(\mathcal{T}_h)\times\mathbb{P}_{k}(\mathcal{F}_h)$,以及它的子空间$\mathring{M}_{l,k} : = \mathbb{P}_{l}(\mathcal{T}_h)\times\mathbb{P}_{k}(\mathring{\mathcal{F}}_h)$,对于$v_h\in M_{l,k}$可以表示为$v_h = (v_0,v_b)$。我们定义其上的内积:
\[
(u_h,v_h)_{0,h} := (u_0,v_0) +\sum_{F\in\mathcal{F}_h}h_F(u_b,v_b)_F,
\]
缩放因子$h_F$确保这两个项具有相同的量级。自然的可以定义范数
\[
\|v_h\|_{0,h}^2 = \|v_0\|^2_{0,h} + \sum_{F\in\mathcal{F}_h}h_F\|v_b\|_{0,h}^2.
\]

对于$\boldsymbol\tau\in\Sigma_h^{-1} $ 定义弱散度算子:$\div_w : \Sigma_h^{-1} \to\mathring{M}_{l,k}$
\[
\div_w\boldsymbol\tau = (\div_T\boldsymbol\tau,-h_F^{-1}[\![\boldsymbol\tau\cdot\boldsymbol n]\!])_{T\in\mathcal{T}_h,F\in\mathring{\mathcal{F}}_h},
\]
显然的,当$\boldsymbol\tau\in \Sigma_h^{-1} \cap H(\div,\Omega)$时,则$\div_w\boldsymbol\tau = (\div\boldsymbol\tau,0)$,在这种情况下,可以写成 $\div_w\boldsymbol\tau = \div\boldsymbol\tau$。

对于$v_h\in\mathring{M}_{l,k}$,定义弱梯度算子$\nabla_w :\mathring{M}_{l,k}\to \Sigma_h^{-1}$:
\[
\begin{aligned}
(\nabla_wv_h,\boldsymbol\tau)_{T} :&= -(v_0,\div\boldsymbol\tau)_T + (v_b,\boldsymbol\tau\cdot\boldsymbol n)_{\partial T}\\
&=(\nabla_Tv_0,\boldsymbol\tau)_T + (v_b-v_0,\boldsymbol\tau\cdot\boldsymbol n)_{\partial T}.
\end{aligned}
\]
由分部积分,可以得到
\[
(\div_w\boldsymbol\tau,v_h)_{0,h} = -(\boldsymbol\tau,\nabla_wv_h),\quad\forall\boldsymbol\tau\in\Sigma_h^{-1},\,v_h\in\mathring{M}_{l,k}.
\]

根据弱算子的定义,可以将\eqref{HMixPoisson1}-\eqref{HMixPoisson2} 改写为:找$\boldsymbol\tau_h\in\Sigma_h^{-1},\,u_h\in\mathring{M}_{l,k}$,使得
\begin{subequations}
\begin{align}
(\boldsymbol\sigma_h, \boldsymbol\tau_h) + (\div_w \boldsymbol\tau_h, u_h)_{0,h} &= 0,\qquad\quad \;\;\; \forall \boldsymbol\tau_h \in \Sigma_h^{-1},\label{HDGPoisson1}\\ 
(\div_w \boldsymbol\sigma_h, v_h)_{0,h}  &= -(f, v_0), \quad \forall  v_h = (v_0,v_b) \in \mathring{M}_{l,k}\label{HDGPoisson2},
\end{align}
\end{subequations}
也等价于
\begin{subequations}
\begin{align}
(\boldsymbol\sigma_h, \boldsymbol\tau_h) - (\boldsymbol\tau_h, \nabla_wu_h) &= 0,\qquad\quad \;\;\; \forall \boldsymbol\tau_h \in \Sigma_h^{-1},\label{HDGPoisson3}\\ 
-( \boldsymbol\sigma_h, \nabla_wv_h) &= -(f, v_0), \quad \forall  v_h = (v_0,v_b) \in \mathring{M}_{l,k}\label{HDGPoisson4},
\end{align}
\end{subequations}

\subsection{WG方法}
由\eqref{HDGPoisson3},我们得到$\boldsymbol\sigma_h = \nabla_wu_h$,则问题变成:找$u_h\in\mathring{M}_{l,k}$,使得
\begin{equation}\label{WGPoisson}
(\nabla_wu_h,\nabla_wv_h) = (f, v_0), \quad \forall  v_h = (v_0,v_b) \in \mathring{M}_{l,k}.
\end{equation}
我们称\eqref{WGPoisson}为无稳定化项的WG数值格式。

我们可以定义$\mathring{M}_{l,k}$上的网格依赖的$H^1$半范:
\[
|v_h|_{1,h}^2 : = \sum_{T\in\mathcal{T}_h}|v_0|_{1,h}^2 + \sum_{T\in\mathcal{T}_h}h_T^{-1}\|v_b-v_0\|_{\partial T}^2,\quad \forall v_h = (v_0,v_b) \in \mathring{M}_{l,k}.
\]
成立下列范数等价性:
\begin{lemma}
对于$ v_h = (v_0,v_b) \in \mathring{M}_{l,k}$,成立
\begin{equation}\label{eq:normequ}
\|\nabla_wv_h\| \eqsim |v_h|_{1,h}.
\end{equation}
\end{lemma}
\begin{proof}
为了简化记号,令$\boldsymbol\tau = \nabla_w v_h$,那么有
\[
\begin{aligned}
\|\nabla_wv_h\|^2&= \sum_{T\in\mathcal{T}_h}(\nabla_Tv_0,\boldsymbol\tau)_{T} + \sum_{T\in\mathcal{T}_h}(v_0-v_b,\boldsymbol\tau\cdot\boldsymbol n)_{\partial T}\\
&\leq \|\nabla_hv_0\|\|\boldsymbol\tau\| + (\sum_{T\in\mathcal{T}_h} h_T^{-1}\|v_0-v_b\|_{0,\partial T}^2)^{\frac{1}{2}}(\sum_{T\in\mathcal{T}_h} h_T\|\boldsymbol\tau\cdot\boldsymbol n\|_{\partial T}^{2})^{\frac{1}{2}}.
\end{aligned}
\]
由此可以推出:
\[
\|\nabla_wv_h\|^2\leq  \sum_{T\in\mathcal{T}_h}|v_0|_{1,h}^2 + \sum_{T\in\mathcal{T}_h}h_T^{-1}\|v_b-v_0\|_{\partial T}^2,
\]
即$ \|\nabla_wv_h\|\leq |v_h|_{1,h}$。我们接下来考虑反方向,取$\boldsymbol\tau\in\Sigma_h^{-1}$所有自由度都为$0$,除 $\forall\;T\in\mathcal{T}_h$
\[
\begin{aligned}
\boldsymbol\tau\cdot\boldsymbol n|_{\partial T} &= h_T^{-1}(v_b-v_0)|_{\partial T},\\
(\boldsymbol\tau,\boldsymbol q)_T &= (\nabla_Tv_0,\boldsymbol q)_T\quad \boldsymbol q\in\nabla\mathbb{P}_{l}(T),
\end{aligned} 
\]
由尺度论证,成立 $\|\boldsymbol\tau\|^2\lesssim\sum_{T\in\mathcal{T}_h}|v_0|_{1,h}^2 + \sum_{T\in\mathcal{T}_h}h_T^{-1}\|v_b-v_0\|_{\partial T}^2 = |v_h|_{1,h}^2$,另一方面
\[
\begin{aligned}
(\nabla_wv_h,\boldsymbol\tau) = \sum_{T\in\mathcal{T}_h}|v_0|_{1,h}^2 + \sum_{T\in\mathcal{T}_h}h_T^{-1}\|v_b-v_0\|_{\partial T}^2  = |v_h|_{1,h}^2.
\end{aligned}
\]
结合上述两个结果,以及Cauchy不等式,得到
$
 |v_h|_{1,h}\lesssim\|\nabla_wv_h\|,
$ 得证。
\end{proof}
在\eqref{WGPoisson}中 ,令$v_0 = 0$,以及$\boldsymbol\sigma_h = \nabla_wu_h$,则有
\[
\begin{aligned}
(\boldsymbol\sigma_h,\nabla_w(0,v_b)) = \sum_{F\in\mathcal{F}_h}([\![\boldsymbol\sigma_h\cdot\boldsymbol n]\!],v_b)_F = 0.
\end{aligned}
\]
这意味着$\boldsymbol\sigma_h\in\Sigma_h$,则$\div_w\boldsymbol\sigma_h = (\div\boldsymbol\sigma_h,0)$,显然有
$(\nabla_wu,\nabla_wv_h) = -(\div\boldsymbol\sigma_h, u_0)$,因此可以将\eqref{WGPoisson}写成混合形式:
\begin{subequations}
\begin{align}
(\boldsymbol\sigma_h, \boldsymbol\tau_h) + (\div\boldsymbol\tau_h, u_0) &= 0,\\ 
( \div\boldsymbol\sigma_h, v_0)&= -(f, v_0).
\end{align}
\end{subequations}
\subsection{虚拟元方法}
$\Sigma_h^{-1}$和$\mathring{M}_{l,k}$实际离散的是$L^2(\Omega;\mathbb{R}^d)$和$H^{-1}(\Omega)$,即离散的原始问题为:找$\boldsymbol\sigma\in L^2(\Omega;\mathbb{R}^d),\, u\in H_0^1(\Omega)$,使得
\[
\begin{aligned}
(\boldsymbol\sigma,\boldsymbol\tau) + \langle\div\boldsymbol\tau,u\rangle &= 0,\qquad\quad\; \forall \boldsymbol\tau\in L^2(\Omega;\mathbb{R}^d),\\
 \langle\div\boldsymbol\sigma,v\rangle &= -(f,v),\quad \forall v\in H_0^1(\Omega) .
\end{aligned}
\]
对于$H_0^1(\Omega)$空间,可以使用非协调虚拟元来进行离散。对于$l=k-1,\,k$,
\[
V^{\mathrm{VE}}_{k,l}(T) : = \{ v\in H^1(T):\Delta v|_T\in\mathbb{P}_{l}(T),\partial_n v|_T\in\mathbb{P}_{k}(F), \, F\in\Delta_{d-1}(T) \}.
\]
显然$\mathbb{P}_{k+1}(T)\in V^{\mathrm{VE}}(T)$,给出虚拟元的自由度:
\begin{subequations}
\begin{align}
(v,q)_F \quad&q \in\mathbb{P}_k(F) \label{eq:DoFVE1}\\ 
 (v,q)_T\quad & q\in\mathbb{P}_{l}(T).\label{eq:DoFVE2}
\end{align}
\end{subequations}
定义全局有限元空间:
\[
\begin{aligned}
\mathring{V}_{k,l}^{\mathrm{VE}}:=\{v\in L^2(\Omega): v|_T\in V^{\mathrm{VE}}(T),\, T\in\mathcal{T}_h; \int_F [\![v]\!]q\,ds = 0, \; q\in\mathbb{P}_{k}(F), \, F\in\mathring{\mathcal{F}}_h\\
\text{ 并且 }
\int_F vq\,ds = 0,\,F\in\mathcal{F}_h\backslash\mathring{\mathcal{F}}_h.
\} 
\end{aligned}
\]
当$k=0,l=-1$时,是CR元, $V^{\mathrm{VE}}_{0,-1}(T) = \mathbb{P}_{1}(T)$;当$k=0,l=0$时,是增强的CR元, $V^{\mathrm{VE}}_{0,0}(T) = \mathbb{P}_{1}(T) + \text{span}\{\boldsymbol x\cdot\boldsymbol x\}$。引入算子$Q_M : \mathring{V}_{k,l}^{\mathrm{VE}}\to\mathring{M}_{l,k}$
\[
Q_Mv_h := (Q_{l,T}v_h,Q_{k,F}v_h)_{T\in\mathcal{T}_h,F\in\mathring{\mathcal{F}}_h} 
\]
根据自由度\eqref{eq:DoFVE1}-\eqref{eq:DoFVE2},这显然是一个双射。定义$L^2$投影算子:$Q_{\Sigma}
:L^2(\Omega;\mathbb{R}^d)\to\Sigma_h^{-1}$,可以得到算子交换性:
\begin{equation}
\nabla_wQ_M v_h = Q_{\Sigma}\nabla_hv_h ,\quad \forall v_h\in \mathring{V}_{k,l}^{\mathrm{VE}}.
\end{equation}
\begin{lemma}
对于$v_h\in \mathring{V}_{k,l}^{\mathrm{VE}},\, \boldsymbol\tau_h\in\Sigma_h^{-1}$,成立
\[
(\nabla_wQ_Mv_h,\boldsymbol\tau_h)_T = (\nabla_hv_h,\boldsymbol\tau_h)_T.
\]
\end{lemma}
\begin{proof}
将等式左侧展开,
 \[
 \begin{aligned}
 (\nabla_wQ_Mv_h,\boldsymbol\tau_h)_T &=  -(Q_{l,T}v_h,\div\boldsymbol\tau_h)_T +(Q_{k,F}v_h,\boldsymbol\tau_h\cdot\boldsymbol n)_{\partial T}\\
 & = -(v_h,\div\boldsymbol\tau_h)_T+(v_h,\boldsymbol\tau_h\cdot\boldsymbol n)_{\partial T}
  \end{aligned}
 \]
 得证。
\end{proof}
根据算子交换性,对于$u_h\in\mathring{M}_{l,k}$,成立
\[
\nabla_w u_h = \nabla_wQ_MQ_M^{-1}u_h = Q_{\Sigma}\nabla_h(Q_M^{-1}u_h),
\]
因此可以将\eqref{WGPoisson}写成:找$u_h\in\mathring{V}_{k,l}^{\mathrm{VE}}$,使得
\begin{equation}\label{eq:VEMPoisson}
(Q_{\Sigma}\nabla_h(u_h),Q_{\Sigma}\nabla_h(v_h)) = (f,Q_{l,h}v_h),\quad\forall v_h\in\mathring{V}_{k,l}^{\mathrm{VE}}.
\end{equation}
当$k =0$时,可以去掉投影算子:找$u_h\in\mathring{V}_{0,l}^{\mathrm{VE}},\, l = -1, 0$,使得
\begin{equation}\label{eq:CRPoisson}
(\nabla_h(u_h),\nabla_h(v_h)) = (f,Q_{l,h}v_h),\quad\forall v_h\in\mathring{V}_{0,l}^{\mathrm{VE}}.
\end{equation}
