% !TEX root = femtensorlecture.tex
\chapter{Finite Element Divdiv Complexes in Two Dimensions}



\section{Divdiv Symmetric Tensor Space}
\subsection{Green's identity}
%$\boldsymbol H(\div\div)$
We start from the Green's identity for smooth functions but on polygons.
\begin{lemma} [Green's identity]\label{lm:Green}
Let $K$ be a polygon, and let $\boldsymbol  \tau\in \mathcal C^2(K; \mathbb S)$ and $v\in H^2(K)$. Then we have
\begin{align}
(\div\div\boldsymbol \tau, v)_K&=(\boldsymbol \tau, \nabla^2v)_K -\sum_{e\in\mathcal E(K)}\sum_{\delta\in\partial e}\sign_{e,\delta}(\boldsymbol  t^{\intercal}\boldsymbol \tau\boldsymbol  n)(\delta)v(\delta) \notag\\
&\quad - \sum_{e\in\mathcal E(K)}\left[(\boldsymbol  n^{\intercal}\boldsymbol \tau\boldsymbol n, \partial_n v)_{e}-(\partial_{t}(\boldsymbol  t^{\intercal}\boldsymbol \tau\boldsymbol n)+\boldsymbol n^{\intercal}\div\boldsymbol \tau,  v)_{e}\right], \label{eq:greenidentitydivdiv}
\end{align}
where
\[
\sign_{e,\delta}:=\begin{cases}
1, & \textrm{ if } \delta \textrm{ is the end point of } e, \\
-1, & \textrm{ if } \delta \textrm{ is the start point of } e.
\end{cases}
\]
\end{lemma}
\begin{proof}
We start from the standard integration by parts
\begin{align*}
\begin{aligned}
(\operatorname{div} \div \boldsymbol  \tau, v)_{K} &=-(\div  \boldsymbol   \tau, \nabla v)_{K}+\sum_{e \in \mathcal E(K)}(\boldsymbol  n^{\intercal} \div  \boldsymbol   \tau, v)_e \\
&=\left(\boldsymbol  \tau, \nabla^{2} v\right)_{K}-\sum_{e \in \mathcal E(K)} (\boldsymbol  \tau \boldsymbol  n, \nabla v)_e+\sum_{e \in \mathcal E(K)}(\boldsymbol  n^{\intercal} \div  \boldsymbol   \tau, v)_e.
\end{aligned}
\end{align*}
Now we expand $(\boldsymbol  \tau \boldsymbol  n, \nabla v)_e=(\boldsymbol  n^{\intercal} \boldsymbol  \tau \boldsymbol  n,  \partial_{n} v)_e+ (\boldsymbol  t^{\intercal} \boldsymbol  \tau \boldsymbol  n, \partial_{t} v)_e$ and apply integration by parts on each edge to the second term
$$
(\boldsymbol  t^{\intercal} \boldsymbol  \tau \boldsymbol  n, \partial_{t} v)_e=\sum_{\delta \in \partial e} \sign_{e,\delta}(\boldsymbol  t^{\intercal}\boldsymbol \tau\boldsymbol  n)(\delta)v(\delta)-(\partial_ t\left( \boldsymbol  t^{\intercal} \boldsymbol  \tau \boldsymbol  n\right), v)_e
$$
to finish the proof.
\end{proof}

In the context of elastic mechanics~\cite{FengShi1996}, $\boldsymbol  n^{\intercal}\boldsymbol \tau\boldsymbol  n$ and $\partial_{t}(\boldsymbol  t^{\intercal}\boldsymbol \tau\boldsymbol  n)+\boldsymbol  n^{\intercal} \div\boldsymbol \tau$ are called
normal bending moment and effective transverse shear force respectively for $\bs\tau$ being a moment.

For a scalar function $\phi$, due to the rotation relation, $\boldsymbol  n^{\intercal} \curl \phi = \boldsymbol  t^{\intercal} \grad \phi =  \partial_t \phi$ and $\boldsymbol  t^{\intercal} \curl \phi = -\boldsymbol  n^{\intercal} \grad \phi = -\partial_n \phi$. For vector and matrix functions, we have the following relations.
\begin{lemma}\label{lm:tauv}
When $\boldsymbol  \tau = \sym\curl \, \boldsymbol  v$, we have the following identities
\begin{align}
\label{eq:trace1} \boldsymbol  n^{\intercal}\boldsymbol \tau\boldsymbol  n &= \boldsymbol  n^{\intercal} \partial_t \boldsymbol  v,\\
\label{eq:trace2--2d} \partial_{t}(\boldsymbol  t^{\intercal}\boldsymbol \tau\boldsymbol  n)+\boldsymbol  n^{\intercal}\div\boldsymbol \tau & =  \partial_t(\boldsymbol  t^{\intercal}\partial_t\boldsymbol  v).
\end{align}
\end{lemma}
\begin{proof}
The first one is a straight forward calculation using $(\curl \, \boldsymbol  v)\boldsymbol n =  \partial_t \boldsymbol  v$. We now focus on the second one.
Since $\div \curl \, \boldsymbol  v = 0$, we have
$$\boldsymbol  n^{\intercal}\div\boldsymbol \tau  = \frac{1}{2}\boldsymbol  n^{\intercal} \div (\curl \, \boldsymbol  v)^{\intercal} = \frac{1}{2}\boldsymbol  n^{\intercal} \curl \div \boldsymbol  v =  \frac{1}{2} \partial_t \div \boldsymbol  v.$$
As $\div \boldsymbol  v = {\rm tr} (\nabla \boldsymbol  v)$ is invariant to the rotation, we can write it as
$$
\div \boldsymbol  v = \boldsymbol  t^{\intercal}\nabla \boldsymbol  v \boldsymbol  t + \boldsymbol  n^{\intercal}\nabla \boldsymbol  v \boldsymbol  n =  \boldsymbol  t^{\intercal}\partial_t\boldsymbol  v + \boldsymbol  n^{\intercal}\partial_n\boldsymbol  v.
$$
Then
$$
 \partial_{t}(\boldsymbol  t^{\intercal}\boldsymbol \tau\boldsymbol  n)+\boldsymbol  n^{\intercal}\div\boldsymbol \tau  =\frac{1}{2}\partial_t[  \boldsymbol  t^{\intercal}\partial_t\boldsymbol  v -  \boldsymbol  n^{\intercal}\partial_n\boldsymbol  v + \div\boldsymbol  v]  =  \partial_t  (\boldsymbol  t^{\intercal}\partial_t\boldsymbol  v),
 $$
 i.e. \eqref{eq:trace2--2d} holds.
\end{proof}

\subsection{Traces}
Next we recall the trace of the space $\boldsymbol{H}(\div{\div },K; \mathbb{S})$ on the boundary of polygon $K$. {Detailed proofs of the following trace operators can be found in \cite[Theorem 2.2]{Amara;Capatina-Papaghiuc;Chatti:2002Bending} for 2-D domains and \cite[Lemma 3.2]{Fuhrer;Heuer;Niemi:2019ultraweak} for both 2-D and 3-D domains.} The normal-normal trace of $\boldsymbol{H}(\div{\div },K; \mathbb{S})$ can be also found in \cite{Sinwel2009,PechsteinSchoeberl2018}.

Define trace space
\begin{align*}
H_{n,0}^{1/2}(\partial K)&:=\{\partial_n v|_{\partial K}: v\in H^2(K)\cap H_0^1(K)\} \\
&\;=\{g\in L^2(\partial K): g|_e\in H_{00}^{1/2}(e)\;\;\forall~e\in\mathcal E(K)\}
\end{align*}
with norm
\[
\|g\|_{H_{n,0}^{1/2}(\partial K)}:=\inf_{v\in H^2(K)\cap H_0^1(K)\atop \partial_n v=g}\|v\|_2.
\]
Let $H_n^{-1/2}(\partial K):=(H_{n,0}^{1/2}(\partial K))'$. Note that for a 2D polygon $K$, and $v\in H^2(K)\cap H_0^1(K)$, the normal derivative $\partial_n v|_{e}\in H_{00}^{1/2}(e)$ for boundary edge $e\in \partial K$ can be derived from the compatible condition for traces on polygonal domains \cite[Theorem 1.5.2.8]{Grisvard:2011Elliptic}.

\begin{lemma}
For any $\boldsymbol \tau\in\boldsymbol{H}(\div{\div },K; \mathbb{S})$,  it holds
\[
\|\boldsymbol  n^{\intercal}\boldsymbol \tau\boldsymbol  n\|_{H_n^{-1/2}(\partial K)}\lesssim \|\boldsymbol{\tau}\|_{\boldsymbol{H}(\div{\div })}.
\]
Conversely, for any $g\in H_n^{-1/2}(\partial K)$, there exists some $\boldsymbol \tau\in\boldsymbol{H}(\div{\div },K; \mathbb{S})$ such that
\[
\boldsymbol  n^{\intercal}\boldsymbol \tau\boldsymbol  n|_{\partial K}=g, \quad
\|\boldsymbol{\tau}\|_{\boldsymbol{H}(\div{\div })} \lesssim \|g\|_{H_n^{-1/2}(\partial K)}.
\]
The hidden constants depend only the shape of the domain $K$.
\end{lemma}

We then consider another part of the trace involving combination of derivatives. Define trace space
\begin{align*}
H_{e,0}^{3/2}(\partial K)&:=\{v|_{\partial K}: v\in H^2(K), \partial_nv|_{\partial K}=0, v(\delta)=0 \textrm{ for each vertex } \delta\in\mathcal V(K)\}
\end{align*}
with norm
\[
\|g\|_{H_{e,0}^{3/2}(\partial K)}:=\inf_{v\in H^2(K)\atop \partial_n v=0, v=g}\|v\|_2.
\]
Let $H_e^{-3/2}(\partial K):=(H_{e,0}^{3/2}(\partial K))'$. Note that since we consider polygon domains, we explicitly impose the condition $v(\delta) = 0$ for each vertex of the polygon. %\LC{ Is the local problem with such condition is still well posed?}

%\LC{On the other hand, for any $g\in H_e^{-3/2}(\partial K)$, consider the following variational problem: find $w\in H^2(K)$ with $\partial_nw|_{\partial K}=0$ and $w|_{\mathcal V(K)}=0$
%such that
%\begin{equation}\label{eq:biharmonic1}
%(\nabla^2w, \nabla^2v)+(w, v)=\langle g, v\rangle_{H_e^{-3/2}(\partial K)\times H_{e,0}^{3/2}(\partial K)} %\quad\forall~v\in H^2(K) \textrm{ with } \partial_nw|_{\partial K}=0 \textrm{ and } w|_{\mathcal V(K)}=0.
%\end{equation}
%for any $v\in H^2(K)$ with $\partial_nv|_{\partial K}=0$ and $v|_{\mathcal V(K)}=0$.}

\begin{lemma}
For any $\boldsymbol \tau\in\boldsymbol{H}(\div{\div },K; \mathbb{S})$,  it holds
\begin{equation}\label{eq:divdivtracee0}
\|\partial_t(\boldsymbol  t^{\intercal}\boldsymbol \tau\boldsymbol  n)+\boldsymbol  n^{\intercal}\div\boldsymbol \tau\|_{H_e^{-3/2}(\partial K)}\lesssim \|\boldsymbol{\tau}\|_{\boldsymbol{H}(\div{\div })}.
\end{equation}
Conversely, for any $g\in H_e^{-3/2}(\partial K)$, there exists some $\boldsymbol \tau\in\boldsymbol{H}(\div{\div },K; \mathbb{S})$ such that
\begin{equation}\label{eq:divdivinvtracee0}
\partial_t(\boldsymbol  t^{\intercal}\boldsymbol \tau\boldsymbol  n)+\boldsymbol  n^{\intercal}\div\boldsymbol \tau=g, \quad
\|\boldsymbol{\tau}\|_{\boldsymbol{H}(\div{\div })} \lesssim \|g\|_{H_e^{-3/2}(\partial K)}.
\end{equation}
The hidden constants depend only the shape of the domain $K$.
\end{lemma}
%\begin{proof}
%Assume $\boldsymbol \tau\in\boldsymbol  C^{\infty}(K;\mathbb S)$.
%For any $v\in H^2(K)$ with $\partial_nv|_{\partial K}=0$ and $v|_{\mathcal V(K)}=0$, it follows from the Green's identity \eqref{eq:greenidentitydivdiv} that %Lemma \ref{lm:Green}
%\begin{equation}\label{eq:20204020}
%(\partial_t(\boldsymbol  t^{\intercal}\boldsymbol \tau\boldsymbol  n)+\boldsymbol  n^{\intercal}\boldsymbol \div\boldsymbol \tau, v)_{\partial K}=-(\boldsymbol \tau, \nabla^2 v)+(\div\boldsymbol \div\boldsymbol \tau, v).
%\end{equation}
%%\begin{align*}
%%(\partial_t(\boldsymbol  t^{\intercal}\boldsymbol \tau\boldsymbol  n)+\boldsymbol  n^{\intercal}\boldsymbol \div\boldsymbol \tau, v)_{\partial K}
%%%&=-(\boldsymbol  t^{\intercal}\boldsymbol \tau\boldsymbol  n, \partial_tv)_{\partial K}+(\boldsymbol  n^{\intercal}\boldsymbol \div\boldsymbol \tau, v)_{\partial K} \\
%%%&=-(\boldsymbol \tau\boldsymbol  n, \nabla v)_{\partial K}+(\boldsymbol  n^{\intercal}\boldsymbol \div\boldsymbol \tau, v)_{\partial K} \\
%%&=-(\boldsymbol \tau, \nabla^2 v)+(\div\boldsymbol \div\boldsymbol \tau, v).
%%\end{align*}
%Thus
%\begin{align*}
%\|\partial_t(\boldsymbol  t^{\intercal}\boldsymbol \tau\boldsymbol  n)+\boldsymbol  n^{\intercal}\boldsymbol \div\boldsymbol \tau\|_{H_e^{-3/2}(\partial K)}&=\sup_{g\in H_{e,0}^{3/2}(\partial K)}\frac{(\partial_t(\boldsymbol  t^{\intercal}\boldsymbol \tau\boldsymbol  n)+\boldsymbol  n^{\intercal}\boldsymbol \div\boldsymbol \tau, g)_{\partial K}}{\|g\|_{H_{e,0}^{3/2}(\partial K)}} \\
%&=\sup_{v\in H^2(K), \partial_nv|_{\partial K}=0\atop v|_{\mathcal V(K)}=0 }\frac{(\partial_t(\boldsymbol  t^{\intercal}\boldsymbol \tau\boldsymbol  n)+\boldsymbol  n^{\intercal}\boldsymbol \div\boldsymbol \tau, v)_{\partial K}}{\|v\|_2} \\
%&=\sup_{v\in H^2(K), \partial_nv|_{\partial K}=0\atop v|_{\mathcal V(K)}=0 }\frac{-(\boldsymbol \tau, \nabla^2 v)+(\div\boldsymbol \div\boldsymbol \tau, v)}{\|v\|_2} \\
%&\lesssim \|\boldsymbol{\tau}\|_{\boldsymbol{H}(\div \boldsymbol{\div })},
%\end{align*}
%which implies the trace inequality \eqref{eq:divdivtracee0} by applying the density argument.
%
%On the other hand, for any $g\in H_e^{-3/2}(\partial K)$, consider the following variational problem: find $w\in H^2(K)$ with $\partial_nw|_{\partial K}=0$ and $w|_{\mathcal V(K)}=0$
%such that
%\begin{equation}\label{eq:biharmonic1}
%(\nabla^2w, \nabla^2v)+(w, v)=\langle g, v\rangle_{H_e^{-3/2}(\partial K)\times H_{e,0}^{3/2}(\partial K)} %\quad\forall~v\in H^2(K) \textrm{ with } \partial_nw|_{\partial K}=0 \textrm{ and } w|_{\mathcal V(K)}=0.
%\end{equation}
%for any $v\in H^2(K)$ with $\partial_nv|_{\partial K}=0$ and $v|_{\mathcal V(K)}=0$.
%By the Lax-Milgram lemma, the problem \eqref{eq:biharmonic1} is well-posed, and we have
%\[
%\Delta^2w=-w,\quad \|\nabla^2w\|_0+\|w\|_0\lesssim \|g\|_{H_e^{-3/2}(\partial K)}.
%\]
%Taking $\boldsymbol \tau=-\nabla^2w$, we get
%\[
%\|\boldsymbol{\tau}\|_{\boldsymbol{H}(\div \boldsymbol{\div })}^2=\|\nabla^2w\|_0^2+\|w\|_0^2 \lesssim \|g\|_{H_e^{-3/2}(\partial K)}^2.
%\]
%Finally applying \eqref{eq:20204020} with the density argument to \eqref{eq:biharmonic1} implies \eqref{eq:divdivinvtracee0}.
%\end{proof}

%Applying the integration by parts and the density argument, we get the following result.
% Green's identity
%for any $v\in H^2(K)$ and $\boldsymbol \tau\in \boldsymbol{H}(\div \boldsymbol{\div },K; \mathbb{S})$ satisfying that $(\boldsymbol  n^{\intercal}\boldsymbol \tau\boldsymbol  n)|_{\partial K}, (\partial_{t_e}(\boldsymbol  t^{\intercal}\boldsymbol \tau\boldsymbol  n)+\boldsymbol  n_e^{\intercal}\boldsymbol \div\boldsymbol \tau)|_{\partial K}\in L^2(\partial K)$, and $\boldsymbol \tau$ is continuous at vertices of $K$,

\subsection{Continuity across the boundary}
We then present a sufficient continuity condition for piecewise smoothing functions to be in $\boldsymbol{H}(\div{\div },\Omega; \mathbb{S})$. Recall that $\mathcal {T}_h$ is a shape regular polygonal mesh of $\Omega$.

\begin{lemma}\label{lem:Hdivdivpatching}
Let $\boldsymbol \tau\in \boldsymbol  L^2(\Omega;\mathbb S)$ such that
\begin{enumerate}[(i)]
\item $\boldsymbol \tau|_K\in \boldsymbol{H}(\div{\div},K; \mathbb{S})$ for each polygon $K\in\mathcal T_h$;

\smallskip
\item $(\boldsymbol  n^{\intercal}\boldsymbol \tau\boldsymbol  n)|_e\in L^2(e)$ is single-valued for each $e\in\mathcal E_h^i$;

\smallskip
\item $(\partial_{t_e}(\boldsymbol  t^{\intercal}\boldsymbol \tau\boldsymbol  n)+\boldsymbol  n_e^{\intercal}\div\boldsymbol \tau)|_e\in L^2(e)$ is single-valued for each $e\in\mathcal E_h^i$;

\smallskip
\item $\boldsymbol \tau(\delta)$ is single-valued for each $\delta\in\mathcal V_h^i$,
\end{enumerate}
then $\boldsymbol \tau\in \boldsymbol{H}(\div{\div},\Omega; \mathbb{S})$.
\end{lemma}
\begin{proof}
%Assume $\boldsymbol \tau\in \boldsymbol  C^{\infty}(\Omega;\mathbb S)$ temporarily.
For any $v\in C_0^{\infty}(\Omega)$, it follows from the Green's identity \eqref{eq:greenidentitydivdiv} that
\begin{align*}
(\boldsymbol \tau, \nabla^2v)&=\sum_{K\in\mathcal T_h}(\div\div\boldsymbol \tau, v)_K+\sum_{K\in\mathcal T_h}\sum_{e\in\mathcal E^i(K)}\sum_{\delta\in\partial e\cap\Omega}\sign_{e,\delta}(\boldsymbol  t^{\intercal}\boldsymbol \tau\boldsymbol  n)(\delta)v(\delta)\\
&\quad+\sum_{K\in\mathcal T_h}\sum_{e\in\mathcal E^i(K)}\left[(\boldsymbol  n^{\intercal}\boldsymbol \tau\boldsymbol  n, \partial_n v)_{e}-(\partial_{t}(\boldsymbol  t^{\intercal}\boldsymbol \tau\boldsymbol  n)+\boldsymbol  n^{\intercal}\div\boldsymbol \tau,  v)_{e}\right].
\end{align*}
As each interior edge is repeated twice in the summation with opposite orientation and the trace of $\tau$ and vertex value $\tau$ is single valued, we get
\[
\langle\div\div\boldsymbol \tau, v\rangle=\sum_{K\in\mathcal T_h}(\div \div\boldsymbol \tau, v)_K,
\]
which ends the proof.
\end{proof}
Besides the continuity of the trace, we also impose the continuity of stress at vertices which is a sufficient but not necessary condition for functions in $\boldsymbol{H}(\div{\div },\Omega; \mathbb{S})$. For example, by the complex \eqref{eq:divdivcomplexL2} and Lemma \ref{lm:tauv}, for $\boldsymbol  \tau = \sym \curl \, \boldsymbol  v$ with $\bs v$ being a Lagrange element function, $\bs \tau\in \boldsymbol{H}(\div{\div },\Omega; \mathbb{S})$ but is not continuous at vertices. Physically, $\boldsymbol  t^{\intercal}\boldsymbol \tau\boldsymbol  n$ represents the torsional moment which may have jump at vertices; see \cite[\S 3.4]{FengShi1996} and \cite[\S 3.4]{HuangShiXu2005}. {Sufficient and necessary conditions are presented in \cite[Proposition 3.6]{Fuhrer;Heuer;Niemi:2019ultraweak}.}

The continuity of stress at vertices is crucial for us to construct $\boldsymbol{H}(\div{\div }, \Omega; \mathbb{S})$ conforming element in the classical triple \cite{Ciarlet1978}, which resembles the $\boldsymbol{H}({\div }, \Omega; \mathbb{S})$ conforming Hu-Zhang element for linear elasticity \cite{HuZhang2015}.
%\LC{The continuity of stress at vertices is crucial for us to construct local bases of $\boldsymbol{H}(\div \boldsymbol{\div },K; \mathbb{S})$ conforming element which resembles the Hu-Zhang $\boldsymbol{H}(\boldsymbol{\div },K; \mathbb{S})$ conforming element for linear elasticity \cite{Hu;Zhang:2015family}. }

\section{Conforming finite element spaces and complex}
In this section we construct conforming finite element spaces for $H(\div {\div},\Omega; \mathbb S)$ on triangles. We first present two polynomial complexes and reveal some decompositions of polynomial tensor and vector spaces. Then we construct the finite element space and prove the unisolvence. We further link standard finite element spaces to construct finite element div-div complex. Finally we extend the construction to the strain complex. 
%For a domain $D$, we denote by $\mathbb P_k(D)$ the polynomial space on $D$ of degree $k$.


\subsection{Finite element spaces for symmetric tensors}
Let $K$ be a triangle, and $b_K$ be the cubic bubble function, i.e., $b_K\in \mathbb P_3(K)\cap H_0^1(K)$.
Take the space of shape functions
%\[
%\boldsymbol \Sigma_{\ell,k}(K):= \mathbb P_{\ell}(K;\mathbb S)+\mathbb C_k^{\oplus}(K;\mathbb S)=\mathbb C_{\ell}(K;\mathbb S)\oplus\mathbb C_k^{\oplus}(K;\mathbb S)
%\]
\[
\boldsymbol \Sigma_{\ell,k}(K):= \mathbb C_{\ell}(K;\mathbb S)\oplus\mathbb C_k^{\oplus}(K;\mathbb S)= \mathbb P_{\ell}(K;\mathbb S)+\mathbb C_k^{\oplus}(K;\mathbb S)
\]
with $k\geq 3$ and $\ell\geq k-1$.
%\begin{align*}
%\boldsymbol \Sigma_{\ell,k}(K)=\{\boldsymbol \tau\in\mathbb P_{k}(K;\mathbb S):&\, \boldsymbol \tau\boldsymbol x^{\perp} \in \mathbb P_{\ell+1}(K;\mathbb R^2), \boldsymbol  n^{\intercal}\boldsymbol \tau\boldsymbol  n|_e\in\mathbb P_{\ell}(e),\\
%&(\partial_{t}(\boldsymbol  t^{\intercal}\boldsymbol \tau\boldsymbol  n)+\boldsymbol  n^{\intercal}\boldsymbol \div\boldsymbol \tau)|_e\in\mathbb P_{\ell-1}(e)\quad\forall~e\in\mathcal E(K) \}
%\end{align*}
%for $\ell\leq k-1$.
 By Lemma \ref{lem:symmpolyspacedirectsum}, we have
\[
\mathbb P_{\min\{\ell,k\}}(K;\mathbb S)\subseteq\boldsymbol \Sigma_{\ell,k}(K) \subseteq \mathbb P_{\max\{\ell,k\}}(K;\mathbb S) \quad\textrm{ and }\quad \boldsymbol \Sigma_{k,k}(K)=\mathbb P_k(K;\mathbb S).
\]
The most interesting cases are $\ell=k-1$ and $\ell = k$ which correspond to RT and BDM $H(\div)$-conforming elements for the vector functions, respectively. 

% It is easy \LC{ Can you write more details? } to check that when $\ell\leq k$,
%\begin{align*}
%\boldsymbol \Sigma_{\ell,k}(K)=\{\boldsymbol \tau\in\mathbb P_{k}(K;\mathbb S):&\, \boldsymbol \tau\boldsymbol  x^{\perp} \in \mathbb P_{\ell+1}(K;\mathbb R^2), \boldsymbol  n^{\intercal}\boldsymbol \tau\boldsymbol  n|_e\in\mathbb P_{\ell}(e),\\
%&(\partial_{t}(\boldsymbol  t^{\intercal}\boldsymbol \tau\boldsymbol  n)+\boldsymbol  n^{\intercal}\boldsymbol \div\boldsymbol \tau)|_e\in\mathbb P_{\ell-1}(e)\quad\forall~e\in\mathcal E(K) \}.
%\end{align*}
The degrees of freedom are given by
\begin{align}
%v(\delta) & \quad\textrm{for each } \delta\in \mathcal V(K), \label{H1ncfmdof1}\\
\boldsymbol \tau (\delta) & \quad\forall~\delta\in \mathcal V(K), \label{Hdivdivfemdof1}\\
(\boldsymbol  n^{\intercal}\boldsymbol \tau\boldsymbol  n, q)_e & \quad\forall~q\in\mathbb P_{\ell-2}(e),  e\in\mathcal E(K),\label{Hdivdivfemdof2}\\
(\partial_{t}(\boldsymbol  t^{\intercal}\boldsymbol \tau\boldsymbol  n)+\boldsymbol  n^{\intercal}\div\boldsymbol \tau, q)_e & \quad\forall~q\in\mathbb P_{\ell-1}(e),  e\in\mathcal E(K),\label{Hdivdivfemdof3}\\
(\boldsymbol \tau, \boldsymbol \varsigma)_K & \quad\forall~\boldsymbol \varsigma\in\nabla^2\mathbb P_{k-2}(K)\oplus \sym (\bs x^{\perp}\otimes \mathbb P_{\ell-2}(K;\mathbb R^2)). \label{Hdivdivfemdof4}
%\sym\curl(b_K\mathbb P_{\ell-2}(K;\mathbb R^2))
\end{align}

Before we prove the unisolvence, we give some characterization of space of shape functions.
\begin{lemma}\label{lm:trace}
For any $\boldsymbol \tau\in\boldsymbol \Sigma_{\ell,k}(K)$, we have 
\begin{enumerate}
 \item $\boldsymbol \tau\boldsymbol x^{\perp} \in \mathbb P_{\ell+1}(K;\mathbb R^2)$ 
 
 \smallskip
 \item $\boldsymbol  n^{\intercal}\boldsymbol \tau\boldsymbol  n|_e\in\mathbb P_{\ell}(e)\quad\forall~e\in\mathcal E(K)$

\smallskip
 \item $(\partial_{t}(\boldsymbol  t^{\intercal}\boldsymbol \tau\boldsymbol  n)+\boldsymbol  n^{\intercal}\div\boldsymbol \tau)|_e\in\mathbb P_{\ell-1}(e)\quad\forall~e\in\mathcal E(K)$.
\end{enumerate}
%
%
%and
%\[
%,\quad .
%\]
\end{lemma}
\begin{proof}
(1) is a direct consequence of the Koszul complex \eqref{eq:divdivKoszulcomplexPoly}. 
Take any $\boldsymbol \tau=\boldsymbol  x\boldsymbol  x^{\intercal}q\in\mathbb C_k^{\oplus}(K;\mathbb S)$ with $q\in\mathbb P_{k-2}(K)$.
Since $\boldsymbol  n^{\intercal}\boldsymbol  x$ is constant on each edge of $K$,
\[
 \boldsymbol  n^{\intercal}\boldsymbol \tau\boldsymbol  n|_e=(\boldsymbol  n^{\intercal}\boldsymbol  x)^2q\in\mathbb P_{k-2}(e),
\]
\[
 (\partial_{t}(\boldsymbol  t^{\intercal}\boldsymbol \tau\boldsymbol  n)+\boldsymbol  n^{\intercal}\div\boldsymbol \tau)|_e=\boldsymbol  n^{\intercal}\boldsymbol  x(\partial_{t}\left(\boldsymbol  t^{\intercal}\boldsymbol x q)+\div(\boldsymbol  x q)+q\right)\in\mathbb P_{k-2}(e).
\]
Thus the results (2) and (3) hold from the requirement $\ell\geq k-1$.
\end{proof}


We now prove the unisolvence as follows.

\begin{lemma}\label{lem:unisovlenHdivdivfem}
The degrees of freedom \eqref{Hdivdivfemdof1}-\eqref{Hdivdivfemdof4} are unisolvent for $\boldsymbol \Sigma_{\ell,k}(K)$.
\end{lemma}
\begin{proof}
We first count the number of the degrees of freedom \eqref{Hdivdivfemdof1}-\eqref{Hdivdivfemdof4}  and the dimension of the space, i.e., $\dim\boldsymbol \Sigma_{\ell,k}(K)$.
Both of them are $$\ell^2+5\ell+3+\frac{1}{2}k(k-1).$$


%\begin{figure}[htbp]
%\begin{center}
%\includegraphics[width=5in]{figures/doftable.pdf}
%%\caption{The number of the degrees of freedom and the dimension of the space.}
%%\label{default}
%\end{center}
%\end{figure}



Then suppose all the degrees of freedom \eqref{Hdivdivfemdof1}-\eqref{Hdivdivfemdof4} applied to $\boldsymbol \tau$ vanish. We are going to prove the function $\boldsymbol \tau = 0$. 

\medskip

\noindent {\em Step 1. Trace is vanished.} By the vanishing degrees of freedom \eqref{Hdivdivfemdof1}-\eqref{Hdivdivfemdof3} and (2)-(3) in Lemma \ref{lm:trace}, we get $(\boldsymbol n^{\intercal}\boldsymbol \tau\boldsymbol n)|_{\partial K}=0$ and $(\partial_{t}(\boldsymbol  t^{\intercal}\boldsymbol \tau\boldsymbol  n)+\boldsymbol  n^{\intercal}\div\boldsymbol \tau)|_{\partial K}=0$.

\medskip

\noindent {\em Step 2. Divdiv is vanished.}
For any $q\in\mathbb P_{k-2}(K)$, it holds from the Green's identity \eqref{eq:greenidentitydivdiv} that
\begin{align*}
(\div\div\boldsymbol \tau, q)_K&=(\boldsymbol \tau, \nabla^2q)_K-\sum_{e\in\mathcal E(K)}\sum_{\delta\in\partial e}\sign_{e,\delta}(\boldsymbol  t^{\intercal}\boldsymbol \tau\boldsymbol  n)(\delta)q(\delta) \\
&\quad-\sum_{e\in\mathcal E(K)}\left[(\boldsymbol  n^{\intercal}\boldsymbol \tau\boldsymbol  n, \partial_n q)_{e}-(\partial_{t}(\boldsymbol  t^{\intercal}\boldsymbol \tau\boldsymbol  n)+\boldsymbol  n^{\intercal}\div\boldsymbol \tau,  q)_{e}\right].
\end{align*}
Since the trace is zero, $\bs \tau$ is zero at vertices, and $(\bs \tau, \nabla^2 q)_K = 0$ from \eqref{Hdivdivfemdof4},  we conclude $\div\div\boldsymbol \tau=0$. 

\medskip

\noindent {\em Step 3. Kernel of divdiv is vanished.}
Thus by the polynomial complex \eqref{eq:divdivcomplexPoly}, there exists $\boldsymbol  v\in\mathbb P_{\ell+1}(K;\mathbb R^2)/\boldsymbol {RT}$ such that
\[
\boldsymbol \tau=\sym\curl \, \boldsymbol  v\quad \textrm{ and }\quad Q_{0}^e(\boldsymbol  n^{\intercal}\boldsymbol  v)=0\quad\forall~e\in\mathcal E(K).
\]
Here we can take $Q_{0}^e(\boldsymbol  n^{\intercal}\boldsymbol  v)=0$ thanks to the degree of freedom of the lowest order Raviart-Thomas element \cite{RaviartThomas1977}. We will prove $\bs v = 0$ by similar procedure. 
%Noting that $\boldsymbol \tau\boldsymbol x^{\perp} \in \mathbb P_{\ell+1}(K;\mathbb R^2)$, by the exactness of the complex \eqref{eq:divdivKoszulcomplexPoly} and Lemma~\ref{lem:symmpolyspacedirectsum}, we have $\boldsymbol  v\in\mathbb P_{\ell+1}(K;\mathbb R^2)$.

By Lemma \ref{lm:tauv}, the fact $(\boldsymbol  n^{\intercal}\boldsymbol \tau\boldsymbol  n)|_{\partial K}=0$ implies
\[
\partial_t(\boldsymbol  n^{\intercal}\boldsymbol  v)|_{\partial K}= (\boldsymbol  n^{\intercal}\boldsymbol \tau\boldsymbol  n)|_{\partial K}=0.
\]
Hence $\boldsymbol  n^{\intercal}\boldsymbol  v|_{\partial K}=0$.
%Hence $\boldsymbol  n^{\intercal}\boldsymbol  v|_e$ is constant on each edge $e\in\mathcal E(K)$.
%Then we can choose $\boldsymbol  v$ satisfying $\boldsymbol  n^{\intercal}\boldsymbol  v|_{\partial K}=0$.
This also means $\boldsymbol  v(\delta)=\boldsymbol 0$ for each $\delta\in\mathcal V(K)$.
%By \eqref{Hdivdivfemdof3},  it holds $(\partial_{t}(\boldsymbol  t^{\intercal}\boldsymbol \tau\boldsymbol  n)+\boldsymbol  n^{\intercal}\boldsymbol \div\boldsymbol \tau)|_{\partial K}=0$.

Again by Lemma \ref{lm:tauv}, since
\begin{equation*}%\label{eq:20200422}
\partial_{t}(\boldsymbol  t^{\intercal}\boldsymbol \tau\boldsymbol  n)+\boldsymbol  n^{\intercal}\div\boldsymbol \tau= \partial_t(\boldsymbol  t^{\intercal}\partial_t\boldsymbol  v)
\end{equation*}
and $(\partial_{t}(\boldsymbol  t^{\intercal}\boldsymbol \tau\boldsymbol  n)+\boldsymbol  n^{\intercal}\div\boldsymbol \tau)|_{\partial K}=0$, we acquire
\[
\partial_{tt}(\boldsymbol  t^{\intercal}\boldsymbol  v)|_{\partial K}=0.
\]
That is $\boldsymbol  t^{\intercal}\boldsymbol  v|_e\in\mathbb P_1(e)$ on each edge $e\in\mathcal E(K)$. Noting that $\boldsymbol  v(\delta)=\boldsymbol 0$ for each $\delta\in\mathcal V(K)$, we get $\boldsymbol  t^{\intercal}\boldsymbol  v|_{\partial K}=0$ and consequently $\boldsymbol  v|_{\partial K}=\boldsymbol 0$, i.e., 
$$
\boldsymbol  v = b_K\psi_{\ell-2},\quad \text{for some } \psi_{\ell-2} \in \mathbb P_{\ell-2}(K;\mathbb R^2).
$$

We then use the fact $\rot: \sym(\boldsymbol x^{\perp}\otimes\mathbb P_{\ell-2}(K;\mathbb R^2)) \to  \mathbb P_{\ell-2}(K;\mathbb R^2)$ is bijection, cf. Lemma \ref{lem:rot}, to find $\phi_{\ell-2}$ s.t. $\rot (\sym\boldsymbol x^{\perp}\otimes\phi_{\ell -2}) = \psi_{\ell-2}$.

Finally we finish the proof by choosing $\boldsymbol \varsigma = \sym (\boldsymbol x^{\perp}\otimes \phi_{\ell -2})$ in \eqref{Hdivdivfemdof4}. The fact 
$$
(\bs \tau, \boldsymbol \varsigma)_K = (\sym\curl b_K\psi_{\ell-2},  \sym (\boldsymbol x^{\perp}\otimes \phi_{\ell -2}))_K = ( b_K\psi_{\ell-2}, \psi_{\ell-2} )_K = 0
$$
will imply $\psi_{\ell-2} = 0$ and consequently $\bs v = 0, \bs \tau = 0$.  
\end{proof}




%\subsection{Reduced finite element space}
%
%
%Similar as the proof of Lemma \ref{lem:unisovlenHdivdivfem},
%we can prove that the degrees of freedom \eqref{Hdivdivreducefemdof1}-\eqref{Hdivdivreducefemdof4} are unisolvent for $\boldsymbol \Sigma_k'(K)$.
%
%\LC{Please count the dimension and comment on the reduction of dimension. The first row is incorrect. }
%\begin{figure}[htbp]
%\begin{center}
%\includegraphics[width=5in]{figures/doftable2.pdf}
%\caption{The number of the degrees of freedom and the dimension of the space for the reduced finite element space.}
%\label{default}
%\end{center}
%\end{figure}


\subsection{Finite element $\div$-$\div$ complex}
Recall the $\div$-$\div$ Hilbert complexes with different regularity
\begin{equation}\label{eq:divdivcomplexL2}
% \resizebox{.9\hsize}{!}{$
\boldsymbol  {RT}\autorightarrow{$\subset$}{} \boldsymbol  H^1(K;\mathbb R^2)\autorightarrow{$\sym\curl$}{} \boldsymbol{H}(\div{\div},K; \mathbb{S}) \autorightarrow{$\div{\div }$}{} L^2(K)\autorightarrow{}{}0,
% $}
\end{equation}
\begin{equation}\label{eq:divdivcomplexH2}
%\resizebox{.9\hsize}{!}{$
\boldsymbol  {RT}\autorightarrow{$\subset$}{} \boldsymbol  H^3(K;\mathbb R^2)\autorightarrow{$\sym\curl$}{} \boldsymbol{H}^2(K; \mathbb{S}) \autorightarrow{$\div{\div}$}{} L^2(K)\autorightarrow{}{}0.
\end{equation}
Both complexes \eqref{eq:divdivcomplexL2} and \eqref{eq:divdivcomplexH2} are exact; see \cite{ChenHuang2018}.

We have constructed finite element spaces for $\boldsymbol{H}(\div{\div},K; \mathbb{S}) $. Now we define a vectorial $H^1$-conforming finite element. Let $\boldsymbol V_{\ell+1}(K):=\mathbb P_{\ell+1}(K;\mathbb R^2)$ with $\ell\geq2$.
The local degrees of freedom are given by
\begin{align}
\boldsymbol  v (\delta), \nabla\boldsymbol  v (\delta) & \quad\forall~\delta\in \mathcal V(K), \label{HermitfemVdof1}\\
(\boldsymbol  v, \boldsymbol  q)_e & \quad\forall~\boldsymbol  q\in\mathbb P_{\ell-3}(e;\mathbb R^2),  e\in\mathcal E(K),\label{HermitfemVdof2}\\
(\boldsymbol  v, \boldsymbol  q)_K & \quad\forall~\boldsymbol  q\in\mathbb P_{\ell-2}(K;\mathbb R^2). \label{HermitfemVdof3}
\end{align}
This finite element is just the vectorial Hermite element \cite{BrennerScott2008,Ciarlet1978}. 

\begin{lemma}
For any triangle $K$, both the polynomial complexes
\begin{equation}\label{eq:divdivcomplexPolyvar}
%\resizebox{.9\hsize}{!}{$
\boldsymbol  {RT}\autorightarrow{$\subset$}{} \boldsymbol V_{\ell+1}(K)\autorightarrow{$\sym\curl$}{} \boldsymbol \Sigma_{\ell,k}(K) \autorightarrow{$\div{\div}$}{} \mathbb P_{k-2}(K)\autorightarrow{}{}0
\end{equation}
and
\begin{equation}\label{eq:divdivcomplexPolyvar0}
%\resizebox{.9\hsize}{!}{$
\boldsymbol 0\autorightarrow{$\subset$}{} \mathring{\boldsymbol V}_{\ell+1}(K)\autorightarrow{$\sym\curl$}{} \mathring{\boldsymbol \Sigma}_{\ell,k}(K) \autorightarrow{$\div{\div}$}{} \mathring{\mathbb P}_{k-2}(K)\autorightarrow{}{}0
\end{equation}
are exact, where
\begin{align*}
\mathring{\boldsymbol V}_{\ell+1}(K)&:=\{\boldsymbol  v\in\boldsymbol V_{\ell+1}(K): \textrm{all degrees of freedom } \eqref{HermitfemVdof1}-\eqref{HermitfemVdof2} \textrm{ vanish}\},
\\
\mathring{\boldsymbol \Sigma}_{\ell,k}(K)&:=\{\boldsymbol  \tau\in\boldsymbol \Sigma_{\ell,k}(K): \textrm{all degrees of freedom } \eqref{Hdivdivfemdof1}-\eqref{Hdivdivfemdof3} \textrm{ vanish}\},
\\
\mathring{\mathbb P}_{k-2}(K)&:=\mathbb P_{k-2}(K)/\mathbb P_{1}(K).
\end{align*}
\end{lemma}
\begin{proof}
The exactness of the complex \eqref{eq:divdivcomplexPolyvar} follows from the exactness of the complex~\eqref{eq:divdivcomplexPoly} and Lemma~\ref{lem:symmpolyspacedirectsum}.

By the proof of Lemma~\ref{lem:unisovlenHdivdivfem}, we have $\mathring{\boldsymbol \Sigma}_{\ell,k}(K)\cap\ker(\div{\div})=\sym \curl\mathring{\boldsymbol V}_{\ell+1}(K)$. This also means
\[
\dim\div{\div}\mathring{\boldsymbol \Sigma}_{\ell,k}(K)=\dim\mathring{\boldsymbol \Sigma}_{\ell,k}(K)-\dim\mathring{\boldsymbol V}_{\ell+1}(K)=\frac{1}{2}k(k-1)-3=\dim\mathring{\mathbb P}_{k-2}(K).
\]
Due to the Green's identity \eqref{eq:greenidentitydivdiv}, we get $\div{\div}\mathring{\boldsymbol \Sigma}_{\ell,k}(K)\subseteq \mathring{\mathbb P}_{k-2}(K)$, which ends the proof.
\end{proof}

To show the commutative diagram for the polynomial complex \eqref{eq:divdivcomplexPolyvar}, we introduce $\boldsymbol \Pi_K: \boldsymbol{H}^2(K; \mathbb{S})\to\boldsymbol \Sigma_{\ell,k}(K)$ be the nodal interpolation operator based on the degrees of freedom \eqref{Hdivdivfemdof1}-\eqref{Hdivdivfemdof4}. We have $\boldsymbol \Pi_K\boldsymbol \tau=\boldsymbol \tau$ for any $\boldsymbol \tau\in\mathbb P_{\min\{\ell,k\}}(K;\mathbb S)$, and
\begin{equation}\label{eq:PiKestimate}
\|\boldsymbol \tau-\boldsymbol \Pi_K\boldsymbol \tau\|_{0,K}+h_K|\boldsymbol \tau-\boldsymbol \Pi_K\boldsymbol \tau|_{1,K}+h_K^2|\boldsymbol \tau-\boldsymbol \Pi_K\boldsymbol \tau|_{2,K}\lesssim h_K^{s}|\boldsymbol \tau|_{s, K}
\end{equation}
for any $\boldsymbol \tau\in\boldsymbol{H}^s(K; \mathbb{S})$
with $2\leq s\leq \min\{\ell,k\}+1$. It follows from the Green's identity \eqref{eq:greenidentitydivdiv} that
\begin{equation}\label{eq:PiKcd}
\div\div(\boldsymbol \Pi_K\boldsymbol \tau)=Q_{k-2}^K\div \div\boldsymbol \tau\quad\forall~\boldsymbol \tau\in\boldsymbol{H}^2(K; \mathbb{S}).
\end{equation}
%for $\ell\geq k-1$. %Hereafter we always assume $\ell\geq k-1$.


%Let the interpolation operator $\boldsymbol  I_K: \boldsymbol  H^3(K;\mathbb R^2)\to\boldsymbol V_{\ell+1}(K)$.
%For $\ell\geq k-1$, take $\boldsymbol \Pi_K:=\widetilde{\boldsymbol \Pi}_K$ which is  the nodal interpolation operator based on the degrees of freedom \eqref{Hdivdivfemdof1}-\eqref{Hdivdivfemdof4}.
%
%The case $2\leq\ell\leq k-2$ requires a twist of the definition.
%For any $\boldsymbol \tau\in\boldsymbol{H}^2(K; \mathbb{S})$,
%employing
%the Green's identity \eqref{eq:greenidentitydivdiv}, we get $Q_{k-2}^K\div\boldsymbol \div(\boldsymbol\tau-\widetilde{\boldsymbol \Pi}_K\boldsymbol\tau)\in\mathring{\mathbb P}_{k-2}(K)$. According to the complex \eqref{eq:divdivcomplexPolyvar0}, there exists $\widetilde{\boldsymbol \tau}\in\mathring{\boldsymbol \Sigma}_{k-2,k}(K)$ satisfying
%\[
%\div\boldsymbol \div\widetilde{\boldsymbol\tau}=Q_{k-2}^K\div\boldsymbol \div(\boldsymbol\tau-\widetilde{\boldsymbol \Pi}_K\boldsymbol\tau),
%\]
%\[
% \|\widetilde{\boldsymbol\tau}\|_{0,K}\lesssim h_K^2\|Q_{k-2}^K\div\boldsymbol \div(\boldsymbol\tau-\widetilde{\boldsymbol \Pi}_K\boldsymbol\tau)\|_{0,K}.
%\]
%Now set $\boldsymbol \Pi_K\boldsymbol \tau:=\widetilde{\boldsymbol \Pi}_K\boldsymbol \tau+\widetilde{\boldsymbol\tau}$. It holds $\boldsymbol \Pi_K\boldsymbol \tau=\boldsymbol \tau$ for any $\boldsymbol \tau\in\mathbb P_{\min\{\ell,k\}}(K;\mathbb S)$.
%For $\ell\geq 2$, we acquire from \eqref{eq:Pitildecd} and \eqref{eq:Pitildeestimate} that
%\begin{equation}\label{eq:PiKcd}
%\div\boldsymbol \div(\boldsymbol \Pi_K\boldsymbol \tau)=Q_{k-2}^K\div\boldsymbol \div\boldsymbol \tau\quad\forall~\boldsymbol \tau\in\boldsymbol{H}^2(K; \mathbb{S}),
%\end{equation}
%\begin{equation}\label{eq:PiKestimate}
%\|\boldsymbol \tau-\boldsymbol \Pi_K\boldsymbol \tau\|_{0,K}+h_K|\boldsymbol \tau-\boldsymbol \Pi_K\boldsymbol \tau|_{1,K}+h_K^2|\boldsymbol \tau-\boldsymbol \Pi_K\boldsymbol \tau|_{2,K}\lesssim h_K^{s}|\boldsymbol \tau|_{s, K}
%\end{equation}
%for any $\boldsymbol \tau\in\boldsymbol{H}^s(K; \mathbb{S})$ with $2\leq s\leq \min\{\ell,k\}+1$.
Let $\widetilde{\boldsymbol  I}_K: \boldsymbol{H}^3(K; \mathbb{R}^2)\to\boldsymbol V_{\ell+1}(K)$
be the nodal interpolation operator based on the degrees of freedom \eqref{HermitfemVdof1}-\eqref{HermitfemVdof3}.
We have $\widetilde{\boldsymbol  I}_K\boldsymbol  q=\boldsymbol  q$ for any $\boldsymbol  q\in\mathbb P_{\ell+1}(K;\mathbb R^2)$, and
\begin{equation}\label{eq:IKtildeestimate}
\|\boldsymbol  v-\widetilde{\boldsymbol  I}_K\boldsymbol  v\|_{0,K}+h_K|\boldsymbol  v-\widetilde{\boldsymbol  I}_K\boldsymbol  v|_{1,K} \lesssim h_K^{s}|\boldsymbol  v|_{s, K} \quad\forall~\boldsymbol  v\in \boldsymbol{H}^s(K; \mathbb{R}^2)
\end{equation}
with $3\leq s\leq \ell+2$.
Then we define $\boldsymbol  I_K: \boldsymbol{H}^3(K; \mathbb{R}^2)\to\boldsymbol V_{\ell+1}(K)$ by modifying $\widetilde{\boldsymbol  I}_K$.
By \eqref{eq:trace1} and \eqref{eq:trace2},
clearly we have $\boldsymbol \Pi_K(\sym\curl \, \boldsymbol  v) - \sym \curl(\widetilde{\boldsymbol  I}_K\boldsymbol  v)\in\mathring{\boldsymbol \Sigma}_{\ell,k}(K)$ for any $\boldsymbol  v\in\boldsymbol{H}^3(K; \mathbb{R}^2)$. And it holds from \eqref{eq:PiKcd} that
\[
\div\div(\boldsymbol \Pi_K(\sym\curl \, \boldsymbol  v) - \sym\curl(\widetilde{\boldsymbol  I}_K\boldsymbol  v))=0.
\]
Thus using the complex \eqref{eq:divdivcomplexPolyvar0}, there exists $\widetilde{\boldsymbol  v}\in\mathring{\boldsymbol V}_{\ell+1}(K)$ satisfying
\[
\sym\curl\widetilde{\boldsymbol  v}=\boldsymbol \Pi_K(\sym\curl \, \boldsymbol  v) - \sym\curl(\widetilde{\boldsymbol  I}_K\boldsymbol  v),
\]
\[
 \|\widetilde{\boldsymbol v}\|_{0,K}\lesssim h_K\|\boldsymbol \Pi_K(\sym\curl \, \boldsymbol  v) - \sym\curl(\widetilde{\boldsymbol  I}_K\boldsymbol  v)\|_{0,K}.
\]
Let $\boldsymbol  I_K\boldsymbol  v:=\widetilde{\boldsymbol  I}_K\boldsymbol  v+\widetilde{\boldsymbol  v}$. Apparently $\boldsymbol  I_K\boldsymbol  q=\boldsymbol  q$ for any $\boldsymbol  q\in\mathbb P_{\ell+1}(K;\mathbb R^2)$, and
\begin{equation}\label{eq:IKcd}
\sym\curl(\boldsymbol  I_K\boldsymbol  v)=\boldsymbol \Pi_K(\sym\curl \, \boldsymbol  v) \quad\forall~\boldsymbol  v\in\boldsymbol{H}^3(K; \mathbb{R}^2).
\end{equation}
It follows from  \eqref{eq:IKtildeestimate} and \eqref{eq:PiKestimate} that
\begin{equation}\label{eq:IKestimate}
\|\boldsymbol  v-\boldsymbol  I_K\boldsymbol  v\|_{0,K}+h_K|\boldsymbol  v-\boldsymbol  I_K\boldsymbol  v|_{1,K} \lesssim h_K^{s}|\boldsymbol  v|_{s, K} \quad\forall~\boldsymbol  v\in \boldsymbol{H}^s(K; \mathbb{R}^2).
\end{equation}
with $3\leq s\leq \ell+2$.

In summary, we have the following commutative diagram for the local finite element complex \eqref{eq:divdivcomplexPolyvar}
$$
\begin{array}{c}
\xymatrix{
  \boldsymbol{RT} \ar[r]^-{\subset} & \boldsymbol  H^3(K;\mathbb R^2) \ar[d]^{\boldsymbol{I}_K} \ar[r]^-{\sym\curl}
                & \boldsymbol{H}^2(K; \mathbb{S}) \ar[d]^{\boldsymbol{\Pi}_K}   \ar[r]^-{\div{\div}} & \ar[d]^{Q_K}L^2(K) \ar[r]^{} & 0 \\
 \boldsymbol{RT} \ar[r]^-{\subset} & \boldsymbol V_{\ell+1}(K) \ar[r]^{\sym\curl}
                &  \boldsymbol \Sigma_{\ell,k}(K)   \ar[r]^{\div{\div}} &  \mathbb P_{k-2}(K) \ar[r]^{}& 0    }
\end{array}
$$
with $Q_K:=Q_{k-2}^K$.

We then glue local finite element spaces to get global conforming spaces.
Define
\begin{align*}
\boldsymbol  V_h:=\{\boldsymbol  v_h\in \boldsymbol  H^1(\Omega;\mathbb R^2):&\, \boldsymbol  v_h|_K\in \mathbb P_{\ell+1}(K;\mathbb R^2) \textrm{ for each } K\in\mathcal T_h, \\
& \textrm{ all the degrees of freedom } \eqref{HermitfemVdof1}-\eqref{HermitfemVdof2}  \textrm{ are single-valued}\},
\end{align*}
\begin{align*}
\boldsymbol\Sigma_h:=\{\boldsymbol \tau_h\in \boldsymbol  L^2(\Omega;\mathbb S):&\, \boldsymbol \tau_h|_K\in \boldsymbol \Sigma_{\ell,k}(K) \textrm{ for each } K\in\mathcal T_h, \\
& \textrm{ all the degrees of freedom } \eqref{Hdivdivfemdof1}-\eqref{Hdivdivfemdof3} \textrm{ are single-valued}\},
\end{align*}
\[
\mathcal Q_h :=\mathbb P_{k-2}(\mathcal T_h)=\{q_h\in L^2(\Omega): q_h|_K\in \mathbb P_{k-2}(K) \textrm{ for each } K\in\mathcal T_h\}.
\]
Due to Lemma~\ref{lem:Hdivdivpatching}, the finite element space $\boldsymbol \Sigma_h\subset\boldsymbol{H}(\div{\div },\Omega; \mathbb{S})$.
Let $\boldsymbol  I_h: \boldsymbol{H}^3(\Omega; \mathbb{S})\to\boldsymbol V_h$, $\boldsymbol \Pi_h: \boldsymbol{H}^2(\Omega; \mathbb{S})\to\boldsymbol\Sigma_h$ and $Q_h^{m}: L^2(\Omega)\to\mathbb P_{m}(\mathcal T_h)$ be defined by $(\boldsymbol  I_h\boldsymbol  v)|_K:=\boldsymbol  I_K(\boldsymbol  v|_K)$, $(\boldsymbol \Pi_h\boldsymbol \tau)|_K:=\boldsymbol \Pi_K(\boldsymbol \tau|_K)$  and $(Q_h^{m} q)|_K:=Q_{m}^K(q|_K)$ for each $K\in\mathcal T_h$, respectively. When the degree is clear from the context, we will simply write the $L^2$-projection $Q_h^{k-2}$ as $Q_h$.

As direct results of \eqref{eq:PiKcd} and \eqref{eq:IKcd}, we have
\begin{equation}\label{eq:Pihcd}
\div\div(\boldsymbol \Pi_h\boldsymbol \tau)=Q_{h}\div\div\boldsymbol \tau\quad\forall~\boldsymbol \tau\in\boldsymbol{H}^2(\Omega; \mathbb{S}),
\end{equation}
\begin{equation}\label{eq:Ihcd}
\sym\curl(\boldsymbol  I_h\boldsymbol  v)=\boldsymbol \Pi_h(\sym\curl \, \boldsymbol  v) \quad\forall~\boldsymbol  v\in\boldsymbol{H}^3(\Omega; \mathbb{R}^2).
\end{equation}

\begin{lemma}
The finite element complex
\begin{equation}\label{eq:divdivcomplexfem}
%\resizebox{.9\hsize}{!}{$
\boldsymbol  {RT}\autorightarrow{$\subset$}{} \boldsymbol  V_h\autorightarrow{$\sym\curl$}{} \boldsymbol \Sigma_h \autorightarrow{$\div{\div}$}{} \mathcal Q_h\autorightarrow{}{}0
\end{equation}
is exact. Moreover, we have the commutative diagram
\begin{equation}\label{eq:divdivcdfem}
\begin{array}{c}
\xymatrix{
  \boldsymbol{RT} \ar[r]^-{\subset} & \boldsymbol  H^3(\Omega;\mathbb R^2) \ar[d]^{\boldsymbol{I}_h} \ar[r]^-{\sym\curl}
                & \boldsymbol{H}^2(\Omega; \mathbb{S}) \ar[d]^{\boldsymbol{\Pi}_h}   \ar[r]^-{\div{\div}} & \ar[d]^{Q_{h}}L^2(\Omega) \ar[r]^{} & 0 \\
 \boldsymbol{RT} \ar[r]^-{\subset} & \boldsymbol V_{h} \ar[r]^{\sym \curl}
                &  \boldsymbol \Sigma_{h}   \ar[r]^{\div{\div}} &  \mathcal Q_h \ar[r]^{}& 0    }
\end{array}.
\end{equation}
\end{lemma}
\begin{proof}
By the complex \eqref{eq:divdivcomplexH2}, for any $q_h\in \mathcal Q_h$, there exists $\boldsymbol \tau \in\boldsymbol{H}^2(\Omega; \mathbb{S})$ satisfying $\div\div\boldsymbol \tau=q_h$. Then it follows from \eqref{eq:Pihcd} that
\[
\div\div(\boldsymbol  \Pi_h\boldsymbol \tau)=Q_h\div\div\boldsymbol \tau=q_h.
\]
Hence $\div{\div}\boldsymbol \Sigma_h = \mathcal Q_h$.
On the other hand, by counting we get
\[
\dim\boldsymbol \Sigma_h=3\#\mathcal V_h+(2\ell-1)\#\mathcal E_h+\ell(\ell-1)\#\mathcal T_h+\frac{1}{2}(k+2)(k-3)\#\mathcal T_h,
\]
\[
\dim\sym\curl \, \boldsymbol  V_h=6\#\mathcal V_h+(2\ell-4)\#\mathcal E_h+\ell(\ell-1)\#\mathcal T_h-3,
\]
\[
\dim\div{\div}\boldsymbol \Sigma_h=\dim\mathbb P_{k-2}(\mathcal T_h)=\frac{1}{2}k(k-1)\#\mathcal T_h.
\]
Here $\#\mathcal S$ means the number of the elements in the finite set $\mathcal S$.
It follows from the Euler's formula $\#\mathcal E_h+1=\#\mathcal V_h+\#\mathcal T_h$ that
\[
\dim\boldsymbol \Sigma_h=\dim\sym\curl \, \boldsymbol  V_h+\dim\div{\div}\boldsymbol \Sigma_h.
\]
Therefore the complex \eqref{eq:divdivcomplexfem} is exact.

The commutative diagram \eqref{eq:divdivcdfem} follows from \eqref{eq:Pihcd} and \eqref{eq:Ihcd}.
\end{proof}

\begin{remark}\rm
Using the smoothing procedure \cite{ArnoldFalkWinther2006}, the natural interpolations in the commutative diagram \eqref{eq:divdivcdfem} can be refined to quasi-interpolations and the top one can be replaced by the complex \eqref{eq:divdivcomplexL2} with minimal regularity.
\end{remark}

\subsection{Conforming finite element spaces for strain complex}
In the application of linear elasticity, the strain complex is more relevant. 
As the rotated version of \eqref{eq:greenidentitydivdiv}, we get the Green's identity
\begin{align}
(\rot\rot\boldsymbol \tau, v)_K&=(\boldsymbol \tau, \curl\curl v)_K+\sum_{e\in\mathcal E(K)}\sum_{\delta\in\partial e}\sign_{e,\delta}(\boldsymbol  n^{\intercal}\boldsymbol \tau\boldsymbol  t)(\delta)v(\delta) \notag\\
&\quad-\sum_{e\in\mathcal E(K)}\left[(\boldsymbol  t^{\intercal}\boldsymbol \tau \boldsymbol  t, \partial_n v)_{e}+(\partial_{t}(\boldsymbol  n^{\intercal}\boldsymbol \tau\boldsymbol  t)-\boldsymbol  t^{\intercal}\rot\boldsymbol \tau,  v)_{e}\right] \label{eq:greenidentityrotrot}
\end{align}
for any $\boldsymbol \tau\in \mathcal C^2(K; \mathbb S)$ and $v\in H^2(K)$.


As a result, we have the following characterization of $\boldsymbol{H}(\rot{\rot },\Omega; \mathbb{S})$.
\begin{lemma}\label{lem:Hrotrotpatching}
Let $\boldsymbol \tau\in \boldsymbol  L^2(\Omega;\mathbb S)$ such that
\begin{enumerate}[(i)]
\item $\boldsymbol \tau|_K\in \boldsymbol{H}(\rot{\rot },K; \mathbb{S})$ for each $K\in\mathcal T_h$;

\smallskip
\item $(\boldsymbol  t^{\intercal}\boldsymbol \tau\boldsymbol  t)|_e\in L^2(e)$ is single-valued for each $e\in\mathcal E_h^i$;

\smallskip
\item $(-\partial_{t_e}(\boldsymbol  n^{\intercal}\boldsymbol \tau\boldsymbol  t)+\boldsymbol  t_e^{\intercal}\rot\boldsymbol \tau)|_e\in L^2(e)$ is single-valued for each $e\in\mathcal E_h^i$;

\smallskip
\item $\boldsymbol \tau(\delta)$ is single-valued for each $\delta\in\mathcal V_h^i$,
\end{enumerate}
then $\boldsymbol \tau\in \boldsymbol{H}(\rot{\rot },\Omega; \mathbb{S})$.
\end{lemma}

%Define the space of shape functions $\boldsymbol \Sigma_{\ell,k}^{\perp}(K)$ with $k\geq 3$ and $\ell\geq k-1$ as follows.
Take the space of shape functions
\[
\boldsymbol \Sigma_{\ell,k}^{\perp}(K):= \mathbb E_{\ell}(K;\mathbb S)\oplus\mathbb E_k^{\oplus}(K;\mathbb S)
\]
with $k\geq 3$ and $\ell\geq k-1$, where recall that the spaces $\mathbb E_{\ell}(K;\mathbb S), \mathbb E_k^{\oplus}(K;\mathbb S)$ are introduced in Remark \ref{rm:Ek}.
%for $\ell\geq k-1$, and
%\begin{align*}
%\boldsymbol \Sigma_{\ell,k}^{\perp}(K)=\{\boldsymbol \tau\in\mathbb P_{k}(K;\mathbb S):&\, \boldsymbol \tau\boldsymbol x \in \mathbb P_{\ell+1}(K;\mathbb R^2), \boldsymbol  t^{\intercal}\boldsymbol \tau\boldsymbol t|_e\in\mathbb P_{\ell}(e),\\
%&(-\partial_{t}(\boldsymbol  n^{\intercal}\boldsymbol \tau\boldsymbol  t)+\boldsymbol  t^{\intercal}\boldsymbol \rot\boldsymbol\tau)|_e\in\mathbb P_{\ell-1}(e)\quad\forall~e\in\mathcal E(K) \}
%\end{align*}
%for $\ell\leq k-2$.
The local degrees of freedom are given by
\begin{align}
%v(\delta) & \quad\textrm{for each } \delta\in \mathcal V(K), \label{H1ncfmdof1}\\
\boldsymbol \tau (\delta) & \quad\forall~\delta\in \mathcal V(K), \label{Hrotrotfemdof1}\\
(\boldsymbol  t^{\intercal}\boldsymbol \tau\boldsymbol  t, q)_e & \quad\forall~q\in\mathbb P_{\ell-2}(e),  e\in\mathcal E(K),\label{Hrotrotfemdof2}\\
(-\partial_{t}(\boldsymbol  n^{\intercal}\boldsymbol \tau\boldsymbol  t)+\boldsymbol  t^{\intercal}\rot\boldsymbol \tau, q)_e & \quad\forall~q\in\mathbb P_{\ell-1}(e),  e\in\mathcal E(K),\label{Hrotrotfemdof3}\\
(\boldsymbol \tau, \boldsymbol \varsigma)_K & \quad\forall~\boldsymbol \varsigma\in\curl\curl \, \mathbb P_{k-2}(K)\oplus
 \sym (\bs x\otimes \mathbb P_{\ell-2}(K;\mathbb R^2)).\label{Hrotrotfemdof4}
%\boldsymbol \defm(b_K\mathbb P_{\ell-2}(K;\mathbb R^2)). 
\end{align}
The degrees of freedom \eqref{Hrotrotfemdof1}-\eqref{Hrotrotfemdof4} are unisolvent for $\boldsymbol \Sigma_{\ell,k}^{\perp}(K)$.


Let
$
\mathbb A:=\begin{pmatrix}
0 & -1\\
1 & 0
\end{pmatrix}
$, then
\[
\boldsymbol  t=\mathbb A \boldsymbol  n,\quad \curl\phi=\mathbb A^{\intercal}\grad\phi,\quad \rot\boldsymbol  v=\div(\mathbb A^{\intercal} \boldsymbol  v), \quad \rot\boldsymbol \tau=\mathbb A\div(\mathbb A^{\intercal}\boldsymbol \tau\mathbb A),
\]
\[
 \rot\rot\boldsymbol \tau=\div\div(\mathbb A^{\intercal}\boldsymbol \tau\mathbb A),\;\; \curl\curl v=\mathbb A^{\intercal}\nabla^2v\mathbb A,\;\; \defm\boldsymbol  v=\mathbb A\sym\curl(\mathbb A^{\intercal}\boldsymbol  v)\mathbb A^{\intercal},
\]
\[
-\partial_{t}(\boldsymbol  n^{\intercal}\boldsymbol \tau\boldsymbol  t)+\boldsymbol  t^{\intercal}\rot\boldsymbol \tau=\partial_{t}(\boldsymbol  t^{\intercal}\mathbb A^{\intercal}\boldsymbol \tau\mathbb A\boldsymbol  n)+\boldsymbol  n^{\intercal}\div(\mathbb A^{\intercal}\boldsymbol \tau\mathbb A)
\]
for sufficiently smooth scalar field $\phi$, vectorial field $\boldsymbol  v$ and tensorial field $\boldsymbol \tau$. Moreover, we have
\[
\boldsymbol  x^{\perp}=\mathbb A^{\intercal} \boldsymbol  x,\quad \mathbb E_{\ell}(K;\mathbb S)=\mathbb A^{\intercal}\mathbb C_{\ell}(K;\mathbb S)\mathbb A,\quad \mathbb E_k^{\oplus}(K;\mathbb S)=\mathbb A^{\intercal}\mathbb C_k^{\oplus}(K;\mathbb S)\mathbb A.
\]
Then the exactness of the complex \eqref{eq:divdivcomplexPolyvar} implies that the local finite element strain complex
\begin{equation}\label{eq:rotrotcomplexPolyvar}
%\resizebox{.9\hsize}{!}{$
\boldsymbol  {RM}\autorightarrow{$\subset$}{} \boldsymbol V_{\ell+1}(K)\autorightarrow{$\defm$}{} \boldsymbol \Sigma_{\ell,k}^{\perp}(K) \autorightarrow{$\rot{\rot}$}{} \mathbb P_{k-2}(K)\autorightarrow{}{}0
\end{equation}
is exact.

Define interpolation operators $\boldsymbol \Pi_K^{\perp}: \boldsymbol{H}^2(K; \mathbb{S})\to\boldsymbol \Sigma_{\ell,k}^{\perp}(K)$ and $\boldsymbol  I_K^{\perp}: \boldsymbol H^3(K;\mathbb R^2)\to\boldsymbol V_{\ell+1}(K)$ as
\[
\boldsymbol \Pi_K^{\perp}\boldsymbol \tau:=\mathbb A(\boldsymbol \Pi_K(\mathbb A^{\intercal}\boldsymbol \tau\mathbb A))\mathbb A^{\intercal}\quad\forall~\boldsymbol \tau\in\boldsymbol{H}^2(K; \mathbb{S}),
\]
\[
\boldsymbol  I_K^{\perp}\boldsymbol  v:=\mathbb A\boldsymbol  I_K(\mathbb A^{\intercal}\boldsymbol  v)\quad\forall~\boldsymbol  v\in \boldsymbol H^3(K;\mathbb R^2).
\]
It follows from \eqref{eq:PiKcd} and \eqref{eq:IKcd} that
\[
\rot\rot\boldsymbol \Pi_K^{\perp}\boldsymbol \tau=\div\div(\boldsymbol \Pi_K(\mathbb A^{\intercal}\boldsymbol \tau\mathbb A))=Q_{k-2}^K\div\div(\mathbb A^{\intercal}\boldsymbol \tau\mathbb A)=Q_{k-2}^K\rot\rot\boldsymbol \tau,
\]
\begin{align*}
\defm(\boldsymbol  I_K^{\perp}\boldsymbol  v)&=\mathbb A\sym\curl(\mathbb A^{\intercal}\boldsymbol  I_K^{\perp}\boldsymbol  v)\mathbb A^{\intercal}=\mathbb A\sym\curl(\boldsymbol  I_K(\mathbb A^{\intercal}\boldsymbol  v))\mathbb A^{\intercal} \\
&=\mathbb A\boldsymbol \Pi_K(\sym\curl(\mathbb A^{\intercal}\boldsymbol  v))\mathbb A^{\intercal} =\mathbb A\boldsymbol \Pi_K(\mathbb A^{\intercal}(\defm\boldsymbol v)\mathbb A)\mathbb A^{\intercal}=\boldsymbol \Pi_K^{\perp}(\defm\boldsymbol v).
\end{align*}
Therefore we have the following commutative diagram for the local finite element complex \eqref{eq:rotrotcomplexPolyvar}
$$
\begin{array}{c}
\xymatrix{
  \boldsymbol{RM} \ar[r]^-{\subset} & \boldsymbol  H^3(K;\mathbb R^2) \ar[d]^{\boldsymbol{I}_K^{\perp}} \ar[r]^-{\defm}
                & \boldsymbol{H}^2(K; \mathbb{S}) \ar[d]^{\boldsymbol{\Pi}_K^{\perp}}   \ar[r]^-{\rot{\rot}} & \ar[d]^{Q_K}L^2(K) \ar[r]^{} & 0 \\
 \boldsymbol{RM} \ar[r]^-{\subset} & \boldsymbol V_{\ell+1}(K) \ar[r]^{\defm}
                &  \boldsymbol \Sigma_{\ell,k}^{\perp}(K)   \ar[r]^{\rot{\rot}} &  \mathbb P_{k-2}(K) \ar[r]^{}& 0    }
\end{array}.
$$

Define
\begin{align*}
\boldsymbol\Sigma_h^{\perp}:=\{\boldsymbol \tau_h\in \boldsymbol  L^2(\Omega;\mathbb S):&\, \boldsymbol \tau_h|_K\in \boldsymbol \Sigma_{\ell,k}^{\perp}(K) \textrm{ for each } K\in\mathcal T_h, \textrm{ all the degrees of }\\
& \qquad\qquad\qquad\;\textrm{ freedom } \eqref{Hrotrotfemdof1}-\eqref{Hrotrotfemdof3} \textrm{ are single-valued}\}.
\end{align*}
Thanks to Lemma~\ref{lem:Hrotrotpatching}, the finite element space $\boldsymbol \Sigma_h^{\perp}\subset\boldsymbol{H}(\rot{\rot},\Omega; \mathbb{S})$.
Let $\boldsymbol  I_h^{\perp}: \boldsymbol{H}^3(\Omega; \mathbb{S})\to\boldsymbol V_h$ and $\boldsymbol \Pi_h^{\perp}: \boldsymbol{H}^2(\Omega; \mathbb{S})\to\boldsymbol\Sigma_h^{\perp}$ be defined by $(\boldsymbol  I_h^{\perp}\boldsymbol  v)|_K:=\boldsymbol  I_K^{\perp}(\boldsymbol  v|_K)$ and $(\boldsymbol \Pi_h^{\perp}\boldsymbol \tau)|_K:=\boldsymbol \Pi_K^{\perp}(\boldsymbol \tau|_K)$, respectively.
Similarly as the commutative diagram \eqref{eq:divdivcdfem}, we have the commutative diagram
\begin{equation*}%\label{eq:rotrotcdfem}
\begin{array}{c}
\xymatrix{
  \boldsymbol{RM} \ar[r]^-{\subset} & \boldsymbol  H^3(\Omega;\mathbb R^2) \ar[d]^{\boldsymbol{I}_h^{\perp}} \ar[r]^-{\defm}
                & \boldsymbol{H}^2(\Omega; \mathbb{S}) \ar[d]^{\boldsymbol{\Pi}_h^{\perp}}   \ar[r]^-{\rot{\rot}} & \ar[d]^{Q_{h}}L^2(\Omega) \ar[r]^{} & 0 \\
 \boldsymbol{RM} \ar[r]^-{\subset} & \boldsymbol V_{h} \ar[r]^{\defm}
                &  \boldsymbol \Sigma_{h}^{\perp}   \ar[r]^{\rot{\rot}} &  \mathcal Q_h \ar[r]^{}& 0    }
\end{array}.
\end{equation*}



\section{Mixed finite element methods for biharmonic equation}

In this section we will apply the $\boldsymbol H(\div\div)$-conforming finite element pair $(\boldsymbol\Sigma_h, \mathcal Q_h)$ to solve the biharmonic equation
\begin{equation}\label{eq:biharmonic}
\begin{cases}
\quad\;\;\,\Delta^2u = -f & \textrm{ in } \Omega,\\
u=\partial_nu=0 & \textrm{ on } \partial\Omega,
\end{cases}
\end{equation}
where $f\in L^2(\Omega)$. A mixed formulation of the biharmonic equation \eqref{eq:biharmonic} is to find $\boldsymbol \sigma\in\boldsymbol{H}(\div{\div },\Omega; \mathbb{S})$ and $u\in L^2(\Omega)$ such that
\begin{align}
(\boldsymbol{\sigma}, \boldsymbol{\tau})+(\div\div\boldsymbol{\tau}, u)&=0 \quad\quad\quad\;\; \forall~\boldsymbol{\tau}\in\boldsymbol{H}(\div{\div},\Omega; \mathbb{S}), \label{mixedform1} \\
(\div\div\boldsymbol{\sigma}, v)&=(f, v) \quad\;\;\; \forall~v\in L^2(\Omega). \label{mixedform2}
\end{align}
Note that Dirichlet-type boundary of $u$ is imposed as natural condition in the mixed formulation.

\subsection{Mixed finite element methods}
Employing the finite element spaces $\boldsymbol\Sigma_h\times\mathcal Q_h$ to discretize $\boldsymbol{H}(\div{\div },\Omega; \mathbb{S})\times L^2(\Omega)$, we propose the following discrete methods for the mixed formulation \eqref{mixedform1}-\eqref{mixedform2}:
find $\boldsymbol \sigma_h\in\boldsymbol\Sigma_h$ and $u_h\in \mathcal Q_h$ such that
\begin{align}
(\boldsymbol{\sigma}_h, \boldsymbol{\tau}_h)+(\div\div\boldsymbol{\tau}_h, u_h)&=0 \quad\quad\quad\;\; \forall~\boldsymbol{\tau}_h\in \boldsymbol\Sigma_h, \label{mfem1} \\
(\div\div\boldsymbol{\sigma}_h, v_h)&=(f, v_h) \quad\; \forall~v_h\in \mathcal Q_h. \label{mfem2}
\end{align}

As a result of \eqref{eq:Pihcd} and \eqref{eq:PiKestimate}, we have the inf-sup condition
\begin{equation*}%\label{eq:discreteinfsup}
\|v_h\|_0\lesssim \sup_{\boldsymbol{\tau}_h\in\boldsymbol\Sigma_h}\frac{(\div\div\boldsymbol{\tau}_h, v_h)}{\|\boldsymbol{\tau}_h\|_{\boldsymbol{H}(\div{\div })}}.
\end{equation*}
By the Babu{\v{s}}ka-Brezzi theory \cite{BoffiBrezziFortin2013}, the following stability result holds
\begin{equation}\label{eq:discretestability}
\|\widetilde{\boldsymbol{\sigma}}_h\|_{\boldsymbol{H}(\div{\div })}+\|\widetilde{u}_h\|_0\lesssim \sup_{\boldsymbol{\tau}_h\in\boldsymbol\Sigma_h\atop v_h\in\mathcal Q_h}\frac{(\widetilde{\boldsymbol{\sigma}}_h, \boldsymbol{\tau}_h)+(\div\div\boldsymbol{\tau}_h, \widetilde{u}_h)+(\div\div\widetilde{\boldsymbol{\sigma}}_h, v_h)}{\|\boldsymbol{\tau}_h\|_{\boldsymbol{H}(\div{\div })} + \|v_h\|_0}
\end{equation}
for any $\widetilde{\boldsymbol{\sigma}}_h\in\boldsymbol\Sigma_h$ and $\widetilde{u}_h\in\mathcal Q_h$.
Hence the mixed finite element method \eqref{mfem1}-\eqref{mfem2} is well-posed.

\begin{theorem}
Let $\boldsymbol \sigma_h\in\boldsymbol\Sigma_h$ and $u_h\in \mathcal Q_h$ be the solution of the mixed finite element methods \eqref{mfem1}-\eqref{mfem2}. Assume $\boldsymbol \sigma\in\boldsymbol{H}^{\min\{\ell,k\}+1}(\Omega; \mathbb{S})$, $u\in H^{k-1}(\Omega)$ and $f\in H^{k-1}(\Omega)$. Then
\begin{align}
\label{eq:errorestimate1}
\|\boldsymbol{\sigma}-\boldsymbol{\sigma}_h\|_0+\|Q_h u-u_h\|_0\lesssim h^{\min\{\ell,k\}+1}|\boldsymbol \sigma|_{\min\{\ell,k\}+1},
\\
\label{eq:errorestimate2}
\|u-u_h\|_0\lesssim h^{\min\{\ell,k\}+1}|\boldsymbol \sigma|_{\min\{\ell,k\}+1}+h^{k-1}|u|_{k-1},
\\
\label{eq:errorestimate3}
\|\boldsymbol{\sigma}-\boldsymbol{\sigma}_h\|_{\boldsymbol{H}(\div{\div })}\lesssim h^{\min\{\ell,k\}+1}|\boldsymbol \sigma|_{\min\{\ell,k\}+1}+h^{k-1}|f|_{k-1}.
\end{align}
\end{theorem}
\begin{proof}
Subtracting \eqref{mfem1}-\eqref{mfem2} from \eqref{mixedform1}-\eqref{mixedform2}, it follows
\[
(\boldsymbol{\sigma}-\boldsymbol{\sigma}_h, \boldsymbol{\tau}_h)+(\div \div\boldsymbol{\tau}_h, u-u_h)+(\div\div(\boldsymbol{\sigma}-\boldsymbol{\sigma}_h), v_h)=0,
\]
which combined with \eqref{eq:Pihcd} yields
\begin{equation}\label{eq:errorequation}
(\boldsymbol{\sigma}-\boldsymbol{\sigma}_h, \boldsymbol{\tau}_h)+(\div\div\boldsymbol{\tau}_h, Q_h u-u_h)+(\div\div(\boldsymbol \Pi_h\boldsymbol{\sigma}-\boldsymbol{\sigma}_h), v_h)=0.
\end{equation}
Taking $\widetilde{\boldsymbol{\sigma}}_h=\boldsymbol \Pi_h\boldsymbol{\sigma}-\boldsymbol{\sigma}_h$ and $\widetilde{u}_h=Q_h u-u_h$ in \eqref{eq:discretestability}, we get
\begin{align*}
\|\boldsymbol \Pi_h\boldsymbol{\sigma}-\boldsymbol{\sigma}_h\|_{\boldsymbol{H}(\div{\div })}+\|Q_h u-u_h\|_0&\lesssim \sup_{\boldsymbol{\tau}_h\in\boldsymbol\Sigma_h\atop v_h\in\mathcal Q_h}\frac{(\boldsymbol \Pi_h\boldsymbol{\sigma}-\boldsymbol{\sigma}, \boldsymbol{\tau}_h)}{\|\boldsymbol{\tau}_h\|_{\boldsymbol{H}(\div{\div })} + \|v_h\|_0} \\
&\leq \|\boldsymbol \Pi_h\boldsymbol{\sigma}-\boldsymbol{\sigma}\|_0.
\end{align*}
Hence we have
\begin{align*}
\|\boldsymbol{\sigma}-\boldsymbol{\sigma}_h\|_0+\|Q_h u-u_h\|_0\lesssim \|\boldsymbol{\sigma}-\boldsymbol \Pi_h\boldsymbol{\sigma}\|_0,
\\
\|u-u_h\|_0\lesssim \|\boldsymbol{\sigma}-\boldsymbol \Pi_h\boldsymbol{\sigma}\|_0+\|u-Q_h u\|_0,
\\
\|\boldsymbol{\sigma}-\boldsymbol{\sigma}_h\|_{\boldsymbol{H}(\div{\div })}\lesssim \|\boldsymbol{\sigma}-\boldsymbol \Pi_h\boldsymbol{\sigma}\|_{\boldsymbol{H}(\div{\div })}.
\end{align*}
Finally we conclude \eqref{eq:errorestimate1}, \eqref{eq:errorestimate2} and \eqref{eq:errorestimate3} from \eqref{eq:PiKestimate}.
\end{proof}

If we are interested in the approximation of stress in $\boldsymbol  H(\div\div)$ norm, it is more economic to chose $\ell = k - 1$.
%In particular, for the lowest order $k = 2$, the d.o.f. for stress is \LC{xxx} and the d.o.f. for displacement is $2\#\mathcal T_h$.
If instead the $L^2$-norm is of concern, $\ell = k$ is a better choice to achieve higher accuracy.

The estimate of $\|Q_hu-u_h\|_0$ in \eqref{eq:errorestimate1} is superconvergent and can be used to postprocess to get a high order approximation of displacement.

\subsection{Superconvergence of displacement in mesh-dependent norm}

%We assume $\ell\geq k-1$ in this subsection.
Equip the space $$H^2(\mathcal T_h):=\{v\in L^2(\Omega): v|_K\in H^2(K) \textrm{ for each } K\in\mathcal T_h\}$$ with squared mesh-dependent norm
\[
|v|_{2,h}^2:=\sum_{K\in\mathcal{T}_h}|v|_{2,K}^2+\sum_{e\in\mathcal E_h}\left(h_e^{-3}\|\llbracket v\rrbracket\|_{0,e}^2+h_e^{-1}\|\llbracket \partial_{n_e}v\rrbracket\|_{0,e}^2\right),
\]
where $\llbracket v\rrbracket$ and $\llbracket \partial_{n_e}v\rrbracket$ are jumps of $v$ and $\partial_{n_e}v$ across $e$ for $e\in \mathcal E_h^i$, and $\llbracket v\rrbracket=v$ and $\llbracket \partial_{n_e}v\rrbracket=\partial_{n_e}v$ for $e\in \mathcal E_h\backslash\mathcal E_h^i$.

\begin{lemma}
It holds the inf-sup condition
\begin{equation}\label{eq:discreteinfsupmd}
|v_h|_{2,h}\lesssim \sup_{\boldsymbol{\tau}_h\in\boldsymbol\Sigma_h}\frac{(\div\div\boldsymbol{\tau}_h, v_h)}{\|\boldsymbol{\tau}_h\|_0}\quad\forall~v_h\in\mathcal Q_h.
\end{equation}
\end{lemma}
\begin{proof}
Let $\boldsymbol{\tau}_h\in\boldsymbol\Sigma_h$ be determined by
\begin{align*}
%v(\delta) & \quad\textrm{for each } \delta\in \mathcal V(K), \label{H1ncfmdof1}\\
\boldsymbol \tau_h (\delta)&=0  \qquad\qquad\quad\;\;\;\forall~\delta\in \mathcal V_h, \\
Q_{\ell-2}^e(\boldsymbol  n^{\intercal}\boldsymbol \tau_h\boldsymbol n)&=-h_e^{-1}\llbracket \partial_{n_e}v_h\rrbracket  \;\;\forall~ e\in\mathcal E_h, \\
\partial_{t_e}(\boldsymbol  t^{\intercal}\boldsymbol \tau_h\boldsymbol  n)+\boldsymbol n_e^{\intercal}\div\boldsymbol \tau_h&=h_e^{-3}\llbracket v_h\rrbracket  \qquad\;\;\;\forall~ e\in\mathcal E_h, \\
(\boldsymbol \tau_h, \boldsymbol \varsigma)_K&=(\nabla^2v_h, \boldsymbol \varsigma)_K  \quad\;\;\forall~\boldsymbol \varsigma\in\nabla^2\mathbb P_{k-2}(K), \\
(\boldsymbol \tau_h, \boldsymbol \varsigma)_K&=0  \qquad\qquad\quad\;\;\;\forall~\boldsymbol \varsigma\in\sym (\bs x^{\perp}\otimes \mathbb P_{\ell-2}(K;\mathbb R^2))
\end{align*}
for each $K\in\mathcal T_h$.
Due to the scaling argument, it holds
\[
\|\boldsymbol \tau_h\|_0\lesssim |v|_{2,h}.
\]
And we get from \eqref{eq:greenidentitydivdiv} that
\[
(\div\div\boldsymbol \tau_h, v_h)=|v|_{2,h}^2.
\]
Hence the inf-sup condition \eqref{eq:discreteinfsupmd} follows.
\end{proof}


An immediate result of the inf-sup condition \eqref{eq:discreteinfsupmd} is the stability result
\begin{equation}\label{eq:discretestabilitymd}
\|\widetilde{\boldsymbol{\sigma}}_h\|_{0}+|\widetilde{u}_h|_{2,h}\lesssim \sup_{\boldsymbol{\tau}_h\in\boldsymbol\Sigma_h\atop v_h\in\mathcal Q_h}\frac{(\widetilde{\boldsymbol{\sigma}}_h, \boldsymbol{\tau}_h)+(\div\div\boldsymbol{\tau}_h, \widetilde{u}_h)+(\div\div\widetilde{\boldsymbol{\sigma}}_h, v_h)}{\|\boldsymbol{\tau}_h\|_{0} + |v_h|_{2,h}}
\end{equation}
for any $\widetilde{\boldsymbol{\sigma}}_h\in\boldsymbol\Sigma_h$ and $\widetilde{u}_h\in\mathcal Q_h$.

\begin{theorem}
Let $\boldsymbol \sigma_h\in\boldsymbol\Sigma_h$ and $u_h\in \mathcal Q_h$ be the solution of the mixed finite element methods \eqref{mfem1}-\eqref{mfem2}. Assume $\boldsymbol \sigma\in\boldsymbol{H}^{\min\{\ell,k\}+1}(\Omega; \mathbb{S})$. Then
\begin{equation}\label{eq:uH2superconvergence}
|Q_hu-u_h|_{2,h}\lesssim h^{\min\{\ell,k\}+1}|\boldsymbol \sigma|_{\min\{\ell,k\}+1}.
\end{equation}
\end{theorem}
\begin{proof}
Taking $\widetilde{\boldsymbol{\sigma}}_h=\boldsymbol \Pi_h\boldsymbol{\sigma}-\boldsymbol{\sigma}_h$ and $\widetilde{u}_h=Q_hu-u_h$ in \eqref{eq:discretestabilitymd}, we get from \eqref{eq:errorequation} that
\[
\|\boldsymbol \Pi_h\boldsymbol{\sigma}-\boldsymbol{\sigma}_h\|_{0}+|Q_hu-u_h|_{2,h}\lesssim \sup_{\boldsymbol{\tau}_h\in\boldsymbol\Sigma_h\atop v_h\in\mathcal Q_h}\frac{(\boldsymbol \Pi_h\boldsymbol{\sigma}-\boldsymbol{\sigma}, \boldsymbol{\tau}_h)}{\|\boldsymbol{\tau}_h\|_{0} + |v_h|_{2,h}}\leq \|\boldsymbol \Pi_h\boldsymbol{\sigma}-\boldsymbol{\sigma}\|_0,
\]
which gives \eqref{eq:uH2superconvergence}.
\end{proof}

The estimate of $|Q_hu-u_h|_{2,h}$ in \eqref{eq:uH2superconvergence} is superconvergent, which is  $\min\{\ell-k,0\}+4$ order higher than the optimal one and will be used to get a high order approximation of displacement by postprocessing.

\subsection{Postprocessing}
%We also assume $\ell\geq k-1$ in this subsection.
Define $u_h^{\ast}\in \mathbb P_{\min\{\ell,k\}+2}(\mathcal T_h)$ as follows: for each $K\in\mathcal T_h$,
\begin{align*}
(\nabla^2u_h^{\ast}, \nabla^2 q)_K&=-(\boldsymbol \sigma_h, \nabla^2 q)_K\quad\forall~q\in\mathbb P_{\min\{\ell,k\}+2}(\mathcal T_h),
\\
(u_h^{\ast}, q)_K&=(u_h, q)_K\qquad\quad\;\forall~q\in\mathbb P_{1}(\mathcal T_h).
\end{align*}
%\LC{Do we need $k\geq 1$?}
Namely we compute the projection of $\boldsymbol  \sigma_h$ in $H^2$ semi-inner product and use $u_h$ to impose the constraint. Recall that $k\geq 3$ and $\ell\geq k-1$. Thus $u_h\in \mathbb P_{k-2}(\mathcal T_h)$ and the local $H^2$-projection is well-defined.

\begin{theorem}
Let $\boldsymbol \sigma_h\in\boldsymbol\Sigma_h$ and $u_h\in \mathcal Q_h$ be the solution of the mixed finite element methods \eqref{mfem1}-\eqref{mfem2}. Assume $u\in H^{\min\{\ell,k\}+3}(\Omega)$. Then
\begin{equation}\label{eq:uastH2superconvergence}
|u-u_h^{\ast}|_{2,h}\lesssim h^{\min\{\ell,k\}+1}|u|_{\min\{\ell,k\}+3}.
\end{equation}
\end{theorem}
\begin{proof}
Let $z=(I-Q_h^1)(Q_h^{\min\{\ell,k\}+2}u-u_h^{\ast})$. By the definition of $u_h^{\ast}$, it follows
\begin{align*}
\|\nabla^2z\|_{0,K}^2&=(\nabla^2(Q_{\min\{\ell,k\}+2}^Ku-u), \nabla^2z)_{K}+(\nabla^2(u-u_h^{\ast}), \nabla^2z)_{K} \\
&=(\nabla^2(Q_{\min\{\ell,k\}+2}^Ku-u), \nabla^2z)_{K}+(\boldsymbol \sigma_h-\boldsymbol \sigma, \nabla^2z)_{K},
\end{align*}
which implies
\begin{equation}\label{eq:20200424-1}
|z|_{2,h}^2\lesssim \sum_{K\in\mathcal T_h}|u-Q_h^{\min\{\ell,k\}+2}u|_{2,K}^2+\|\boldsymbol \sigma-\boldsymbol \sigma_h\|_{0}^2.
\end{equation}
Since $Q_h^1(Q_h^{\min\{\ell,k\}+2}u-u_h^{\ast})=Q_h^1(Q_h^{k-2}u-u_h)$ and
\[
|(I-Q_h^1)(Q_h^{k-2}u-u_h)|_{2,h}^2\lesssim \sum_{K\in\mathcal T_h}|Q_h^{k-2}u-u_h|_{2,K}^2,
\]
we get
\begin{equation}\label{eq:20200424-2}
|Q_h^1(Q_h^{\min\{\ell,k\}+2}u-u_h^{\ast})|_{2,h}\lesssim |Q_h^1(Q_h^{k-2}u-u_h)|_{2,h}\lesssim |Q_h^{k-2}u-u_h|_{2,h}.
\end{equation}
Combining \eqref{eq:20200424-1} and \eqref{eq:20200424-2} gives
\[
|Q_h^{\min\{\ell,k\}+2}u-u_h^{\ast}|_{2,h}\leq |u-Q_h^{\min\{\ell,k\}+2}u|_{2,h}+\|\boldsymbol \sigma-\boldsymbol \sigma_h\|_{0}+|Q_h^{k-2}u-u_h|_{2,h}.
\]
Finally \eqref{eq:uastH2superconvergence} follows from \eqref{eq:errorestimate1} and \eqref{eq:uH2superconvergence}.
\end{proof}


\subsection{Hybridization}

In this subsection we consider a partial hybridization of the mixed finite element methods \eqref{mfem1}-\eqref{mfem2} by relaxing the continuity of the effective transverse shear force.
To this end, let
\begin{align*}
\widetilde{\boldsymbol\Sigma}_h:=\{\boldsymbol \tau_h\in \boldsymbol  L^2(\Omega;\mathbb S):&\, \boldsymbol \tau_h|_K\in \boldsymbol \Sigma_{\ell,k}(K) \textrm{ for each } K\in\mathcal T_h, \\
& \textrm{ all the degrees of freedom } \eqref{Hdivdivfemdof1}-\eqref{Hdivdivfemdof2} \textrm{ are single-valued}\},
\end{align*}
\begin{align*}
\Lambda_h:=\{\mu_h\in L^2(\mathcal E_h):& \, \mu_h|_e\in\mathbb P_{\ell-1}(e) \textrm{ for each } e\in\mathcal E_h^i,\\
 & \,\textrm{and } \mu_h|_e=0 \textrm{ for each } e\in\mathcal E_h\backslash\mathcal E_h^i \}.
\end{align*}

\begin{lemma}
Let $\boldsymbol \sigma_h\in\boldsymbol\Sigma_h$ and $u_h\in \mathcal Q_h$ be the solution of the mixed finite element methods \eqref{mfem1}-\eqref{mfem2}.
Let $(\widetilde{\boldsymbol\sigma}_h, \widetilde{u}_h, \lambda_h)\in\widetilde{\boldsymbol\Sigma}_h\times\mathcal Q_h\times\Lambda_h$ satisfy the partial hybridized mixed finite methods
\begin{align}
(\widetilde{\boldsymbol\sigma}_h, \boldsymbol{\tau}_h)+b_h(\boldsymbol{\tau}_h, \widetilde{u}_h, \lambda_h)&=0 \quad\quad\quad\;\; \forall~\boldsymbol{\tau}_h\in \widetilde{\boldsymbol\Sigma}_h, \label{mfemhybrid1} \\
b_h(\widetilde{\boldsymbol\sigma}_h, v_h, \mu_h)&=(f, v_h) \quad\; \forall~v_h\in \mathcal Q_h,\; \mu_h\in\Lambda_h, \label{mfemhybrid2}
\end{align}
where
\[
b_h(\boldsymbol{\tau}_h, v_h, \mu_h):=\sum_{K\in\mathcal T_h}(\div\div\boldsymbol \tau_h, v_h)_K-\sum_{K\in\mathcal T_h}(\partial_{t}(\boldsymbol  t^{\intercal}\boldsymbol \tau_h\boldsymbol  n)+\boldsymbol  n^{\intercal}\div\boldsymbol \tau_h,  \mu_h)_{\partial K}.
\]
Then $\widetilde{\boldsymbol\sigma}_h=\boldsymbol \sigma_h$ and $\widetilde{u}_h=u_h$.
\end{lemma}
\begin{proof}
First show the unisolvence of the discrete methods \eqref{mfemhybrid1}-\eqref{mfemhybrid2}. Assume $f$ is zero.
Due to \eqref{mfemhybrid2} with $v_h=0$, we get $\widetilde{\boldsymbol\sigma}_h\in\boldsymbol\Sigma_h$.
Hence $(\widetilde{\boldsymbol\sigma}_h, \widetilde{u}_h)\in\boldsymbol\Sigma_h\times\mathcal Q_h$ satisfies the mixed finite element methods \eqref{mfem1}-\eqref{mfem2} with $f=0$. Then $\widetilde{\boldsymbol\sigma}_h=\boldsymbol0$ and $\widetilde{u}_h=0$ follows from the unisolvence of the mixed methods \eqref{mfem1}-\eqref{mfem2}.  Thus \eqref{mfemhybrid1} becomes
\[
\sum_{K\in\mathcal T_h}(\partial_{t}(\boldsymbol  t^{\intercal}\boldsymbol \tau_h\boldsymbol  n)+\boldsymbol  n^{\intercal}\div\boldsymbol \tau_h,  \lambda_h)_{\partial K}=0\quad\forall~\boldsymbol\tau_h\in\boldsymbol\Sigma_h.
\]
Now taking $\boldsymbol{\tau}_h\in\boldsymbol\Sigma_h$ such that
\begin{align*}
\boldsymbol \tau_h (\delta)&=0  \quad\;\forall~\delta\in \mathcal V_h, \\
Q_{\ell-2}^e(\boldsymbol  n^{\intercal}\boldsymbol \tau_h\boldsymbol n)&=0  \;\;\;\;\;\forall~ e\in\mathcal E_h, \\
\partial_{t_e}(\boldsymbol  t^{\intercal}\boldsymbol \tau_h\boldsymbol  n)+\boldsymbol n_e^{\intercal}\div\boldsymbol \tau_h&=\lambda_h  \;\;\,\forall~ e\in\mathcal E_h, \\
(\boldsymbol \tau_h, \boldsymbol \varsigma)_K&=0  \quad\;\forall~\boldsymbol \varsigma\in\nabla^2\mathbb P_{k-2}(K)\oplus\sym (\bs x^{\perp}\otimes \mathbb P_{\ell-2}(K;\mathbb R^2))
\end{align*}
for each $K\in\mathcal T_h$, we acquire $\lambda_h=0$.

For general $f\in L^2(\Omega)$, it also follows from \eqref{mfemhybrid2} that $\widetilde{\boldsymbol\sigma}_h\in\boldsymbol\Sigma_h$. And then $(\widetilde{\boldsymbol\sigma}_h, \widetilde{u}_h)\in\boldsymbol\Sigma_h\times\mathcal Q_h$ satisfies the mixed finite element methods \eqref{mfem1}-\eqref{mfem2}. Thus $\widetilde{\boldsymbol\sigma}_h=\boldsymbol \sigma_h$ and $\widetilde{u}_h=u_h$.
\end{proof}

The space of shape functions for $\widetilde{\boldsymbol\Sigma}_h$ is still $\boldsymbol \Sigma_{\ell,k}(K)$.
The local degrees of freedom are %given by
\begin{align*}
%v(\delta) & \quad\textrm{for each } \delta\in \mathcal V(K), \label{H1ncfmdof1}\\
\boldsymbol \tau (\delta) & \quad\forall~\delta\in \mathcal V(K), \\%\label{Hdivdivfemhybriddof1}\\
(\boldsymbol  n^{\intercal}\boldsymbol \tau\boldsymbol  n, q)_e & \quad\forall~q\in\mathbb P_{\ell-2}(e),  e\in\mathcal E(K), \\%\label{Hdivdivfemhybriddof2}\\
(\partial_{t}(\boldsymbol  t^{\intercal}\boldsymbol \tau\boldsymbol  n)+\boldsymbol  n^{\intercal}\div\boldsymbol \tau, q)_e & \quad\forall~q\in\mathbb P_{\ell-1}(e),  e\in\mathring{\mathcal E}(K), \\%\label{Hdivdivfemhybriddof3}\\
(\boldsymbol \tau, \boldsymbol \varsigma)_K & \quad\forall~\boldsymbol \varsigma\in\nabla^2\mathbb P_{k-2}(K)\oplus \sym (\bs x^{\perp}\otimes \mathbb P_{\ell-2}(K;\mathbb R^2)). %\label{Hdivdivfemhybriddof4}
\end{align*}
Here notation $e\in\mathring{\mathcal E}(K)$ means $(\partial_{t}(\boldsymbol  t^{\intercal}\boldsymbol \tau\boldsymbol  n)+\boldsymbol  n^{\intercal}\div\boldsymbol \tau, q)_e$ are the interior degrees of freedom, i.e., $(\partial_{t}(\boldsymbol  t^{\intercal}\boldsymbol \tau\boldsymbol  n)+\boldsymbol  n^{\intercal}\div\boldsymbol \tau, q)_e$ are double-valued on each edge $e\in\mathcal E_h^i$.

When $\ell=k$, we can take the following degrees of freedom
\begin{align*}
%v(\delta) & \quad\textrm{for each } \delta\in \mathcal V(K), \label{H1ncfmdof1}\\
\boldsymbol \tau (\delta) & \quad\forall~\delta\in \mathcal V(K), \\%\label{Hdivdivfemhybriddof1}\\
(\boldsymbol  n^{\intercal}\boldsymbol \tau\boldsymbol  n, q)_e & \quad\forall~q\in\mathbb P_{k-2}(e),  e\in\mathcal E(K), \\%\label{Hdivdivfemhybriddof2}\\
(\boldsymbol  t^{\intercal}\boldsymbol \tau\boldsymbol  n, q)_e,\; (\boldsymbol  t^{\intercal}\boldsymbol \tau\boldsymbol  t, q)_e & \quad\forall~q\in\mathbb P_{k-2}(e),  e\in\mathring{\mathcal E}(K), \\%\label{Hdivdivfemhybriddof2}\\
(\boldsymbol \tau, \boldsymbol \varsigma)_K & \quad\forall~\boldsymbol \varsigma\in \mathbb P_{k-3}(K;\mathbb S). %\label{Hdivdivfemhybriddof4}
\end{align*}
These are exactly the tensor version of the local degrees of freedom for the Lagrange element.
Therefore we can adopt the standard Lagrange element basis to implement the hybridized mixed finite element methods \eqref{mfemhybrid1}-\eqref{mfemhybrid2}.

