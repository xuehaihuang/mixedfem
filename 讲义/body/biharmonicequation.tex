% !TEX root = lecture.tex
\chapter{重调和方程混合元方法}

设 $\Omega\subset\mathbb R^d$ ($d\geq 2$) 是有界多面体区域. 具有齐次 Dirichlet 边界条件的重调和方程如下所示:
\begin{equation}\label{eq:biharmonic}
\begin{cases}
\Delta^2u=f\quad \textrm{在 }\,\Omega\textrm{ 中}, \\
u|_{\partial\Omega}=\partial_{n}u|_{\partial\Omega}=0.
\end{cases}
\end{equation}
重调和方程 \eqref{eq:biharmonic} 在连续介质力学中广泛出现,例如线性弹性理论和 Stokes 流问题. 在二维情形下,该问题对应于 Kirchhoff 薄板模型,用以描述在垂直载荷 $f$ 作用下弹性薄板的弯曲 \cite{FengShi1996,Reddy2006}. 

设 $f\in H^{-2}(\Omega)$. 重调和方程 \eqref{eq:biharmonic} 的变分形式为:寻找 $u\in H_0^2(\Omega)$,使得
\begin{equation}\label{eq:biharmonicvariatonalform}
(\nabla^2 u, \nabla^2 v) = \langle f, v\rangle \quad \forall\, v\in H_0^2(\Omega).
\end{equation}
由于对任意 $v\in H_0^2(\Omega)$ 有 $\|v\|_2 \lesssim |v|_2$,因此变分问题 \eqref{eq:biharmonicvariatonalform} 是适定的. 

根据非光滑区域的椭圆正则性理论(参见 \cite{BlumRannacher1980, Grisvard1985, Dauge1988, Grisvard1992}),存在 $\alpha\in (\frac{1}{2},1]$,使得当 $f\in H^{-2+\alpha}(\Omega)$ 时,问题 \eqref{eq:biharmonicvariatonalform} 的解 $u$ 属于 $H^{2+\alpha}(\Omega)$ 并满足
\[
\|u\|_{H^{2+\alpha}(\Omega)}\lesssim \|f\|_{H^{-2+\alpha}(\Omega)}.
\]
当 $\Omega$ 为凸区域时,我们可以取 $\alpha=1$,即
\[
\|u\|_{H^{3}(\Omega)}\lesssim \|f\|_{H^{-1}(\Omega)}.
\]

\section{混合变分问题}

\subsection{ $H^{-1}(\div\div,\Omega; \mathbb{S})$ 空间和分布混合变分形式}

定义 Hilbert 空间 (参见 \cite{PechsteinSchoberl2011}):
\[
H^{-1}(\div{\div },\Omega;\mathbb{S}):=\{\boldsymbol{\tau}\in L^{2}(\Omega; \mathbb{S}): \div \div\boldsymbol{\tau}\in H^{-1}(\Omega)\},
\]
其范数平方定义为 
$
\|\boldsymbol{\tau}\|_{H^{-1}(\div\div)}^2:=\|\boldsymbol{\tau}\|^2+\|\div \div\boldsymbol{\tau}\|_{-1}^2
$. 

先给出空间 $H^{-1}(\div{\div },\Omega;\mathbb{S})$ 的如下直和分解.
\begin{lemma}
成立直和分解:
\begin{equation}\label{Hn1divdivSdecomp}
H^{-1}(\div{\div },\Omega;\mathbb{S}) = \iota(H_0^1(\Omega)) \oplus \big(H^{-1}(\div\div,\Omega; \mathbb{S})\cap \ker(\div\div)\big),
\end{equation}
其中 $\iota (v) = v \boldsymbol I_{d\times d}$,以及 
$$
H^{-1}(\div\div,\Omega; \mathbb{S})\cap \ker(\div\div)=\{\boldsymbol{\tau}\in L^{2}(\Omega; \mathbb{S}): \div\div\boldsymbol{\tau}=0\}
$$ 
\end{lemma}
\begin{proof}
显然 \eqref{Hn1divdivSdecomp} 式右端是直和(因 $\operatorname{div}\operatorname{div}(v \boldsymbol I)=\Delta v=0$ 且 $v\in H_0^1(\Omega)$ 蕴含 $v=0$),且包含于左端. 

下证分解成立. 任取 $\boldsymbol{\tau}\in H^{-1}(\operatorname{div}\operatorname{div},\Omega;\mathbb{S})$. 由于拉普拉斯算子 $\Delta: H_0^1(\Omega) \to H^{-1}(\Omega)$ 是同构映射,存在唯一的 $v\in H_0^1(\Omega)$ 满足 $\Delta v=\operatorname{div}\operatorname{div} \boldsymbol{\tau}$. 于是
$$
\operatorname{div}\operatorname{div}(\boldsymbol{\tau}-\iota(v)) = \operatorname{div}\operatorname{div}\boldsymbol{\tau} - \Delta v = 0.
$$
因此,$\boldsymbol{\tau}-\iota(v)$ 属于核空间 $\ker(\operatorname{div}\operatorname{div})$. 证毕. 
\end{proof}

若 $\Omega$ 为可缩区域,则核空间 $H^{-1}(\operatorname{div}\operatorname{div},\Omega; \mathbb{S})\cap \ker(\operatorname{div}\operatorname{div})$ 可被显式刻画.
特别地,当 $d=2$ 时,\eqref{Hn1divdivSdecomp} 中的直和分解可表示为:
$$
H^{-1}(\operatorname{div}\operatorname{div},\Omega;\mathbb{S}) = \iota(H_0^1(\Omega)) \oplus \operatorname{sym}\operatorname{curl} H^1(\Omega;\mathbb R^2).
$$
当 $d=3$ 时,该分解为:
$$
H^{-1}(\operatorname{div}\operatorname{div},\Omega;\mathbb{S}) = \iota(H_0^1(\Omega)) \oplus \operatorname{sym}\operatorname{curl} H^1(\Omega;\mathbb T).
$$

下面讨论空间 $H^{-1}(\div{\div },\Omega;\mathbb{S})$ 在边界上的迹.
令
\[
H_{n,0}^{1/2}(\Gamma):=\{\partial_{n}\varphi: \varphi\in H^2(\Omega)\cap H_0^1(\Omega)\},
\]
并赋予范数
\[
\|w\|_{H_{n,0}^{1/2}(\Gamma)}:=\inf_{\varphi\in H^2(\Omega)\cap H_0^1(\Omega), \partial_{n}\varphi=w}\|\varphi\|_{H^2(\Omega)}.
\]
对任意的 $\varphi\in H^2(\Omega)\cap H_0^1(\Omega)$,显然有 $(\partial_{n}\varphi)|_{\Gamma}\in H_{n,0}^{1/2}(\Gamma)$,且满足
$$
\|\partial_{n}\varphi\|_{H_{n,0}^{1/2}(\Gamma)}\leq \|\varphi\|_{H^2(\Omega)}\quad\forall\,\varphi\in H^2(\Omega)\cap H_0^1(\Omega).
$$
定义
\[
H_{n}^{-1/2}(\Gamma):=\left(H_{n,0}^{1/2}(\Gamma)\right)'.
\]

\begin{lemma}[定理 3.34,文献 \cite{Sinwel2009}]
设 $\Omega\subset\mathbb R^d$ 是具有光滑或 Lipschitz 边界 $\Gamma=\partial\Omega$ 的有界区域. 
\begin{enumerate}[label=(\alph*)]
\item 迹算子 $\bs\tau\in H^{-1}(\div\div,\Omega; \mathbb{S})\to (\boldsymbol{n}^{\intercal}\bs\tau\boldsymbol{n})|_{\Gamma}\in H_{n}^{-1/2}(\Gamma)$ 是有界的:
\begin{equation}\label{nntracenormbound}
\|\boldsymbol{n}^{\intercal}\bs\tau\boldsymbol{n}\|_{H_{n}^{-1/2}(\Gamma)}\lesssim \|\boldsymbol\tau\|_{H^{-1}(\div\div)}\quad\forall~\boldsymbol\tau\in H^{-1}(\div{\div },\Omega;\mathbb{S}).
\end{equation}
\item 迹算子 $\bs\tau\in H^{-1}(\div\div,\Omega; \mathbb{S})\to (\boldsymbol{n}^{\intercal}\bs\tau\boldsymbol{n})|_{\Gamma}\in H_{n}^{-1/2}(\Gamma)$ 是满射,即对于任意 $g\in H_{n}^{-1/2}(\Gamma)$,存在 $\boldsymbol\tau\in H^{-1}(\div{\div },\Omega;
\mathbb{S})$ 使得
\[
(\boldsymbol{n}^{\intercal}\bs\tau\boldsymbol{n})|_{\Gamma}=g\quad\textrm{且}\quad \|\boldsymbol\tau\|_{H^{-1}(\div\div)}\lesssim \|g\|_{H_{n}^{-1/2}(\Gamma)}.
\]
\end{enumerate}
\end{lemma}
\begin{proof}
(a)
根据范数 $\|\cdot\|_{H_{n}^{-1/2}(\Gamma)}$ 和 $\|\cdot\|_{H_{n,0}^{1/2}(\Gamma)}$ 的定义,对于任意 $\boldsymbol\tau\in C^{\infty}(\Omega;\mathbb{S})$,有
\begin{align*}
\|\boldsymbol{n}^{\intercal}\bs\tau\boldsymbol{n}\|_{H_{n}^{-1/2}(\Gamma)}&=\sup_{w\in H_{n,0}^{1/2}(\Gamma)}\frac{\langle \boldsymbol{n}^{\intercal}\bs\tau\boldsymbol{n}, w\rangle}{\|w\|_{H_{n,0}^{1/2}(\Gamma)}}=\sup_{\varphi\in H^2(\Omega)\cap H_0^1(\Omega)}\frac{\langle \boldsymbol{n}^{\intercal}\bs\tau\boldsymbol{n}, \partial_{n}\varphi\rangle}{\|\varphi\|_{H^2(\Omega)}}.
\end{align*}
由 $\varphi\in H^2(\Omega)\cap H_0^1(\Omega)$ 和分部积分公式可得
\begin{align*}
\langle \boldsymbol{n}^{\intercal}\bs\tau\boldsymbol{n}, \partial_{n}\varphi\rangle&=(\bs\tau\boldsymbol{n}, \nabla\varphi)_{\partial\Omega}=(\boldsymbol\tau, \nabla^2\varphi)+(\div\boldsymbol\tau, \nabla\varphi) \\
&=(\boldsymbol\tau, \nabla^2\varphi) - (\div\div\boldsymbol\tau, \varphi)\lesssim \|\boldsymbol\tau\|_{H^{-1}(\div\div)}\|\varphi\|_{H^2(\Omega)}.
\end{align*}
故有
\begin{equation*}
\|\boldsymbol{n}^{\intercal}\bs\tau\boldsymbol{n}\|_{H_{n}^{-1/2}(\Gamma)} \lesssim \|\boldsymbol\tau\|_{H^{-1}(\div\div)}\quad\forall~\boldsymbol\tau\in C^{\infty}(\Omega;\mathbb{S}).
\end{equation*}
因此,由稠密性技巧可知 \eqref{nntracenormbound} 成立.

(b) 给定 $g\in H_{n}^{-1/2}(\Gamma)$,令 $w\in H^2(\Omega)\cap H_0^1(\Omega)$ 是以下重调和方程的解:
\begin{equation*}
\begin{cases}
\Delta^2w+w = 0\quad \textrm{在 }\,\Omega\textrm{ 中}, \\
\partial_{nn}w|_{\Gamma}=g.
\end{cases}
\end{equation*}
其变分形式为
\begin{equation*}
(\nabla^2w, \nabla^2v) + (w,v) = \langle g, \partial_{n}v\rangle \quad \forall\, v\in H^2(\Omega)\cap H_0^1(\Omega).
\end{equation*}
由 Lax-Milgram 引理可知该变分问题是适定的,且有
$$
\|w\|_2\lesssim \|g\|_{H_{n}^{-1/2}(\Gamma)}.
$$
令 $\boldsymbol{\tau}=\nabla^2w \in L^2(\Omega;\mathbb S)$,则 $\div\div\boldsymbol{\tau}=\Delta^2w=-w$. 因此 $(\boldsymbol{n}^{\intercal}\bs\tau\boldsymbol{n})|_{\Gamma}=g$,且有
\begin{equation*}
\|\boldsymbol\tau\|_{H^{-1}(\div\div)}\leq \|\boldsymbol\tau\|+\|\div\div\boldsymbol{\tau}\|_{-1}=\|\nabla^2w\|+\|w\|_{-1}\lesssim \|g\|_{H_{n}^{-1/2}(\Gamma)}.
\end{equation*}
得证.
\end{proof}

\begin{lemma}[定理 2.1 \cite{PechsteinSchoberl2011} 和 定理 3.36 \cite{Sinwel2009}]\label{lem:Hdivdivcontinuous}
设 $D$, $D_1$ 和 $D_2$ 是 $\mathbb R^d$ 中的三个开有界区域. 
假设 $\bar{D}=\bar{D}_1\cup \bar{D}_2$ 且 $D_1\cap D_2=\varnothing$. 设 $\bs\tau\in  L^2(D;\mathbb S)$. 设 $\bs\tau_i:= \bs\tau |_{D_i}\in H(\div, D_i;\mathbb S)$, 其中 $i=1,2$;并设
$(\div_F\boldsymbol\tau_i\boldsymbol{n})|_{\partial D_i}\in H^{-1/2}(\partial D_i)$. 
如果法向-法向分量在界面 $\partial D_1\cap\partial D_2$ 上的跳量 $[\boldsymbol{n}^{\intercal}\bs\tau\boldsymbol{n}]$ 等于零,则 $\bs\tau\in H^{-1}(\div\div,D; \mathbb{S})$. 
\end{lemma}
\begin{proof}
对任意 $v\in C_0^{\infty}(D)$,
由分部积分公式和跳量 $[\boldsymbol{n}^{\intercal}\bs\tau\boldsymbol{n}]$ 在界面 $\partial D_1\cap\partial D_2$ 上等于零可得,
\begin{align*}
\langle \div\div\boldsymbol\tau, v\rangle & =(\boldsymbol\tau, \nabla^2v) = -\sum_{i=1}^2(\div\boldsymbol\tau_i, \nabla v)_{D_i}+\sum_{i=1}^2(\boldsymbol\tau_i\boldsymbol{n}, \nabla v)_{\partial D_i} \\
&= -\sum_{i=1}^2(\div\boldsymbol\tau_i, \nabla v)_{D_i}+\sum_{i=1}^2(\Pi_F\boldsymbol\tau_i\boldsymbol{n}, \nabla_F v)_{\partial D_i} \\
&= -\sum_{i=1}^2(\div\boldsymbol\tau_i, \nabla v)_{D_i}-\sum_{i=1}^2(\div_F\boldsymbol\tau_i\boldsymbol{n}, v)_{\partial D_i} \\
&\leq \sum_{i=1}^2\|\div\boldsymbol\tau_i\|_{D_i}\|\nabla v\|_{D_i} + \sum_{i=1}^2\|\div_F\boldsymbol\tau_i\boldsymbol{n}\|_{H^{-1/2}(\partial D_i)}\|v\|_{H^{1/2}(\partial D_i)}.
\end{align*} 
由迹不等式知,存在依赖于区域 $D_i$ 的常数 $C_i$ 使得
\begin{equation*}
\langle \div\div\boldsymbol\tau, v\rangle\leq \sum_{i=1}^2\|\div\boldsymbol\tau_i\|_{D_i}\|\nabla v\|_{D_i} + \sum_{i=1}^2C_i\|\div_F\boldsymbol\tau_i\boldsymbol{n}\|_{H^{-1/2}(\partial D_i)}\|v\|_{1, D_i}.
\end{equation*}
故由稠密性技巧可知 $\bs\tau\in H^{-1}(\div\div,D; \mathbb{S})$.
\end{proof}

\begin{remark}
条件 $(\div_F\boldsymbol\tau_i\boldsymbol{n})|_{\partial D_i}\in H^{-1/2}(\partial D_i)$ 是对切向法向分量 $\Pi_F\bs\tau_{i}\boldsymbol{n}$ 在 $d-2$ 维单纯形处的单元内连续性约束. 如果稍微减弱正则性要求,该约束消失,可以构造简单的离散格式. 
\end{remark}


设 $f\in H^{-1}(\Omega)$. 
引入变量 $\bs\sigma:=-\nabla^2 u$,将重调和方程 \eqref{eq:biharmonic} 重写为:
\begin{equation}\label{eq:biharmonic2variables}
\begin{cases}
\bs\sigma =-\nabla^2 u\qquad\;\;\; \textrm{在 }\,\Omega\textrm{ 中}, \\
\div\div\bs\sigma=-f\quad\; \textrm{在 }\,\Omega\textrm{ 中}, \\
u|_{\partial\Omega}=\partial_{n}u|_{\partial\Omega}=0.
\end{cases}
\end{equation}

重调和方程 \eqref{eq:biharmonic2variables} 的一种分布混合变分形式为:
% Hellan-Herrmann-Johnson (HHJ) 混合格式 \cite{KrendlRafetsederZulehner2016,Hellan1967,Herrmann1967,Johnson1973} 是
找 $(\boldsymbol{\sigma} , u)\in H^{-1}(\div\div,\Omega; \mathbb{S})\times H_0^1(\Omega)$ 使得:
\begin{subequations}\label{eq:hhjmixedformulation}
\begin{align}
(\boldsymbol\sigma, \boldsymbol\tau)+ \langle \div\div\boldsymbol\tau, u\rangle  & =0 \quad\quad\quad\;\;
\forall~\boldsymbol\tau\in H^{-1}(\div\div,\Omega; \mathbb{S}), \label{eq:hhjmixedformulation1}\\
\langle \div\div\boldsymbol\sigma, v\rangle & =-\langle f, v\rangle  \quad \forall~v\in H_0^1(\Omega). \label{eq:hhjmixedformulation2}
\end{align}
\end{subequations}
% 其中
% \[
% a(\boldsymbol\sigma, \boldsymbol\tau):=(\boldsymbol\sigma, \boldsymbol\tau),\quad b(\boldsymbol\tau, v):=\langle \div\div\boldsymbol\tau, v\rangle.
% \]

显然,对于任意 $\boldsymbol\sigma, \boldsymbol\tau\in H^{-1}(\div\div,\Omega; \mathbb{S})$ 和 $v\in H_0^1(\Omega)$,有
\[
(\boldsymbol\sigma, \boldsymbol\tau)\leq\|\boldsymbol\sigma\|\|\boldsymbol\tau\|\leq\|\boldsymbol\sigma\|_{H^{-1}(\div\div)}\|\boldsymbol\tau\|_{H^{-1}(\div\div)},
\]
\[
\langle \div\div\boldsymbol\tau, v\rangle \leq\|\div\div\boldsymbol\tau\|_{-1}|v|_1\leq \|\boldsymbol\tau\|_{H^{-1}(\div\div)}|v|_1.
\]
在算子 $\div\div: H^{-1}(\div\div,\Omega; \mathbb{S})\to H^{-1}(\Omega)$ 的核空间
% $$
% H^{-1}(\div\div,\Omega; \mathbb{S})\cap \ker(\div\div)=\{\boldsymbol{\tau}\in L^{2}(\Omega; \mathbb{S}): \div\div\boldsymbol{\tau}=0\}
% $$ 
上具有强制性:
\[
(\boldsymbol\tau, \boldsymbol\tau)=\|\boldsymbol\tau\|^2=\|\boldsymbol\tau\|_{H^{-1}(\div\div)}^2\quad\forall~\boldsymbol\tau\in H^{-1}(\div\div,\Omega; \mathbb{S})\cap \ker(\div\div).
\]

按照 \cite{BoffiBrezziFortin2013,BrezziRaviart1977} 中的证明思路,对于 $v\in H_0^1(\Omega)$,我们可以看到:
\[
\langle \div\div(-\iota(v)), v\rangle =-\langle \div\div(\iota(v)), v\rangle=-\langle \Delta v, v\rangle=|v|_1^2,
\]
\[
\|-\iota(v)\|_{H^{-1}(\div\div)}^2=\|\iota(v)\|^2+|v|_1^2\lesssim |v|_1^2.
\]
因此,我们得到如下的 inf-sup 条件:
\[
\sup_{\bs\tau\in H^{-1}(\div\div,\Omega; \mathbb{S})}\frac{\langle \div\div\boldsymbol\tau, v\rangle}{\|\boldsymbol\tau\|_{H^{-1}(\div\div)}}\geq\frac{\langle \div\div(-\iota(v)), v\rangle}{\|-\iota(v)\|_{H^{-1}(\div\div)}}\gtrsim |v|_1.
\]

由 Brezzi 理论可知,分布混合变分形式 \eqref{eq:hhjmixedformulation} 是适定的. 

\begin{theorem}[推论 2.3,文献 \cite{KrendlRafetsederZulehner2016}]
问题 \eqref{eq:biharmonicvariatonalform} 与分布混合变分形式 \eqref{eq:hhjmixedformulation} 是完全等价的. 即,如果 $u\in H_0^2(\Omega)$ 是问题 \eqref{eq:biharmonicvariatonalform} 的解,则 $\bs\sigma=-\nabla^2 u\in H^{-1}(\div\div,\Omega; \mathbb{S})$ 且 $(\bs\sigma, u)$ 是分布混合变分形式 \eqref{eq:hhjmixedformulation} 的解. 反之,如果 $(\bs\sigma, u)\in H^{-1}(\div\div,\Omega; \mathbb{S})\times H_0^1(\Omega)$ 是分布混合变分形式 \eqref{eq:hhjmixedformulation} 的解,则 $u\in H_0^2(\Omega)$ 且 $u$ 是问题 \eqref{eq:biharmonicvariatonalform} 的解. 
\end{theorem}
\begin{proof}
两个问题均是唯一可解的. 因此,只需证明若 $u$ 解问题 \eqref{eq:biharmonicvariatonalform},则令 $\bs\sigma=-\nabla^2 u$,$(\bs\sigma, u)$ 满足分布混合变分形式 \eqref{eq:hhjmixedformulation} 即可. 
假设 $u\in H_0^2(\Omega)$ 是问题 \eqref{eq:biharmonicvariatonalform} 的解. 
显然 $\bs\sigma\in L^{2}(\Omega; \mathbb{S})$ 且
\[
(\bs\sigma, \nabla^2v)=-\langle f, v\rangle\quad\forall~v\in H_0^2(\Omega),
\]
这意味着在分布意义下 $\div\div\bs\sigma=-f\in H^{-1}(\Omega)$. 
因此,$\bs\sigma\in H^{-1}(\div\div,\Omega; \mathbb{S})$ 且式 \eqref{eq:hhjmixedformulation2} 成立. 

根据 $\div\div\bs\tau$ 的分布定义,我们有
\[
\langle \div\div\boldsymbol\tau, v\rangle=(\boldsymbol\tau, \nabla^2v)\quad\forall~v\in C_0^{\infty}(\Omega).
\]
由于 $C_0^{\infty}(\Omega)$ 在 $H_0^2(\Omega)$ 中稠密,取 $v=u$ 可得
\[
\langle \div\div\boldsymbol\tau, u\rangle=(\boldsymbol\tau, \nabla^2u)=-(\boldsymbol\tau, \bs\sigma),
\]
这证明了 \eqref{eq:hhjmixedformulation1}. 
\end{proof}


\subsection{ $H(\div\div,\Omega; \mathbb{S})$ 空间和混合变分形式}

在本节中,假设 $f\in L^{2}(\Omega)$. 
定义 Hilbert 空间 %(参见 \cite{PechsteinSchoberl2011}):
\[
H(\div{\div },\Omega;
\mathbb{S}):=\{\boldsymbol{\tau}\in L^{2}(\Omega; \mathbb{S}): \div \div\boldsymbol{\tau}\in H(\Omega)\},
\]
其范数平方定义为 
$
\|\boldsymbol{\tau}\|_{H(\div\div)}^2:=\|\boldsymbol{\tau}\|^2+\|\div \div\boldsymbol{\tau}\|^2
$. 

下面探讨分段光滑的张量函数属于 Sobolev 空间 $H(\div{\div },\Omega; \mathbb{S})$ 所需满足的连续性条件.
设 $K$ 是一个多面体区域.
迹算子 $\tr^{\div\div}\bs \sigma$ 在分布意义下定义为如下:
$$
\langle \tr^{\div\div}\bs \sigma, \tr^{\nabla^2} v \rangle_{\partial K}:= (\div\div\boldsymbol \sigma, v)_K - (\boldsymbol \sigma, \nabla^2v)_K.
$$
随后,我们将 $\tr^{\div\div}\bs \sigma$ 和 $\tr^{\nabla^2} v$ 进一步分解为两个 $(d-1)$ 维面迹算子和一个 $(d-2)$ 维面迹算子,以揭示其在分片结构下的连续性要求.

回顾文献~\cite{ChenHuang2022a,ChenHuang2022} 中给出的 $\div\div$ 算子的格林公式.
\begin{lemma}[文献~\cite{ChenHuang2022} 引理 5.2]
对于任意 $\boldsymbol \sigma\in \mathcal C^2(K; \mathbb S)$ 和 $v\in H^2(K)$,成立如下格林公式:
\begin{align}
&\notag  (\div\div\boldsymbol \sigma, v)_K=(\boldsymbol \sigma, \nabla^2v)_K \\
&- \sum_{F\in\partial K}\left[( \tr_1(\bs \sigma), \tr_1(v))_{F} -  ( \tr_2(\bs \sigma), \tr_2(v))_F\right] -\sum_{e\in\Delta_{d-2}(K)}(\tr_e(\bs \sigma), \tr_e(v))_e, \label{eq:greenidentitydivdiv} 
\end{align}
其中
\begin{align*}
&\tr_1(\bs \sigma) = \boldsymbol  n_{\partial K}^{\intercal}\boldsymbol \sigma\boldsymbol  n_{\partial K}, &  & \tr_1( v) = \partial_n v\mid_{\partial K},\\
&\tr_2(\bs \sigma) =  \boldsymbol n_{\partial K}^{\intercal}\div \boldsymbol \sigma +  \div_F(\boldsymbol\sigma \boldsymbol n_{\partial K}),&  & \tr_2(v) = v\mid_{\partial K},\\
&\tr_e(\bs \sigma) = \sum_{F\in\partial K,e\in \partial F}\boldsymbol n_{F,e}^{\intercal}\boldsymbol \sigma \boldsymbol n_{\partial K}, &  & \tr_e(v) = v \mid_{\Delta_{d-2}(K)}.
\end{align*}
\end{lemma}
\begin{proof}
应用分部积分公式可得:
\begin{align*}
\begin{aligned}
(\operatorname{div}\div \boldsymbol \tau, v)_{K} &=-(\div\boldsymbol \tau, \nabla v)_{K}+\sum_{F \in \partial K}(\boldsymbol n^{\intercal}\div \boldsymbol \tau, v)_F \\
&=\left(\boldsymbol \tau, \nabla^{2} v\right)_{K}-\sum_{F \in \partial K} (\boldsymbol \tau \boldsymbol n, \nabla v)_F+\sum_{F \in \partial K}(\boldsymbol n^{\intercal} \div \boldsymbol \tau, v)_F.
\end{aligned}
\end{align*}
接着我们利用分解 $\nabla v = \partial_nv\boldsymbol n+ \nabla_F v$ 并应用 Stokes 定理得到:
\begin{align*}
(\boldsymbol \tau \boldsymbol n, \nabla v)_F&=(\boldsymbol \tau \boldsymbol n, \partial_nv\boldsymbol n+ \nabla_F v)_F \\
&=(\boldsymbol n^{\intercal}\boldsymbol \tau \boldsymbol n, \partial_nv)_F -( \div_F(\boldsymbol \tau\boldsymbol n), v)_F + \sum_{e\in\partial F} (\boldsymbol n_{F,e}^{\intercal}\boldsymbol \tau \boldsymbol n, v)_e.
\end{align*}
结合上述两个等式即可推导出格林公式~\eqref{eq:greenidentitydivdiv}. 
\end{proof}

设 $\mathcal{T}_h$ 是 $\Omega$ 的一个多面体网格剖分. 记 $\mathcal{F}_h$ 和 $\mathcal{E}_h$ 分别是 $\mathcal{T}_h$ 中所有 $(d-1)$ 维面和 $(d-2)$ 维面的集合,并分别记 $\mathring{\mathcal F}_h$ 和 $\mathring{\mathcal E}_h$ 为其内部$(d-1)$ 维面和内部$(d-2)$ 维面的集合. 
当在所有单元上求和时,假设测试函数 $v$ 足够光滑(例如 $v\in C^2(\Omega)$),我们可以合并内部面和棱上的项. 对于内部面 $F\in \mathring{\mathcal F}_h$,记包含 $F$ 的两个单元为 $K_1, K_2$. 
引入如下跳量定义:
\begin{align*}
&[\tr_1(\bs \sigma)]_F : = \bs n_{\partial K_1}^{\intercal}\bs \sigma \bs n_{\partial K_1} \mid _{F} - \bs n_{\partial K_2}^{\intercal}\bs \sigma \bs n_{\partial K_2} \mid _{F},  \\
&[\tr_2(\bs \sigma)]_F := (\boldsymbol n_{\partial K_1}^{\intercal}\div \boldsymbol \sigma +  \div_F(\boldsymbol\sigma \boldsymbol n_{\partial K_1}))\mid _{F} + (\boldsymbol n_{\partial K_2}^{\intercal}\div \boldsymbol \sigma +  \div_F(\boldsymbol\sigma \boldsymbol n_{\partial K_2}))\mid _{F},  &  &\\
& [ \tr_e(\bs \sigma)]|_e := \sum_{K\in \omega_e} \sum_{F\in\partial K, e\in\partial F} (\boldsymbol n_{F,e}^{\intercal}\boldsymbol \sigma \boldsymbol n_{\partial K})|_e.  &  &
\end{align*}

回顾空间 $H(\div{\div },K; \mathbb{S})$ 在多面体 $K$ 的边界上 的迹(参见文献~\cite[Lemma 3.2]{FuehrerHeuerNiemi2019} 以及 \cite{Sinwel2009,PechsteinSchoeberl2018}). 
令 $H_{00}^{1/2}(F)$ 为 $\mathcal C_0^{\infty}(F)$ 关于范数 $\|\cdot\|_{H^{1/2}(\partial K)}$ 的闭包,该空间包含所有属于 $H^{1/2}(F)$ 且其零延拓至整个边界 $\partial K$ 后属于 $H^{1/2}(\partial K)$ 的函数. 
定义如下迹空间:
\begin{align*}
H_{n,0}^{1/2}(\partial K)&:=\{\partial_n v|_{\partial K}: v\in H^2(K)\cap H_0^1(K)\} \\
&\;=\{g\in L^2(\partial K): g|_F\in H_{00}^{1/2}(F)\;\;\forall~F\in\mathcal F(K)\},
\end{align*}
其范数为
\[
\|g\|_{H_{n,0}^{1/2}(\partial K)}:=\inf_{v\in H^2(K)\cap H_0^1(K)\atop \partial_n v=g}\|v\|_2,
\]
以及
\begin{align*}
H_{t,0}^{3/2}(\partial K)&:=\{v|_{\partial K}: v\in H^2(K), \partial_nv|_{\partial K}=0, v|_{e}=0\;\;\forall~e\in\mathcal E(K) \},
\end{align*}
其范数为
\[
\|g\|_{H_{t,0}^{3/2}(\partial K)}:=\inf_{v\in H^2(K)\atop \partial_n v=0, v=g}\|v\|_2.
\]
对于 $\tr_1$,令 $H_n^{-1/2}(\partial K):=(H_{n,0}^{1/2}(\partial K))'$;对于 $\tr_2$,令 $H_t^{-3/2}(\partial K):=(H_{t,0}^{3/2}(\partial K))'$.


%\begin{lemma}
%For any $\boldsymbol \tau\in\boldsymbol{H}(\div \boldsymbol{\div },K; \mathbb{S})$,  it holds
%\[
%\|\boldsymbol  n^{\intercal}\boldsymbol \tau\boldsymbol  n\|_{H_n^{-1/2}(\partial K)} \lesssim \|\boldsymbol{\tau}\|_{\boldsymbol{H}(\div \boldsymbol{\div })}.
%\]
%Conversely, for any $g_n\in H_n^{-1/2}(\partial K)$, there exists some $\boldsymbol \tau\in\boldsymbol{H}(\div \boldsymbol{\div },K; \mathbb{S})$ such that
%\[
%\boldsymbol  n^{\intercal}\boldsymbol \tau\boldsymbol  n|_{\partial K}=g_n, \quad 
%\|\boldsymbol{\tau}\|_{\boldsymbol{H}(\div \boldsymbol{\div })} \lesssim \|g_n\|_{H_n^{-1/2}(\partial K)}.
%\]
%The hidden constants depend only the shape of the domain $K$.
%\end{lemma}

\begin{lemma}[文献~\cite{FuehrerHeuerNiemi2019} 引理 3.2]\label{lem:Hdivdivtrace}
对于任意 $\boldsymbol \tau\in\boldsymbol{H}(\div{\div },K; \mathbb{S})$,成立:
\[
\|\boldsymbol n^{\intercal}\boldsymbol \tau\boldsymbol n\|_{H_n^{-1/2}(\partial K)} + \|2\div_F(\boldsymbol\tau \boldsymbol n)+ \partial_n(\boldsymbol n^{\intercal} \boldsymbol \tau\boldsymbol n)\|_{H_t^{-3/2}(\partial K)}\lesssim \|\boldsymbol{\tau}\|_{\boldsymbol{H}(\div{\div },K)}.
\]
反之,对于任意 $g_n\in H_n^{-1/2}(\partial K)$ 和 $g_t\in H_t^{-3/2}(\partial K)$,存在 $\boldsymbol \tau\in\boldsymbol{H}(\div{\div },K; \mathbb{S})$ 使得
\[
\boldsymbol n^{\intercal}\boldsymbol \tau\boldsymbol n|_{\partial K}=g_n, \quad 2\div_F(\boldsymbol\tau \boldsymbol n)+ \partial_n(\boldsymbol n^{\intercal} \boldsymbol \tau\boldsymbol n)=g_t,
\]
且满足
\[
\|\boldsymbol{\tau}\|_{\boldsymbol{H}(\div{\div },K)} \lesssim \|g_n\|_{H_n^{-1/2}(\partial K)}+\|g_t\|_{H_t^{-3/2}(\partial K)},
\]
其中的隐含常数仅依赖于区域 $K$ 的形状.
\end{lemma}



注意,格林公式~\eqref{eq:greenidentitydivdiv} 中的项 $(\boldsymbol n_{F,e}^{\intercal}\boldsymbol \tau \boldsymbol n, v)_e$ 并未被引理~\ref{lem:Hdivdivtrace} 所覆盖. 事实上,$\boldsymbol{H}(\div{\div },K; \mathbb{S})$ 的迹的完整刻画由 $(\div{\div }\bs\tau, v)_K-\left(\boldsymbol \tau, \nabla^{2} v\right)_{K}$ 给出,而该表达式无法被等价地解耦(参见文献~\cite[引理~3.2]{FuehrerHeuerNiemi2019}).
然而,在施加额外光滑性假设的情况下,可以将此迹在各面上局部化.


下面给出分段光滑函数属于 $\boldsymbol{H}(\div{\div },\Omega; \mathbb{S})$ 的一个充分条件. 

\begin{lemma}[文献~\cite{FuehrerHeuerNiemi2019} 命题 3.6]\label{lm:divdivconforming}
设 $\bs \sigma \in L^2(\Omega;\mathbb S)$ 且对任意 $K\in \mathcal T_h$ 有 $\bs \sigma|_K\in H^{2}(K;\mathbb S)$. 则 $\bs \sigma \in H(\div\div,\Omega;\mathbb S)$ 当且仅当满足以下条件:
\begin{enumerate}
\item $[\tr_1(\bs \sigma)]_F = 0$ 对于所有 $F\in \mathring{\mathcal F}_h$ 成立;
\smallskip
\item $[\tr_2(\bs \sigma)]_F = 0$ 对于所有 $F\in \mathring{\mathcal F}_h$ 成立;
\smallskip
\item $[ \tr_e(\bs \sigma)]|_e = 0$ 对于所有 $e\in \mathring{\mathcal E}_h$ 成立. 
\end{enumerate}
\end{lemma}


在构造 $H(\operatorname{div}\operatorname{div})$-协调有限元时,强制跳跃条件
$[ \operatorname{tr}_e(\boldsymbol{\sigma}) ]|_e = 0$
是非常困难的,因为该约束是施加在与 $e$ 相邻的宏单元上的. 
将 $\boldsymbol{\sigma}$ 在 $e\in \mathring{\mathcal E}_h$ 的法平面 $\mathcal N_e$ 上的投影保持连续,是一个充分但绝非必要的条件. 更具体地,作为一个 $(d-2)$-维子单纯形,边 $e$ 的法平面 $\mathcal N_e$ 的维数为 2. 为了施加该连续性条件,我们为每条边 $e\in \Delta_1(\mathcal{T}_h)$ 选取两个正交于 $e$ 的标准正交方向 ${\bs n_1, \bs n_2}$. 需要强调的是,$\mathcal N_e$ 仅由 $e$ 决定,而与包含 $e$ 的单元无关. 

我们将 $\mathcal N_e$ 上的 $2\times 2$ 对称矩阵空间记为 $\mathbb S(\mathcal N_e)$,并将
\[
Q_{\mathcal N_e}(\bs \sigma):=(\bs n_i^{\intercal}\bs \sigma\bs n_j)_{i,j=1,2}
\]
定义为 $\boldsymbol{\sigma}\in \mathbb S$ 在 $\mathbb S(\mathcal N_e)$ 上的投影. 



\begin{lemma}\label{lm:edgejump}
设 $\boldsymbol{\sigma}\in L^2(\Omega;\mathbb S)$,且对每个单元 $K\in \mathcal T_h$ 有 $\boldsymbol{\sigma}|_K \in H^{2}(K;\mathbb S)$. 
若 $Q_{\mathcal N_e}(\boldsymbol{\sigma})$ 在 $e$ 上连续,则对所有 $e\in \mathring{\mathcal E}_h$ 都有
$[ \operatorname{tr}_e(\boldsymbol{\sigma}) ]|_e = 0$. 
\end{lemma}
\begin{proof}
对于任意包含 $e\in \mathring{\mathcal E}_h$ 的面 $F$,由于 $F$ 也是内部面,因此在 $\omega_e$ 中恰有两个单元 $K_1$ 与 $K_2$ 满足 $F \subset \partial K_i,\ i=1,2$. 
法向量 $\bs n_{F,e}$ 由 $F$ 的定向诱导,该定向独立于单元,而 $\bs n_{\partial K}$ 是依赖于包含 $F$ 的单元 $K$ 的外法向,且 $\bs n_{\partial K_1}\mid_F = -\bs n_{\partial K_2}\mid _F$. 
由此可得
$(\boldsymbol n_{F,e}^{\intercal}\boldsymbol \sigma \boldsymbol n_{\partial K_1} + \boldsymbol n_{F,e}^{\intercal}\boldsymbol \sigma \boldsymbol n_{\partial K_2})|_e = 0$,
从而推出
$[ \operatorname{tr}_e(\boldsymbol{\sigma}) ]|_e = 0$. 
这完成了证明. 
\end{proof}


重调和方程 \eqref{eq:biharmonic2variables} 的一种混合变分形式为:
% Hellan-Herrmann-Johnson (HHJ) 混合格式 \cite{KrendlRafetsederZulehner2016,Hellan1967,Herrmann1967,Johnson1973} 是
找 $(\boldsymbol{\sigma} , u)\in H(\div\div,\Omega; \mathbb{S})\times L^2(\Omega)$ 使得:
\begin{subequations}\label{eq:divdivmixedformulation}
\begin{align}
(\boldsymbol\sigma, \boldsymbol\tau)+ (\div\div\boldsymbol\tau, u)  & =0 \quad\quad\quad\;\;
\forall~\boldsymbol\tau\in H(\div\div,\Omega; \mathbb{S}), \label{eq:divdivmixedformulation1}\\
(\div\div\boldsymbol\sigma, v) & =-(f, v)  \quad \forall~v\in L^2(\Omega). \label{eq:divdivmixedformulation2}
\end{align}
\end{subequations}
% 其中
% \[
% a(\boldsymbol\sigma, \boldsymbol\tau):=(\boldsymbol\sigma, \boldsymbol\tau),\quad b(\boldsymbol\tau, v):=\langle \div\div\boldsymbol\tau, v\rangle.
% \]

显然,对于任意 $\boldsymbol\sigma, \boldsymbol\tau\in H(\div\div,\Omega; \mathbb{S})$ 和 $v\in L^2(\Omega)$,有
\[
(\boldsymbol\sigma, \boldsymbol\tau)\leq\|\boldsymbol\sigma\|\|\boldsymbol\tau\|\leq\|\boldsymbol\sigma\|_{H(\div\div)}\|\boldsymbol\tau\|_{H(\div\div)},
\]
\[
(\div\div\boldsymbol\tau, v) \leq\|\div\div\boldsymbol\tau\| \|v\|\leq \|\boldsymbol\tau\|_{H(\div\div)}\|v\|.
\]

\begin{lemma}
成立
$$
\div\div H(\div\div,\Omega; \mathbb{S})=L^2(\Omega),
$$
即成立如下的 inf-sup 条件: 
\begin{equation}\label{eq:divdivinfsup}
  \|v\|\lesssim \sup_{\bs\tau\in H(\div\div,\Omega; \mathbb{S})}\frac{(\div\div\boldsymbol\tau, v)}{\|\boldsymbol\tau\|_{H(\div\div)}}\qquad \forall~v\in L^2(\Omega).
\end{equation}
\end{lemma}
\begin{proof}
显然 $\div\div H(\div\div,\Omega; \mathbb{S})\subseteq L^2(\Omega)$, 故只需证明 inf-sup 条件 \eqref{eq:divdivinfsup} 即可.

任取 $v\in L^2(\Omega)$. 由 $\div H^1(\Omega;\mathbb R^d)=L^2(\Omega)$ 知,存在 $\boldsymbol{w}\in H^1(\Omega;\mathbb R^d)$ 使得
\begin{equation*}
\div\boldsymbol{w}=v,\quad \|\boldsymbol{w}\|_{1}\lesssim \|v\|.
\end{equation*}
进一步,由$\div H^2(\Omega;\mathbb M)=H^1(\Omega;\mathbb R^d)$ 知,存在 $\boldsymbol{\tau}\in H^2(\Omega;\mathbb M)$ 使得
\begin{equation*}
\div\boldsymbol{\tau}=\boldsymbol{w},\quad \|\boldsymbol{\tau}\|_{2}\lesssim \|\boldsymbol{w}\|_1.
\end{equation*}
从而,有
\begin{equation*}
\div\div\boldsymbol{\tau}=v,\quad \|\boldsymbol{\tau}\|_{H(\div\div)}\lesssim \|\boldsymbol{\tau}\|_2\lesssim \|\boldsymbol{w}\|_1\lesssim \|v\|.
\end{equation*}
故 inf-sup 条件 \eqref{eq:divdivinfsup} 成立.
\end{proof}


\begin{theorem}
混合变分形式 \eqref{eq:divdivmixedformulation} 是适定的,且与问题 \eqref{eq:biharmonicvariatonalform} 等价. 即,如果 $u\in H_0^2(\Omega)$ 是问题 \eqref{eq:biharmonicvariatonalform} 的解,则 $\bs\sigma=-\nabla^2 u\in H(\div\div,\Omega; \mathbb{S})$ 且 $(\bs\sigma, u)$ 是混合变分形式 \eqref{eq:divdivmixedformulation} 的解. 反之,如果 $(\bs\sigma, u)\in H(\div\div,\Omega; \mathbb{S})\times L^2(\Omega)$ 是混合变分形式 \eqref{eq:divdivmixedformulation} 的解,则 $u\in H_0^2(\Omega)$ 且 $u$ 是问题 \eqref{eq:biharmonicvariatonalform} 的解. 
\end{theorem}
\begin{proof}
在算子 $\div\div: H(\div\div,\Omega; \mathbb{S})\to L^2(\Omega)$ 的核空间
$$
H(\div\div,\Omega; \mathbb{S})\cap \ker(\div\div)=\{\boldsymbol{\tau}\in L^{2}(\Omega; \mathbb{S}): \div\div\boldsymbol{\tau}=0\}
$$ 
上具有强制性:
\[
(\boldsymbol\tau, \boldsymbol\tau)=\|\boldsymbol\tau\|^2=\|\boldsymbol\tau\|_{H(\div\div)}^2\quad\forall~\boldsymbol\tau\in H(\div\div,\Omega; \mathbb{S})\cap \ker(\div\div).
\]
结合 inf-sup 条件 \eqref{eq:divdivinfsup} ,
由 Brezzi 理论可得混合变分形式 \eqref{eq:divdivmixedformulation} 的适定性. 

两个问题均是唯一可解的. 因此,只需证明若 $u$ 解问题 \eqref{eq:biharmonicvariatonalform},则令 $\bs\sigma=-\nabla^2 u$,$(\bs\sigma, u)$ 满足混合变分形式 \eqref{eq:divdivmixedformulation} 即可. 
假设 $u\in H_0^2(\Omega)$ 是问题 \eqref{eq:biharmonicvariatonalform} 的解. 
显然 $\bs\sigma\in L^{2}(\Omega; \mathbb{S})$ 且
\[
(\bs\sigma, \nabla^2v)=-(f, v)\quad\forall~v\in H_0^2(\Omega),
\]
这意味着在分布意义下 $\div\div\bs\sigma=-f\in L^{2}(\Omega)$. 
因此,$\bs\sigma\in H(\div\div,\Omega; \mathbb{S})$ 且式 \eqref{eq:divdivmixedformulation2} 成立. 

根据 $\div\div\bs\tau$ 的分布定义,我们有
\[
(\div\div\boldsymbol\tau, v) =(\boldsymbol\tau, \nabla^2v)\quad\forall~v\in C_0^{\infty}(\Omega).
\]
由于 $C_0^{\infty}(\Omega)$ 在 $H_0^2(\Omega)$ 中稠密,取 $v=u$ 可得
\[
(\div\div\boldsymbol\tau, u)=(\boldsymbol\tau, \nabla^2u)=-(\boldsymbol\tau, \bs\sigma),
\]
这证明了 \eqref{eq:divdivmixedformulation1}. 
\end{proof}

\section{$H(\div)$ 协调对称张量有限元}\label{sec:divS}
本节我们将构造$H(\div)$ 协调对称张量有限元 \cite{ChenHuang2022}. 
% 对于空间 $V = \mathbb P_k(T;\mathbb S)$,我们构造的单元与 Hu 在文献~\cite{Hu2015a} 中构造的单元略有不同. 此外,我们还构造了一类新的 $\mathbb P_{k+1}^{-}(T;\mathbb S)$ 型有限元. 对称 $H(\div)$-协调单元的迹空间似乎难以刻画,取而代之的是,我们通过识别泡函数空间,从而只需处理迹空间的对偶空间. 

% \subsection{散度算子}
\begin{lemma}\label{lem:divsymtensoronto}
设 $k\geq 0$,以及 $D$ 是 $d$ 维可缩区域. 算子 $\div: \sym (\mathbb H_{k}(D;\mathbb R^d)\boldsymbol x^{\intercal}) \to \mathbb H_{k}(D;\mathbb R^d)$ 是双射,从而 $\div: \mathbb P_{k+1}(D;\mathbb S) \to \mathbb P_{k}(D;\mathbb R^d)$ 是满射. 
\end{lemma}
\begin{proof}
注意到
\begin{align*}
\div(\sym(\mathbb H_{k}(D;\mathbb R^d)\boldsymbol x^{\intercal})) \subseteq \mathbb H_{k}(D;\mathbb R^d), \\
 \dim(\sym(\mathbb H_{k}(D;\mathbb R^d)\boldsymbol x^{\intercal}))=\dim\mathbb H_{k}(D;\mathbb R^d),
\end{align*}
只需证明 $\sym(\mathbb H_{k}(D;\mathbb R^d)\boldsymbol x^{\intercal})\cap\ker(\div)=\{\boldsymbol0\}$ 即可. 即:对于任意满足 $\div\sym(\boldsymbol q\boldsymbol x^{\intercal})=\boldsymbol0$ 的 $\boldsymbol q\in\mathbb H_{k}(D;\mathbb R^d)$,我们要证明 $\boldsymbol q =\boldsymbol 0$. 

%As $\sym(\boldsymbol  x \boldsymbol q)$ is symmetry, we also have $\div\sym(\boldsymbol  x \boldsymbol q)=0$. 
由 \eqref{eq:Hkdiv} 可知,
\begin{align*}
2\div \sym(\boldsymbol q\boldsymbol x^{\intercal}) & = \div (\boldsymbol q\boldsymbol x^{\intercal}) +\div (\boldsymbol x\boldsymbol q^{\intercal}) = (k+d)\boldsymbol q + (\grad \boldsymbol x) \boldsymbol q + (\div \boldsymbol q) \boldsymbol x \\
&= (k+d+1)\boldsymbol q+ (\div \boldsymbol q) \boldsymbol x.
\end{align*}
由 $\div \sym(\boldsymbol q\boldsymbol x^{\intercal})=\boldsymbol 0$ 可推得
\begin{equation}\label{eq:divsymxq}
(k+d+1)\boldsymbol q+ (\div \boldsymbol q) \boldsymbol x=\boldsymbol 0.
\end{equation}
对 \eqref{eq:divsymxq} 两边应用散度算子 $\div$,结合 \eqref{eq:Hkdiv} 可得
$$
2(k+d)\div\boldsymbol q=0.
$$
因此 $\div\boldsymbol q=0$,结合 \eqref{eq:divsymxq} 即得 $\boldsymbol q=\boldsymbol 0$. 
\end{proof}

\begin{lemma}[\cite{ChenHuHuang2018} Section 3.1]
集合 $\{\boldsymbol{T}_{i,j}:=\boldsymbol{t}_{i,j}\boldsymbol{t}_{i,j}^{\intercal}: 0\leq i<j\leq d\}$ 与 $\{\boldsymbol{N}_{i,j}:=-\operatorname{sym}(\nabla\lambda_i\otimes\nabla\lambda_j): 0\leq i<j\leq d\}$ 均构成 $\mathbb S$ 的一组基,且它们相互对偶,即满足
\begin{equation*}
\boldsymbol{T}_{i,j}:\boldsymbol{N}_{\ell,m} = \delta_{i,\ell}\delta_{j,m}, \quad \forall~0\leq i<j\leq d,\ 0\leq \ell<m\leq d.
\end{equation*}
\end{lemma} 

\begin{proof}
根据 $\boldsymbol{T}_{i,j}$ 和 $\boldsymbol{N}_{\ell,m}$ 的定义,并利用对称张量双点积(Frobenius 内积)的性质,我们有
$$
\boldsymbol{T}_{i,j}:\boldsymbol{N}_{\ell,m} = -\boldsymbol{t}_{i,j}^{\intercal}\sym(\nabla\lambda_\ell\otimes\nabla\lambda_m)\boldsymbol{t}_{i,j} = -(\boldsymbol{t}_{i,j}^{\intercal}\nabla\lambda_\ell)(\boldsymbol{t}_{i,j}^{\intercal}\nabla\lambda_m) = (\delta_{i,\ell}-\delta_{j,\ell})(\delta_{j,m}-\delta_{i,m}).
$$
注意到 $i<j$ 以及 $\ell<m$,因此 $\boldsymbol{T}_{i,j}:\boldsymbol{N}_{\ell,m} = \delta_{i,\ell}\delta_{j,m}$.

由于 $\mathbb S$ 的维数为 $\frac{d(d+1)}{2}$,而下标对 $(i,j)$ 的数量恰好与之相等,因此这两个集合均为 $\mathbb S$ 的基. 
\end{proof}



\subsection{泡函数空间}

设 $T$ 是 $d$ 维单纯形,以及整数 $k\geq1$. 
定义 $H(\div, T; \mathbb{S})$ 中的 $k$ 次多项式泡函数空间为
$$
\mathbb{B}_{k}(\div, T; \mathbb{S}):=\left\{\boldsymbol{\tau}\in \mathbb{P}_{k}(T; \mathbb{S}): \boldsymbol{\tau}\boldsymbol{n}|_{\partial K}=\boldsymbol{0}\right\}.
$$
易知 $\mathbb{B}_{1}(\div, T; \mathbb{S})$ 仅为零空间. 文献~\cite[Lemma 2.2]{Hu2015} 给出了 $\mathbb{B}_{k}(\div, T; \mathbb{S})$ 的如下刻画. 
\begin{lemma}\label{lm:bubbledof}
对于 $k\geq 2$,成立
\begin{equation}\label{eq:symtensorbubble}
    \mathbb{B}_{k}(\operatorname{div},T; \mathbb{S}) = \bigoplus_{0\leq i<j\leq d}\lambda_i\lambda_j \mathbb P_{k-2}(T)\boldsymbol{T}_{i,j}.
\end{equation}
因此,
\[
\dim \mathbb{B}_{k}(\operatorname{div},T; \mathbb{S}) = \dim \mathbb P_{k-2}(T;\mathbb S) = \frac{d(d+1)}{2} \binom{d+k-2}{d}.
\]
\end{lemma}

\begin{proof}
首先,容易验证包含关系 
\[
\bigoplus_{0\leq i<j\leq d}\lambda_i\lambda_j \mathbb P_{k-2}(T)\boldsymbol{T}_{i,j} \subseteq \mathbb{B}_{k}(\operatorname{div},T; \mathbb{S}).
\]
另一方面,由于 $\{\boldsymbol{T}_{i,j}\}_{0\leq i<j\leq d}$ 构成了 $\mathbb S$ 的一组基,对于任意 $\boldsymbol \tau \in \mathbb{B}_{k}(\operatorname{div},T; \mathbb{S})$,我们可以将其展开为 
\[
\boldsymbol \tau = \sum_{0\leq i<j\leq d} c_{ij}\boldsymbol T_{i,j},\quad \textrm{ 其中 } \; c_{ij}=\boldsymbol \tau:\boldsymbol{N}_{i,j}\in\mathbb P_k(T).
\]
应用等式 \eqref{eq:tangradlambdadual} 可得,
$$
(\boldsymbol \tau\nabla\lambda_{\ell})|_{F_{\ell}} = 
\sum_{0\leq i<j\leq d} c_{ij}|_{F_{\ell}}\boldsymbol{t}_{i,j}(\delta_{j,\ell}-\delta_{\ell,i}) = \sum_{0\leq i<\ell} c_{i\ell}|_{F_{\ell}}\boldsymbol{t}_{i,\ell} - \sum_{\ell<j\leq d} c_{\ell j}|_{F_{\ell}}\boldsymbol{t}_{\ell,j},\quad \ell=0,1,\ldots, d.
$$
由 $(\boldsymbol \tau\nabla\lambda_{\ell})|_{F_{\ell}} = 0$ 以及向量组 $\{\boldsymbol{t}_{i,\ell}: 0\leq i\leq d, i\neq \ell\}$ 是 $\mathbb R^d$ 的一组基可知:
\[
c_{i\ell}|_{F_{\ell}} = 0 \quad (0\leq i<\ell), \qquad c_{\ell j}|_{F_{\ell}} = 0 \quad (\ell<j\leq d).
\]
综上所述,对于所有的 $0\leq i<j\leq d$,当 $\ell = i$ 或 $\ell = j$ 时,均有 $c_{ij}|_{F_{\ell}}=0$. 
这表明 $c_{ij}$ 必须含有因子 $\lambda_i$ 和 $\lambda_j$,即 $c_{ij} = \lambda_i\lambda_j q_{ij}$,其中 $q_{ij}\in \mathbb P_{k-2}(T)$. 
证毕. 
\end{proof}


\begin{lemma}\label{lem:HuZhanginteriordof}
对于 $k\geq2$,有
$$
\mathbb{B}_{k}'(\div, T; \mathbb{S}) = \mathcal N (\mathbb{P}_{k-2}(T; \mathbb{S})).
$$
即 $\boldsymbol{\tau}\in\mathbb{B}_{k}(\div, T; \mathbb{S})$ 由以下自由度唯一确定:
$$
(\boldsymbol{\tau}, \boldsymbol{q})_T, \quad\forall~\boldsymbol{q}\in\mathbb{P}_{k-2}(T; \mathbb{S}).
$$ 
\end{lemma}
\begin{proof}
假设 $\boldsymbol{\tau} \in \mathbb{B}_{k}(\operatorname{div}, T; \mathbb{S})$ 满足
\begin{equation}\label{eq:vanishing_moments}
    (\boldsymbol{\tau}, \boldsymbol{q})_T = 0, \quad \forall~\boldsymbol{q}\in\mathbb{P}_{k-2}(T; \mathbb{S}).
\end{equation}
由引理 \ref{lm:bubbledof} 中的分解 \eqref{eq:symtensorbubble} 可知,存在 $q_{ij}\in \mathbb P_{k-2}(T)$ ($0\leq i<j\leq d$) 使得
\[
\boldsymbol{\tau}=\sum_{0\leq i<j\leq d}\lambda_i\lambda_j q_{ij}\boldsymbol{T}_{i,j}.
\]
注意到对称张量组 $\{\boldsymbol{N}_{i,j}\}_{0\leq i<j\leq d}$ 与 $\{\boldsymbol{T}_{i,j}\}_{0\leq i<j\leq d}$ 关于 Frobenius 内积构成对偶基. 
选取测试函数 $\boldsymbol{q}=\sum_{0\leq i<j\leq d} q_{ij}\boldsymbol{N}_{i,j}\in\mathbb{P}_{k-2}(T; \mathbb{S})$ 并代入 \eqref{eq:vanishing_moments},我们得到
\[
0 = (\boldsymbol \tau, \boldsymbol{q})_T = \sum_{0\leq i<j\leq d} \int_T \lambda_i\lambda_j q^2_{ij} \,\mathrm{d}x.
\]
由于权函数 $\lambda_i\lambda_j$ 在单纯形 $T$ 内部恒正,上述等式蕴含对于所有 $i,j$ 均有 $q_{ij}=0$,进而 $\boldsymbol{\tau}=\boldsymbol{0}$. 
最后,由维数计数可知结论成立. 
\end{proof}

文献~\cite[Section 4.3]{ChenHuang2024a} 给出了 $\mathbb B_k(\div, T;\mathbb S)$ 和 $\mathbb B_k'(\div, T;\mathbb S)$ 的另一种刻画. 

\subsection{迹空间}
由于对称性约束,迹映射 $\operatorname{tr}^{\operatorname{div}}: \mathbb P_k(T; \mathbb S) \to \mathbb P_{k}(\partial T;\mathbb R^{d})$ 并非满射. 我们需要在低维子单纯形上施加特定的相容性条件. 幸运的是,我们仅需知道其像空间的维数. 


\begin{lemma} 
设整数 $k\geq 1$,则有
\begin{align*}
\dim {\rm tr}^{\div}( \mathbb P_k(T; \mathbb S)) &= \dim \mathbb P_k(T; \mathbb S) - \dim \mathbb{B}_{k}(\div,T; \mathbb{S}) \\
&= \dim \mathbb H_k(T; \mathbb S) + \dim \mathbb H_{k-1}(T; \mathbb S)\\
&= \frac{1}{2}d(d+1) \left [{k+d-1 \choose d-1} + {k+d-2 \choose d-1}\right ].
\end{align*}
\end{lemma}

下面我们阐述 $H(\operatorname{div};\mathbb S)$ 有限元因对称性引起的超光滑性. 
考虑由两个 $(d-1)$ 维面 $F, F'\in \Delta_{d-1}(T)$ 共享的 $\ell$ 维子单纯形 $e\in \Delta_{\ell}(T)$ (其中 $\ell=0,1,\ldots,d-2$). 
利用 $\boldsymbol \tau$ 的对称性,分量 $(\boldsymbol n_{F}^{\intercal}\boldsymbol \tau\boldsymbol n_{F'})|_e$ 可由 $(\boldsymbol \tau\boldsymbol n_{F})|_{F}$ 和 $(\boldsymbol \tau\boldsymbol n_{F'})|_{F'}$ 共同确定. 
这意味着 $e$ 上形如 $\boldsymbol n_{i}^{\intercal}\boldsymbol \tau\boldsymbol n_{j}$ ($i,j=1,\ldots, d-\ell$) 的自由度. 
特别地,对于 $0$ 维顶点 $\delta$,其自由度即为整个张量值 $\boldsymbol \tau(\delta)$. 

限制在面 $F\in \Delta_{d-1}(T)$ 上的迹 $\boldsymbol \tau \boldsymbol n$ 可进一步分解为两个分量:
\begin{enumerate}[label=(\arabic*)]
    \item 法向-法向分量 $\boldsymbol n^{\intercal}\boldsymbol \tau\boldsymbol n$:该分量主要由各个维度单纯形上的自由度 $\boldsymbol n_{i}^{\intercal}\boldsymbol \tau\boldsymbol n_{j}$ 所确定;
    \item 切向-法向分量 $\Pi_F\boldsymbol \tau\boldsymbol n$:一旦其在 $F$ 的边界 $\partial F$ 上的法向迹(即 $\operatorname{tr}^{\operatorname{div}_F}(\Pi_F\boldsymbol \tau\boldsymbol n) = \boldsymbol n_{F,e}^{\intercal}\boldsymbol \tau\boldsymbol n$)被确定,该分量将由 $F$ 上的内部矩唯一确定. 
\end{enumerate}


\begin{lemma}\label{lem:divSboundarydofs}
如下自由度给出了
 $$
 ( {\rm tr}^{\div}( \mathbb P_k(T; \mathbb S)) )'
 $$
的一组基:
\begin{subequations}\label{HdivSfemdof}
\begin{align}
\boldsymbol \tau (\delta), & \quad~\delta\in \Delta_{0}(T), \label{HdivSfemdof1}\\
(\boldsymbol n_i^{\intercal}\boldsymbol \tau\boldsymbol n_j, q)_f, & \quad~q\in\mathbb P_{k-\ell-1}(f),  f\in\Delta_{\ell}(T),\;  \label{HdivSfemdof2}\\
&\quad\quad i,j=1,\ldots, d-\ell, \textrm{ 和 }\; \ell=1,\ldots, d-1, \notag\\
(\Pi_F\boldsymbol \tau\boldsymbol n, \boldsymbol q)_F, & \quad~\boldsymbol q\in {\rm ND}_{k-2}(F),  F\in\Delta_{d-1}(T).\label{HdivSfemdof3}
\end{align}
\end{subequations}
\end{lemma}
\begin{proof}
首先计算自由度个数.
对于 $\ell=0,1,\ldots, d-2$,我们将 $(\boldsymbol n_i^{\intercal}\boldsymbol \tau\boldsymbol n_j)_{1\leq i,j\leq d-\ell}$ 进行重新分配:
当 $i=j$ 时,将 $\boldsymbol n_i^{\intercal}\boldsymbol \tau\boldsymbol n_i$ 分配到面 $F_i$ 上;
当 $i\neq j$ 时,将 $\boldsymbol n_i^{\intercal}\boldsymbol \tau\boldsymbol n_j$ 分配到 $(d-2)$ 维的面(脊)$e_{ij} = F_i \cap F_j$ 上. 
结合 $F_i$ 和 $e_{ij}$ 上 Lagrange 元的唯一可解性,可知自由度 \eqref{HdivSfemdof1}-\eqref{HdivSfemdof2} 的总个数为
$$
{d+1\choose2}{k+d-2\choose d-2} + (d+1){k+d-1\choose d-1}.
$$
另一方面,注意到 $F$ 为 $(d-1)$ 维单纯形,其上的第一类 N\'ed\'elec 空间维数为
$$
\dim {\rm ND}_{k-2}(F)= (d-1)\dim\mathbb P_{k-1}(F) - \dim\mathbb H_{k}(F)= (d-1){k+d-2\choose d-1} - {k+d-2\choose d-2}
$$
将上述两部分相加,并利用组合恒等式 $\binom{k+d-1}{d-1} = \binom{k+d-2}{d-1} + \binom{k+d-2}{d-2}$,可得自由度 \eqref{HdivSfemdof} 的总个数为:
\begin{align*}
&\quad {d+1\choose2}{k+d-2\choose d-2} + (d+1){k+d-1\choose d-1} + (d^2-1){k+d-2\choose d-1} - (d+1){k+d-2\choose d-2} \\
& = {d+1\choose2}{k+d-2\choose d-2} + (d+1)d{k+d-2\choose d-1} = \frac{1}{2}d(d+1) \left [{k+d-1 \choose d-1} + {k+d-2 \choose d-1}\right ].
\end{align*}
这恰好等于 $\dim {\rm tr}^{\div}( \mathbb P_k(T; \mathbb S))$.

接着证明若所有自由度 \eqref{HdivSfemdof} 均为零,则 $\boldsymbol \tau = \boldsymbol 0$. 
由于 $\boldsymbol n_i^{\intercal}\boldsymbol \tau\boldsymbol n_j$ 在边或面上是多项式,由自由度 \eqref{HdivSfemdof1}-\eqref{HdivSfemdof2} 取值为零以及 Lagrange 元的唯一可解性,可得
\[
\boldsymbol n_i^{\intercal}\boldsymbol\tau\boldsymbol n_j|_f=0, \quad\forall~f\in\Delta_{\ell}(T), \; i,j=1,\ldots, d-\ell, \quad \text{和} \quad \ell=1,\ldots, d-1.
\]
这意味着对于任意面 $F\in\Delta_{d-1}(T)$ 及其边界 $e\in\Delta_{d-2}(F)$,有
\begin{equation}\label{eq:vanishing_traces}
\boldsymbol n^{\intercal}\boldsymbol\tau\boldsymbol n|_F=0, \qquad \boldsymbol n_{F,e}^{\intercal}\boldsymbol\tau\boldsymbol n|_e = (\Pi_F\boldsymbol\tau\boldsymbol n)\cdot \boldsymbol n_{F,e} |_e = 0.
\end{equation}
注意 $\Pi_F\boldsymbol\tau\boldsymbol n|_F \in \mathbb P_k(F; \mathbb{R}^{d-1})$. 
对于切向分量 $\Pi_F\boldsymbol\tau\boldsymbol n$,其在 $\partial F$ 上的法向迹由 \eqref{eq:vanishing_traces} 的第二个等式确定为零;其在 $F$ 内部的矩由自由度 \eqref{HdivSfemdof3} 确定为零. 
由 BDM 单元在 $F$ 上的唯一可解性可知,$\Pi_F\boldsymbol\tau\boldsymbol n|_F=\boldsymbol 0$. 
最后,结合 \eqref{eq:vanishing_traces} 中法向分量为零的结论,即得 $\boldsymbol\tau\boldsymbol n|_F=\boldsymbol 0$. 
\end{proof}

\begin{remark}\rm
\label{hudofs}
% 作为比较,
文献~\cite{Hu2015} 中 Hu-Zhuang 元在边界上的自由度为
\begin{align*}
\boldsymbol \tau (\delta), & \quad~\delta\in \Delta_{0}(T), \\
(\boldsymbol  n_i^{\intercal}\boldsymbol \tau\boldsymbol n_j, q)_f, & \quad~q\in\mathbb P_{k-\ell-1}(f),  f\in\Delta_{\ell}(T),\;  \\
&\quad\quad i,j=1,\ldots, d-\ell, \textrm{ 和 }\; \ell=1,\ldots, d-1, \notag\\
(\boldsymbol t_i^{\intercal}\boldsymbol \tau\boldsymbol n_j, q)_f, & \quad~q\in\mathbb P_{k-\ell-1}(f),  f\in\Delta_{\ell}(T),\;  \\
&\quad\quad i=1,\ldots, \ell,j=1,\ldots, d-\ell, \textrm{ 和 }\; \ell=1,\ldots, d-1.
\end{align*}
区别在于施加切向-法向分量的方式不同. 
\end{remark}

\subsection{泡函数空间的分解}\label{subsec:newdivSfem}
为了构造 $H(\div, T;\mathbb S)$ 有限元,由 $\mathcal N (\mathbb{P}_{k-2}(T; \mathbb{S}))$ 给出的内部自由度已足够. 对于 $H(\div\div, T;\mathbb S)$ 有限元的构造,我们使用 $\div$ 算子将 $\mathbb B_k(\div, T;\mathbb S)$ 分解为
\begin{equation*}
E_{0,k}(\mathbb S) := \mathbb B_k(\div, T; \mathbb S)\cap \ker(\div), \quad E_{0,k}^{\bot}(\mathbb S) := \mathbb B_k(\div, T; \mathbb S)/ E_{0,k}(\mathbb S).
\end{equation*}
%
在不引起混淆的情况下,我们将 $E_{0,k}(\mathbb S)$ 和 $E_{0,k}^{\bot}(\mathbb S)$ 分别简记为 $E_{0}(\mathbb S)$ 和 $E_{0}^{\bot}(\mathbb S)$. 
% 如前所述,
我们可以通过 $\div^*$ 来刻画 $E_{0,k}^{\bot}(\mathbb S)$ 的对偶空间,即限制在泡函数空间上的 $-\defm:=-\sym \grad$,并可扩展到 $H^1(T; \mathbb R^d)$. 
\begin{lemma}\label{lem:E0Sbot}
设整数 $k\geq2$. 映射
 $$
 \div: E_{0,k}^{\bot}(\mathbb S) \to \mathbb P_{k-1,{\rm RM}}^{\bot}:=\mathbb P_{k-1}(T;\mathbb R^d)/\ker(\defm)
 $$
 是双射,从而
\begin{align*}
(E_{0,k}^{\bot}(\mathbb S))' &= \mathcal N(\defm \mathbb P_{k-1}(T;\mathbb R^d)),\\
\dim E_{0,k}^{\bot}(\mathbb S)  &=  d{k+d-1\choose k-1} -\frac{1}{2}(d^2+d).
\end{align*}
\end{lemma}
\begin{proof}
我们采用文献~\cite[Theorem~2.2]{Hu2015} 中的方法证明 $\div\mathbb B_k(\div, T; \mathbb S)=\mathbb P_{k-1,{\rm RM}}^{\bot}$. 

首先,包含关系 $ \div( \mathbb B_k(\div, T;\mathbb S)) \subseteq \mathbb P_{k-1,{\rm RM}}^{\bot}$ 可以通过分部积分证明:
$$
(\div \boldsymbol \tau, \boldsymbol v)_T = -( \boldsymbol \tau, \defm \boldsymbol v)_T = 0 \quad \forall~\boldsymbol v\in \ker(\defm).
$$
若 $ \div( \mathbb B_k(\div, T;\mathbb S)) \neq \mathbb P_{k-1,{\rm RM}}^{\bot}$,则存在非零函数 $\boldsymbol v\in  \mathbb P_{k-1,{\rm RM}}^{\bot}$ 满足 $\boldsymbol v\perp \div( \mathbb B_k(\div, T; \mathbb S))$,这等价于 $\defm \boldsymbol v \perp \mathbb B_k(\div, T; \mathbb S)$. 在基 $\{\boldsymbol N_{i,j}, 0\leq i<j\leq d\}$ 下展开对称矩阵 $\defm \boldsymbol v $ 为 $\defm \boldsymbol v = \sum\limits_{0\leq i<j\leq d} q_{ij} {\boldsymbol N}_{i,j}$,其中 $q_{ij}\in\mathbb P_{k-2}(T)$. 令 $\boldsymbol \tau_v = \sum\limits_{0\leq i<j\leq d} q_{ij} \lambda_i \lambda_j \boldsymbol T_{i,j} \in \mathbb B_k(\div, T;\mathbb S)$. 我们有
$$
(\defm \boldsymbol v, \boldsymbol \tau_v)_T = \sum_{0\leq i<j\leq d} \int_Tq_{ij}^2 \lambda_{i}\lambda_j \dx = 0,
$$
这意味着对于所有 $0\leq i< j \leq d$,都有 $q_{ij} = 0$,即 $\defm \boldsymbol v = 0$. 由于 $\boldsymbol v\in  \mathbb P_{k-1,{\rm RM}}^{\bot}$,故 $\boldsymbol v = 0$. 

因为 $\div E_{0,k}^{\bot}(\mathbb S)=\div\mathbb B_k(\div, T; \mathbb S)$,所以映射
 $\div: E_{0,k}^{\bot}(\mathbb S) \to \mathbb P_{k-1,{\rm RM}}^{\bot}$
 是双射. 

 对于 $\boldsymbol v\in E_{0,k}^{\bot}(\mathbb S)$,若对所有 $\boldsymbol q\in \mathbb P_{k-1}(T;\mathbb R^d)$ 都有 $(\boldsymbol v, \defm\boldsymbol q)_T = 0$,则 $\div \boldsymbol v = \boldsymbol 0$,即 $\boldsymbol v\in E_{0,k}(\mathbb S)$. 于是 $\boldsymbol v\in E_{0,k}(\mathbb S)\cap E_{0,k}^{\bot}(\mathbb S) = \{\boldsymbol 0\}$. 因此 $E_{0,k}^{\bot}(\mathbb S)$ 由 $\mathcal N(\defm \mathbb P_{k-1}(T;\mathbb R^d))$ 确定. 由由维数匹配可知结论成立. 
\end{proof}
 

接着我们转向空间 $E_{0,k}(\mathbb S) $. 如果使用原始方法,我们需要前一个空间的泡函数空间和微分算子. 例如,在二维情形下~\cite{ArnoldWinther2002},有
$
E_{0,k}(\mathbb S)=\curl\curl(\mathbb P_{k+2}(T)\cap H_0^2(T))
$;
在三维情形下~\cite{ArnoldAwanouWinther2008,ChenHuang2022b},有
$$
E_{0,k}(\mathbb S)=\textrm{inc}\, \mathbb B_{k+2} ({\rm inc},T;\mathbb S),
$$
其中
\begin{align*}
\mathbb B_{k+2} ({\rm inc},T;\mathbb S):=\{&\boldsymbol\tau\in\mathbb P_{k+2}(T;\mathbb S): \boldsymbol n\times\boldsymbol\tau\times\boldsymbol n=\boldsymbol 0, \\
&2\defm_F(\boldsymbol n\cdot\boldsymbol\tau\Pi_F)-\Pi_F\partial_n\boldsymbol\tau\Pi_F=\boldsymbol 0\;\;\forall~F\in\Delta_{d-1}(T) \}.
\end{align*} 
这种基于张量势的刻画很难推广到任意维数. 

取而代之的是,我们使用对偶方法来刻画 $E_{0,k}'(\mathbb S)$. 为此,对于连通区域 $D$,记刚体运动空间为
\[
\textrm{RM}:={\rm ND}_{0}(D)=\{\boldsymbol c+\boldsymbol A\boldsymbol x: \;\boldsymbol c\in\mathbb R^d,\; \boldsymbol A\in \mathbb K\},
\]
其中 $\mathbb K$ 为反对称矩阵空间. 
定义投影算子 $\boldsymbol \pi_{\textrm{RM}}: \mathcal C^1(D; \mathbb R^d)\to \textrm{RM}$ 为
$$
\boldsymbol \pi_{\textrm{RM}}\boldsymbol  v:=\boldsymbol  v(\boldsymbol 0)+(\skw(\nabla\boldsymbol v))(\boldsymbol 0)\boldsymbol x.
$$
显然,对于所有 $\boldsymbol v\in\textrm{RM}$,有 $\boldsymbol \pi_{\textrm{RM}}\boldsymbol v=\boldsymbol v$. 
此外,我们定义映射 $\cdot\boldsymbol x: \mathbb P_{k}(D;\mathbb S) \to \mathbb P_{k+1}(D;\mathbb R^d)$ 为 $\boldsymbol \tau \mapsto \boldsymbol \tau \boldsymbol x$,即矩阵向量乘积. 


我们将建立如下短正合列:
\begin{equation*}%\label{eq:deRhamcomplex3dPolydouble}
%\resizebox{.91\hsize}{!}{$
\xymatrix{
\textrm{RM}\ar@<0.4ex>[r]^-{\subset} & \mathbb P_{k+1}(D;\mathbb R^d)\ar@<0.4ex>[r]^-{\defm} \ar@<0.4ex>[l]^-{\boldsymbol \pi_{\textrm{RM}}}  & \defm\mathbb P_{k+1}(D; \mathbb R^d)\ar@<0.4ex>[l]^-{\cdot \boldsymbol x}  
\ar@<0.4ex>[r]^-{}& \boldsymbol0 \ar@<0.4ex>[l]^-{},
%& 0 \ar@<0.4ex>[l]^-{\supset} }.
}
%$}
\end{equation*}
并由此导出一个空间分解. 

\begin{lemma}\label{lem:xdefkernal}
设整数 $k\geq0$. 若 $\boldsymbol q\in \mathbb P_{k+1}(D;\mathbb R^d)$ 满足 $(\defm\boldsymbol q)\boldsymbol x=\boldsymbol0$,则 $\boldsymbol q\in\textrm{RM}$. 
\end{lemma}
\begin{proof}
由于
$
\boldsymbol x^{\intercal}(\boldsymbol x\cdot\nabla)\boldsymbol q=\boldsymbol x^{\intercal}(\nabla\boldsymbol q)\boldsymbol x=\boldsymbol x^{\intercal}(\defm\boldsymbol q)\boldsymbol x=0
$,
我们得到
$$(\boldsymbol x\cdot\nabla)(\boldsymbol x^{\intercal}\boldsymbol q)=\boldsymbol x^{\intercal}(\boldsymbol x\cdot\nabla)\boldsymbol q+\boldsymbol x^{\intercal}\boldsymbol q=\boldsymbol x^{\intercal}\boldsymbol q.$$
由 \eqref{eq:Hkxgrad} 可知,这意味着 $\boldsymbol x^{\intercal}\boldsymbol q\in\mathbb P_1(D)$. 
另一方面,注意到 $(\nabla\boldsymbol q)\boldsymbol x = \nabla(\boldsymbol x^{\intercal}\boldsymbol q)-\boldsymbol q$,我们有
$$(\boldsymbol x\cdot\nabla)\boldsymbol q + (\nabla(\boldsymbol x^{\intercal}\boldsymbol q)-\boldsymbol q)=(\nabla\boldsymbol q)^{\intercal}\boldsymbol x + (\nabla\boldsymbol q)\boldsymbol x=2(\defm\boldsymbol q)\boldsymbol x=\boldsymbol0,$$
这蕴含 $(\boldsymbol x\cdot\nabla)\boldsymbol q - \boldsymbol q = -\nabla(\boldsymbol x^{\intercal}\boldsymbol q)\in \mathbb P_0(D;\mathbb R^d)$. 
因此 $\boldsymbol q\in\mathbb P_1(D;\mathbb R^d)$. 假设 $\boldsymbol q=\boldsymbol A\boldsymbol x+\boldsymbol C$,其中 $\boldsymbol A\in\mathbb M$,$\boldsymbol C\in\mathbb R^d$. 那么
$$
\boldsymbol x^{\intercal}(\sym\boldsymbol A)\boldsymbol x+\boldsymbol x^{\intercal}\boldsymbol C=\boldsymbol x^{\intercal}\boldsymbol A\boldsymbol x+\boldsymbol x^{\intercal}\boldsymbol C=\boldsymbol x^{\intercal}\boldsymbol q\in\mathbb P_1(D),
$$
这意味着 $\sym\boldsymbol A=\boldsymbol0$. 因此 $\boldsymbol A\in\mathbb K$ 且 $\boldsymbol q\in\textrm{RM}$. 
\end{proof}   

\begin{lemma}
设整数 $k\geq0$. 成立
\begin{equation}\label{eq:xdefimage}
\left(\defm\mathbb P_{k+1}(D;\mathbb R^d)\right)\boldsymbol x=\mathbb P_{k}(D;\mathbb S)\boldsymbol x=\mathbb P_{k+1}(D;\mathbb R^d)\cap\ker(\boldsymbol \pi_{\rm RM}).
\end{equation}
\end{lemma}
\begin{proof}
对于任意 $\boldsymbol\tau\in \mathbb P_{k}(D;\mathbb S)$,有
$$
\boldsymbol \pi_{\rm RM}(\boldsymbol\tau\boldsymbol x)= (\skw(\nabla(\boldsymbol\tau\boldsymbol x)))(\boldsymbol 0)\boldsymbol x= \skw(\boldsymbol\tau(\boldsymbol 0))\boldsymbol x=\boldsymbol0.
$$
因此 $\mathbb P_{k}(D;\mathbb S)\boldsymbol x\subseteq\mathbb P_{k+1}(D;\mathbb R^d)\cap\ker(\boldsymbol \pi_{\rm RM})$. 
另一方面,由引理~\ref{lem:xdefkernal} 可得
$$
\dim\left(\left(\defm\mathbb P_{k+1}(D;\mathbb R^d)\right)\boldsymbol x\right)=\dim\mathbb P_{k+1}(D;\mathbb R^d)-\dim\textrm{RM},
$$
这等于 $\mathbb P_{k+1}(D;\mathbb R^d)\cap\ker(\boldsymbol \pi_{\rm RM})$ 的维数. 
故 \eqref{eq:xdefimage} 成立. 
\end{proof}

\begin{corollary}
设整数 $k\geq0$. 我们有如下空间分解:
\begin{equation}\label{eq:symTensorPolySpacedecomp}
\mathbb P_{k}(D;\mathbb S)=\defm \mathbb P_{k+1}(D;\mathbb R^d) \oplus (\ker (\cdot\boldsymbol x)\cap \mathbb P_{k}(D;\mathbb S)).
\end{equation}
\end{corollary}
\begin{proof}
由 引理~\ref{lem:xdefkernal} 可知 $\defm \mathbb P_{k+1}(D;\mathbb R^d) \cap (\ker (\cdot\boldsymbol x)\cap \mathbb P_{k}(D;\mathbb S))=\{\boldsymbol0\}$. 
由于 \eqref{eq:xdefimage},
\begin{align*}
&\quad\dim\defm \mathbb P_{k+1}(D;\mathbb R^d) +\dim(\ker (\cdot\boldsymbol x)\cap \mathbb P_{k}(D;\mathbb S)) \\
&=\dim\defm \mathbb P_{k+1}(D;\mathbb R^d)+\dim\mathbb P_{k}(D;\mathbb S)-\dim(\mathbb P_{k}(D;\mathbb S)\boldsymbol x) \\
&=\dim\mathbb P_{k+1}(D;\mathbb R^d)-\dim\textrm{RM}+\dim\mathbb P_{k}(D;\mathbb S)-\dim(\mathbb P_{k}(D;\mathbb S)\boldsymbol x) \\
&=\dim\mathbb P_{k}(D;\mathbb S),
\end{align*}
即得 \eqref{eq:symTensorPolySpacedecomp}. 
\end{proof}


\begin{lemma}\label{lem:E0S}
设整数 $k\geq2$. 则有
\begin{equation}\label{eq:E0Sdualcharac} 
E_{0,k}'(\mathbb S) = \mathcal N(\ker (\cdot\boldsymbol x)\cap \mathbb P_{k-2}(T;\mathbb S)).
\end{equation}
即函数 \(\boldsymbol\tau\in E_{0,k}(\mathbb S)\) 由下列自由度唯一确定:
\begin{equation}\label{eq:20251206}
(\boldsymbol\tau, \boldsymbol q)_T,\quad\forall~\boldsymbol q\in \ker (\cdot\boldsymbol x)\cap \mathbb P_{k-2}(T;\mathbb S).
\end{equation}  
此外,空间的维数为
$$
\dim E_{0,k}(\mathbb S)= \frac{d(d+1)}{2} {k-2+d \choose d} - d {d+k-1 \choose d} + \frac{d(d+1)}{2}.
$$
\end{lemma}
\begin{proof}
根据空间分解 \eqref{eq:symTensorPolySpacedecomp}、引理~\ref{lem:HuZhanginteriordof} 以及引理~\ref{lem:E0Sbot},可得
\begin{align*}
\dim(\ker (\cdot\boldsymbol x)\cap \mathbb P_{k-2}(T;\mathbb S))&=\dim\mathbb P_{k-2}(T;\mathbb S)-\dim\defm \mathbb P_{k-1}(T;\mathbb R^d) \\
&=\dim\mathbb B_k(\div, T; \mathbb S)-\dim E_{0,k}^{\bot}(\mathbb S)=\dim E_{0,k}(\mathbb S),
\end{align*}
即维数相符. 由此亦可得出 \(\dim E_{0,k}(\mathbb S)\) 的具体表达式. 

现设 \(\boldsymbol\tau\in E_{0,k}(\mathbb S)\) 使得由 \eqref{eq:20251206} 定义的自由度为零. 应用分部积分,可得
\begin{equation*}
(\boldsymbol\tau, \defm\boldsymbol q)_T =-(\div\boldsymbol\tau, \boldsymbol q)_T=0,\quad\forall\,\boldsymbol q\in\mathbb P_{k-1}(T;\mathbb R^d). 
\end{equation*}
结合空间分解 \eqref{eq:symTensorPolySpacedecomp} 可知,
\begin{equation*}
(\boldsymbol\tau, \boldsymbol q)_T =0,\quad\forall\,\boldsymbol q\in\mathbb P_{k-2}(T;\mathbb S). 
\end{equation*}
进一步由引理 \ref{lem:HuZhanginteriordof} 可得 \(\boldsymbol\tau=0\),证毕. 
\end{proof}

\begin{remark}\rm 
在二维和三维情形下,我们有 (参见~\cite{ChenHuang2020,ChenHuang2022b})
\begin{equation*}
%\mathbb E_{k}^{\oplus}(D)
\ker (\cdot\boldsymbol x)\cap \mathbb P_{k}(D;\mathbb S)=\begin{cases}
\boldsymbol x^{\perp}(\boldsymbol x^{\perp})^{\intercal}\mathbb P_{k-2}(D), & \textrm{ for }  d=2,\\
\boldsymbol x\times\mathbb P_{k-2}(D;\mathbb S)\times\boldsymbol x, & \textrm{ for } d=3,
\end{cases}
\end{equation*}
其中 $\boldsymbol x^{\perp}:=\begin{pmatrix}
x_2\\ -x_1
\end{pmatrix}$. 但这推广到任意维数并不容易,也非必要. 
找到 $\ker (\cdot\boldsymbol x)\cap \mathbb P_{k-1}(T;\mathbb S)$ 显式基的一种计算方法如下:分别找一组 $\mathbb P_{k-1}(T;\mathbb S)$ 和 $\mathbb P_{k}(T;\mathbb R^d)$ 的基,然后形成算子 $\cdot\boldsymbol x$ 的矩阵表示 $X$,随后可以通过代数方法找到零空间 $\ker(X)$. 
\end{remark}

\subsection{$H(\div; \mathbb S)$-协调有限元}

结合 引理 \ref{lem:divSboundarydofs}, \ref{lem:E0Sbot}, \ref{lem:E0S} 和空间分解~\eqref{eq:symTensorPolySpacedecomp},我们得到 $H(\div; \mathbb S)$-协调有限元的自由度. 
\begin{theorem}\label{thm:SBDMunisolvence}
取形函数空间 $V(\mathbb S) = \mathbb P_k(T;\mathbb S)$,其中 $k\geq d+1$. 
如下自由度
\begin{subequations}\label{HdivSBDMfemdof}
\begin{align}
\boldsymbol \tau (\delta), & \quad~\delta\in \Delta_{0}(T), \label{HdivSBDMfemdof1}\\
(\boldsymbol n_i^{\intercal}\boldsymbol \tau\boldsymbol n_j, q)_f, & \quad~q\in\mathbb P_{k-\ell-1}(f),  f\in\Delta_{\ell}(T),\;  \label{HdivSBDMfemdof2}\\
&\quad\quad i,j=1,\ldots, d-\ell, \textrm{ 和 }\; \ell=1,\ldots, d-1, \notag\\
(\Pi_F\boldsymbol \tau\boldsymbol n, \boldsymbol q)_F, & \quad~\boldsymbol q\in {\rm ND}_{k-2}(F),  F\in\Delta_{d-1}(T),\label{HdivSBDMfemdof3} \\
(\boldsymbol \tau, \boldsymbol q)_T, &\quad~\boldsymbol q\in \mathbb P_{k-2}(T;\mathbb S)
%\resizebox{8.45cm}{!}{$\mathbb P_{k-2}(K;\mathbb S)=\defm \mathbb P_{k-1}(K;\mathbb R^d) \oplus (\ker (\cdot \boldsymbol x)\cap \mathbb P_{k-2}(K;\mathbb S))$} 
\label{HdivSBDMfemdof4}
\end{align}
\end{subequations}
对于 $\mathbb P_k(T;\mathbb S)$ 是唯一可解的. 
 最后一个自由度 \eqref{HdivSBDMfemdof4} 可以替换为
 \begin{align}
 (\div\boldsymbol \tau, \boldsymbol q)_T, &\quad~\boldsymbol q\in \mathbb P_{k-1}(T;\mathbb R^d)/\textrm{RM}, \label{HdivSdef}\\
 (\boldsymbol \tau, \boldsymbol q)_T, &\quad~\boldsymbol q\in \ker (\cdot\boldsymbol x)\cap \mathbb P_{k-2}(T;\mathbb S) \label{HdivSker}.
 \end{align} 
\end{theorem}
全局有限元空间 $\boldsymbol V_h(\div;\mathbb S)\subset H(\div, \Omega; \mathbb S)$ 定义为
\begin{align*}
\boldsymbol V_h(\div;\mathbb S):=\{\boldsymbol \tau\in L^2(\Omega;\mathbb S): &\,\boldsymbol \tau|_T\in\mathbb P_k(T;\mathbb S) \textrm{ 对每个 } T\in\mathcal T_h, \\
&\textrm{自由度 \eqref{HdivSBDMfemdof1}-\eqref{HdivSBDMfemdof3} 是单值的} \}.    
\end{align*}
显然,由 引理~\ref{lem:divSboundarydofs} 的证明可知 $\boldsymbol V_h(\div;\mathbb S)\subset H(\div, \Omega; \mathbb S)$. 

%Especially, the degrees of freedom \eqref{HdivSBDMfemdof1}-\eqref{HdivSBDMfemdof4} are
%\begin{align*}
%\boldsymbol \tau (\delta) & \quad\forall~\delta\in \mathcal V(K), \\
%(\boldsymbol  n^{\intercal}\boldsymbol \tau\boldsymbol n, q)_e & \quad\forall~q\in\mathbb P_{k-2}(e),  e\in\Delta_{d-1}(T), \\
%(\boldsymbol t^{\intercal}\boldsymbol \tau\boldsymbol n, \boldsymbol q)_e & \quad\forall~\boldsymbol q\in \mathbb P_{k-2}(e),  e\in\Delta_{d-1}(T), \\
%(\boldsymbol \tau, \boldsymbol q)_K &\quad \forall~\boldsymbol q\in \mathbb P_{k-2}(K;\mathbb S)
%\end{align*}
%in two dimensions, and
对于最重要的三维情形,自由度 \eqref{HdivSBDMfemdof} 变为
\begin{align*}
\boldsymbol \tau (\delta), & \quad~\delta\in \Delta_{0}(T), \\
(\boldsymbol  n_i^{\intercal}\boldsymbol \tau\boldsymbol n_j, q)_e, & \quad~q\in\mathbb P_{k-2}(e),  e\in\Delta_1(T), i,j=1,2,\\
(\boldsymbol  n^{\intercal}\boldsymbol \tau\boldsymbol n, q)_F, & \quad~q\in\mathbb P_{k-3}(F),  F\in\Delta_2(T),\\
(\Pi_F\boldsymbol \tau\boldsymbol n, \boldsymbol q)_F, & \quad~\boldsymbol q\in {\rm ND}_{k-2}(F),  F\in\Delta_2(T), \\
(\boldsymbol \tau, \boldsymbol q)_T, &\quad~\boldsymbol q\in \mathbb P_{k-2}(T;\mathbb S),
\end{align*}
这与三维 Hu-Zhang 单元~\cite{HuZhang2015} 不同. 

唯一可解性对于 \(k\geq 1\) 均成立. 然而,条件 \(k\geq d+1\) 确保了包含每个面 \(F\in\Delta_2(T)\) 上的自由度 \((\boldsymbol \tau\boldsymbol n, \boldsymbol q)_F\)(对于所有 \(\boldsymbol q\in \mathbb P_{1}(F; \mathbb R^{d-1})\)). 由此可知,全局 \(H(\div; \mathbb S)\)-协调有限元空间的散度包含分片刚体运动(\(\textrm{RM}\))空间. 结合 \(\div\mathbb B_k(\div, T; \mathbb S)=\mathbb P_{k-1,{\rm RM}}^{\bot}\),可推导出如下离散 inf-sup 条件. 


\begin{lemma}\label{lm:divSinfsup}
设 $k\geq d+1$. 则如下 inf-sup 条件成立:
$$
\|\boldsymbol p_h\|\lesssim\sup_{\boldsymbol\tau_h\in \boldsymbol V_h(\div;\mathbb S)}\frac{(\div\boldsymbol\tau_h, \boldsymbol p_h)}{\|\boldsymbol\tau_h\|_{H(\div)}}\qquad\forall~\boldsymbol p_h\in\mathbb P_{k-1}(\mathcal T_h; \mathbb{R}^d),
$$
其中 $\mathbb P_{k-1}(\mathcal T_h; \mathbb{R}^d):=\{\boldsymbol p_h\in L^2(\Omega;\mathbb{R}^d): \boldsymbol p_h|_K\in\mathbb P_{k-1}(T; \mathbb{R}^d), \forall\, T\in\mathcal T_h\}$. 
\end{lemma}
\begin{proof}
任取 $\boldsymbol p_h\in\mathbb P_{k-1}(\mathcal T_h; \mathbb{R}^d)$,存在 $\boldsymbol\tau\in H^1(\Omega;\mathbb S)$ 使得~\cite{CostabelMcIntosh2010}
$$
\div\boldsymbol\tau=\boldsymbol p_h,\quad \|\boldsymbol\tau\|_1\lesssim \|\boldsymbol p_h\|.
$$
定义 \(\boldsymbol\tau_h\in \boldsymbol V_h(\div;\mathbb S)\),使其满足如下条件:
\begin{align*}
(\boldsymbol n^{\intercal}\boldsymbol\tau_h\boldsymbol n, q)_F &= (\boldsymbol n^{\intercal}\boldsymbol\tau\boldsymbol n, q)_F, && \forall~q\in\mathbb P_1(F), F\in\Delta_2(T),\\
(\Pi_F\boldsymbol\tau_h\boldsymbol n, \boldsymbol q)_F &= (\Pi_F\boldsymbol\tau\boldsymbol n, \boldsymbol q)_F, && \forall~\boldsymbol q\in\mathbb P_1(F;\mathbb R^{d-1}), F\in\Delta_2(T),\\
(\div\boldsymbol\tau_h, \boldsymbol q)_T &= (\div\boldsymbol\tau, \boldsymbol q)_T = (\boldsymbol p_h, \boldsymbol q)_T, && \forall~\boldsymbol q\in\mathbb P_{k-1}(T;\mathbb R^d)/\textrm{RM},
\end{align*}
并且令其在 \(T\in\mathcal T_h\) 上其余的自由度 \eqref{HdivSBDMfemdof1}-\eqref{HdivSBDMfemdof3} 和 \eqref{HdivSdef}-\eqref{HdivSker} 均为零. 利用尺度变换论证,可得
\begin{equation}\label{eq:20220120}  
\|\boldsymbol\tau_h\|_{0}\lesssim \|\boldsymbol\tau\|_{1}\lesssim \|\boldsymbol p_h\|.
\end{equation}
应用分部积分公式,有
$$
(\div\boldsymbol\tau_h, \boldsymbol q)_T=(\div\boldsymbol\tau, \boldsymbol q)_T=(\boldsymbol p_h, \boldsymbol q)_T\quad\forall~\boldsymbol q\in\textrm{RM}.
$$
因此
$$
(\div\boldsymbol\tau_h, \boldsymbol q)_T=(\div\boldsymbol\tau, \boldsymbol q)_T=(\boldsymbol p_h, \boldsymbol q)_T\quad\forall~\boldsymbol q\in\mathbb P_{k-1}(T;\mathbb R^d),
$$
这蕴含了 \(\div\boldsymbol\tau_h=\boldsymbol p_h\). 结合 \eqref{eq:20220120},即可推导出所述 inf-sup 条件. 
\end{proof}


\section{面向 $\div\div$ 算子的对称张量有限元}

这一节构造法向-法向连续的对称张量有限元和 $H(\div\div)$-协调的对称张量有限元.

\subsection{法向-法向连续的对称张量有限元}

首先探讨法向-法向连续的对称张量有限元空间,用于离散空间 $H^{-1}(\div{\div },\Omega;\mathbb{S})$. 
% 文献 \cite{PechsteinSchoberl2011} 中已经给出了一种构造方法. 在此,我们给出一种不同的构造,该方法利用内蕴基,从而避免了到参考单元的复杂变换. 

设 $\boldsymbol{t}_{i,j}$ 为从顶点 $\texttt{v}_i$ 指向顶点 $\texttt{v}_j$ 的向量,即 $\boldsymbol{t}_{i,j}= \texttt{v}_j - \texttt{v}_i$. 
我们将形函数空间选取为 $\mathbb P_k(T; \mathbb S) := \mathbb P_k(T) \otimes \mathbb S$. 自由度定义如下:
\begin{subequations}\label{eq:hhjnd}
 \begin{align}
\int_F \boldsymbol{n}^{\intercal}\boldsymbol{\tau} \boldsymbol{n}\, q \, \mathrm{d} S, &\quad q\in \mathbb P_k(F), F\in \Delta_{d-1}(T),\label{eq:dofFnn}\\
\int_T \boldsymbol{\tau}: \boldsymbol{q} \, \mathrm{d}x, &\quad \boldsymbol{q}\in  \mathbb P_{k-1}(T; \mathbb S), \label{eq:dofTk-1}\\
\int_{F_i} \boldsymbol{n}^{\intercal}\boldsymbol{\tau}\boldsymbol{t}_{0,j}\, q \, \mathrm{d} S, &\quad q\in \mathbb P_k(F_i), i=1,\ldots, d-2, j=i+1, \ldots, d. \label{eq:dofFnt}
\end{align}
\end{subequations}
这里,$F_i\in\Delta_{d-1}(T)$ 是与顶点 $\texttt{v}_i$ 相对的面.
自由度 \eqref{eq:dofFnt} 被视为单纯形 $T$ 的内部自由度,即如果在两个相邻单元中选取该自由度,其值将是双值的(非连续的). 

\begin{lemma}\label{lem:Sbasis}
我们有如下分解:
\begin{equation}\label{eq:Sbasis}
\mathbb{S}=\bigoplus_{i=0}^{d}\operatorname{span}\{\boldsymbol{n}_i\boldsymbol{n}_i^{\intercal}\} \oplus \bigoplus_{i=1}^{d-2}\bigoplus_{j=i+1}^{d}\operatorname{span}\{\operatorname{sym}(\boldsymbol{n}_i\boldsymbol{t}_{0,j}^{\intercal})\}.
\end{equation}
\end{lemma}
\begin{proof}
式 \eqref{eq:Sbasis} 右端张量的个数为
$$
d+1 + \sum_{i=1}^{d-2}(d - i) = \frac{1}{2}d(d+1) = \dim\mathbb S.
$$
故我们只需证明 \eqref{eq:Sbasis} 右端的张量是线性无关的. 
假设
\begin{equation}\label{eq:Sbasisproof1}  
\sum_{i=0}^{d}C_{ii}\boldsymbol{n}_i\boldsymbol{n}_i^{\intercal} + \sum_{i=1}^{d-2}\sum_{j=i+1}^{d}C_{ij}\operatorname{sym}(\boldsymbol{n}_i\boldsymbol{t}_{0,j}^{\intercal})=\boldsymbol{0},
\end{equation}
其中 $C_{ij}\in\mathbb R$. 将 \eqref{eq:Sbasisproof1} 式乘以 $\boldsymbol{t}_{0, d}$,可得
$$
C_{00}\boldsymbol{n}_0(\boldsymbol{n}_0^{\intercal}\boldsymbol{t}_{0, d}) + C_{dd}\boldsymbol{n}_d(\boldsymbol{n}_d^{\intercal}\boldsymbol{t}_{0, d}) + \frac{1}{2}\sum_{i=1}^{d-2}\sum_{j=i+1}^{d}C_{ij}\boldsymbol{n}_i(\boldsymbol{t}_{0,j}^{\intercal}\boldsymbol{t}_{0,d})=\boldsymbol{0}.
$$
注意到 $\{\boldsymbol{n}_0, \boldsymbol{n}_1, \ldots, \boldsymbol{n}_{d-2}, \boldsymbol{n}_{d}\}$ 构成了 $\mathbb R^d$ 的一组基,因此我们得到 $C_{dd}=C_{00}=0$. 
再将 \eqref{eq:Sbasisproof1} 式乘以 $\boldsymbol{t}_{0, d-1}$,可得
$$
C_{d-1,d-1}\boldsymbol{n}_{d-1}(\boldsymbol{n}_{d-1}^{\intercal}\boldsymbol{t}_{0,d-1}) + \frac{1}{2}\sum_{i=1}^{d-2}\sum_{j=i+1}^{d}C_{ij}\boldsymbol{n}_i(\boldsymbol{t}_{0,j}^{\intercal}\boldsymbol{t}_{0,d-1})=\boldsymbol{0}.
$$
从而 $C_{d-1,d-1}=0$. 

现在开始关于 $i$ 从 $d-2$ 到 $1$ 进行数学归纳.
先考虑 $i=d-2$ 的情形. 将 \eqref{eq:Sbasisproof1} 式乘以 $\boldsymbol{t}_{0, j}$,其中 $j=d-1, d$,考察 $\boldsymbol{n}_{d-2}$ 的系数可得
$$
(C_{d-2,d-1}\boldsymbol{t}_{0,d-1}+C_{d-2,d}\boldsymbol{t}_{0,d})^{\intercal}(\boldsymbol{t}_{0,d-1},\boldsymbol{t}_{0,d})=0,
$$
这意味着 $C_{d-2,d-1}=C_{d-2,d}=0$. 再将 \eqref{eq:Sbasisproof1} 式乘以 $\boldsymbol{t}_{0, d-2}$,可得
$$
C_{d-2,d-2}\boldsymbol{n}_{d-2}(\boldsymbol{n}_{d-2}^{\intercal}\boldsymbol{t}_{0,d-2}) + \frac{1}{2}\sum_{i=1}^{d-3}\sum_{j=i+1}^{d}C_{ij}\boldsymbol{n}_i(\boldsymbol{t}_{0,j}^{\intercal}\boldsymbol{t}_{0,d-2})=\boldsymbol{0}.
$$
从而 $C_{d-2,d-2}=0$. 

让 $i$ 从 $d-3$ 递减到 $1$,重复此过程:通过乘以 $\boldsymbol{t}_{0, j}$ ($j=i+1,\ldots, d$),我们可以得到 $C_{i,i+1}=C_{i,i+2}=\ldots=C_{i,d}=0$. 再乘以 $\boldsymbol{t}_{0, i}$ 可得 $C_{i, i}=0$. 最终,我们得出对于所有 $i=1,\ldots, d-2$ 和 $j=i,i+1,\ldots, d$,都有 $C_{ij}=0$. 
\end{proof}

当 $d=2$ 时,我们有
$$
\mathbb{S}=\operatorname{span}\{\boldsymbol{n}_0\boldsymbol{n}_0^{\intercal}, \boldsymbol{n}_1\boldsymbol{n}_1^{\intercal}, \boldsymbol{n}_2\boldsymbol{n}_2^{\intercal}\}.
$$

\begin{lemma}
自由度 \eqref{eq:hhjnd} 对于空间 $\mathbb P_k(T; \mathbb S)$ 是唯一可解的. 
\end{lemma}

\begin{proof}
首先验证维数的一致性. 注意到多项式空间的维数满足恒等式
$$
\dim \mathbb P_k(T) = \dim \mathbb P_{k-1}(T) + \dim \mathbb P_k(F).
$$
自由度 \eqref{eq:dofFnn} 和 \eqref{eq:dofFnt} 的总数为 $\binom{d+1}{2}\times \dim \mathbb P_k(F)$. 因此,总自由度个数为
$$
(\dim \mathbb P_{k-1}(T) + \dim \mathbb P_k(F))\times \binom{d+1}{2} = \dim (\mathbb P_k(T) \otimes \mathbb S).
$$
这与空间 $\mathbb P_k(T; \mathbb S)$ 的维数相符. 

接下来证明线性无关性. 任取 $\boldsymbol{\tau} \in \mathbb P_k(T; \mathbb S)$,假设其所有自由度 \eqref{eq:hhjnd} 均取零值,只需证明 $\boldsymbol{\tau} = \boldsymbol{0}$. 

设 $\{\boldsymbol{N}_m\}_{m=1}^{\dim \mathbb S}$ 为引理 \ref{lem:Sbasis} 中 $\mathbb S$ 的基底,并记 $\{\boldsymbol{N}_m'\}_{m=1}^{\dim \mathbb S}$ 为其对偶基 (即满足 $\boldsymbol{N}_l : \boldsymbol{N}_m' = \delta_{l, m}$). 我们可以将 $\boldsymbol{\tau}$ 展开为:
$$
\boldsymbol{\tau} = \sum_{m=1}^{\dim \mathbb S} c_m(\boldsymbol{x}) \boldsymbol{N}_m', \quad \text{其中 }\; c_m(\boldsymbol{x}) = \boldsymbol{\tau}:\boldsymbol{N}_m \in \mathbb P_k(T).
$$
根据自由度 \eqref{eq:dofFnn} 和 \eqref{eq:dofFnt} 的定义以及引理 \ref{lem:Sbasis} 的构造,每一个基底张量 $\boldsymbol{N}_m$ 都唯一关联一个特定的面 $F_{j_m} \in \Delta_{d-1}(T)$,使得对应的边界自由度决定系数 $c_m$ 在该面上的迹. 由于边界自由度为零,必有 
$$
c_m(\boldsymbol{x}) \big|_{F_{j_m}} = 0.
$$
这意味着 $c_m(\boldsymbol{x})$ 含有对应面 $F_{j_m}$ 的重心坐标 $\lambda_{j_m}$ 作为因子,即
$$
c_m(\boldsymbol{x}) = \lambda_{j_m}(\boldsymbol{x}) \, q_m(\boldsymbol{x}), \quad \text{其中 }\; q_m \in \mathbb P_{k-1}(T).
$$
此时 $\boldsymbol{\tau}$ 可表示为 $\boldsymbol{\tau} = \sum_{m} \lambda_{j_m} q_m \boldsymbol{N}_m'$. 

利用内部自由度 \eqref{eq:dofTk-1},我们针对每个 $m$ 选取测试函数 $\boldsymbol{q} = q_m \boldsymbol{N}_m \in \mathbb P_{k-1}(T; \mathbb S)$. 由自由度取值为零可得:
$$
0 = \int_T \boldsymbol{\tau} : (q_m \boldsymbol{N}_m) \, \mathrm{d}x 
= \int_T \left( \sum_{l} \lambda_{j_l} q_l \boldsymbol{N}_l' \right) : (q_m \boldsymbol{N}_m) \, \mathrm{d}x 
= \int_T \lambda_{j_m} q_m^2 \, \mathrm{d}x.
$$
这里利用了对偶基的性质 $\boldsymbol{N}_l' : \boldsymbol{N}_m = \delta_{l,m}$ 消去了求和符号. 由于重心坐标 $\lambda_{j_m}$ 在单元 $T$ 内部恒大于零,上述积分意味着 $q_m \equiv 0$. 

因为 $m$ 是任意的,故所有系数 $c_m$ 均为零,从而 $\boldsymbol{\tau} = \boldsymbol{0}$. 
\end{proof}


引入泡函数空间
$$
\mathbb B_k^{nn}(T;\mathbb S):=\{\boldsymbol{\tau}\in\mathbb{P}_k(T;\mathbb S): \text{自由度 \eqref{eq:dofFnn} 取零值}\}.
$$
自由度 \eqref{eq:hhjnd} 可以替换为:
\begin{subequations}\label{eq:hhjndnew}
 \begin{align}
\int_F \boldsymbol{n}^{\intercal}\boldsymbol{\tau} \boldsymbol{n}\, q \, \mathrm{d} S, &\quad q\in \mathbb P_k(F), F\in \Delta_{d-1}(T),\label{eq:divdivSdofF}\\
\int_T \boldsymbol{\tau}: \boldsymbol{q} \, \mathrm{d}x, &\quad \boldsymbol{q}\in\mathbb B_k^{nn}(T;\mathbb S). \label{eq:divdivSdofT}
\end{align}
\end{subequations}

\subsection{$H(\div\div)\cap H(\div)$ 协调对称张量有限元}
本节构造 $H(\div\div)\cap H(\div)$ 协调对称张量有限元 \cite[Sections 5.2-5.3]{ChenHuang2022}. 

%Consider the sequence
%\begin{equation}
%H(\operatorname{div}, \mathbb{S})\cap H(\operatorname{div}, \div, \mathbb{S}) \stackrel{\text { div }}{\longrightarrow} H(\operatorname{div}, \mathbb{R}^d) \stackrel{\text { div }}{\longrightarrow} L^{2}
%\end{equation}


\begin{lemma}\label{lem:divdivonto}
设 $k\geq 1$ 为整数,则算子
\begin{equation*}
\div\div:  \boldsymbol x\boldsymbol x^{\intercal}\mathbb H_{k-1}(D) \to \mathbb H_{k-1}(D)
\end{equation*}
是双射. 由此可知,映射 $\div\div: \mathbb P_{k+1}(D;\mathbb S)\to \mathbb P_{k-1}(D)$ 是满射. 
\end{lemma}

\begin{proof}
根据 \eqref{eq:Hkdiv},对于任意 $q\in\mathbb H_{k-1}(D)$,我们有
$$
\div\div(\boldsymbol x\boldsymbol x^{\intercal}q)=\div((k+d)\boldsymbol x q)=(k+d)(k+d-1)q,
$$
证毕. 
\end{proof}

基于格林公式 \eqref{eq:greenidentitydivdiv},只需强加 $\boldsymbol \tau \boldsymbol n$ 和 $\boldsymbol n^{\intercal}\div \boldsymbol \tau$ 的连续性,即可使构造的有限元空间满足 $H(\operatorname{div}, \mathbb{S})\cap H(\operatorname{div}\div, \mathbb{S})$-协调性. 文献~\cite{HuMaZhang2021} 提出了此类方法,用于构造二维和三维的 $H(\operatorname{div}, \mathbb{S})\cap H(\operatorname{div}\div, \mathbb{S})$-协调有限元. 
% 关于该空间分解的示意图,请参见图 \ref{fig:divdivcomp}. 
% \begin{figure}[htbp]
% \begin{center}
% \includegraphics[width=10.5cm]{figures/divdivanddivconforming.pdf}
% \caption{$H(\div\div)\cap H(\div)$ 协调有限元的 $\mathbb P_k(T;\mathbb S)$ 空间分解. }
% \label{fig:divdivcomp}
% \end{center}
% \end{figure}

子空间 $\tr^{\div}(\mathbb P_k(T;\mathbb S))$ 和 $E_0(\mathbb S)$ 保持不变. 空间 $\div E_0^{\bot}(\mathbb S) = \mathbb P_{k-1,\mathrm{RM}}^{\bot}$ 将通过迹算子被进一步分解. 定义
$$
F_0(\mathbb S)\subseteq E_0^{\bot}(\mathbb S), \quad \text{满足} \quad \div F_0(\mathbb S) = \mathbb B_{k-1}(\div, T)\cap \textrm{RM}^{\bot},
$$ 
以及
$F_{\tr}(\mathbb S)\subseteq E_0^{\bot}(\mathbb S)$ 满足 ${\rm tr}^{\div}(\div F_{\tr}(\mathbb S)) = {\rm tr}^{\div}(\div E_0^{\bot}) = {\rm tr}^{\div}(\mathbb P_{k-1,\mathrm{RM}}^{\bot})$. 
由于 $\div$ 算子限制在 $E_0^{\bot}(\mathbb S)$ 上是双射,上述定义是良定的. 这里我们定义
$$
\mathbb B_{k-1}(\div, T)\cap \textrm{RM}^{\bot}:=\{\boldsymbol v\in\mathbb B_{k-1}(\div, T): (\boldsymbol v, \boldsymbol q)_T=0, \;\forall\,\boldsymbol q\in\textrm{RM}\}.
$$

%Since $\div \mathbb P_k(T;\mathbb S)=\mathbb P_{k-1}(K,\mathbb R^d)$, we have
%$$
%{\rm tr}^{\div}(\div \mathbb P_k(T;\mathbb S)) = {\rm tr}^{\div}(\mathbb P_{k-1}(K,\mathbb R^d)).
%$$ 

\begin{lemma}\label{lm:Ftr}
对于整数 $k\geq 3$,成立
 $$
 {\rm tr}^{\div}(\div F_{\tr}(\mathbb S)) 
% =  {\rm tr}^{\div}(\div E_0^{\bot}(\mathbb S)) 
   = {\rm tr}^{\div}(\mathbb P_{k-1}(T;\mathbb R^d)).
 $$ 
 因此,$({\rm tr}^{\div}(\div F_{\tr}(\mathbb S)))'=\mathcal N( \mathbb P_{k-1}(\Delta_{d-1}(T)))$.
% \mnote{ give the form.}
\end{lemma}
\begin{proof}
根据定义,$ {\rm tr}^{\div}(\div F_{\tr}(\mathbb S)) = {\rm tr}^{\div}(\mathbb P_{k-1,\mathrm{RM}}^{\bot})\subseteq{\rm tr}^{\div}(\mathbb P_{k-1}(T;\mathbb R^d))$. 
另一方面,给定一个迹 $p\in {\rm tr}^{\div}(\mathbb P_{k-1}(T;\mathbb R^d))$,由 BDM 元的唯一可解性,我们可以找到一个 $\boldsymbol v\in \mathbb P_{k-1}(T;\mathbb R^d)$ 使得在 $\partial T$ 上 $\boldsymbol v\cdot\boldsymbol n = p$,且 $\boldsymbol v \perp \textrm{RM}$. 这是因为当 $k\geq 3$ 时,$\textrm{RM}={\rm ND}_{0}(T)\subseteq {\rm ND}_{k-3}(T)$. 
\end{proof}

\begin{lemma}\label{lm:F0star}
对于整数 $k\geq 3$,我们有
$$
F_0'(\mathbb S) = \mathcal N(\defm ({\rm ND}_{k-3}(T))).
$$
\end{lemma}
\begin{proof}
任取 $\boldsymbol \tau \in F_0(\mathbb S)$,即 $\boldsymbol \tau$ 满足
$$
(\boldsymbol \tau\boldsymbol n)|_{\partial T}= 0, \quad \boldsymbol n^{\intercal} \div \boldsymbol \tau |_{\partial T} = 0, \quad \boldsymbol \tau \perp  E_0(\mathbb S).
$$ 
假设
$$
(\boldsymbol \tau, \defm \boldsymbol q)_T = 0, \quad \forall~\boldsymbol q\in {\rm ND}_{k-3}(T).
$$
%then $\boldsymbol \tau \in $. 
注意到 $\boldsymbol v =\div \boldsymbol \tau \in \mathbb B_{k-1}(\div,T)$,且对于所有 $\boldsymbol q\in {\rm ND}_{k-3}(T)$ 均有 $(\boldsymbol v, \boldsymbol q)_T = 0$,则由定理 \ref{thm:BDMunisolvence} 可知 $\boldsymbol v = \boldsymbol 0$. 因此 $\div \boldsymbol \tau = \boldsymbol 0$,即 $\boldsymbol \tau\in E_0(\mathbb S)$. 由于 $\boldsymbol \tau \perp  E_0(\mathbb S)$,故 $\boldsymbol \tau = \boldsymbol 0$. 

最后,通过维数计数:
$$
\dim F_0(\mathbb S) = \dim \mathbb B_{k-1}(\div, T) - \dim \textrm{RM} = \dim {\rm ND}_{k-3}(T) - \dim \ker(\defm),
$$
即可完成证明. 
\end{proof}

我们将构造总结为以下定理. 
\begin{theorem}\label{th:divdivcapdiv}
令 $V(\div\div^+;\mathbb S) := \mathbb P_k(T;\mathbb S)$,其中 $k\geq \max\{d,3\}$. 则以下自由度集合确定了一个 $H(\div\div; \mathbb S)\cap H(\div; \mathbb S)$-协调有限元:
\begin{subequations}\label{HdivdivplusSfemdof}
\begin{align}
\boldsymbol \tau (\delta), & \quad~\delta\in \Delta_{0}(T), \label{HdivdivplusSfemdof1}\\
(\boldsymbol n_i^{\intercal}\boldsymbol \tau\boldsymbol n_j, q)_f, & \quad~q\in\mathbb P_{k-\ell-1}(f),  f\in\Delta_{\ell}(T),\;  \label{HdivdivplusSfemdof2}\\
&\quad\quad i,j=1,\ldots, d-\ell, \textrm{ 和 }\; \ell=1,\ldots, d-1, \notag\\
(\Pi_F\boldsymbol \tau\boldsymbol n, \boldsymbol q)_F, & \quad~\boldsymbol q\in {\rm ND}_{k-2}(F),  F\in\Delta_{d-1}(T),\label{HdivdivplusSfemdof3}\\
(\boldsymbol n^{\intercal}\div\boldsymbol\tau, q)_F, &\quad~q\in\mathbb P_{k-1}(F), \, F\in \Delta_{d-1}(T),\label{HdivdivplusSfemdof4}\\
(\div\div\boldsymbol \tau, q)_T, &\quad~q\in \mathbb P_{k-2}(T)/\mathbb P_{1}(T),\label{HdivdivplusSfemdof51}\\
(\div\boldsymbol \tau, \boldsymbol q)_T, &\quad~\boldsymbol q\in \big(\mathbb P_{k-3}(T; \mathbb K)/\mathbb P_{0}(T; \mathbb K)\big)\boldsymbol x,\label{HdivdivplusSfemdof52}\\
(\boldsymbol \tau, \boldsymbol q)_T, &\quad~\boldsymbol q\in  \ker (\cdot\boldsymbol x)\cap \mathbb P_{k-2}(T;\mathbb S). \label{HdivdivplusSfemdof6}
\end{align}
\end{subequations}
\end{theorem}
\begin{proof}
由引理~\ref{lem:divSboundarydofs},自由度 \eqref{HdivdivplusSfemdof1}-\eqref{HdivdivplusSfemdof3} 为零意味着 $\boldsymbol \tau \boldsymbol n |_{\partial T}= \boldsymbol0$. 接着应用引理~\ref{lm:Ftr}-\ref{lm:F0star},
由自由度 \eqref{HdivdivplusSfemdof4}-\eqref{HdivdivplusSfemdof52} 为零可得 $\boldsymbol \tau \in E_0(\mathbb S)$. 
最后,结合 \eqref{eq:E0Sdualcharac} 和 \eqref{HdivdivplusSfemdof6} 可知 $\boldsymbol \tau = \boldsymbol0$. 

接下来进行维数计数. 与 BDM 型 $H(\div,\mathbb S)$ 元(参见定理 \ref{thm:SBDMunisolvence})的自由度相比,区别在于 \eqref{HdivSdef} 与 \eqref{HdivdivplusSfemdof4}-\eqref{HdivdivplusSfemdof52}. 
根据 BDM $H(\div)$-协调元的唯一可解性,我们有
$$
\dim \mathbb P_{k-1}(T; \mathbb R^d) = \dim {\rm ND}_{k-3}(T) + \sum_{F\in \Delta_{d-1}(T)}\dim\mathbb P_{k-1}(F).
$$
因此,自由度 \eqref{HdivdivplusSfemdof1}-\eqref{HdivdivplusSfemdof6} 的总数等于 $\dim\mathbb P_k(T;\mathbb S)$. 
%Due to Theorem \ref{thm:SBDMunisolvence}, the number of degrees of freedom \eqref{HdivdivSfemdof1}-\eqref{HdivdivSfemdof6} is
%\begin{align*}
%&\quad\dim\mathbb P_k(K;\mathbb S)-\dim\mathbb H_{k-1}(K;\mathbb R^d)-\dim\mathbb H_{k-1}(K) + \sum_{F\in \Delta_{d-1}(T)}\dim\mathbb P_{k-1}(F) \\
%&=\dim\mathbb P_k(K;\mathbb S) - (d+1){ k + d - 2 \choose k-1}+(d+1){ k + d - 2 \choose k-1}=\dim\mathbb P_k(K;\mathbb S),
%\end{align*}
%as required.
% Then dimension count which will match as $\div \mathbb P_{k}(K;\mathbb S) = \mathbb P_{k-1,{\rm RM}}^{\bot}$ and $\defm ND$ removes the RM. 
\end{proof}

全局有限元空间 $\boldsymbol V_h(\div\div^+;\mathbb S)\subset H(\div\div, \Omega; \mathbb S)\cap H(\div, \Omega; \mathbb S)$ 定义如下:
\begin{align*}
\boldsymbol V_h(\div\div^+;\mathbb S):=\{&\boldsymbol \tau\in L^2(\Omega;\mathbb S): \boldsymbol \tau|_T\in \mathbb P_k(T;\mathbb S), \, \forall\,T\in\mathcal T_h; \textrm{自由度 \eqref{HdivdivplusSfemdof1}-\eqref{HdivdivplusSfemdof4} 是单值的} \}.    
\end{align*}

条件 $k\geq d$ 确保了每个面 $F\in\Delta_{d-1}(T)$ 上的自由度 $(\boldsymbol n^{\intercal}\boldsymbol \tau\boldsymbol n, q)_F$ ($\forall q\in \mathbb P_{0}(F)$) 被包含在内,
由此 $\div\div\boldsymbol V_h(\div\div^+;\mathbb S)$ 空间将包含所有分片线性函数. 

\begin{lemma}\label{lm:divdivdivSinfsup}
设 $k\geq \max\{d,3\}$. 则成立如下 inf-sup 条件:
$$
\|p_h\|\lesssim\sup_{\boldsymbol\tau_h\in \boldsymbol V_h(\div\div^+;\mathbb S)}
\frac{(\div\div\boldsymbol{\tau}_h, p_h)}{\|\boldsymbol\tau_h\|_{H(\div)}+
\|\div\div\boldsymbol{\tau}_h\|}\qquad\forall~p_h\in\mathbb P_{k-2}(\mathcal T_h),
$$
其中 $\mathbb P_{k-2}(\mathcal T_h):=\{p_h\in L^2(\Omega): p_h|_T\in\mathbb P_{k-2}(T), \, \forall\,T\in\mathcal T_h\}$. 
\end{lemma}
\begin{proof}
对于 $p_h\in\mathbb P_{k-2}(\mathcal T_h)$,存在 $\boldsymbol\tau\in H^2(\Omega;\mathbb S)$ 使得~\cite{CostabelMcIntosh2010}
$$
\div\boldsymbol\tau=p_h,\quad \|\boldsymbol\tau\|_2\lesssim \|p_h\|.
$$
令 $\boldsymbol\tau_h\in \boldsymbol V_h(\div\div^+;\mathbb S)$,使得除了以下自由度外,其余所有自由度 \eqref{HdivdivplusSfemdof1}-\eqref{HdivdivplusSfemdof6} 均为零:
\begin{align*}
(\boldsymbol n^{\intercal}\boldsymbol\tau_h\boldsymbol n, q)_F &= (\boldsymbol n^{\intercal}\boldsymbol\tau\boldsymbol n, q)_F, & \forall~q\in\mathbb P_0(F), \, F\in\Delta_{d-1}(T),\\
(\Pi_F\boldsymbol\tau_h\boldsymbol n, \boldsymbol q)_F &= (\Pi_F\boldsymbol\tau\boldsymbol n, \boldsymbol q)_F, & \forall~\boldsymbol q\in\mathbb P_0(F; \mathbb R^{d-1}), \, F\in\Delta_{d-1}(T),\\
(\boldsymbol n^{\intercal}\div\boldsymbol\tau_h, q)_F &= (\boldsymbol n^{\intercal}\div\boldsymbol\tau, q)_F, & \forall~q\in\mathbb P_1(F;\mathbb R^d), \, F\in\Delta_{d-1}(T), \\
(\div\div\boldsymbol\tau_h, q)_T &= (\div\div\boldsymbol\tau, q)_T=(p_h, q)_T, & \forall~q\in \mathbb P_{k-2}(T)/\mathbb P_{1}(T).
\end{align*}
通过尺度论证,我们有
\begin{equation}\label{eq:202201201}  
\|\boldsymbol\tau_h\|_{H(\div)}\lesssim \|\boldsymbol\tau\|_{2}\lesssim \|p_h\|.
\end{equation}
应用分部积分可得,
$$
(\div\div\boldsymbol\tau_h, q)_T=(\div\div\boldsymbol\tau, q)_T=(p_h, q)_T, \quad \forall~q\in\mathbb P_1(T).
$$
因此
$$
(\div\div\boldsymbol\tau_h, q)_T=(\div\div\boldsymbol\tau, q)_T=(p_h, q)_T, \quad \forall~q\in\mathbb P_{k-2}(T),
$$
这意味着 $\div\div\boldsymbol\tau_h=p_h$. 据此,结合 \eqref{eq:202201201} 即可导出 inf-sup 条件. 
\end{proof}



进一步,构造 $\mathbb P_{k}^{+}(\mathbb S)$ 型 $H(\div\div; \mathbb S)\cap H(\div; \mathbb S)$ 协调有限元.

形状函数空间定义为
\begin{equation*}
V^{+}(\div\div^+;\mathbb S):=\mathbb P_k(T;\mathbb S) \oplus \boldsymbol x\boldsymbol x^{\intercal}\mathbb H_{k-1}(T),
\end{equation*}
其中 $k\geq \max\{d,3\}$. 
自由度选取如下:
% \begin{align}
% \boldsymbol \tau (\delta) & \quad\forall~\delta\in \mathcal V(K), \label{HdivdivSRTfemdof1}\\
% (\boldsymbol  n_i^{\intercal}\boldsymbol \tau\boldsymbol n_j, q)_f & \quad\forall~q\in\mathbb P_{k+r-d-1}(f),  f\in\mathcal F^r(K),\;  \label{HdivdivSRTfemdof2}\\
% &\quad\quad i,j=1,\cdots, r, \textrm{ 且 } r=1,\cdots, d-1, \notag\\
% (\Pi_F\boldsymbol \tau\boldsymbol n, \boldsymbol q)_F & \quad\forall~\boldsymbol q\in {\rm ND}_{k-2}(F),  F\in\mathcal F^{1}(K),\label{HdivdivSRTfemdof3}\\
% (\boldsymbol  n^{\intercal}\div \boldsymbol \tau, p)_F &\quad\forall~p\in\mathbb P_{k-1}(F), F\in \mathcal F^{1}(K),\label{HdivdivSRTfemdof4}\\
% (\div\div\boldsymbol \tau, q)_K &\quad \forall~q\in \mathbb P_{k-1}(K)/\mathbb P_{1}(K),\label{HdivdivSRTfemdof51}\\
% (\div\boldsymbol \tau, \boldsymbol q)_K &\quad \forall~\boldsymbol q\in \big(\mathbb P_{k-3}(K; \mathbb K)/\mathbb P_{0}(K; \mathbb K)\big)\boldsymbol x,\label{HdivdivSRTfemdof52}\\
% (\boldsymbol \tau, \boldsymbol q)_K &\quad \forall~\boldsymbol q\in  \ker (\cdot\boldsymbol x)\cap \mathbb P_{k-2}(K;\mathbb S) \label{HdivdivSRTfemdof6}.
% \end{align}
\begin{subequations}\label{HdivdivplusSRTfemdof}
\begin{align}
\boldsymbol \tau (\delta), & \quad~\delta\in \Delta_{0}(T), \label{HdivdivplusSRTfemdof1}\\
(\boldsymbol n_i^{\intercal}\boldsymbol \tau\boldsymbol n_j, q)_f, & \quad~q\in\mathbb P_{k-\ell-1}(f),  f\in\Delta_{\ell}(T),\;  \label{HdivdivplusSRTfemdof2}\\
&\quad\quad i,j=1,\ldots, d-\ell, \textrm{ 和 }\; \ell=1,\ldots, d-1, \notag\\
(\Pi_F\boldsymbol \tau\boldsymbol n, \boldsymbol q)_F, & \quad~\boldsymbol q\in {\rm ND}_{k-2}(F),  F\in\Delta_{d-1}(T),\label{HdivdivplusSRTfemdof3}\\
(\boldsymbol n^{\intercal}\div\boldsymbol\tau, q)_F, &\quad~q\in\mathbb P_{k-1}(F), \, F\in \Delta_{d-1}(T),\label{HdivdivplusSRTfemdof4}\\
(\div\div\boldsymbol \tau, q)_T, &\quad~q\in \mathbb P_{k-1}(T)/\mathbb P_{1}(T),\label{HdivdivplusSRTfemdof51}\\
(\div\boldsymbol \tau, \boldsymbol q)_T, &\quad~\boldsymbol q\in \big(\mathbb P_{k-3}(T; \mathbb K)/\mathbb P_{0}(T; \mathbb K)\big)\boldsymbol x,\label{HdivdivplusSRTfemdof52}\\
(\boldsymbol \tau, \boldsymbol q)_T, &\quad~\boldsymbol q\in  \ker (\cdot\boldsymbol x)\cap \mathbb P_{k-2}(T;\mathbb S). \label{HdivdivplusSRTfemdof6}
\end{align}
\end{subequations}

% 由引理~\ref{lm:VPfem},取 $\dd=\div\div$,$V=\mathbb P_k(K;\mathbb S)$,$\mathbb H=\boldsymbol x\boldsymbol x^{\intercal}\mathbb H_{k-1}(K)$,$\mathbb P=\mathbb P_{k-1}(K)/\mathbb P_{1}(K)$ 以及 $\mathbb Q=\ker (\cdot\boldsymbol x)\cap \mathbb P_{k-2}(K;\mathbb S)$,即可得到 $\mathbb P_{k+1}^{-}(\mathbb S)$ 型 $H(\div\div; \mathbb S)\cap H(\div; \mathbb S)$ 协调元. 利用 $\div\div\mathbb B^+=\mathbb P_{k-1}(K)/\mathbb P_{1}(K)$ 和 $\nabla^2(\mathbb P+\dd\mathbb H)=\nabla^2\mathbb P_{k-1}(K)$ 这一事实,可知假设 (H5) 成立. 

由于增加了 $\boldsymbol x\boldsymbol x^{\intercal}\mathbb H_{k-1}(T)$,$\div\div$ 算子的值域从 $\mathbb P_{k-2}(T)$ 扩充至 $\mathbb P_{k-1}(T)$. 自由度 $(\div\div\boldsymbol \tau, q)_T $ 中的测试函数空间也从 $\mathbb P_{k-2}(T)/\mathbb P_{1}(T)$ 增加到 $\mathbb P_{k-1}(T)/\mathbb P_{1}(T)$. 
因此,自由度 \eqref{HdivdivplusSRTfemdof} 的总数等于 $\dim V^{+}(\div\div^+;\mathbb S)$. 
然而,由于 $(\boldsymbol x\boldsymbol x^{\intercal}\mathbb H_{k-1}(T))\boldsymbol n|_F\in \mathbb P_{k}(F;\mathbb R^d)$,边界自由度保持不变. 

预期使用 $\mathbb P_{k}^{+}(\mathbb S)$ 型对称元离散重调和问题时,其离散弯矩的 $\div\div$ 收敛阶将比 $\mathbb P_k(\mathbb S)$ 型对称元高一阶,且计算成本未显著增加;参见~\cite[Section 4]{ChenHuang2020a}. 在求解线性代数方程组时,所有内部自由度均可逐单元消除(静态凝聚). 

\begin{lemma}\label{lem:E0SRT}
令 $\boldsymbol\tau\in V^{+}(\div\div^+;\mathbb S)$. 
若自由度 \eqref{HdivdivplusSRTfemdof1}-\eqref{HdivdivplusSRTfemdof52} 均为零,则 $\boldsymbol \tau \in E_0(\mathbb S)$. 
\end{lemma}
\begin{proof}
由于 $\boldsymbol x\cdot \boldsymbol n$ 在每个 $(d-1)$ 维面上均为常数,因此迹 $\boldsymbol\tau\boldsymbol n|_F\in\mathbb P_k(F;\mathbb R^{d})$ 和 $(\boldsymbol  n^{\intercal}\div \boldsymbol \tau)|_F\in\mathbb P_{k-1}(F)$ 保持不变. 
由定理~\ref{th:divdivcapdiv} 可得 ${\rm tr}^{\rm div}\boldsymbol\tau=\boldsymbol0$ 且 ${\rm tr}^{\rm div}(\div \boldsymbol \tau)=0$. 

应用格林公式 \eqref{eq:greenidentitydivdiv},可得
$$
(\div\div\boldsymbol\tau , v)_T=(\boldsymbol\tau, \nabla^2 v)_T=0 \quad \forall~v \in \mathbb P_{1}(T).
$$
因此,由自由度~\eqref{HdivdivplusSRTfemdof51} 为零可知 $\div\div\boldsymbol\tau=0$,结合引理~\ref{lem:divdivonto} 意味着 $\boldsymbol\tau\in\mathbb P_k(T;\mathbb S)$. 
最后,由引理~\ref{lm:F0star} 及自由度~\eqref{HdivdivplusSRTfemdof52} 为零,可得 $\boldsymbol \tau \in E_0(\mathbb S)$. 
%  Applying the Green's identity \eqref{eq:greenidentitydivdiv}, we get from the vanishing degrees of freedom~\eqref{HdivdivSRTfemdof51} that
% $$
% (\div\div\boldsymbol\tau , v)_K=(\boldsymbol\tau, \nabla^2 v)_K=0 \quad \forall~v \in \mathbb P_{k-1}(K),
% $$
% as $\nabla^2v = \defm \nabla v \in \defm  \mathbb P_{k-1}(K;\mathbb R^d)$. 
% Hence $\div\div\boldsymbol\tau=0$, which combined with Lemma~\ref{lem:divdivonto} implies $\boldsymbol\tau\in\mathbb P_k(K;\mathbb S)$.
% Finally we achieve from Lemma~\ref{lm:F0star} and the vanishing degrees of freedom~\eqref{HdivdivSRTfemdof52} that $\boldsymbol \tau \in E_0(\mathbb S)$.
\end{proof}

% ----------------------------------------------------------------

% \begin{lemma}\label{lem:divdivSRTboundarydofs}
% Let $F\in\mathcal F^1(K)$ and $\boldsymbol\tau\in\mathbb P_k(K;\mathbb S)+\boldsymbol x\boldsymbol x^{\intercal}\mathbb H_{k-1}(K)$. If all the degrees of freedom \eqref{HdivdivSRTfemdof1}-\eqref{HdivdivSRTfemdof4} restricted to $F$ vanish, then $\boldsymbol\tau\boldsymbol n|_F=\boldsymbol0$ and $(\boldsymbol  n^{\intercal}\div \boldsymbol \tau)|_F=0$.
% \end{lemma}
% \begin{proof}
% Since $\boldsymbol\tau\boldsymbol n|_F\in\mathbb P_k(F;\mathbb R^{d-1})$ and $(\boldsymbol  n^{\intercal}\div \boldsymbol \tau)|_F\in\mathbb P_{k-1}(F)$, we conclude the results from Theorem~\ref{th:divdivcapdiv}.
% \end{proof}

结合引理~\ref{lem:E0SRT}、\eqref{eq:E0Sdualcharac} 以及自由度 \eqref{HdivdivplusSRTfemdof6},可证得 $\mathbb P_{k+1}^{-}(\mathbb S)$ 型 $H(\div\div; \mathbb S)\cap H(\div; \mathbb S)$ 协调元的唯一可解性. 
\begin{theorem}\label{th:P-S}
自由度 \eqref{HdivdivplusSRTfemdof} 可唯一确定空间 $V^{+}(\div\div^+;\mathbb S) = \mathbb P_k(K;\mathbb S) \oplus \boldsymbol x\boldsymbol x^{\intercal}\mathbb H_{k-1}(K)$. 
\end{theorem}


有限元空间 $\boldsymbol V_h^+(\div\div^+;\mathbb S)\subset H(\div\div, \Omega; \mathbb S)\cap H(\div, \Omega; \mathbb S)$ 定义如下:
\begin{align*}
\boldsymbol V_h^+(\div\div^+;\mathbb S)&:=\{\boldsymbol \tau\in L^2(\Omega;\mathbb S): \text{对于任意 }\, T\in\mathcal T_h, \boldsymbol \tau|_T\in \mathbb P_k(T;\mathbb S)\oplus \boldsymbol x\boldsymbol x^{\intercal}\mathbb H_{k-1}(T), \\
&\qquad \textrm{且自由度 \eqref{HdivdivplusSRTfemdof1}-\eqref{HdivdivplusSRTfemdof4} 是单值的} \}.    
\end{align*}
% \begin{proof}
% Thanks to Theorem~\ref{th:divdivcapdiv},
% the number of degrees of freedom \eqref{HdivdivSRTfemdof1}-\eqref{HdivdivSRTfemdof6} equals to $\dim \mathbb P_k(K;\mathbb S)+\dim\mathbb H_{k-1}(K)$.

% Take any $\boldsymbol\tau\in\mathbb P_k(K;\mathbb S)+\boldsymbol x\boldsymbol x^{\intercal}\mathbb H_{k-1}(K)$ and suppose all the degrees of freedom \eqref{HdivdivSRTfemdof1}-\eqref{HdivdivSRTfemdof6} vanish. It follows from Lemma~\ref{lem:divdivSRTboundarydofs} that $\boldsymbol\tau\boldsymbol n|_{\partial K}=\boldsymbol0$ and $(\boldsymbol  n^{\intercal}\div \boldsymbol \tau)|_{\partial K}=0$.
% Then we get from the Green's identity \eqref{eq:greenidentitydivdiv} and the vanishing degrees of freedom \eqref{HdivdivSRTfemdof5} that
% $$
% (\div\div\boldsymbol\tau , v)_K=(\boldsymbol\tau, \nabla^2 v)_K=0, \quad \forall v \in \mathbb P_{k-1}(K)
% $$
% as $\nabla^2v = \defm \nabla v \in \defm  \mathbb P_{k-1}(K;\mathbb R^d)$. 

% Hence $\div\div\boldsymbol\tau=\boldsymbol0$, which implies $\boldsymbol\tau\in\mathbb P_k(K;\mathbb S)$.
% Therefore we conclude $\boldsymbol\tau=\boldsymbol0$ from Theorem~\ref{th:divdivcapdiv}.
% \end{proof}

与引理~\ref{lm:divdivdivSinfsup} 类似,我们有如下 inf-sup 条件. 
\begin{lemma}\label{lm:divdivdivSminfsup}
设 $k\geq \max\{d,3\}$. 成立如下条件:
$$
\|p_h\|\lesssim\sup_{\boldsymbol\tau_h\in \boldsymbol V_h^+(\div\div^+;\mathbb S)}
\frac{(\div\div\boldsymbol{\tau}_h, p_h)}{\|\boldsymbol\tau_h\|_{H(\div)}+
\|\div\div\boldsymbol{\tau}_h\|_{0}}\qquad\forall~p_h\in\mathbb P_{k-1}(\mathcal T_h).
$$
\end{lemma}









\subsection{$H(\div\div)$ 协调对称张量有限元}\label{subsec:newdivdivSfem}
本节构造$H(\div\div)$ 协调对称张量有限元 \cite[Section 5.4]{ChenHuang2022}. 

同时要求 $\boldsymbol \tau \boldsymbol n$ 和 $\boldsymbol n^{\intercal}\div \boldsymbol \tau $ 连续,虽然是函数属于 $H(\div\div, \Omega; \mathbb S)$ 的充分条件,但并非必要条件. 
基于格林公式 \eqref{eq:greenidentitydivdiv},除了 $\boldsymbol n^{\intercal}\boldsymbol\tau\boldsymbol n$ 之外,只需保证组合量 $\boldsymbol n^{\intercal}\div \boldsymbol \tau + \div_F(\boldsymbol\tau \boldsymbol n)$ 的连续性即可. 

\begin{theorem}\label{thm:divdivS}
取 $V(\div\div;\mathbb S):=\mathbb P_k(T;\mathbb S)$(其中 $k\geq\max\{d,3\}$)作为形函数空间. 
其自由度定义如下:
% \begin{align}
% \boldsymbol \tau (\delta), & \quad\forall~\delta\in \mathcal V(K), \label{newHdivdivSfemdof1}\\
% (\boldsymbol  n_i^{\intercal}\boldsymbol \tau\boldsymbol n_j, q)_f, & \quad\forall~q\in\mathbb P_{k+r-d-1}(f), \, \forall f\in\mathcal F^r(K),\;  \label{newHdivdivSfemdof2}\\
% &\quad\quad i,j=1,\cdots, r, \text{ 且 } r=1,\cdots, d-1, \notag\\
% (\Pi_F\boldsymbol \tau\boldsymbol n, \boldsymbol q)_F, & \quad\forall~\boldsymbol q\in \mathrm{ND}_{k-2}(F), \, \forall F\in\Delta_{d-1}(T),\label{newHdivdivSfemdof4}\\
% (\boldsymbol n^{\intercal}\div \boldsymbol \tau +  \div_F(\boldsymbol\tau \boldsymbol n), p)_F, &\quad\forall~p\in\mathbb P_{k-1}(F), \, \forall F\in \Delta_{d-1}(T),\label{newHdivdivSfemdof3}\\
% (\boldsymbol \tau, \defm\boldsymbol q)_K, &\quad \forall~\boldsymbol q\in \mathrm{ND}_{k-3}(K),\notag\\%\label{newHdivdivSfemdof5}\\
% (\boldsymbol \tau, \boldsymbol q)_K, &\quad \forall~\boldsymbol q\in  \ker (\cdot\boldsymbol x)\cap \mathbb P_{k-2}(K;\mathbb S). \notag% \label{newHdivdivSfemdof6}.
% \end{align}
\begin{subequations}\label{eq:divdivS}
\begin{align}
\boldsymbol \tau (\texttt{v}), & \quad~\texttt{v}\in \Delta_0(T), \label{HdivdivSfemdof1}\\
(\boldsymbol n_i^{\intercal}\boldsymbol \tau\boldsymbol n_j, q)_f, &\quad~q\in\mathbb P_{k-\ell-1}(f),  f\in\Delta_{\ell}(T),\;  \label{HdivdivSfemdof2}\\
&\quad\quad i,j=1,\ldots, d-\ell, \textrm{ 和 }\; \ell=1,\ldots, d-1, \notag\\
(\boldsymbol n^{\intercal}\div \boldsymbol \tau +  \div_F(\boldsymbol\tau \boldsymbol n), q)_F, &\quad~q\in\mathbb P_{k-1}(F), F\in \partial T,\label{HdivdivSfemdof3}\\
(\Pi_F\boldsymbol \tau\boldsymbol n, \boldsymbol q)_F, & \quad~\boldsymbol q\in {\rm ND}_{k-2}(F),  F\in\partial T,\label{HdivdivSfemdof4}\\
(\boldsymbol \tau, \defm\boldsymbol q)_T, &\quad~\boldsymbol q\in {\rm ND}_{k-3}(T)\backslash {\rm RM},\label{HdivdivSfemdof5}\\
(\boldsymbol \tau, \boldsymbol q)_T, &\quad~\boldsymbol q\in  \ker (\cdot\boldsymbol x)\cap \mathbb P_{k-2}(T;\mathbb S), \label{HdivdivSfemdof6}
\end{align}
\end{subequations}
其中,自由度 \eqref{HdivdivSfemdof4} 被视单纯形 $T$ 的内部自由度,即它在单元之间不需要满足单值(连续)条件. 
\end{theorem}

\begin{proof}
根据引理 \ref{lem:divSboundarydofs},项 $\div_F(\boldsymbol \tau\boldsymbol n)$ 可由自由度 \eqref{HdivdivSfemdof1}、\eqref{HdivdivSfemdof2} 和 \eqref{HdivdivSfemdof4} 确定. 结合自由度 \eqref{HdivdivSfemdof3} 进行线性组合,即可确定迹 $\boldsymbol n^{\intercal}\div \boldsymbol \tau$. 进而,由定理 \ref{th:divdivcapdiv} 可得唯一可解性. 
\end{proof}

有限元空间 $\boldsymbol V_h(\div\div)$ 定义如下:
\begin{align*}
\boldsymbol V_h(\div\div,\Omega;\mathbb S):=\{&\boldsymbol \tau\in L^2(\Omega;\mathbb S): \boldsymbol \tau|_K\in\mathbb P_k(T;\mathbb S), \, \forall\,T\in\mathcal T_h; \textrm{自由度 \eqref{HdivdivSfemdof1}-\eqref{HdivdivSfemdof3} 是单值的} \}.    
\end{align*}
由于 $\boldsymbol n^{\intercal}\boldsymbol\tau\boldsymbol n$ 和 $\boldsymbol n^{\intercal}\div \boldsymbol \tau + \div_F(\boldsymbol\tau \boldsymbol n)$ 是连续的,因此 $\boldsymbol V_h(\div\div)\subset H(\div\div, \Omega; \mathbb S)$;参见文献~\cite[Lemma 4.4]{ChenHuang2022a}. 

最后,我们介绍一种 $\mathbb P_{k}^{+}(\mathbb S)$ 型的 $H(\div\div; \mathbb S)$-协调元. 

\begin{theorem}
设整数 $k\geq \max\{d,3\}$. 
取形函数空间为 $$V^{+}(\div\div;\mathbb S):=\mathbb P_k(T;\mathbb S) \oplus \boldsymbol x\boldsymbol x^{\intercal}\mathbb H_{k-1}(T).$$
其自由度定义如下:
\begin{subequations}\label{eq:divdivSRT}
\begin{align}
\boldsymbol \tau (\texttt{v}), & \quad~\texttt{v}\in \Delta_0(T), \label{HdivdivSRTfemdof1}\\
(\boldsymbol n_i^{\intercal}\boldsymbol \tau\boldsymbol n_j, q)_f, &\quad~q\in\mathbb P_{k-\ell-1}(f),  f\in\Delta_{\ell}(T),\;  \label{HdivdivSRTfemdof2}\\
&\quad\quad i,j=1,\ldots, d-\ell, \textrm{ 和 }\; \ell=1,\ldots, d-1, \notag\\
(\boldsymbol n^{\intercal}\div \boldsymbol \tau +  \div_F(\boldsymbol\tau \boldsymbol n), q)_F, &\quad~q\in\mathbb P_{k-1}(F), F\in \partial T,\label{HdivdivSRTfemdof3}\\
(\Pi_F\boldsymbol \tau\boldsymbol n, \boldsymbol q)_F, & \quad~\boldsymbol q\in {\rm ND}_{k-2}(F),  F\in\partial T,\label{HdivdivSRTfemdof4}\\
(\boldsymbol \tau, \defm\boldsymbol q)_T, &\quad~\boldsymbol q\in \mathbb P_{k-2}(T;\mathbb R^d)\backslash {\rm RM},\label{HdivdivSRTfemdof5}\\
(\boldsymbol \tau, \boldsymbol q)_T, &\quad~\boldsymbol q\in  \ker (\cdot\boldsymbol x)\cap \mathbb P_{k-2}(T;\mathbb S), \label{HdivdivSRTfemdof6}
\end{align}
\end{subequations}
同样,自由度 \eqref{HdivdivSRTfemdof4} 被视为单纯形 $T$ 的内部自由度,即它在单元之间不需要满足单值(连续)条件. 
\end{theorem}

\begin{proof}
根据引理 \ref{lem:divSboundarydofs},项 $\div_F(\boldsymbol \tau\boldsymbol n)$ 可由自由度 \eqref{HdivdivSRTfemdof1}、\eqref{HdivdivSRTfemdof2} 和 \eqref{HdivdivSRTfemdof4} 确定. 结合自由度 \eqref{HdivdivSRTfemdof3} 进行线性组合,即可确定迹 $\boldsymbol n^{\intercal}\div \boldsymbol \tau$. 进而,由定理 \ref{th:P-S} 可得唯一可解性. 
\end{proof}

全局有限元空间 $\boldsymbol V_h^{+}(\div\div)\subset H(\div\div, \Omega; \mathbb S)$ 定义为:
\begin{align*}
\boldsymbol V_h^{+}(\div\div,\Omega;\mathbb S):=\{&\boldsymbol \tau\in L^2(\Omega;\mathbb S): \boldsymbol \tau|_T\in V^{+}(\div\div;\mathbb S), \, \forall\,T\in\mathcal T_h; \textrm{自由度 \eqref{HdivdivSRTfemdof1}-\eqref{HdivdivSRTfemdof3} 是单值的} \}.    
\end{align*}

最后,我们列出 divdiv 协调元的 inf-sup 条件. 
\begin{lemma}
设 $k\geq \max\{d,3\}$. 
成立如下 inf-sup 条件:
\begin{equation}\label{eq:divdivinfsup1}
\|p_h\|\lesssim\sup_{\boldsymbol\tau_h\in \boldsymbol V_h(\div\div,\Omega;\mathbb S)}
\frac{(\div\div\boldsymbol{\tau}_h, p_h)}{\|\boldsymbol\tau_h\|_{0}+
\|\div\div\boldsymbol{\tau}_h\|_{0}}\qquad\forall~p_h\in\mathbb P_{k-2}(\mathcal T_h),
\end{equation}
\begin{equation}\label{eq:divdivinfsup2}
\|p_h\|\lesssim\sup_{\boldsymbol\tau_h\in \boldsymbol V_h^{+}(\div\div,\Omega;\mathbb S)}
\frac{(\div\div\boldsymbol{\tau}_h, p_h)}{\|\boldsymbol\tau_h\|_{0}+
\|\div\div\boldsymbol{\tau}_h\|_{0}}\qquad\forall~p_h\in\mathbb P_{k-1}(\mathcal T_h).
\end{equation}
\end{lemma}

\begin{proof}
由于 $\|\boldsymbol\tau_h\|_{0}\leq \|\boldsymbol\tau_h\|_{H(\div)}$ 且 $\boldsymbol V_h(\div\div^+;\mathbb S)\subseteq \boldsymbol V_h(\div\div,\Omega;\mathbb S)$,inf-sup 条件 \eqref{eq:divdivinfsup1} 可由引理~\ref{lm:divdivdivSinfsup} 直接推出. 同理,由 $\boldsymbol V_h^+(\div\div^+;\mathbb S)\subseteq \boldsymbol V_h^+(\div\div,\Omega;\mathbb S)$ 及引理~\ref{lm:divdivdivSminfsup} 可得 inf-sup 条件 \eqref{eq:divdivinfsup2}. 
\end{proof}



\subsection{可杂交化 $H(\div\div)$ 协调对称张量有限元}
由自由度 \eqref{eq:divdivS} 所定义的 $H(\div\div)$ 协调对称张量有限元,由于要求多项式阶数满足 $k\geq \max\{d,3\}$ 且自由度形式相对复杂,在实际实现中具有较大的挑战性.
% 在二维情形下,一种 $H(\div\div)\cap H(\div)$-协调元已成功实现,并应用$H(\div; \mathbb S)$ 协调的 Hu-Zhang 元的基底来离散重调和方程,详见~\cite{HuMaZhang2021}. 

本节将提出一种可杂交化的 $H(\div\div)$ 协调对称张量有限元 \cite{ChenHuang2025}.   
对于任意 $k\geq 3$,单元 $T$ 上的形函数空间取为 $\mathbb P_k(T;\mathbb S)$,其自由度给定为 \eqref{eq:newdivdivS}:
\begin{subequations}\label{eq:newdivdivS}
\begin{align}
(\tr_e(\bs \tau), q)_e, &\quad q\in \mathbb P_k(e), e\in \Delta_{d-2}(T),\label{eq:newdivdivdof1}\\
(\bs n^{\intercal}\bs \tau \bs n, q )_F, &\quad q\in \mathbb P_k(F), F\in \partial T,\label{eq:newdivdivdof2}\\
( \tr_2(\bs \tau), q)_F, &\quad q\in\mathbb P_{k-1}(F), F\in \partial T,\label{eq:newdivdivdof3}\\
(\Pi_F\boldsymbol \tau\boldsymbol n, \boldsymbol q)_F, & \quad \boldsymbol q\in {\rm ND}_{k-2}(F),  F\in\partial T,\label{eq:newdivdivdof4}\\
(\boldsymbol \tau, \defm\boldsymbol q)_T, &\quad \boldsymbol q\in {\rm ND}_{k-3}(T),\label{eq:newdivdivdof5}\\
(\boldsymbol \tau, \boldsymbol q)_T, &\quad \boldsymbol q\in  \ker (\cdot\boldsymbol x)\cap \mathbb P_{k-2}(T;\mathbb S), \label{eq:newdivdivdof6}
\end{align}
\end{subequations}
其中
$$
\tr_e(\bs \tau) = \sum_{F\in\partial T,e\in \partial F}\boldsymbol n_{F,e}^{\intercal}\boldsymbol \tau \boldsymbol n_{\partial T},\quad \tr_2(\bs \tau) =  \boldsymbol n_{\partial T}^{\intercal}\div \boldsymbol \tau +  \div_F(\boldsymbol\tau \boldsymbol n_{\partial T}).
$$
这里,$\bs n_{F,e}$ 表示由面 $F$ 的定向诱导的 $e$ 在 $F$ 上的法方向. 
与自由度 \eqref{eq:divdivS} 相比,区别在于自由度 \eqref{HdivdivSfemdof1}-\eqref{HdivdivSfemdof2} 被重新分配到了棱和面上,形成了自由度 \eqref{eq:newdivdivdof1}-\eqref{eq:newdivdivdof2}. 

我们现在简要解释这一重分配过程. 不失一般性,考虑顶点 $\texttt{v}_0$. 选择 $\{\bs n_{F_i}, i=1,\ldots, d\}$ 作为 $\mathbb R^d$ 的一组基,其中 $F_i$ 是包含 $\texttt{v}_0$ 的 $(d-1)$ 维面($i=1,\ldots, d$). 自由度 $\boldsymbol \tau (\texttt{v}_0)\in \mathbb S$ 由对称矩阵 $( \bs n_{F_i}^{\intercal}\boldsymbol \tau (\texttt{v}_0)\bs n_{F_j})_{i,j=1,\ldots, d}$ 确定. 我们将对角项 $\bs n_{F_i}\boldsymbol \tau (\texttt{v}_0)\bs n_{F_i}$ 分配给面 $F_i$ ($i=1,\ldots, d$),将非对角项 $\bs n_{F_i}\boldsymbol \tau (\texttt{v}_0)\bs n_{F_j}$ ($1\leq i< j\leq d$) 分配给 $(d-2)$ 维面 $e_{ij} = F_i\cap F_j$. 
这种重分配可以推广到自由度 \eqref{HdivdivSfemdof2}. 对于低维子单纯形 $f\in \Delta_r(T), r=1,\ldots, d-1$,使用 $\{\bs n_{F_i}, f\in \Delta_r(F_i), i=1,\ldots, d-r\}$ 作为 $f$ 的法平面 $\mathcal N_f$ 的基. 我们可以将对角项 $\bs n_{F_i}^{\intercal}\boldsymbol \tau\bs n_{F_i}|_f$ 分配给面 $F_i$,将非对角项 $\bs n_{F_i}^{\intercal}\boldsymbol \tau\bs n_{F_j}|_f$ 分配给 $(d-2)$ 维面 $e_{ij}= F_i\cap F_j$. 

重分配后,我们将自由度合并. 一个多项式 $u\in \mathbb P_k(T)$ 可由下式确定:
\begin{equation}\label{eq:PkDoF}
(u, q)_T, \quad q\in \mathbb P_k(T).
\end{equation}
回顾~\cite[(2.6)]{ArnoldFalkWinther2009} 中 Lagrange 元的几何分解:
\begin{align}
\label{eq:Prdec}
\mathbb P_k(T) &= \Oplus_{r = 0}^{d}\Oplus_{f\in \Delta_{r}(T)} b_f\mathbb P_{k - (r +1)} (f),
\end{align}
其中 $b_f\in\mathbb P_{r+1}(f)$ (且 $b_f|_{\partial f} = 0$) 是 $f$ 上的 $(r+1)$ 次多项式泡函数. 
基于 \eqref{eq:Prdec},自由度 \eqref{eq:PkDoF} 可以分解为:
\begin{equation}\label{eq:PkdecDoF}
(u, q)_f, \quad q\in\mathbb P_{k-r-1}(f),  f\in\Delta_{r}(T),\;  r=0,1,\ldots, d.
\end{equation}
反之,\eqref{eq:PkdecDoF} 中的自由度可以合并为 \eqref{eq:PkDoF}. 

重分配后,我们在 $(d-1)$ 维面和 $(d-2)$ 维面上合并自由度. 例如,在 $(d-1)$ 维面 $F$ 上,我们将拥有自由度:
\begin{equation}\label{eq:nnface}
(\boldsymbol  n_F^{\intercal}\boldsymbol \tau\boldsymbol n_F, q)_f,  \quad q\in\mathbb P_{k-r-1}(f),  f\in\Delta_{r}(F),\;  r=0,1,\ldots, d-1.
\end{equation}
根据 Lagrange 元的分解 \eqref{eq:Prdec},我们可以将 \eqref{eq:nnface} 合并到自由度 \eqref{eq:newdivdivdof2}. 类似地,在由 $F_1$ 和 $F_2$ 共享的 $(d-2)$ 维面 $e$ 上,我们将 $\boldsymbol  n_{F_1}^{\intercal}\boldsymbol \tau\boldsymbol n_{F_2}$ 的自由度合并为:
\begin{equation}\label{eq:nnedge}
(\boldsymbol  n_{F_1}^{\intercal}\boldsymbol \tau\boldsymbol n_{F_2}, q)_e,\quad q\in \mathbb P_k(e), e\in \Delta_{d-2}(T).
\end{equation}
为了从自由度 \eqref{eq:nnedge} 切换到边跳跃自由度 \eqref{eq:newdivdivdof1},我们需要以下引理. 

\begin{lemma}\label{lem:Snnbasis}
对于 $(d-2)$ 维面 $e\in\Delta_{d-2}(T)$,设 $F_1$ 和 $F_2$ 是 $\Delta_{d-1}(T)$ 中共享 $e$ 的两个 $(d-1)$ 维面,且 $\bs n_{F_i}=\bs n_{F_i,\partial T}$ ($i=1,2$). 则
$$\{\boldsymbol{n}_{F_1}\otimes\boldsymbol{n}_{F_1}, \boldsymbol{n}_{F_2}\otimes\boldsymbol{n}_{F_2}, \sym(\boldsymbol{n}_{F_1,e}\otimes\boldsymbol{n}_{F_1})+\sym(\boldsymbol{n}_{F_2,e}\otimes\boldsymbol{n}_{F_2})\}$$ 
和 
$$\{\boldsymbol{n}_{F_1}\otimes\boldsymbol{n}_{F_1}, \boldsymbol{n}_{F_2}\otimes\boldsymbol{n}_{F_2}, \sym(\boldsymbol{n}_{F_1}\otimes\boldsymbol{n}_{F_2})\}$$ 
是 $e$ 的法平面上的对称矩阵空间 $\mathbb S(\mathcal N_e)$ 的基. 
\end{lemma}
\begin{proof}
显然,
$$
\mathbb S(\mathcal N_e) = \operatorname{span}\{\boldsymbol{n}_{F_1}\otimes\boldsymbol{n}_{F_1}, \boldsymbol{n}_{F_2}\otimes\boldsymbol{n}_{F_2}, \sym(\boldsymbol{n}_{F_1}\otimes\boldsymbol{n}_{F_2})\},
$$
且 $\sym(\boldsymbol{n}_{F_1,e}\otimes\boldsymbol{n}_{F_1})+\sym(\boldsymbol{n}_{F_2,e}\otimes\boldsymbol{n}_{F_2})\in \mathbb S(\mathcal N_e)$. 

现在我们证明 $\boldsymbol{n}_{F_1}\otimes\boldsymbol{n}_{F_1}$, $\boldsymbol{n}_{F_2}\otimes\boldsymbol{n}_{F_2}$ 和 $\sym(\boldsymbol{n}_{F_1,e}\otimes\boldsymbol{n}_{F_1})+\sym(\boldsymbol{n}_{F_2,e}\otimes\boldsymbol{n}_{F_2})$ 是线性无关的. 假设常数 $c_1, c_2$ 和 $c_3$ 满足
$$
c_1\boldsymbol{n}_{F_1}\otimes\boldsymbol{n}_{F_1}+c_2\boldsymbol{n}_{F_2}\otimes\boldsymbol{n}_{F_2}+c_3\big(\sym(\boldsymbol{n}_{F_1,e}\otimes\boldsymbol{n}_{F_1})+\sym(\boldsymbol{n}_{F_2,e}\otimes\boldsymbol{n}_{F_2})\big)=0.
$$
我们来证明 $c_1=c_2=c_3=0$. 在上述方程两边乘以 $\sym(\boldsymbol{n}_{F_1,e}\otimes\boldsymbol{n}_{F_2,e})$,我们得到
$$
\frac{1}{2}c_3(\boldsymbol{n}_{F_1}\cdot\boldsymbol{n}_{F_2,e} + \boldsymbol{n}_{F_2}\cdot\boldsymbol{n}_{F_1,e})=0.
$$
注意到 $\boldsymbol{n}_{F_1}\cdot\boldsymbol{n}_{F_2,e}$ 和 $\boldsymbol{n}_{F_2}\cdot\boldsymbol{n}_{F_1,e}$ 均为正值,故 $c_3=0$. 这蕴含
$$
c_1\boldsymbol{n}_{F_1}\otimes\boldsymbol{n}_{F_1}+c_2\boldsymbol{n}_{F_2}\otimes\boldsymbol{n}_{F_2}=0.
$$
因此,$c_1=c_2=0$. 
\end{proof}


下面证明其唯一可解性. 回顾二项式系数符号 ${n\choose k}$,若 $n\geq 0$ 且 $k<0$,则规定 ${n\choose k} := 0$. 
\begin{lemma}
对于 $k\geq 3$,自由度 \eqref{eq:newdivdivS} 对空间 $\mathbb P_k(T;\mathbb S)$ 是唯一可解的. 
\end{lemma}
\begin{proof}
对于 $d$ 维单纯形 $T$,$r$ 维子单纯形的个数为 ${d+1 \choose r+1}$. 空间 $\mathbb P_{k-r-1}(f)$(其中 $\dim f = r$)的维数为 ${ k-1 \choose k - r - 1}$,这也适用于 $r\geq k$(此时维数为 0). $f$ 的法平面 $\mathcal N_f$ 维数为 $d - r$,其上的对称张量空间维数为 ${ d -r +1 \choose 2}$,可分为非对角部分和对角部分,即 ${d-r+1\choose2}={d-r\choose2}+d-r$. 
自由度 \eqref{HdivdivSfemdof1}-\eqref{HdivdivSfemdof2} 的数量为
\begin{align}
\notag &\quad\sum_{r=0}^{d-1}{d+1\choose r+1}{k-1\choose k-r-1}{d-r+1\choose2} \\
\label{eq:redistribution} &=\frac{1}{2}d(d+1)\sum_{r=0}^{d-2}{d-1\choose r+1}{k-1\choose k-r-1}+(d+1)\sum_{r=0}^{d-1}{d\choose r+1}{k-1\choose k-r-1} \\
\label{eq:merge} &=\frac{1}{2}d(d+1){k+d-2\choose k}+(d+1){k+d-1\choose k},
\end{align}
这等于自由度 \eqref{eq:newdivdivdof1}-\eqref{eq:newdivdivdof2} 的数量. 
因此,自由度 \eqref{eq:newdivdivS} 的数量与自由度 \eqref{eq:divdivS} 的数量相匹配,根据定理 \ref{thm:divdivS},这也等于空间 $\mathbb P_k(T;\mathbb S)$ 的维数. 在上述推导中,\eqref{eq:redistribution} 对应于将自由度重分配到 $(d-2)$ 维面和 $(d-1)$ 维面,而 \eqref{eq:merge} 对应于 $(d-2)$ 维面和 $(d-1)$ 维面上 Lagrange 元自由度的合并. 

设 $\boldsymbol{\tau} \in \mathbb{P}_k(T;\mathbb{S})$,并假设由 \eqref{eq:newdivdivS} 定义的所有自由度均取值为零. 利用引理 \ref{lem:Snnbasis} 可知,自由度 \eqref{eq:newdivdivdof1}-\eqref{eq:newdivdivdof2} 为零蕴含了自由度 \eqref{eq:nnedge} 亦为零. 进一步地,结合 \eqref{eq:nnedge} 与 \eqref{eq:newdivdivdof2} 为零的事实,可推出自由度 \eqref{HdivdivSfemdof1}-\eqref{HdivdivSfemdof2} 全都为零. 因此,根据定理 \ref{thm:divdivS} 所述的唯一可解性,我们断定自由度 \eqref{eq:newdivdivS} 在 $\mathbb{P}_k(T;\mathbb{S})$ 上是唯一可解的. 
\end{proof}

首先定义全局空间:
\begin{align*}
\Sigma_{k}^{\operatorname{div}\operatorname{div}-} := \big\{ \boldsymbol{\tau}\in L^2(\Omega;\mathbb{S}) :\, & \boldsymbol{\tau}|_T\in \mathbb{P}_{k}(T;\mathbb{S}) \quad \forall\,T\in\mathcal{T}_h,  \text{ 且自由度 \eqref{eq:newdivdivdof2}-\eqref{eq:newdivdivdof3} 是单值的} \big\}.
\end{align*}
根据构造,对于 $\boldsymbol{\tau}\in \Sigma_{k}^{\operatorname{div}\operatorname{div}-}$,其迹 $\operatorname{tr}_1(\boldsymbol{\tau})$ 和 $\operatorname{tr}_2(\boldsymbol{\tau})$ 均是连续的. 然而,棱上的跳跃 $[\operatorname{tr}_e(\boldsymbol{\tau})]|_{e}$ 未必为零,这导致 $\Sigma_{k}^{\operatorname{div}\operatorname{div}-}$ 并非 $H(\operatorname{div}\operatorname{div})$-协调的(参见引理 \ref{lm:divdivconforming}). 
事实上,边跳跃条件 $[\operatorname{tr}_e(\boldsymbol{\tau})]|_{e} = 0$ 涉及棱 $e$ 周围的单元片 $\omega_e$. 在每个单元内部,$\operatorname{tr}_e(\boldsymbol{\tau})$ 本身可能不为零,且不同单元贡献的迹通常各不相同. 因此,在定义 $\Sigma_{k}^{\operatorname{div}\operatorname{div}-}$ 时,自由度 \eqref{eq:newdivdivdof1} 不是单值的. 

为此,我们定义如下子空间:
\[
\Sigma_{k}^{\operatorname{div}\operatorname{div}} := \big\{ \boldsymbol{\tau}\in \Sigma_{k}^{\operatorname{div}\operatorname{div}-}: [\operatorname{tr}_e(\boldsymbol{\tau})]|_{e} = 0 \quad \forall\,e\in \mathring{\mathcal{E}}_h \big\}.
\]
即我们在单元局部的棱迹自由度上增加如下约束:
\[
\operatorname{tr}_e^{T_1}(\boldsymbol{\tau})+ \operatorname{tr}_e^{T_2}(\boldsymbol{\tau}) + \cdots + \operatorname{tr}_e^{T_{|\omega_e|}}(\boldsymbol{\tau}) \big|_e = 0,
\]
从而获得一个 $H(\operatorname{div}\operatorname{div})$-协调的有限元子空间. 

令 $I_h^{\operatorname{div}\operatorname{div}}: H^2(\Omega;\mathbb{S})\to \Sigma_{k}^{\operatorname{div}\operatorname{div}-}$ 为基于自由度 \eqref{eq:newdivdivS} 定义的典范插值算子(canonical interpolation operator). 即对于 \eqref{eq:newdivdivS} 中的所有自由度泛函 $N$,均有 $N(I_h^{\operatorname{div}\operatorname{div}}\boldsymbol{\tau}) = N(\boldsymbol{\tau})$. 为简化记号,我们将 $I_h^{\operatorname{div}\operatorname{div}}\boldsymbol{\tau}$ 简记为 $\boldsymbol{\tau}_I$. 
注意到
$$
[\tr_e(\bs \tau_I)]|_e=Q_{k,e}([\tr_e(\bs \tau)]|_e)=0\quad\forall~e\in\mathring{\mathcal{E}}_h, \bs\tau\in H^2(\Omega;\mathbb S), 
$$
因此确实有 $\bs \tau_I \in \Sigma_{k}^{\div\div}$. 
\begin{lemma}
$I_h^{\operatorname{div}\operatorname{div}}$ 是一个 Fortin 算子. 即对于 $\boldsymbol{\tau}\in H^2(\Omega;\mathbb{S})$,满足:
\begin{equation}\label{eq:IdivdivQcd}
\operatorname{div}\operatorname{div}(\boldsymbol{\tau}_I) = Q_{k-2}(\operatorname{div}\operatorname{div}\boldsymbol{\tau}).
\end{equation}
\end{lemma}
\begin{proof}
该结论可利用 Green 公式 \eqref{eq:greenidentitydivdiv} 以及 $I_h^{\operatorname{div}\operatorname{div}}$ 的定义直接推得. 
\end{proof}

利用 Fortin 算子,我们可以建立如下 inf-sup 条件. 
\begin{lemma}
对于 $k\geq 3$,成立如下 inf-sup 条件:
\begin{equation}\label{eq:newdivdivinfsup}
\inf_{p_h\in V^{-1}_{k-2}}\sup_{\boldsymbol{\tau}_h\in \Sigma_{k}^{\operatorname{div}\operatorname{div}}}
\frac{(\operatorname{div}\operatorname{div}\boldsymbol{\tau}_h, p_h)}{\|\boldsymbol{\tau}_h\|_{\operatorname{div}\operatorname{div}} \|p_h\|} = \alpha > 0.
\end{equation}
\end{lemma}
\begin{proof}
对于 $p_h\in V^{-1}_{k-2}$,根据复形正则性理论~\cite{ArnoldHu2021,PaulyZulehner2020},存在函数 $\boldsymbol{\tau}\in H^2(\Omega;\mathbb{S})$ 使得
\[
\|\boldsymbol{\tau}\|_2 \lesssim \|p_h\|, \quad \operatorname{div}\operatorname{div}\boldsymbol{\tau} = p_h.
\]
令 $\boldsymbol{\tau}_h = \boldsymbol{\tau}_I \in \Sigma_{k}^{\operatorname{div}\operatorname{div}}$. 由 \eqref{eq:IdivdivQcd} 可知,
\[
\operatorname{div}\operatorname{div}\boldsymbol{\tau}_h = Q_{k-2}(\operatorname{div}\operatorname{div}\boldsymbol{\tau}) = p_h.
\]
应用尺度论证,可得
\[
\|\boldsymbol{\tau}_h\|_{\operatorname{div}\operatorname{div}} \lesssim \|\boldsymbol{\tau}\|_2 \lesssim \|p_h\|.
\]
从而 \eqref{eq:newdivdivinfsup} 得证. 
\end{proof}

与文献~\cite{ChenHuang2022,ChenHuang2020a,ChenHuang2022a,HuangZhangZhouZhu2024,HuLiangMa2022,HuLiangMaZhang2024,HuMaZhang2021} 中构造的现有 $H(\operatorname{div}\operatorname{div})$-协调元相比,我们并不强加低维子单纯形上的法空间连续性,因此不需要条件 $k\geq d$ 来保证 inf-sup 条件. 然而,为了确保自由度 \eqref{eq:newdivdivdof5} 满足 $\operatorname{RM} = \ker(\operatorname{def})\subseteq \operatorname{ND}_{k-3}(T)$,我们仍需假设 $k\geq 3$. 下述注记进一步指出了 $k\leq 2$ 时 $\mathbb{P}_{k}(T;\mathbb{S})$ 空间的不可行性. 


\begin{remark}\label{rm:k=2}\rm
考虑线性多项式 $v\in \mathbb{P}_1(T)$. 利用恒等式 \eqref{eq:greenidentitydivdiv} 及事实 $\nabla^2 v = 0$,对于 $\boldsymbol{\tau}\in\mathbb{P}_k(T; \mathbb{S})$,有
\begin{equation}\label{eq:greenidentityP1}
(\operatorname{div}\operatorname{div}\boldsymbol{\tau}, v)_T = \sum_{F\in\partial T}\left[(\operatorname{tr}_2(\boldsymbol{\tau}), v)_F - (\boldsymbol{n}^{\intercal}\boldsymbol{\tau}\boldsymbol{n}, \partial_n v)_{F}\right] - \sum_{F\in\partial T}\sum_{e\in\partial F}(\boldsymbol{n}_{F,e}^{\intercal}\boldsymbol{\tau} \boldsymbol{n}, v)_e. 
\end{equation}
当 $k\leq 2$ 时,对于 $\boldsymbol{\tau}\in\mathbb{P}_k(T; \mathbb{S})$,有 $\operatorname{div}\operatorname{div}\boldsymbol{\tau}\in\mathbb{P}_0(T)$. 我们总可以选取一个非零函数 $v\in\mathbb{P}_1(T)\cap L_0^2(T)$(即均值为零的线性函数),使得 $(\operatorname{div}\operatorname{div}\boldsymbol{\tau}, v)_T = 0$. 这将导致
\[
\sum_{F\in\partial T}\left[(\operatorname{tr}_2(\boldsymbol{\tau}), v)_F - (\boldsymbol{n}^{\intercal}\boldsymbol{\tau}\boldsymbol{n}, \partial_n v)_{F}\right] - \sum_{F\in\partial T}\sum_{e\in\partial F}(\boldsymbol{n}_{F,e}^{\intercal}\boldsymbol{\tau} \boldsymbol{n}, v)_e = 0. 
\]
这意味着当 $k\leq 2$ 时,迹自由度 \eqref{eq:newdivdivdof1}-\eqref{eq:newdivdivdof3} 之间存在线性相关性,从而不是线性无关的. 这也反映了 $\operatorname{div}\operatorname{div}$ 算子的离散值域应当包含分片一次多项式空间.
\end{remark}


\subsubsection{Raviart-Thomas 型 $H(\div\div)$ 协调对称张量有限元}\label{sec:divdivconformingRT}
我们通过增加高阶内部矩来扩充 $\operatorname{div}\operatorname{div}$ 算子的值域. 
取 $k\geq 2$,定义形函数空间为:
\[
\Sigma_{k^+}(T;\mathbb{S}) := \mathbb{P}_k(T;\mathbb{S}) \oplus \boldsymbol{x}\boldsymbol{x}^{\intercal}\mathbb{H}_{k-1}(T).
\]
附加分量 $\boldsymbol{x}\boldsymbol{x}^{\intercal}\mathbb{H}_{k-1}(T)$ 将 $\operatorname{div}\operatorname{div}$ 算子的值域扩充至 $\mathbb{P}_{k-1}(T)$(注意到 $\operatorname{div}\operatorname{div}(\boldsymbol{x}\boldsymbol{x}^{\intercal}\mathbb{H}_{k-1}(T)) = \mathbb{H}_{k-1}(T)$). 相较于原值域 $\operatorname{div}\operatorname{div} \mathbb{P}_k(T;\mathbb{S}) = \mathbb{P}_{k-2}(T)$,其阶数提高了一阶. 

对于 $k\geq 3$,其自由度定义与 \eqref{eq:newdivdivS} 基本一致,唯一的区别在于我们将 \eqref{eq:newdivdivdof5} 中的内部自由度扩充为:
\begin{equation}
(\boldsymbol{\tau}, \operatorname{def}\boldsymbol{q})_T, \quad \text{其中} \quad \boldsymbol{q}\in\mathbb{P}_{k-2}(T;\mathbb{R}^d). \label{eq:newdivdivdofRT5}
\end{equation}
由此,自由度 $(\boldsymbol{\tau}, \operatorname{def}\boldsymbol{q})_T$ 的测试空间由 \eqref{eq:newdivdivdof5} 中的 $\boldsymbol{q} \in \operatorname{ND}_{k-3}(T) = \nabla \mathbb{P}_{k-2}(T)\oplus\mathbb{P}_{k-3}(T;\mathbb{K})\boldsymbol{x}$ 扩充至 $\mathbb{P}_{k-2}(T;\mathbb{R}^d) = \nabla \mathbb{P}_{k-1}(T)\oplus\mathbb{P}_{k-3}(T;\mathbb{K})\boldsymbol{x}$. 
此外,由于 $(\boldsymbol{x}\boldsymbol{x}^{\intercal}\mathbb{H}_{k-1}(T))\boldsymbol{n}|_F \in \mathbb{P}_{k}(F;\mathbb{R}^d)$,所有边界自由度 \eqref{eq:newdivdivdof1}-\eqref{eq:newdivdivdof4} 保持不变. 

对于 $k=2$ 的情形,自由度 \eqref{eq:newdivdivdofRT5} 涉及的空间满足 $\ker(\operatorname{def}) = \operatorname{ND}_{0}(T) \not\subseteq \mathbb{P}_{0}(T;\mathbb{R}^d)$. 针对 $\Sigma_{2^+}(T;\mathbb{S})$,我们提出如下自由度集合,这是对文献~\cite{ChenHuang2022a} 中构造的 $H(\operatorname{div}\operatorname{div})$-协调有限元进行重分配后的推广:
\begin{subequations}\label{eq:newdivdivSk2RT}
\begin{align}
(\operatorname{tr}_e(\boldsymbol{\tau}), q)_e, &\quad q\in \mathbb{P}_2(e), \quad \forall e\in \Delta_{d-2}(T),\label{eq:newdivdivdofk2RT1}\\
(\boldsymbol{n}^{\intercal}\boldsymbol{\tau} \boldsymbol{n}, q )_F, &\quad q\in \mathbb{P}_2(F), \quad \forall F\in \partial T,\label{eq:newdivdivdofk2RT2}\\
(\operatorname{tr}_2(\boldsymbol{\tau}), q)_F, &\quad q\in\mathbb{P}_{1}(F), \quad \forall F\in \partial T,\label{eq:newdivdivdofk2RT3}\\
(\Pi_f\boldsymbol{\tau}\boldsymbol{n}_{F_r}, \boldsymbol{q})_{f}, & \quad \boldsymbol{q}\in \mathbb{B}_2^{\operatorname{div}}(f), \quad \forall f= f_{0:r-2}\in \Delta_{r-2}(F_r), \, r= 3,\ldots, d,
\label{eq:newdivdivdofk2RT4}\\
(\boldsymbol{\tau}, \boldsymbol{q})_T, &\quad \boldsymbol{q}\in \ker(\boldsymbol{x}^{\intercal}\cdot\boldsymbol{x})\cap \mathbb{P}_{1}(T;\mathbb{S}),\label{eq:newdivdivdofk2RT5}
\end{align}
\end{subequations}
其中 $f_{0:r} = \operatorname{conv}(\texttt{v}_0, \texttt{v}_1, \ldots, \texttt{v}_{r})$ 表示由顶点 $\{\texttt{v}_0, \texttt{v}_1, \ldots, \texttt{v}_{r}\}$ 张成的 $r$ 维单纯形. 关于该空间唯一可解性的证明参见文献 \cite{ChenHuang2025} 中的定理 A.5.

对于 $k\geq 2$,定义全局空间:
\begin{align*}
\Sigma_{k^+}^{\operatorname{div}\operatorname{div}} &:= \big\{ \boldsymbol{\tau}\in L^2(\Omega;\mathbb{S}): \,  \boldsymbol{\tau}|_T\in \Sigma_{k^+}(T;\mathbb{S}) \quad \forall\,T\in\mathcal{T}_h, \\
&\qquad \text{ 自由度 \eqref{eq:newdivdivdof2}-\eqref{eq:newdivdivdof3} 是单值的}, \; [\operatorname{tr}_e(\boldsymbol{\tau})]|_e = 0 \quad \forall\,e\in \mathring{\mathcal{E}}_h \big\}.
\end{align*}
显然我们有 $\Sigma_{k^+}^{\operatorname{div}\operatorname{div}}\subset H(\operatorname{div}\operatorname{div}, \Omega;\mathbb{S})$. 

类似于利用典范插值算子 $I_h^{\operatorname{div}\operatorname{div}}$ 证明 \eqref{eq:newdivdivinfsup} 的过程,对于 $k\geq 2$,我们有如下 inf-sup 条件:
\begin{equation}\label{eq:newdivdivRTinfsup}
\inf_{p_h\in V^{-1}_{k-1}}\sup_{\boldsymbol{\tau}_h\in \Sigma_{k^+}^{\operatorname{div}\operatorname{div}}}
\frac{(\operatorname{div}\operatorname{div}\boldsymbol{\tau}_h, p_h)}{\|\boldsymbol{\tau}_h\|_{\operatorname{div}\operatorname{div}} \|p_h\|} = \alpha > 0.
\end{equation}

\subsubsection{Raviart-Thomas 型 $H(\div\div)$ 协调线性对称张量有限元}\label{sec:lowerdivdivconforming}

针对 $k=1$ 的情形,我们通过添加特定的二次和三次多项式来扩充 $\mathbb{P}_1(T;\mathbb{S})$ 空间. 取形函数空间为:
\begin{equation}
\Sigma_{1^{++}}(T;\mathbb{S}) := \mathbb{P}_1(T;\mathbb{S}) \oplus \operatorname{sym}(\boldsymbol{x}\otimes \mathbb{H}_1(T;\mathbb{R}^d)) \oplus \boldsymbol{x}\boldsymbol{x}^{\intercal}\mathbb{H}_{1}(T).
\label{eq:sigma1+}
\end{equation}
注意到值域 $\operatorname{div}\operatorname{div}(\boldsymbol{x}\boldsymbol{x}^{\intercal}\mathbb{H}_{1}(T)) = \mathbb{H}_{1}(T)$ 且 $\operatorname{div}\operatorname{div} \operatorname{sym}(\boldsymbol{x}\otimes \mathbb{H}_1(T;\mathbb{R}^d)) = \mathbb{P}_0(T)$. 因此,该空间的散度-散度像空间为 $\operatorname{div}\operatorname{div} \Sigma_{1^{++}}(T;\mathbb{S}) = \mathbb{P}_1(T)$. 

当 $\boldsymbol{\tau}\in \Sigma_{1^{++}}(T;\mathbb{S})$ 时,易知对于 $e\in \Delta_{d-2}(T)$ 有 $\operatorname{tr}_e(\boldsymbol{\tau})\in\mathbb{P}_1(e)$,且对于 $F\in \partial T$ 有 $(\boldsymbol{n}^{\intercal}\boldsymbol{\tau} \boldsymbol{n})|_F$ 和 $\operatorname{tr}_2(\boldsymbol{\tau})|_F$ 均属于 $\mathbb{P}_{1}(F)$. 据此,我们提出如下自由度:
\begin{subequations}\label{eq:newdivdivSreduce}
\begin{align}
(\operatorname{tr}_e(\boldsymbol{\tau}), q)_e, &\quad q\in \mathbb{P}_1(e), \; e\in \Delta_{d-2}(T),\label{eq:newdivdivSreduce1}\\
(\boldsymbol{n}^{\intercal}\boldsymbol{\tau} \boldsymbol{n}, q )_F, &\quad q\in \mathbb{P}_1(F), \; F\in \partial T,\label{eq:newdivdivSreduce2}\\
( \operatorname{tr}_2(\boldsymbol{\tau}), q)_F, &\quad q\in\mathbb{P}_{1}(F), \; F\in \partial T.\label{eq:newdivdivSreduce3}
\end{align}
\end{subequations}

\begin{lemma}\label{lm:1++}
自由度 \eqref{eq:newdivdivSreduce} 在空间 $\Sigma_{1^{++}}(T;\mathbb{S})$ 上是唯一可解的. 
\end{lemma}
\begin{proof}
自由度 \eqref{eq:newdivdivSreduce1}-\eqref{eq:newdivdivSreduce2} 实质上是 $\mathbb{P}_1(T; \mathbb{S})$ 的顶点自由度的重分配. 形函数空间 \eqref{eq:sigma1+} 中的扩充部分维数为 $d^2 + d$,而自由度 \eqref{eq:newdivdivSreduce3} 的数量为 $(d+1)d$. 因此,自由度 \eqref{eq:newdivdivSreduce} 的总数恰好等于 $\dim\Sigma_{1^{++}}(T;\mathbb{S}) = \frac{1}{2}d(d+1)(d+3)$. 

设 $\boldsymbol{\tau}\in\Sigma_{1^{++}}(T;\mathbb{S})$,并假设所有自由度 \eqref{eq:newdivdivSreduce} 均为零. 这意味着:
\begin{equation}\label{eq:vanishingtrace}    
\operatorname{tr}_1(\boldsymbol{\tau})=0,\quad \operatorname{tr}_2(\boldsymbol{\tau})=0,\quad Q_{\mathcal{N}_e}(\boldsymbol{\tau})=0 \quad \forall\,e\in\Delta_{d-2}(T).
\end{equation}
利用分部积分可推导出 $\operatorname{div}\operatorname{div}\boldsymbol{\tau}=0$. 由此可知 $\boldsymbol{\tau} \in \mathbb{P}_1(T;\mathbb{S}) + \operatorname{sym}(\boldsymbol{x}\otimes \mathbb{P}_1(T;\mathbb{R}^d))$. 

令 $\boldsymbol{\tau} = \boldsymbol{\tau}_1 + \operatorname{sym}(\boldsymbol{x}\otimes\boldsymbol{q})$,其中 $\boldsymbol{\tau}_1\in\mathbb{P}_1(T;\mathbb{S})$ 且 $\boldsymbol{q}\in \mathbb{H}_1(T;\mathbb{R}^d)$. 由 $\operatorname{div}\operatorname{div}\boldsymbol{\tau}=0$ 可知 $\operatorname{div}\boldsymbol{q}=0$. 
由于 $\operatorname{tr}_2(\boldsymbol{\tau}_1)$ 是分片常数,结合式 \eqref{eq:vanishingtrace} 中 $\operatorname{tr}_2(\boldsymbol{\tau})=0$ 的事实,对于任意面 $F\in\partial T$,必有 $\operatorname{tr}_2(\operatorname{sym}(\boldsymbol{x}\otimes\boldsymbol{q}))|_F\in\mathbb{P}_0(F)$. 利用恒等式 $\operatorname{div}(\boldsymbol{x}\boldsymbol{q}^{\intercal}) = \boldsymbol{q} + \boldsymbol{x}\operatorname{div}\boldsymbol{q}$,$\operatorname{div}(\boldsymbol{q}\boldsymbol{x}^{\intercal}) = (d+1)\boldsymbol{q}$,以及 $\operatorname{div}_F(\boldsymbol{x}\boldsymbol{q}\cdot\boldsymbol{n}) = d\boldsymbol{q}\cdot\boldsymbol{n}$,我们得到:
\[
\operatorname{tr}_2(\operatorname{sym}(\boldsymbol{x}\otimes\boldsymbol{q}))|_F = (d+1)\boldsymbol{q}\cdot\boldsymbol{n} + \frac{1}{2}\boldsymbol{x}\cdot\boldsymbol{n}(\operatorname{div}\boldsymbol{q} + \operatorname{div}_F\boldsymbol{q}) \in \mathbb{P}_0(F).
\]
这表明 $(\boldsymbol{q}\cdot\boldsymbol{n})|_F \in \mathbb{P}_0(F)$,即 $\boldsymbol{q}$ 属于最低次 Raviart-Thomas 空间 $\textrm{RT}$. 由于 $\boldsymbol{q} \in \mathbb{H}_1(T;\mathbb{R}^d) \cap \ker(\operatorname{div})$,从而推得 $\boldsymbol{q}=0$. 
此时 $\boldsymbol{\tau} \in \mathbb{P}_1(T;\mathbb{S})$. 式 \eqref{eq:vanishingtrace} 中的第三个等式意味着 $\boldsymbol{\tau}$ 在 $T$ 的所有顶点上为零,因此 $\boldsymbol{\tau}=0$. 
\end{proof} 

定义全局空间:
\begin{align*}
\Sigma_{1^{++}}^{\operatorname{div}\operatorname{div}} &:= \big\{ \boldsymbol{\tau}\in L^2(\Omega;\mathbb{S}):  \boldsymbol{\tau}|_T\in \Sigma_{1^{++}}(T;\mathbb{S}) \quad \forall\,T\in\mathcal{T}_h, \\
&\qquad \text{ 自由度 \eqref{eq:newdivdivSreduce2}-\eqref{eq:newdivdivSreduce3} 是单值的}, \; [\operatorname{tr}_e(\boldsymbol{\tau})]|_e = 0 \quad \forall\,e\in \mathring{\mathcal{E}}_h \big\}.
\end{align*}
显然 $\Sigma_{1^{++}}^{\operatorname{div}\operatorname{div}} \subset H(\operatorname{div}\operatorname{div}, \Omega;\mathbb{S})$. 
再次利用典范插值算子 $I_h^{\operatorname{div}\operatorname{div}}$,可建立如下 inf-sup 条件:
\begin{equation}\label{eq:newdivdivk1infsup}
\inf_{p_h\in V^{-1}_{1}}\sup_{\boldsymbol{\tau}_h\in \Sigma_{1^{++}}^{\operatorname{div}\operatorname{div}}}
\frac{(\operatorname{div}\operatorname{div}\boldsymbol{\tau}_h, p_h)}{\|\boldsymbol{\tau}_h\|_{\operatorname{div}\operatorname{div}} \|p_h\|} = \alpha > 0.
\end{equation}
在二维情形 $d=2$,文献~\cite{FuehrerHeuer2025} 给出了该有限元的构造.


\section{Hellan-Herrmann-Johnson 混合元方法}

设 $\Omega \subset \mathbb{R}^2$. 依据引理 \ref{lem:Hdivdivcontinuous},我们引入如下具有法向-法向连续性的 Sobolev 空间:
\[
\Sigma^{nn}(\mathcal{T}_h;\mathbb{S}) := \left\{ \bs\tau\in H^1(\mathcal{T}_h;\mathbb{S}) : \llbracket \boldsymbol{n}^{\intercal}\bs\tau\boldsymbol{n} \rrbracket|_e = 0 \quad \forall\,e\in\mathcal{E}_h^i \right\}.
\]
这里,分片 $H^1$ 空间定义为 $H^1(\mathcal{T}_h;\mathbb{S}) := \{ \bs\tau\in L^2(\mathcal{T}_h;\mathbb{S}) : \bs\tau|_T\in H^1(T;\mathbb{S}) \quad \forall\,T\in\mathcal{T}_h \}$. 

\begin{lemma}
对于任意 $\bs\tau\in\Sigma^{nn}(\mathcal{T}_h;\mathbb{S})$ 和 $v\in H_0^{2}(\Omega)$,下述格林公式成立:
\begin{equation}\label{eq:divdivnngreen}
\begin{aligned}
\langle \div\div\bs\tau, v\rangle &= -\sum_{T\in\mathcal{T}_h}(\div\bs\tau, \nabla v)_T + \sum_{T\in\mathcal{T}_h}(\boldsymbol{t}^{\intercal}\bs\tau\boldsymbol{n}, \partial_t v)_{\partial T} \\
&= \sum_{T\in\mathcal{T}_h}(\bs\tau, \nabla^2 v)_T - \sum_{T\in\mathcal{T}_h}(\boldsymbol{n}^{\intercal}\bs\tau\boldsymbol{n}, \partial_n v)_{\partial T}.
\end{aligned}
\end{equation}
\end{lemma}

\begin{proof}
对于任意 $\bs\tau\in\Sigma^{nn}(\mathcal{T}_h;\mathbb{S})$ 和光滑函数 $v\in C_0^{\infty}(\Omega)$,应用分部积分公式可得:
\begin{align*}
\langle \div\div\bs\tau, v\rangle &= (\bs\tau, \nabla^2v) = -\sum_{T\in\mathcal{T}_h}(\div\bs\tau, \nabla v)_T + \sum_{T\in\mathcal{T}_h}(\bs\tau\boldsymbol{n}, \nabla v)_{\partial T} \\
&= -\sum_{T\in\mathcal{T}_h}(\div\bs\tau, \nabla v)_T + \sum_{T\in\mathcal{T}_h}(\boldsymbol{t}^{\intercal}\bs\tau\boldsymbol{n}, \partial_t v)_{\partial T} + \sum_{T\in\mathcal{T}_h}(\boldsymbol{n}^{\intercal}\bs\tau\boldsymbol{n}, \partial_{n} v)_{\partial T}.
\end{align*}
由于 $\bs\tau$ 具有法向-法向连续性 $\llbracket \boldsymbol{n}^{\intercal}\bs\tau\boldsymbol{n}\rrbracket = 0$,且 $v$ 在 $\Omega$ 内部光滑,上式最后一项的求和为零. 因此,
\[
\langle \div\div\bs\tau, v\rangle = -\sum_{T\in\mathcal{T}_h}(\div\bs\tau, \nabla v)_T + \sum_{T\in\mathcal{T}_h}(\boldsymbol{t}^{\intercal}\bs\tau\boldsymbol{n}, \partial_t v)_{\partial T}.
\]
最后,鉴于 $C_0^{\infty}(\Omega)$ 在 $H_0^{2}(\Omega)$ 中的稠密性,等式 \eqref{eq:divdivnngreen} 对任意 $v\in H_0^{2}(\Omega)$ 均成立. 
\end{proof}


设 $k\geq 1$,我们定义如下有限元空间:
\begin{align*}
\Sigma_{k-1}^{nn}(\mathcal{T}_h;\mathbb{S})&:=\{\bs\tau\in \Sigma^{nn}(\mathcal{T}_h;\mathbb{S}):\bs\tau\in \mathbb P_{k-1}(T;\mathbb S)\quad\forall~T\in\mathcal T_h\},
\\
\mathring{V}_k^{\grad}(\mathcal{T}_h)&:=\left\{v\in H_0^1(\Omega): v|_T\in \mathbb P_{k}(T)\quad\forall~T\in\mathcal T_h\right\}.
\end{align*}
给定 $T\in\mathcal T_h$,由自由度 \eqref{eq:hhjnd} 知,空间 $\Sigma_{k-1}^{nn}(\mathcal{T}_h;\mathbb{S})$ 的局部自由度为:
\begin{itemize}
    \item 边自由度:
    \[
    \int_e (\boldsymbol{n}^{\intercal}\bs\tau\boldsymbol{n}) q \dd s, \quad \, q\in \mathbb{P}_{k-1}(e), \ e\in\Delta_1(T);
    \]
    \item 内部自由度:
    \[
    \int_T \bs\tau : \bs q \dx, \quad \, \bs q\in \mathbb{P}_{k-2}(T; \mathbb{S}).
    \]
\end{itemize}
其相应的基函数构造如下:
\begin{itemize}
    \item 关联于连接顶点 $\texttt{v}_i$ 和 $\texttt{v}_j$ 的边的基函数:
    \[
    \lambda_i^{\ell}\lambda_j^{k-1-\ell}\operatorname{sym}(\boldsymbol{t}_i\otimes\boldsymbol{t}_j), \quad \ell=0,1,\ldots, k-1;
    \]
    \item 关联于单元内部的基函数:
    \[
    \lambda_0^{\alpha_0}\lambda_1^{\alpha_1}\lambda_2^{\alpha_2}\lambda_{\ell}\operatorname{sym}(\boldsymbol{t}_i\otimes \boldsymbol{t}_j),
    \]
    其中 $(ij\ell)$ 是 $(012)$ 的循环置换, 指数满足 $\alpha_0+\alpha_1+\alpha_2=k-2$. 
\end{itemize}
$k$ 次 Lagrange 元空间 $\mathring{V}_k^{\grad}(\mathcal{T}_h)$ 的局部自由度为:
\begin{equation*}
v(\delta),\quad \delta\in\Delta_0(T);\quad \int_evq\dd s,\quad q\in \mathbb P_{k-2}(e),  e\in\Delta_1(T);\quad \int_Tvq\dx,\quad q\in \mathbb P_{k-3}(T).
\end{equation*}
% \begin{itemize}
% \item $v(\delta)$ \qquad\; $\forall\,\delta\in\Delta_0(T)$;
% \item $\int_evq\dd s\quad\forall\,q\in \mathbb P_{k-2}(e), e\in\Delta_1(T)$;
% \item $\int_Tvq\dx\quad\forall\,q\in \mathbb P_{k-3}(T)$.
% \end{itemize}

为了建立离散混合变分形式,我们引入弱双散度算子 $(\div\div)_w: \Sigma^{nn}(\mathcal{T}_h;\mathbb{S})\to \mathring{V}_k^{\grad}(\mathcal{T}_h)$,其定义如下:
\begin{equation*}
((\div\div)_w\boldsymbol{\tau}, v) := -\sum_{T\in\mathcal{T}_h}(\div\boldsymbol{\tau}, \nabla v)_T + \sum_{T\in\mathcal{T}_h}(\boldsymbol{t}^{\intercal}\boldsymbol{\tau}\boldsymbol{n}, \partial_t v)_{\partial T}.
\end{equation*}
应用分部积分公式,该定义等价于
\begin{equation*}
((\div\div)_w\boldsymbol{\tau}, v) = \sum_{T\in\mathcal{T}_h}(\boldsymbol{\tau}, \nabla^2v)_T - \sum_{T\in\mathcal{T}_h}(\boldsymbol{n}^{\intercal}\boldsymbol{\tau}\boldsymbol{n}, \partial_n v)_{\partial T}.
\end{equation*}


基于分布混合变分形式 \eqref{eq:hhjmixedformulation} 的 Hellan-Herrmann-Johnson (HHJ) 混合有限元方法为: 找 $\boldsymbol{\sigma}_h\in \Sigma_{k-1}^{nn}(\mathcal{T}_h;\mathbb{S})$ 和 $u_h\in \mathring{V}_k^{\grad}(\mathcal{T}_h)$ 满足
\begin{subequations}\label{eq:hhjmfem}
\begin{align}
(\boldsymbol\sigma_h, \boldsymbol\tau_h)+ ((\div\div)_w\boldsymbol\tau_h, u_h) & =0 \quad\quad\quad\quad \forall~\boldsymbol\tau_h\in\Sigma_{k-1}^{nn}(\mathcal{T}_h;\mathbb{S}), \label{eq:hhjmfem1}\\
((\div\div)_w\boldsymbol\sigma_h, v_h) & =-\langle f, v_h\rangle  \quad \forall~v_h\in \mathring{V}_k^{\grad}(\mathcal{T}_h). \label{eq:hhjmfem2}
\end{align}
\end{subequations}
% 其中
% \[
% b_h(\boldsymbol\sigma_h, v_h):=-\sum_{T\in\mathcal T_h}(\div\boldsymbol\sigma_h, \nabla v_h)_T+\sum_{T\in\mathcal T_h}(\boldsymbol{t}^{\intercal}\boldsymbol\sigma_h\boldsymbol{n}, \partial_t v_h)_{\partial T}.
% \]
% 应用分部积分,有
% \[
% b_h(\boldsymbol\sigma_h, v_h)=\sum_{T\in\mathcal T_h}(\boldsymbol\sigma_h, \nabla^2v_h)_T-\sum_{T\in\mathcal T_h}(\boldsymbol{n}^{\intercal}\boldsymbol\sigma_h\boldsymbol{n}, \partial_{n} v_h)_{\partial T}.
% \]

为此,我们引入如下网格依赖范数:
\begin{align*}
\|\boldsymbol{\tau}\|_{0,h}^2 &:= \|\boldsymbol{\tau}\|^2 + \sum_{e\in\Delta_1(\mathcal{T}_h)}h_e\|\boldsymbol{n}^{\intercal}\boldsymbol{\tau}\boldsymbol{n}\|_{0,e}^2, \quad \forall~\boldsymbol{\tau}\in\Sigma^{nn}(\mathcal{T}_h;\mathbb{S}),\\
\|v\|_{2,h}^2 &:= \sum_{T\in\mathcal{T}_h}|v|_{2,T}^2 + \sum_{e\in\Delta_1(\mathcal{T}_h)}h_e^{-1}\|\llbracket\partial_{n_e}v\rrbracket\|_{0,e}^2, \quad \forall~v\in V:=H^2(\mathcal{T}_h)\cap H_0^1(\Omega).
\end{align*}
由离散 Poincar\'e 不等式可得
\begin{equation}\label{eq:H2brokenpoincareineqlty}
\|v\|_1\lesssim \|v\|_{2,h},\quad \forall~v\in V.
\end{equation}
同时引入离散双线性形式 $b_h(\cdot, \cdot): H^1(\mathcal{T}_h;\mathbb{S})\times H^2(\mathcal{T}_h)\to\mathbb{R}$,定义为:
\[
b_h(\boldsymbol{\tau}, v) := -\sum_{T\in\mathcal{T}_h}(\div\boldsymbol{\tau}, \nabla v)_T + \sum_{T\in\mathcal{T}_h}(\boldsymbol{t}^{\intercal}\boldsymbol{\tau}\boldsymbol{n}, \partial_t v)_{\partial T}
= \sum_{T\in\mathcal{T}_h}(\boldsymbol{\tau}, \nabla^2v)_T - \sum_{T\in\mathcal{T}_h}(\boldsymbol{n}^{\intercal}\boldsymbol{\tau}\boldsymbol{n}, \partial_n v)_{\partial T}.
\]
显然,成立
\[
((\div\div)_w\boldsymbol\tau, v)=b_h(\boldsymbol\tau, v)\quad\forall\,\boldsymbol\tau\in\Sigma^{nn}(\mathcal{T}_h;\mathbb{S}),\; v\in\mathring{V}_k^{\grad}(\mathcal{T}_h).
\]

\subsection{插值算子}

为了推导 HHJ 混合元方法 \eqref{eq:hhjmfem} 的误差估计,我们需要引入特定的插值算子. 
首先,定义算子 $I_h^{nn}: \Sigma^{nn}(\mathcal{T}_h;\mathbb{S})\to \Sigma_{k-1}^{nn}(\mathcal{T}_h;\mathbb{S})$ 如下(参见 \cite{BabuvskaOsbornPitkaranta1980,FalkOsborn1980,Comodi1989,BoffiBrezziFortin2013}):对于给定 $\boldsymbol{\tau}\in\Sigma^{nn}(\mathcal{T}_h;\mathbb{S})$,在任意单元 $T\in \mathcal{T}_h$ 及其任意边 $e$ 上满足
\[
\int_e\boldsymbol{n}^{\intercal}(\boldsymbol{\tau}-I_h^{nn}\boldsymbol{\tau})\boldsymbol{n}\,q\dd s = 0, \quad \forall~q\in \mathbb{P}_{k-1}(e),
\]
\[
\int_T(\boldsymbol{\tau}-I_h^{nn}\boldsymbol{\tau}):\boldsymbol{q}\dx = 0, \quad \forall~\boldsymbol{q}\in \mathbb{P}_{k-2}(T;\mathbb{S}).
\]
其次,定义算子 $I_h^{\grad}: V\to \mathring{V}_k^{\grad}(\mathcal{T}_h)$ 如下(参见 \cite{BabuvskaOsbornPitkaranta1980,FalkOsborn1980,Comodi1989,Stenberg1991}):对于给定 $w\in V$,在任意单元 $T\in \mathcal{T}_h$、任意顶点 $\delta$ 和任意边 $e$ 上满足
\[
I_h^{\grad}w(\delta) = w(\delta),
\]
\[
\int_e(w-I_h^{\grad}w)v\dd s = 0, \quad \forall~v\in \mathbb{P}_{k-2}(e),
\]
\[
\int_T(w-I_h^{\grad}w)v\dx = 0, \quad \forall~v\in \mathbb{P}_{k-3}(T).
\]

以下引理总结了插值算子 $I_h^{nn}$ 和 $I_h^{\grad}$ 的逼近性质(参见 \cite{BabuvskaOsbornPitkaranta1980,FalkOsborn1980,Comodi1989,Stenberg1991}). 

\begin{lemma}\label{lem:hhjinterpolation}
对于任意非负整数 $m$,若 $v\in H^{m+3}(\Omega)$ 且 $\boldsymbol{\tau}\in H^{m+1}(\Omega; \mathbb{S})$,则对于任意 $T\in\mathcal{T}_h$,成立如下误差估计:
\begin{align}
\label{eq:hhjInnestimate}
\|\boldsymbol{\tau}-I_h^{nn}\boldsymbol{\tau}\|_{0,T} + h_T^{1/2}|\boldsymbol{\tau}-I_h^{nn}\boldsymbol{\tau}|_{0,\partial T} &\lesssim h_T^{\min\{m+1,k\}}|\boldsymbol{\tau}|_{m+1,T}, \\
\label{eq:hhjIgradestimate}
h_T^{3/2}\|\nabla(v-I_h^{\grad}v)\|_{0,\partial T}+\sum_{i=0}^2h_T^i|v-I_h^{\grad}v|_{i,T}  &\lesssim h^{\min\{m+2,k\}+1}_T|v|_{m+3,T}.
\end{align}
\end{lemma}

此外,上述插值算子还满足如下恒等式. 

\begin{lemma}
成立如下交换性质:
\begin{equation}\label{eq:bpi}
(\div\div)_w(I_h^{nn}\boldsymbol{\tau}) = (\div\div)_w\boldsymbol{\tau}, \quad \forall~\boldsymbol{\tau}\in \Sigma^{nn}(\mathcal{T}_h;\mathbb{S}),
\end{equation}
以及
\begin{equation}\label{eq:bp0}
b_h(\boldsymbol{\tau}, v-I_h^{\grad}v) = 0, \quad \forall~\boldsymbol{\tau}\in \Sigma_{k-1}^{nn}(\mathcal{T}_h;\mathbb{S}), \, v\in V.
\end{equation}
\end{lemma}

\begin{proof}
利用 $(\div\div)_w$ 和 $b_h(\cdot, \cdot)$ 的表达式,结合插值算子 $I_h^{nn}$ 和 $I_h^{\grad}$ 的定义(即对特定多项式空间的积分为零),直接推导即可得证. 
\end{proof}


\subsection{适定性和误差分析}

\begin{lemma}[引理 4.2,文献 \cite{HuangHuangXu2011}]
成立如下 inf-sup 条件:
\begin{equation}\label{eq:HHJinfsup}
\|v_h\|_{2,h}\lesssim \sup_{\bs\tau_h\in\Sigma_{k-1}^{nn}(\mathcal{T}_h;\mathbb{S})}\frac{((\div\div)_w\bs\tau_h, v_h)}{\|\bs\tau_h\|_{0,h}}\quad\forall~v_h\in \mathring{V}_k^{\grad}(\mathcal{T}_h).
\end{equation}
\end{lemma}

\begin{proof}
设 $\bs\tau_h\in\Sigma_{k-1}^{nn}(\mathcal{T}_h;\mathbb{S})$ 使得对于任意 $T\in\mathcal T_h$ 和 $e\in\Delta_1(\mathcal{T}_h)$,满足
\begin{align*}
\boldsymbol{n}^{\intercal}\bs\tau\boldsymbol{n}&=-h_e^{-1}\llbracket\partial_{n_e}v_h\rrbracket\quad\textrm{ 在 }\, e \textrm{ 上},
\\
\int_T\bs\tau_h:\boldsymbol{q}\dx&=\int_T\nabla^2v_h:\boldsymbol{q}\dx\quad\forall~\boldsymbol{q}\in \mathbb P_{k-2}(T;\mathbb S).
\end{align*}
利用尺度变换论证,可得
\begin{equation}\label{eq:temp20180429-1}
\|\bs\tau_h\|_{0,h}\lesssim \|v_h\|_{2,h}.
\end{equation}
此外,我们有
\[
((\div\div)_w\bs\tau_h, v_h)=\sum_{T\in\mathcal T_h}(\boldsymbol\tau_h, \nabla^2v_h)_T+\sum_{T\in\mathcal T_h}(\boldsymbol{n}^{\intercal}\bs\tau_h\boldsymbol{n}, \partial_{n} v_h)_{\partial T}=\|v_h\|_{2,h}^2,
\]
结合 \eqref{eq:temp20180429-1} 即推出 \eqref{eq:HHJinfsup}. 
\end{proof}

\begin{theorem}
HHJ 混合方法 \eqref{eq:hhjmfem} 是适定的,且满足
\begin{equation}\label{eq:divdivsigmah}
(\div\div)_w(I_h^{nn}\boldsymbol{\sigma}-\boldsymbol{\sigma}_h)=(\div\div)_w(\boldsymbol{\sigma}-\boldsymbol{\sigma}_h)=0.
\end{equation}
\end{theorem}

\begin{proof}
显然,双线性形式 $(\cdot, \cdot)$ 和 $((\div\div)_w\cdot, \cdot)$ 关于网格依赖范数 $\|\cdot\|_{0,h}$ 和 $\|\cdot\|_{2,h}$ 是连续的. 
根据逆不等式,我们有
\[
\|\bs\tau_h\|_{0,h}^2\lesssim \|\bs\tau_h\|^2=(\bs\tau_h,\bs\tau_h)\quad\forall~\bs\tau_h\in\Sigma_{k-1}^{nn}(\mathcal{T}_h;\mathbb{S}).
\]
结合离散 inf-sup 条件 \eqref{eq:HHJinfsup},由 Brezzi 理论可得 HHJ 混合方法 \eqref{eq:hhjmfem} 的适定性.

对任意 $v\in \mathring{V}_k^{\grad}(\mathcal{T}_h)$,
由 \eqref{eq:hhjmfem2} 和 \eqref{eq:hhjmixedformulation2} 可得,
\begin{equation*}
((\div\div)_w(\boldsymbol{\sigma}-\boldsymbol{\sigma}_h),v) = -(\div\boldsymbol{\sigma}, \nabla v) + \langle f, v\rangle = -(\div\boldsymbol{\sigma}, \nabla v) - \langle \div\div\boldsymbol{\sigma}, v\rangle=0.
\end{equation*}
再结合 \eqref{eq:bpi} 可知 \eqref{eq:divdivsigmah} 成立.
\end{proof}

\begin{theorem}
设 $(\boldsymbol{\sigma} , u)\in H^{-1}(\div\div,\Omega; \mathbb{S})\times H_0^1(\Omega)$ 是 HHJ 分布混合变分形式 \eqref{eq:hhjmixedformulation} 的解,$(\boldsymbol{\sigma}_h, u_h)\in\Sigma_{k-1}^{nn}(\mathcal{T}_h;\mathbb{S})\times \mathring{V}_k^{\grad}(\mathcal{T}_h)$ 是 HHJ 混合有限元方法 \eqref{eq:hhjmfem} 的解. 
假设对于某个非负整数 $m$,有 $\boldsymbol{\sigma}\in H^{m+1}(\Omega; \mathbb{S})$ 和 $u\in H^{m+3}(\Omega)$. 
则成立如下误差估计:
\begin{equation}\label{eq:hhjenergyerror}
\|\bs\sigma-\bs\sigma_h\|_{0,h}+\|I_h^{\grad}u-u_h\|_{2,h}\lesssim h^{\min\{m+1,k\}}\|\bs\sigma\|_{m+1},
\end{equation}
\begin{equation}\label{eq:hhjuhH2}
\|u-u_h\|_{2,h}\lesssim h^{\min\{m+1,k-1\}}\|u\|_{m+3},
\end{equation}
\begin{equation}\label{eq:hhjuhH1}
|u-u_h|_{1}\lesssim h^{\min\{m+1,k\}}\|u\|_{m+3}.
\end{equation}
\end{theorem}

\begin{proof}
利用分部积分,由问题 \eqref{eq:biharmonic2variables} 可得
\begin{align*}
(\boldsymbol\sigma, \boldsymbol\tau_h)+ b_h(\boldsymbol\tau_h, u) & =0 \quad\quad \forall~\boldsymbol\tau_h\in\Sigma_{k-1}^{nn}(\mathcal{T}_h;\mathbb{S}).
\end{align*}
将上式减去 \eqref{eq:hhjmfem1},得到如下正交性:
\begin{align*}
(\boldsymbol\sigma-\boldsymbol\sigma_h, \boldsymbol\tau_h)+ b_h(\boldsymbol\tau_h, u-u_h) & =0 \quad\quad \forall~\boldsymbol\tau_h\in\Sigma_{k-1}^{nn}(\mathcal{T}_h;\mathbb{S}). 
\end{align*}
结合 \eqref{eq:bp0} 可知,
\begin{align}
(\boldsymbol\sigma-\boldsymbol\sigma_h, \boldsymbol\tau_h)+ ((\div\div)_w\boldsymbol\tau_h, I_h^{\grad}u-u_h) & =0 \quad\quad \forall~\boldsymbol\tau_h\in\Sigma_{k-1}^{nn}(\mathcal{T}_h;\mathbb{S}). \label{eq:temp20180502-1}
\end{align}
在上式中取 $\boldsymbol\tau_h=I_h^{nn}\boldsymbol{\sigma}-\boldsymbol{\sigma}_h$ 可得正交性:
\begin{equation*}
(\boldsymbol\sigma-\boldsymbol\sigma_h, I_h^{nn}\boldsymbol{\sigma}-\boldsymbol{\sigma}_h)=0.
\end{equation*}
因此,
$$
\|I_h^{nn}\boldsymbol{\sigma}-\boldsymbol{\sigma}_h\|^2=(I_h^{nn}\boldsymbol{\sigma}-\boldsymbol{\sigma}, I_h^{nn}\boldsymbol{\sigma}-\boldsymbol{\sigma}_h)\leq\|I_h^{nn}\boldsymbol{\sigma}-\boldsymbol{\sigma}\|\|I_h^{nn}\boldsymbol{\sigma}-\boldsymbol{\sigma}_h\|.
$$
利用逆不等式,我们有
\begin{equation*}
\|I_h^{nn}\boldsymbol{\sigma}-\boldsymbol{\sigma}_h\|_{0,h}\lesssim \|I_h^{nn}\boldsymbol{\sigma}-\boldsymbol{\sigma}_h\|\leq \|I_h^{nn}\boldsymbol{\sigma}-\boldsymbol{\sigma}\|.
\end{equation*}
结合三角不等式和插值误差估计 \eqref{eq:hhjInnestimate} 可得,
$$
\|\bs\sigma-\bs\sigma_h\|_{0,h}\lesssim h^{\min\{m+1,k\}}\|\bs\sigma\|_{m+1}.
$$
由离散 inf-sup 条件 \eqref{eq:HHJinfsup} 以及 \eqref{eq:temp20180502-1},有
\begin{align*}
\|I_h^{\grad}u-u_h\|_{2,h}&\lesssim \sup_{\bs\tau\in\Sigma_{k-1}^{nn}(\mathcal{T}_h;\mathbb{S})}\frac{((\div\div)_w\bs\tau, I_h^{\grad}u-u_h)}{\|\bs\tau\|_{0,h}} \\
&= \sup_{\bs\tau\in\Sigma_{k-1}^{nn}(\mathcal{T}_h;\mathbb{S})}\frac{(\boldsymbol\sigma_h-\boldsymbol\sigma, \boldsymbol\tau_h)}{\|\bs\tau\|_{0,h}}\leq\|\boldsymbol\sigma_h-\boldsymbol\sigma\|.
\end{align*}
结合上述两个等式即可推出 \eqref{eq:hhjenergyerror}. 

利用三角不等式、\eqref{eq:hhjenergyerror} 以及 \eqref{eq:hhjIgradestimate} 可得 \eqref{eq:hhjuhH2}. 

利用 \eqref{eq:H2brokenpoincareineqlty} 和 \eqref{eq:hhjenergyerror} 可得
\begin{equation*}
|I_h^{\grad}u-u_h|_{1}\lesssim\|I_h^{\grad}u-u_h\|_{2,h}\lesssim h^{\min\{m+1,k\}}\|\bs\sigma\|_{m+1}.
\end{equation*}
结合三角不等式、上式以及 \eqref{eq:hhjIgradestimate} 即可证得 \eqref{eq:hhjuhH1}. 
\end{proof}

利用对偶论证,我们可以进一步建立 $u_h$ 和 $I_h^{\grad}u$ 之间在 $H^1$ 范数下的超收敛误差估计. 
设 $(\widetilde{\boldsymbol{\sigma}},\widetilde{u})$ 是如下对偶问题的解:
\begin{equation}\label{eq:hhjdual1}
\left\{
\begin{array}{ll}
\widetilde{\boldsymbol{\sigma}}=-\nabla^2\widetilde{u} & \text{在}\ \Omega\ \text{中}, \\
\div\div\widetilde{\boldsymbol{\sigma}}
=\Delta(I_h^{\grad}u-u_h) & \text{在}\ \Omega\ \text{中}, \\
\widetilde{u}=\partial_{n}\widetilde{u}=0 & \text{在}\ \partial
\Omega\ \text{上}.
\end{array}
\right.
\end{equation}
由于 $\triangle(I_h^{\grad}u-u_h)\not\in L^2(\Omega)$,\eqref{eq:hhjdual1} 的第二个方程应由以下弱形式关系理解:
\begin{equation}
\int_\Omega (\div\widetilde{\boldsymbol{\sigma}})\cdot
\nabla v\dx=\int_\Omega \nabla(I_h^{\grad}u-u_h)\cdot
\nabla v\dx\quad \forall\,v\in H^1_0(\Omega).
\label{hhjdual2}
\end{equation}
假设 $\widetilde{u}\in H^3(\Omega)\cap H_0^2(\Omega)$ 且满足正则性估计:
\begin{equation}
\label{hhjregularity}
\|\widetilde{\boldsymbol{\sigma}}\|_{1}+\|\widetilde{u}\|_{3}\lesssim |I_h^{\grad}u-u_h|_{1}.
\end{equation}
当 $\Omega$ 是凸有界多边形区域时,此正则性结果已在 \cite{Dauge1988,Grisvard1992} 中获得. 

\begin{theorem}
设 $(\boldsymbol{\sigma} , u)\in H^{-1}(\div\div,\Omega; \mathbb{S})\times H_0^1(\Omega)$ 是 HHJ 分布混合变分形式 \eqref{eq:hhjmixedformulation} 的解,$(\boldsymbol{\sigma}_h, u_h)\in \Sigma_{k-1}^{nn}(\mathcal{T}_h;\mathbb{S})\times \mathring{V}_k^{\grad}(\mathcal{T}_h)$ 是 HHJ 混合有限元方法 \eqref{eq:hhjmfem} 的解. 
假设正则性条件 \eqref{hhjregularity} 成立,且对于某个非负整数 $m$,$\boldsymbol{\sigma}\in  H^{m+1}(\Omega; \mathbb{S})$ 和 $u\in H^{m+3}(\Omega)$. 
则
\begin{equation}\label{eq:hhjH1supererror}
|I_h^{\grad}u-u_h|_{1}\lesssim h^{\min\{m+1,k\}+1}(\|\bs\sigma\|_{m+1}+\delta_{k1}\|f\|).
\end{equation}
\end{theorem}
\begin{proof}
在 \eqref{hhjdual2} 中取 $v=I_h^{\grad}u-u_h$,由 \eqref{eq:bpi}-\eqref{eq:bp0} 和 \eqref{eq:temp20180502-1} 可得
\begin{align*}
|I_h^{\grad}u-u_h|_1^2&=(\div\widetilde{\boldsymbol{\sigma}}, \nabla(I_h^{\grad}u-u_h))=-b_h(\widetilde{\boldsymbol{\sigma}}, I_h^{\grad}u-u_h)\\
&=-b_h(I_h^{nn}\widetilde{\boldsymbol{\sigma}}, I_h^{\grad}u-u_h)=-b_h(I_h^{nn}\widetilde{\boldsymbol{\sigma}}, u-u_h) \\
&=(\boldsymbol\sigma-\boldsymbol\sigma_h, I_h^{nn}\widetilde{\boldsymbol{\sigma}}).
\end{align*}
利用 \eqref{eq:hhjdual1}, \eqref{eq:divdivsigmah} 和 \eqref{eq:bp0},我们有
\begin{align*}
(\boldsymbol\sigma-\boldsymbol\sigma_h, \widetilde{\boldsymbol{\sigma}})&=-(\boldsymbol\sigma-\boldsymbol\sigma_h, \nabla^2\widetilde{u}) =-b_h(\boldsymbol\sigma-\boldsymbol\sigma_h, \widetilde{u}) \\
&=-b_h(\boldsymbol\sigma-\boldsymbol\sigma_h, \widetilde{u}-I_h^{\grad}\widetilde{u}) =-b_h(\boldsymbol\sigma, \widetilde{u}-I_h^{\grad}\widetilde{u}).
\end{align*}
因此,
\begin{align}
|I_h^{\grad}u-u_h|_1^2&=(\boldsymbol\sigma-\boldsymbol\sigma_h, I_h^{nn}\widetilde{\boldsymbol{\sigma}})=(\boldsymbol\sigma-\boldsymbol\sigma_h, I_h^{nn}\widetilde{\boldsymbol{\sigma}}-\widetilde{\boldsymbol{\sigma}})+(\boldsymbol\sigma-\boldsymbol\sigma_h, \widetilde{\boldsymbol{\sigma}}) \notag\\
&=(\boldsymbol\sigma-\boldsymbol\sigma_h, I_h^{nn}\widetilde{\boldsymbol{\sigma}}-\widetilde{\boldsymbol{\sigma}})-b_h(\boldsymbol\sigma, \widetilde{u}-I_h^{\grad}\widetilde{u}).\label{eq:temp20180502-3}
\end{align}

若 $k=1$,由 \eqref{eq:temp20180502-3},\eqref{eq:hhjmixedformulation2},引理 \ref{lem:hhjinterpolation} 和 \eqref{eq:hhjenergyerror} 可得
\begin{align*}
|I_h^{\grad}u-u_h|_1^2&=(\boldsymbol\sigma-\boldsymbol\sigma_h, I_h^{nn}\widetilde{\boldsymbol{\sigma}}-\widetilde{\boldsymbol{\sigma}})+(f, \widetilde{u}-I_h^{\grad}\widetilde{u}) \\
&\lesssim \|\boldsymbol\sigma-\boldsymbol\sigma_h\|\|I_h^{nn}\widetilde{\boldsymbol{\sigma}}-\widetilde{\boldsymbol{\sigma}}\| + \|f\|\|\widetilde{u}-I_h^{\grad}\widetilde{u}\|
\\
&\lesssim h\|\boldsymbol\sigma-\boldsymbol\sigma_h\|\|\widetilde{\boldsymbol{\sigma}}\|_1 +h^2\|f\|\|\widetilde{u}\|_2\lesssim (h\|\boldsymbol\sigma-\boldsymbol\sigma_h\|+h^2\|f\|)(\|\widetilde{\boldsymbol{\sigma}}\|_1 +\|\widetilde{u}\|_2) \\
&\lesssim h^2(\|\boldsymbol\sigma\|_1+\|f\|)(\|\widetilde{\boldsymbol{\sigma}}\|_1 +\|\widetilde{u}\|_2)
\end{align*}
由 \eqref{hhjregularity} 可知
\[
|I_h^{\grad}u-u_h|_1\lesssim h^2(\|\boldsymbol\sigma\|_1+\|f\|).
\]

若 $k\geq2$,由 \eqref{eq:temp20180502-3}, \eqref{eq:bp0} 和引理 \ref{lem:hhjinterpolation},
\begin{align*}
|I_h^{\grad}u-u_h|_1^2&=(\boldsymbol\sigma-\boldsymbol\sigma_h, I_h^{nn}\widetilde{\boldsymbol{\sigma}}-\widetilde{\boldsymbol{\sigma}})-b_h(\boldsymbol\sigma, \widetilde{u}-I_h^{\grad}\widetilde{u}) \\
&=(\boldsymbol\sigma-\boldsymbol\sigma_h, I_h^{nn}\widetilde{\boldsymbol{\sigma}}-\widetilde{\boldsymbol{\sigma}})-b_h(\boldsymbol\sigma-I_h^{nn}\boldsymbol\sigma, \widetilde{u}-I_h^{\grad}\widetilde{u}) \\
&\lesssim \|\boldsymbol\sigma-\boldsymbol\sigma_h\|\|I_h^{nn}\widetilde{\boldsymbol{\sigma}}-\widetilde{\boldsymbol{\sigma}}\|+\|\boldsymbol\sigma-I_h^{nn}\boldsymbol\sigma\|_{0,h}\|\widetilde{u}-I_h^{\grad}\widetilde{u}\|_{2,h} \\
&\lesssim h\|\boldsymbol\sigma-\boldsymbol\sigma_h\|\|\widetilde{\boldsymbol{\sigma}}\|_1+h\|\boldsymbol\sigma-I_h^{nn}\boldsymbol\sigma\|_{0,h}\|\widetilde{u}\|_{3} \\
&\lesssim h(\|\boldsymbol\sigma-\boldsymbol\sigma_h\|+\|\boldsymbol\sigma-I_h^{nn}\boldsymbol\sigma\|_{0,h})(\|\widetilde{\boldsymbol{\sigma}}\|_1+\|\widetilde{u}\|_{3}).
\end{align*}
由 \eqref{hhjregularity} 和 \eqref{eq:hhjenergyerror} 可得
\begin{align*}
|I_h^{\grad}u-u_h|_1\lesssim h(\|\boldsymbol\sigma-\boldsymbol\sigma_h\|+\|\boldsymbol\sigma-I_h^{nn}\boldsymbol\sigma\|_{0,h})\lesssim h^{\min\{m+1,k\}+1}\|\bs\sigma\|_{m+1}.
\end{align*}
证毕.
\end{proof}

\subsection{后处理}
本节我们将利用 \eqref{eq:hhjenergyerror} 中的最优力矩误差估计和 \eqref{eq:hhjH1supererror} 中的超收敛结果,构造挠度 $u$ 的一个新的超收敛逼近. 
对于任意整数 $l\geq1$,令 $I_h^{l}$ 为关于 $\mathcal{T}_h$ 的分片 $l$ 次 Lagrange 插值算子 \cite{Ciarlet1978,BrennerScott2008}. 
记
\[
\mathbb P_{k+1}(\mathcal{T}_h):=\left\{v\in L^2(\Omega): v|_T\in \mathbb P_{k+1}(T), \quad \forall\,T\in\mathcal{T}_h\right\}.
\]
基于该空间,我们定义 $u$ 的一个新的分片逼近 $u_h^{\ast}\in \mathbb P_{k+1}(\mathcal{T}_h)$ 为如下局部问题的解:对于任意 $T\in\mathcal{T}_h$,
\begin{equation}\label{eq:postprocess1}
u_h^{\ast}(\texttt{v}_i)=u_h(\texttt{v}_i),\quad i=1,2,3,
\end{equation}
% 且满足
\begin{align}
\int_T\nabla^2u_h^{\ast}:\nabla^2v\dx=-\int_T\boldsymbol{\sigma}_h:\nabla^2v\dx \label{eq:postprocess2}
\end{align}
对于任意满足 $v(\texttt{v}_i)=0~(i=1,2,3)$ 的 $v\in \mathbb P_{k+1}(T)$ 均成立,其中 $\{\texttt{v}_i\}_{i=1}^3$ 表示单元 $T$ 的三个顶点. 
由条件 \eqref{eq:postprocess1} 以及 $I_h^1$ 的性质,我们有
\begin{equation}\label{eq:postproestimate1}
|I_h^1(I_h^{k+1}u-u_h^{\ast})|_{1}=|I_h^1(I_h^{\grad}u-u_h)|_{1}\lesssim |I_h^{\grad}u-u_h|_{1}.
\end{equation}
令 $z:=(I-I_h^1)(I_h^{k+1}u-u_h^{\ast})$. 
易知 $I_h^1z=0$,$z\in \mathbb P_{k+1}(\mathcal{T}_h)$,且在剖分 $\mathcal{T}_h$ 的每个顶点 $\texttt{v}$ 处均有 $z(\texttt{v})=0$. 
因此,根据 $I_h^1$ 的插值误差估计可知
\begin{equation}\label{eq:z1}
\|z\|_{0, T}+h_T|z|_{1,T}=\|z-I_h^1z\|_{0,T}+h_T|z-I_h^1z|_{1,T}\lesssim h_T^2|z|_{2,T}.
\end{equation}

\begin{theorem}\label{thm:uuhstar1}
假设正则性条件 \eqref{hhjregularity} 成立,且对于某个非负整数 $m$,有 $\boldsymbol{\sigma}\in H^{m+1}(\Omega; \mathbb{S})$ 以及 $u\in H^{m+3}(\Omega)$. 则成立如下估计:
\[
|u-u_h^{\ast}|_{1,h}\lesssim h^{\min\{m+1,k\}+1}(\|\boldsymbol{\sigma}\|_{m+1}+\|u\|_{m+3}+\delta_{k1}\|f\|).
\]
\end{theorem}
\begin{proof}
在 \eqref{eq:postprocess2} 中选取 $v=z$,并结合问题 \eqref{eq:biharmonic2variables} 的第一个方程,可得
\[
\int_T\nabla^2(u-u_h^{\ast}):\nabla^2z\dx=-\int_T(\boldsymbol{\sigma}-\boldsymbol{\sigma}_h):\nabla^2z\dx.
\]
回顾 $z$ 的定义,我们有
\begin{align*}
\int_T\nabla^2z:\nabla^2z\dx 
&=\int_T\nabla^2(I_h^{k+1}u-u_h^{\ast}):\nabla^2z\dx \\
&=\int_T\nabla^2(I_h^{k+1}u-u):\nabla^2z\dx+\int_T\nabla^2(u-u_h^{\ast}):\nabla^2z\dx \\
&=\int_T\nabla^2(I_h^{k+1}u-u):\nabla^2z\dx-\int_T(\boldsymbol{\sigma}-\boldsymbol{\sigma}_h):\nabla^2z\dx.
\end{align*}
另一方面,利用 Cauchy-Schwarz 不等式,
\[
|z|_{2,T}^2\lesssim  |I_h^{k+1}u-u|_{2,T}|z|_{2,T} + \|\boldsymbol{\sigma}-\boldsymbol{\sigma}_h\|_{0,T}|z|_{2,T},
\]
结合 \eqref{eq:z1} 可得
\[
|z|_{1,T}\lesssim h_T|z|_{2,T}\lesssim h_T|I_h^{k+1}u-u|_{2,T} + h_T\|\boldsymbol{\sigma}-\boldsymbol{\sigma}_h\|_{0,T}.
\]
由三角不等式及 \eqref{eq:postproestimate1},有
\begin{align*}
|u-u_h^{\ast}|_{1,h} &\leq |u-I_h^{k+1}u|_{1,h}+|I_h^1(I_h^{k+1}u-u_h^{\ast})|_{1} +|z|_{1,h} \\
&\lesssim |u-I_h^{k+1}u|_{1,h} + |I_h^{\grad}u-u_h|_{1} + h(|u-I_h^{k+1}u|_{2,h}+\|\boldsymbol{\sigma}-\boldsymbol{\sigma}_h\|).
\end{align*}
最后,结合 $I_h^{k+1}$ 的插值误差估计、\eqref{eq:hhjH1supererror} 以及 \eqref{eq:hhjenergyerror} 即可证得结论. 
\end{proof}


\subsection{杂交化}

本小节我们将对 HHJ 分布混合变分形式 \eqref{eq:hhjmixedformulation} 进行杂交化处理 \cite{ArnoldBrezzi1985,HuangHuang2016,HuangHuang2019}. 
引入两个有限元空间:
\begin{align*}
\mathbb P_{k-1}(\mathcal{T}_h; \mathbb{S})&:=\{\boldsymbol{\tau}_h\in L^{2}(\Omega ; \mathbb{S}): \boldsymbol{\tau}_h|_T\in \mathbb P_{k-1}(T ; \mathbb{S}) \quad\forall\,T \in \mathcal{T}_{h}\}, \\
M_h&:=\{\mu_h\in L^2(\mathcal{E}_{h}): \mu_h|_e\in \mathbb P_{k-1}(e) \quad\forall\, e \in \Delta_1(\mathring{\mathcal{T}}_{h}), \text { 且 }\, \mu_h=0 \text { 在 } \Delta_1(\mathcal{T}_{h})\backslash\Delta_1(\mathring{\mathcal{T}}_{h}) \text{ 上}\}.
\end{align*}
在乘子空间 $M_h$ 上赋予如下平方范数:
$$
\|\mu_h\|_{\alpha,h}^2:=\sum_{T\in\mathcal T_h}\sum_{e\in\Delta_1(T)}h_e^{-2\alpha}\|\mu_h\|_{e}^2,\quad \alpha=\pm1/2.
$$

定义弱 Hessian 算子 $\nabla_w^2: \mathring{V}_k^{\grad}(\mathcal{T}_h)\times M_h\to\mathbb P_{k-1}(\mathcal{T}_h; \mathbb{S})$ 为
$$
(\nabla_w^2(v, \mu), \boldsymbol\tau)_T = (\boldsymbol{\tau}, \nabla^2v)_T + (\boldsymbol{n}^{\intercal}\boldsymbol{\tau}\boldsymbol{n}_e, \mu-\partial_{n_e}v)_{\partial T}\quad\forall~\boldsymbol\tau\in\mathbb P_{k-1}(T; \mathbb{S}),\, T\in\mathcal{T}_h.
$$

\begin{lemma}
成立如下范数等价性:
\begin{equation}\label{eq:weakHessiannormequiv}
\|\nabla_w^2(v, \mu)\|^2\eqsim 
\|\nabla_h^2v\|^2+\sum_{T\in\mathcal T_h}\sum_{e\in\Delta_1(T)}h_e^{-1}\|\mu-\partial_{n_e}v\|_e^2, \quad\forall~v\in \mathring{V}_k^{\grad}(\mathcal{T}_h),\mu\in M_h.
\end{equation}
\end{lemma}
\begin{proof}
由弱 Hessian 算子的定义和逆不等式可得
$$
\|\nabla_w^2(v, \mu)\|^2\lesssim 
\|\nabla_h^2v\|^2+\sum_{T\in\mathcal T_h}\sum_{e\in\Delta_1(T)}h_e^{-1}\|\mu-\partial_{n_e}v\|_e^2.
$$
接下来证明反向不等式.

取 $\boldsymbol{\tau}\in\mathbb P_{k-1}(\mathcal{T}_h; \mathbb{S})$ 使得
\begin{align*}
(\boldsymbol{n}^{\intercal}\boldsymbol{\tau}\boldsymbol{n}_e)|_{e}&=h_e^{-1}(\mu-\partial_{n_e}v),\qquad e\in\Delta_1(T),\, T\in\mathcal T_h,
\\
\int_T\boldsymbol{\tau}:\boldsymbol{q}\dx&=\int_T\nabla^2v:\boldsymbol{q}\dx, \quad\forall~\boldsymbol{q}\in \mathbb P_{k-2}(T;\mathbb S),\, T\in\mathcal T_h.
\end{align*}
则有
\begin{equation*}
\|\boldsymbol{\tau}\|^2\lesssim \|\nabla_h^2v\|^2 + \sum_{T\in\mathcal T_h}\sum_{e\in\Delta_1(T)}h_e^{-1}\|\mu-\partial_{n_e}v\|_e^2,
\end{equation*}
以及
\[
\|\nabla_h^2v\|^2 + \sum_{T\in\mathcal T_h}\sum_{e\in\Delta_1(T)}h_e^{-1}\|\mu-\partial_{n_e}v\|_e^2 = (\nabla_w^2(v, \mu), \boldsymbol\tau) \leq\|\nabla_w^2(v, \mu)\| \|\boldsymbol\tau\|.
\]
因此范数等价性 \eqref{eq:weakHessiannormequiv} 成立. 
\end{proof}

由范数等价性 \eqref{eq:weakHessiannormequiv} 可知,
\begin{equation*}
\|v\|_{2,h}\lesssim \|\nabla_w^2(v, \mu)\|,\quad\forall\,v\in \mathring{V}_k^{\grad}(\mathcal{T}_h),\mu\in M_h.
\end{equation*}
因此,$\|\nabla_w^2(v, \mu)\|$ 是离散空间 $\mathring{V}_k^{\grad}(\mathcal{T}_h)\times M_h$ 上的范数.

HHJ 混合元方法 \eqref{eq:hhjmfem} 的杂交化形式为:求 $(u_h,\lambda_h)\in \mathring{V}_k^{\grad}(\mathcal{T}_h)\times M_h$ 使得
\begin{equation}\label{hhjHybrid}
(\nabla_w^2(u_h,\lambda_h), \nabla_w^2(v_h,\mu_h))=(f, v_h)\quad\forall~v_h\in \mathring{V}_k^{\grad}(\mathcal{T}_h), 
\mu_h\in M_h.
\end{equation}

\begin{theorem}
杂交混合有限元方法 \eqref{hhjHybrid} 是适定的,且等价于 HHJ 混合元方法 \eqref{eq:hhjmfem}. 具体而言:设 $(u_h,\lambda_h)\in \mathring{V}_k^{\grad}(\mathcal{T}_h)\times M_h$ 是杂交混合有限元方法 \eqref{hhjHybrid} 的解; 令 $\boldsymbol{\sigma}_h=-\nabla_w^2(u_h,\lambda_h)$,则 $\boldsymbol{\sigma}_h\in\Sigma_{k-1}^{nn}(\mathcal{T}_h;\mathbb{S})$,且 $(\boldsymbol{\sigma}_h,u_h)$ 是 HHJ 混合元方法 \eqref{eq:hhjmfem} 的解.
\end{theorem}
\begin{proof}
由于 $\|\nabla_w^2(v, \mu)\|$ 是离散空间 $\mathring{V}_k^{\grad}(\mathcal{T}_h)\times M_h$ 上的范数,由 Lax-Milgram 引理可得杂交混合有限元方法 \eqref{hhjHybrid} 的适定性.

接着证明等价性. 在 \eqref{hhjHybrid} 中取 $v_h=0$,可得
\begin{equation*}
\sum_{T\in\mathcal{T}_h}(\boldsymbol{n}^{\intercal}\boldsymbol{\sigma}_h\boldsymbol{n}_e, \mu_h)_{\partial T} = 0 \quad\forall~\mu_h\in M_h.
\end{equation*}
由上式可知,$\boldsymbol{\sigma}_h\in\Sigma_{k-1}^{nn}(\mathcal{T}_h;\mathbb{S})$. 进而,\eqref{hhjHybrid} 即可转化为 \eqref{eq:hhjmfem2}. 另一方面,对任意的 $\boldsymbol{\tau}_h\in\Sigma_{k-1}^{nn}(\mathcal{T}_h;\mathbb{S})$ 成立
$$
(\boldsymbol{\sigma}_h,\boldsymbol{\tau}_h) = -(\nabla_w^2(u_h,\lambda_h), \boldsymbol{\tau}_h) = -\sum_{T\in\mathcal{T}_h}(\boldsymbol{\tau}_h, \nabla^2u_h)_T + \sum_{T\in\mathcal{T}_h}(\boldsymbol{n}^{\intercal}\boldsymbol{\tau}_h\boldsymbol{n}, \partial_n u_h)_{\partial T} = -((\div\div)_w\boldsymbol{\tau}_h,u_h).
$$
证毕.
\end{proof}

\begin{theorem}
设 $(\boldsymbol{\sigma} , u)\in H^{-1}(\div\div,\Omega; \mathbb{S})\times H_0^1(\Omega)$ 是 HHJ 分布混合变分形式 \eqref{eq:hhjmixedformulation} 的解,$(u_h,\lambda_h)$ 是杂交混合有限元方法 \eqref{hhjHybrid} 的解. 
假设对于某个非负整数 $m$,$\boldsymbol{\sigma}\in  H^{m+1}(\Omega; \mathbb{S})$ 且 $u\in H^{m+3}(\Omega)$. 
则
\begin{align}
\label{eq:errorestimate3}\|\partial_{n_e}(I_h^{\grad}u-u_h)-(Q_{\mathcal{E}_h}^{k-1}(\partial_{n_e}u)-\lambda_h)\|_{1/2,h}&\lesssim h^{\min\{m+1,k\}}\|\bs\sigma\|_{m+1}, \\
\label{eq:errorestimate4}
\|\partial_{n_e}u_h-\lambda_h\|_{-1/2,h}&\lesssim   h^{\min\{m+2,k\}}\|u\|_{m+3}.
\end{align}
当 $\Omega$ 是凸区域时,我们有
\begin{equation}\label{eq:errorestimate5}
\|Q_{\mathcal{E}_h}^{k-1}(\partial_{n_e}u)-\lambda_h\|_{-1/2,h}\lesssim h^{\min\{m+1,k\}+1}(\|\bs\sigma\|_{m+1}+\delta_{k1}\|f\|).
\end{equation}
\end{theorem}
\begin{proof}
对任意的 $\boldsymbol{\tau}\in \mathbb P_{k-1}(T; \mathbb{S})$, 由 $I_h^{\grad}$ 的定义可得,
\begin{align*}
(\nabla_w^2(I_h^{\grad}u, Q_{\mathcal{E}_h}^{k-1}(\partial_{n_e}u)), \boldsymbol\tau)_T &= -(\div\boldsymbol{\tau}, \nabla (I_h^{\grad}u))_T + (\boldsymbol{t}^{\intercal}\boldsymbol{\tau}\boldsymbol{n}, \partial_{t}(I_h^{\grad}u))_{\partial T} + (\boldsymbol{n}^{\intercal}\boldsymbol{\tau}\boldsymbol{n}, \partial_{n}u)_{\partial T} \\
&=-(\div\boldsymbol{\tau}, \nabla u)_T + (\boldsymbol{\tau}\boldsymbol{n}, \nabla u)_{\partial T} = (\nabla^2u, \boldsymbol{\tau})_T=-(\boldsymbol{\sigma}, \boldsymbol{\tau})_T.
\end{align*}
因此,对任意的 $\boldsymbol{\tau}\in \mathbb P_{k-1}(\mathcal{T}_h; \mathbb{S})$, 成立
$$
(\nabla_w^2(I_h^{\grad}u-u_h, Q_{\mathcal{E}_h}^{k-1}(\partial_{n_e}u)-\lambda_h), \boldsymbol\tau) = (\boldsymbol{\sigma}_h-\boldsymbol{\sigma}, \boldsymbol{\tau}).
$$
从而,由 \eqref{eq:hhjenergyerror} 可得,
$$
\|\nabla_w^2(I_h^{\grad}u-u_h, Q_{\mathcal{E}_h}^{k-1}(\partial_{n_e}u)-\lambda_h)\|\leq\|\boldsymbol{\sigma}_h-\boldsymbol{\sigma}\|\lesssim h^{\min\{m+1,k\}}\|\bs\sigma\|_{m+1}.
$$
由范数等价性 \eqref{eq:weakHessiannormequiv} 可知,
\begin{align*}
\|\partial_{n_e}(I_h^{\grad}u-u_h)-(Q_{\mathcal{E}_h}^{k-1}(\partial_{n_e}u)-\lambda_h)\|_{1/2,h}&\lesssim \|\boldsymbol{\sigma}-I_h^{nn}\boldsymbol{\sigma}\|,
\end{align*}
结合 $I_h^{nn}$ 的插值误差估计表明 \eqref{eq:errorestimate3} 成立. 
\eqref{eq:errorestimate4} 由三角不等式、\eqref{eq:errorestimate3} 以及 $I_h^{\grad}$ 和 $Q_{\mathcal{E}_h}^{k-1}$ 的插值误差估计得到. 
最后,\eqref{eq:errorestimate5} 由 \eqref{eq:errorestimate3},\eqref{eq:hhjH1supererror} 和逆不等式得到. 
\end{proof}


\subsection{HHJ 方法与修正 Morley 元方法的等价性}
回顾 Morley 元空间定义:
\begin{align*}
V_h^M := \Big\{&v_h\in L^2(\Omega): v_h|_T\in \mathbb P_2(T),\;
\forall~T \in
\mathcal{T}_h; \displaystyle\int_e\llbracket\partial_{n} v_h\rrbracket\dd s = 0,\; \forall~e \in \Delta_1(\mathcal{T}_h); \\
&
\quad\;
v_h \textrm{ 在每个顶点 } \texttt{v}\in \Delta_0(\mathcal{T}_h) \textrm{ 处连续}; v_h(\texttt{v})=0, \;\forall~ \texttt{v}\in\Delta_0(\mathcal{T}_h) \cap \partial\Omega
\Big\}.
\end{align*}
修正 Morley 元方法为:求 $w_h\in V_h^M$ 使得
\begin{equation}\label{eq:modiMorley4Biharmonic}
(\nabla_h^2 w_h, \nabla_h^2 v_h)=(f, I_h^1v_h), \quad\forall~v_h\in V_h^M.
\end{equation}

\begin{theorem}
设 $w_h\in V_h^M$ 是 Morley 元方法 \eqref{eq:modiMorley4Biharmonic} 的解. 
则 $(I_h^1w_h, Q_{\mathcal{E}_h}^0(\partial_{n_e}w_h))\in \mathring{V}_1^{\grad}(\mathcal{T}_h)\times M_h$ 是 $k=1$ 时杂交混合有限元方法 \eqref{hhjHybrid} 的解. 
\end{theorem}
\begin{proof}
对任意的 $\boldsymbol{\tau}\in \mathbb P_{0}(\mathcal{T}_h; \mathbb{S})$,我们有
\begin{align*}
(\nabla_w^2(I_h^1w_h, Q_{\mathcal{E}_h}^0(\partial_{n_e}w_h)), \boldsymbol\tau) & = \sum_{T\in\mathcal{T}_h} (\partial_{n}(w_h-I_h^1w_h), \boldsymbol{n}^{\intercal}\boldsymbol{\tau}\boldsymbol{n})_{\partial T} \\
& = (\nabla_h^2w_h, \boldsymbol\tau) -\sum_{T\in\mathcal{T}_h} (\partial_{t}(w_h-I_h^1w_h), \boldsymbol{t}^{\intercal}\boldsymbol{\tau}\boldsymbol{n})_{\partial T} = (\nabla_h^2w_h, \boldsymbol\tau).
\end{align*}
上述推导中利用了如下事实:由于 $w_h$ 与 $I_h^1w_h$ 在顶点处重合,且 $\boldsymbol{\tau}$ 为分片常数张量,因此切向导数项的积分为零. 
即成立如下等式:
$$
\nabla_w^2(I_h^1w_h, Q_{\mathcal{E}_h}^0(\partial_{n_e}w_h)) = \nabla_h^2w_h.
$$
从而,将此式代入 Morley 元方法 \eqref{eq:modiMorley4Biharmonic} 可得
$$
(\nabla_w^2(I_h^1w_h, Q_{\mathcal{E}_h}^0(\partial_{n_e}w_h)), \nabla_w^2(I_h^1v_h, Q_{\mathcal{E}_h}^0(\partial_{n_e}v_h))) = (\nabla_h^2 w_h, \nabla_h^2 v_h)=(f, I_h^1v_h), \quad\forall~v_h\in V_h^M.
$$
最后,考虑到空间 $V_h^M$ 与 $\mathring{V}_1^{\grad}(\mathcal{T}_h)\times M_h$ 之间通过映射 $v_h \mapsto (I_h^1v_h, Q_{\mathcal{E}_h}^0(\partial_{n_e}v_h))$ 构成的自由度一一对应关系,可知结论成立. 
\end{proof}


% \subsection{HHJ 混合方法的 Hilbert 复形}

% 在本节中,我们将推导 HHJ 混合方法~\eqref{mfem1}-\eqref{mfem2} 的正合序列及交换图. 

% %The first operator $B = \div \boldsymbol \div$ maps a symmetric tensor function to a vector function.

% 对于向量值函数 $\boldsymbol{\phi}=(\phi_1, \phi_2)^{T}$,记 $\boldsymbol{\phi}^{\perp}:=(-\phi_2, \phi_1)^T$ 为垂直于 $\boldsymbol{\phi}$ 的向量. 
% 标准对称梯度算子定义为
% \[
% \boldsymbol{\varepsilon}(\boldsymbol{\phi})=\frac{1}{2}\left(\boldsymbol{\nabla}\boldsymbol{\phi}+(\boldsymbol{\nabla}\boldsymbol{\phi})^T\right).
% \]
% 对称旋度算子类似地定义为
% $$
% \nabla ^s\times\boldsymbol \phi :=
% %\boldsymbol{L}^T\boldsymbol{\varepsilon}(\boldsymbol \phi^{\bot})\boldsymbol{L} = \frac{1}{2}\boldsymbol{L}^T(\nabla \boldsymbol \phi^{\bot} + \nabla^{T} \boldsymbol \phi^{\bot})\boldsymbol{L} =
% \frac{1}{2}\left ( \mathbf{curl} \boldsymbol \phi +  (\mathbf{curl}\boldsymbol \phi)^T\right ).
% $$
% %with $\boldsymbol{L}=\left(
% %\begin{array}{cc}
% %0 & -1 \\
% %1 & 0
% %\end{array}
% %\right)$.
% %
% %Set
% %\[
% %\mathbf{Curl}\boldsymbol{\phi}=\left(
% %\begin{array}{ll}
% %\partial_2\phi_1 &  -\partial_1\phi_1 \\
% %\partial_2\phi_2 &  -\partial_1\phi_2 \\
% %\end{array}
% %\right), \quad \boldsymbol{\varepsilon}^{\bot}(\boldsymbol{\phi})= \frac{1}{2}\left(
% %\begin{array}{cc}
% %0 &  \boldsymbol{\nabla}\cdot\boldsymbol{\phi} \\
% %-\boldsymbol{\nabla}\cdot\boldsymbol{\phi} &  0 \\
% %\end{array}
% %\right) + \mathbf{Curl}\boldsymbol{\phi}.
% %\]
% %Then we have
% %\[
% %M_n(\boldsymbol{\varepsilon}^{\bot}(\boldsymbol{\phi}))=\partial_{\boldsymbol{t}}(\boldsymbol{\phi}\cdot\boldsymbol{n}), \quad M_{nt}(\boldsymbol{\varepsilon}^{\bot}(\boldsymbol{\phi}))=-\frac{1}{2}\boldsymbol{\nabla}\cdot\boldsymbol{\phi} + \partial_{\boldsymbol{t}}(\boldsymbol{\phi}\cdot\boldsymbol{t}).
% %\]
% 令
% \[
% \overline{\boldsymbol{P}}_1(\Omega; \mathbb{R}^2):=\textrm{span}\left\{
% \left(\begin{array}{c}
% 0 \\
% 1
% \end{array}
% \right), \left(\begin{array}{c}
% 1 \\
% 0
% \end{array}
% \right),  \left(\begin{array}{c}
% x_1 \\
% x_2
% \end{array}
% \right)\right\}.
% \]
% 易见,$\overline{\boldsymbol{P}}_1^{\textrm{Rot}}(\Omega; \mathbb{R}^2)$ 正是刚体运动空间,其中
% \[
% \overline{\boldsymbol{P}}_1^{\textrm{Rot}}(\Omega; \mathbb{R}^2):=\{\boldsymbol{\phi}\in\boldsymbol{L}^2(\Omega; \mathbb{R}^2): \boldsymbol{\phi}^{\perp}\in\overline{\boldsymbol{P}}_1(\Omega; \mathbb{R}^2)\}.
% \]


% \begin{lemma}\label{lem:1}
% 如下 Kirchhoff 板的序列
% % \begin{equation}\label{exactsequence1}
% % \overline{\boldsymbol{P}}_1(\Omega; \mathbb{R}^2)\autorightarrow{$\subset$}{} \boldsymbol{C}^{\infty}(\Omega; \mathbb{R}^2)\autorightarrow{$\nabla ^s\times$}{} \boldsymbol{C}^{\infty}(\Omega; \mathbb{S}) \autorightarrow{$\mathrm{div}\boldsymbol{\mathrm{div}}$}{} C^{\infty}(\Omega)\autorightarrow{}{}0
% % \end{equation}
% 是一个正合复形. 
% \end{lemma}
% \begin{proof}
% 通过直接计算可知 \eqref{exactsequence1} 是一个复形,即 $\nabla ^s\times(\overline{\boldsymbol{P}}_1) = 0$ 且 $\mathrm{div}\boldsymbol{\mathrm{div}}~\nabla ^s\times = 0$. 接下来我们要验证其正合性. 

% 首先证明 $\ker(\nabla ^s\times)=\overline{\boldsymbol{P}}_1(\Omega; \mathbb{R}^2)$. 对于任意满足 $\nabla ^s\times\boldsymbol{\phi}=\boldsymbol{0}$ 的 $\boldsymbol{\phi}\in\boldsymbol{C}^{\infty}(\Omega; \mathbb{R}^2)$,有
% \[
% \nabla ^s\times\boldsymbol{\phi} = \boldsymbol{L}^T\boldsymbol{\varepsilon}(\boldsymbol \phi^{\bot})\boldsymbol{L}=\boldsymbol{0},
% \]
% 其中 $\boldsymbol{L}=\left(
% \begin{array}{cc}
% 0 & -1 \\
% 1 & 0
% \end{array}
% \right)$. 
% 因此我们有
% \[
% \boldsymbol{\varepsilon}(\boldsymbol \phi^{\bot})=\boldsymbol{0},
% \]
% 这意味着 $\boldsymbol{\phi}\in\overline{\boldsymbol{P}}_1(\Omega; \mathbb{R}^2)$. 

% 接下来,利用文献~\cite[引理~1]{Beirao-da-VeigaNiiranenStenberg2007} 和 \cite[引理~3.1]{HuangHuangXu2011} 中采用的类似论证,证明 $\ker(\mathrm{div}\boldsymbol{\mathrm{div}})=\nabla ^s\times\boldsymbol{C}^{\infty}(\Omega; \mathbb{R}^2)$. 
% 首先,通过直接计算可知 $\nabla ^s\times\boldsymbol{C}^{\infty}(\Omega; \mathbb{R}^2) \subset \ker(\mathrm{div}\boldsymbol{\mathrm{div}})$. 
% 对于任意 $\boldsymbol{\tau}\in \ker(\mathrm{div}\boldsymbol{\mathrm{div}})$,存在 $v\in C^{\infty}(\Omega)$ 使得
% $\boldsymbol{\mathrm{div}}\boldsymbol{\tau}=\mathbf{curl}v=-\boldsymbol{\mathrm{div}}(v\boldsymbol{L}),$
% 这意味着
% $\boldsymbol{\mathrm{div}}(\boldsymbol{\tau}+v\boldsymbol{L})=0.$
% 因此,存在向量函数 $\boldsymbol{\phi}\in\boldsymbol{C}^{\infty}(\Omega; \mathbb{R}^2)$ 满足
% \[
% \boldsymbol{\tau}+v\boldsymbol{L}=\mathbf{curl}\boldsymbol{\phi}.
% \]
% 由于 $\boldsymbol{\tau}$ 是对称的,故有 $\boldsymbol{\tau}=\nabla ^s\times\boldsymbol{\phi}$. 
% 从而 $\ker(\mathrm{div}\boldsymbol{\mathrm{div}})\subset \nabla ^s\times\boldsymbol{C}^{\infty}(\Omega; \mathbb{R}^2)$. 

% 最后证明 $\mathrm{div}\boldsymbol{\mathrm{div}}\boldsymbol{C}^{\infty}(\Omega; \mathbb{S})=C^{\infty}(\Omega)$. 由文献 \cite[p. 405]{ArnoldWinther2002} 中的弹性力学复形可知,散度算子 $\boldsymbol \div :\boldsymbol{C}^{\infty}(\Omega; \mathbb{S})\to \boldsymbol{C}^{\infty}(\Omega; \mathbb{R}^2)$ 是满射. 此外,根据文献 \cite[p. 27]{ArnoldFalkWinther2006} 中的 de Rham 复形,散度算子 $\div :\boldsymbol{C}^{\infty}(\Omega; \mathbb{R}^2)\to C^{\infty}(\Omega)$ 也是满射. 因此我们有 $\mathrm{div}\boldsymbol{\mathrm{div}}\boldsymbol{C}^{\infty}(\Omega; \mathbb{S})=C^{\infty}(\Omega)$. 
% \end{proof}
% %\LC{write complex involving Sobolev spaces like $H^1\to H(\div\div)\to L^2$}
% %is exact.
% %\begin{remark}
% %This is a rotated version of the elasticity complex in two dimensions
% %$$
% %\overline{\boldsymbol{P}}_1^{\bot}(\Omega; \mathbb{R}^2)\autorightarrow{$\subset$}{} \boldsymbol{C}^{\infty}(\Omega; \mathbb{R}^2)\autorightarrow{$\boldsymbol{\varepsilon}$}{} \boldsymbol{C}^{\infty}(\Omega; \mathbb{S}) \autorightarrow{$\curl \curl$}{} C^{\infty}(\Omega)\autorightarrow{}{}0
% %$$
% %\end{remark}

% 接着,我们推导一个正则性要求较低的正合序列. 为此,定义 $B: \mathcal V\to \mathcal P^{\prime}$ 为
% $$
% \langle B\boldsymbol{\tau}, v\rangle :=b(\boldsymbol{\tau}, v) \quad \forall~ v\in \mathcal P.
% $$
% 对于任意 $(\boldsymbol{\tau}, v)\in\mathcal V\times \mathcal P$ 且 $v\in H_0^2(\Omega)$,由分部积分及 $\left [M_n(\boldsymbol{\tau}) \right ]|_{\mathcal{E}_h^i}=0$ 这一事实可得
% \begin{align}
% \langle B\boldsymbol{\tau}, v\rangle =& \int_{\Omega}\boldsymbol{\tau}:\boldsymbol{\nabla}^2v\, {\rm d}x - \sum_{K\in\mathcal{T}_h}\int_{\partial K}(\boldsymbol{\tau}\boldsymbol{n})\cdot\boldsymbol{\nabla}v\, {\rm d}s+\sum_{K\in\mathcal{T}_h}\int_{\partial K}M_{nt}(\boldsymbol{\tau})\partial_{\boldsymbol{t}}v\, {\rm d}s \notag\\
% =&\int_{\Omega}\boldsymbol{\tau}:\boldsymbol{\nabla}^2v\, {\rm d}x - \sum_{K\in\mathcal{T}_h}\int_{\partial K}M_{n}(\boldsymbol{\tau})\partial_{\boldsymbol{n}}v\, {\rm d}s \notag\\
% =&\int_{\Omega}\boldsymbol{\tau}:\boldsymbol{\nabla}^2v\, {\rm d}x=\langle \div\boldsymbol \div\boldsymbol{\tau}, v\rangle_{H^{-2}(\Omega)\times H_0^2(\Omega)}. \label{eq:temp8}
% \end{align}
% 另一方面,对于任意 $(\boldsymbol{\tau}, v)\in\mathcal V\times \mathcal P$ 且 $\boldsymbol{\tau}\in \boldsymbol H(\boldsymbol \div,\Omega; \mathbb S): = \{\boldsymbol \tau \in L^2(\Omega; \mathbb S): \boldsymbol\div \boldsymbol \tau \in \boldsymbol{L}^{2}(\Omega; \mathbb{R}^2)\}$,由于 $v\in\mathcal P$ 意味着 $v$ 在 $\Omega$ 内连续,由 $\left [M_{nt}(\boldsymbol{\tau}) \right ]|_{\mathcal{E}_h^i}=0$ 可得
% \begin{align*}
% \langle B\boldsymbol{\tau}, v\rangle =&-\int_{\Omega}(\boldsymbol \div\boldsymbol{\tau})\cdot\boldsymbol{\nabla}v\, {\rm d}x + \sum_{K\in\mathcal{T}_h}\int_{\partial K}M_{nt}(\boldsymbol{\tau})\partial_{\boldsymbol{t}}v\, {\rm d}s \\
% =&-\int_{\Omega}(\boldsymbol \div\boldsymbol{\tau})\cdot\boldsymbol{\nabla}v\, {\rm d}x=\langle \div\boldsymbol \div\boldsymbol{\tau}, v\rangle_{H^{-1}(\Omega)\times H_0^1(\Omega)}.
% \end{align*}
% 因此,双线性形式 $b(\cdot,\cdot)$ 既可以定义在 $\boldsymbol H(\boldsymbol \div,\Omega; \mathbb S)\times H_0^1(\Omega)$ 上,此时 $B=\div\boldsymbol \div$ 在 $H^{-1}(\Omega)$ 分布意义下成立;也可以定义在 $\boldsymbol L^2(\Omega)\times H^{2}_0(\Omega)$ 上,此时 $B=\div\boldsymbol \div$ 在 $H^{-2}(\Omega)$ 分布意义下成立. 然而,构造 $\boldsymbol H(\boldsymbol \div,\Omega; \mathbb S)$ 或 $H^{2}_0(\Omega)$ 的协调有限元空间十分困难. 直到本世纪,采用多项式形函数的 $\boldsymbol H(\boldsymbol \div,\Omega; \mathbb S)$ 协调混合有限元才在 \cite{Hu2015a,HuZhang2015c, HuZhang2015, ChenHuHuang2017, HuZhang2016, ArnoldWinther2002, AdamsCockburn2005, ArnoldAwanouWinther2008} 中被构造出来,而针对一般形状正则非结构网格的高效快速求解器直到最近才在 \cite{ChenHuHuang2017a} 中得到开发. 
% 我们对这两个空间的光滑性进行了折衷,将双线性形式 $b(\cdot,\cdot)$ 理解为定义在 $\mathcal V\times \mathcal P$ 上,从而有
% $$
% \mathrm{div}\boldsymbol{\mathrm{div}}: \boldsymbol{H}^{-1}(\mathrm{div}\boldsymbol{\mathrm{div}},\Omega; \mathbb{S}) \to  H^{-1}(\Omega),
% $$
% 其中空间 $\boldsymbol{H}^{-1}(\div\boldsymbol \div,\Omega; \mathbb S): = \{\boldsymbol \tau \in L^2(\Omega; \mathbb S): \div \boldsymbol\div \boldsymbol \tau \in H^{-1}(\Omega)\}$ 最早由 \cite{PechsteinSchoberl2011} 引入. 

% 利用与引理~\ref{lem:1} 类似的论证,我们可以获得如下 Kirchhoff 板的 Hilbert 序列. 
% \begin{lemma}
% 如下 Kirchhoff 板的 Hilbert 序列
% % \begin{equation}\label{eq:hhjexactsequencecontinuous}
% % \overline{\boldsymbol{P}}_1(\Omega; \mathbb{R}^2)\autorightarrow{$\subset$}{} \boldsymbol{H}^1(\Omega; \mathbb{R}^2)\autorightarrow{$\nabla ^s\times$}{} \boldsymbol{H}^{-1}(\mathrm{div}\boldsymbol{\mathrm{div}},\Omega; \mathbb{S}) \autorightarrow{$\mathrm{div}\boldsymbol{\mathrm{div}}$}{} H^{-1}(\Omega)\autorightarrow{}{}0
% % \end{equation}
% 是一个正合复形. 
% \end{lemma}


% \begin{remark}\rm
% Kirchhoff 板的一个光滑性较低的正合 Hilbert 序列是
% % \begin{equation}\label{eq:lesssmoothexactsequencecontinuous}
% % \overline{\boldsymbol{P}}_1(\Omega; \mathbb{R}^2)\autorightarrow{$\subset$}{} \boldsymbol{L}^2(\Omega; \mathbb{R}^2)\autorightarrow{$\nabla ^s\times$}{} \boldsymbol{H}^{-2}(\mathrm{div}\boldsymbol{\mathrm{div}},\Omega; \mathbb{S}) \autorightarrow{$\mathrm{div}\boldsymbol{\mathrm{div}}$}{} H^{-2}(\Omega)\autorightarrow{}{}0,
% % \end{equation}
% 其中 $\boldsymbol{H}^{-2}(\div\boldsymbol \div,\Omega; \mathbb S): = \{\boldsymbol \tau \in \boldsymbol H^{-1}(\Omega; \mathbb S): \div \boldsymbol\div \boldsymbol \tau \in H^{-2}(\Omega)\}$. 
% 然而,$\boldsymbol{H}^{-2}(\div\boldsymbol \div,\Omega; \mathbb S)$ 的有限元空间很难构造. 事实上,在 HHJ 混合方法中,空间 $\mathcal V$ 和 $\mathcal V_h$ 也都不是 $\boldsymbol{H}^{-1}(\div\boldsymbol \div,\Omega; \mathbb S)$ 的子空间. 也就是说,HHJ 混合方法本质上仍是一种非协调方法. $\Box$
% \end{remark}

% \begin{remark}\rm
% 序列 \eqref{eq:lesssmoothexactsequencecontinuous} 的对偶复形为
% % \[
% % 0\autorightarrow{$\subset$}{} H_0^2(\Omega)\autorightarrow{$\boldsymbol\nabla^2$}{} \boldsymbol{H}_0(\boldsymbol{\mathrm{rot}},\Omega; \mathbb{S}) \autorightarrow{$\boldsymbol{\mathrm{rot}}$}{} \boldsymbol{L}_0^2(\Omega; \mathbb{R}^2)\autorightarrow{}{}0,
% % \]
% 其中
% \[
% \boldsymbol H_0(\mathbf{rot},\Omega; \mathbb S): = \{\boldsymbol \tau \in L^2(\Omega; \mathbb S): \mathbf{rot} \boldsymbol \tau \in \boldsymbol{L}^{2}(\Omega; \mathbb{R}^2),\, \textrm{ 且在 }\,\partial \Omega \textrm{ 上 }\,\boldsymbol \tau\boldsymbol t=\boldsymbol 0 \},
% \]
% \[
% \boldsymbol{L}_0^2(\Omega; \mathbb{R}^2): = \{\boldsymbol \phi \in \boldsymbol{L}^2(\Omega; \mathbb{R}^2): \int_{\Omega} \boldsymbol \phi\,\, {\rm d}x=\boldsymbol 0\}.
% \]
% 值得注意的是,上述最后一个正合序列是二维弹性力学复形~\cite[(2.1)]{ArnoldWinther2002} 的旋转形式. $\Box$
% \end{remark}

% \medskip

% 在离散层面,我们将针对前述有限元空间推导类似的各个正合序列. 为此,首先讨论两个微分算子 $\nabla ^s\times$ 和 $\mathrm{div}\boldsymbol{\mathrm{div}}$ 的离散化. 由于 $\nabla ^s\times$ 仅要求 $H^1$ 的光滑性,可以通过选择有限元空间 $\mathcal S_h \subset H^1$ 来自然地离散化. 难点在于算子 $\mathrm{div}\boldsymbol{\mathrm{div}}$ 的离散化. 首先,我们可以将 $B : \mathcal{V}_h\to \mathcal{P}_h^{\prime}$ 理解为
% \[
% \langle B\boldsymbol{\tau}, v\rangle :=b(\boldsymbol{\tau}, v) \quad \forall~ v\in \mathcal{P}_h.
% \]
% %The space $\mathcal V$ is a
% %\[
% %\overline{\boldsymbol{P}}_1(\Omega; \mathbb{R}^2)\autorightarrow{$\subset$}{} \boldsymbol{H}^1(\Omega; \mathbb{R}^2)\autorightarrow{$\boldsymbol{\varepsilon}^{\bot}$}{} \boldsymbol{H}^{-1}(\div\boldsymbol{\div},\Omega; \mathbb{S}) \autorightarrow{$\boldsymbol{\nabla}\cdot(\boldsymbol{\nabla}\cdot)$}{} H^{-1}(\Omega)\autorightarrow{}{}0
% %\]
% %
% 利用 $L^2$ 内积诱导的 Riesz 表示定理,我们可以将 $\mathcal{P}_h^{\prime}$ 等同于 $\mathcal I_h^{\grad}$,并最终定义 $(\mathrm{div}\boldsymbol{\mathrm{div}})_h: \mathcal{V}_h\to \mathcal{P}_h$ 如下:对于任意 $\boldsymbol{\tau}\in\mathcal{V}_h$,$(\mathrm{div}\boldsymbol{\mathrm{div}})_h\boldsymbol{\tau}\in\mathcal{P}_h$ 由下式唯一确定
% \[
% \int_{\Omega}(\mathrm{div}\boldsymbol{\mathrm{div}})_h\boldsymbol{\tau}\, v\,{\rm d}x=b(\boldsymbol{\tau}, v) \quad \forall~v\in\mathcal{P}_h.
% \]

% 为了给出交换图,我们需要引入一些插值算子. 
% 令 $Q_h$ 为从 $L^2(\Omega)$ 到 $\mathcal{P}_h$ 的 $L^2$ 正交投影算子,由于 $\mathcal I_h^{\grad} \subset H_0^1(\Omega)$,该算子可延拓为 $H^{-1}(\Omega) \to \mathcal I_h^{\grad}$. 


% 对于任意单元 $K\in \mathcal{T}_h$,定义 $I_K: H^2(K)\to P_r(K)$ 如下 (参见 \cite{BabuvskaOsbornPitkaranta1980,FalkOsborn1980, Comodi1989,Stenberg1991}):给定 $w\in H^2(K)$,对于 $K$ 的任意顶点 $a$ 和任意边 $e$,
% \begin{align*}
% I_Kw(a)&=w(a),\\
% \int_e(w-I_Kw)v\, {\rm d}s&=0 \quad \forall~v\in P_{r-2}(e),\\
% \int_K(w-I_Kw)v\, \, {\rm d}x&=0 \quad \forall~v\in P_{r-3}(K).
% \end{align*}
% 相应的全局插值算子 $I_h$ 定义为
% \[
% (I_h)|_K:=I_K \quad\textrm{ 对所有 } K\in\mathcal{T}_h.
% \]
% 令 $\boldsymbol{I}_K=I_K\times I_K$, $\boldsymbol{I}_h=I_h\times I_h$. 

% \smallskip
% \begin{lemma}\label{lem: Bcomforming}
% $(\mathrm{div}\boldsymbol{\mathrm{div}})_h$ 是算子 $B$ 的协调离散,意指 $\ker ((\mathrm{div}\boldsymbol{\mathrm{div}})_h)\subset \ker B$. 
% \end{lemma}
% \begin{proof}
% 由 $I_h$ 的定义,我们有 (参见~\cite[p.~1058]{BabuvskaOsbornPitkaranta1980})
% \begin{equation}\label{eq:temp4}
%  b(\boldsymbol \tau_h, v) =  b(\boldsymbol \tau_h, I_hv), \quad \forall \boldsymbol{\tau}_h\in \mathcal{V}_h, v\in \mathcal{P}.
% \end{equation}

% 对于任意 $\boldsymbol{\tau}\in \ker ((\mathrm{div}\boldsymbol{\mathrm{div}})_h)$,由 \eqref{eq:temp4} 可知,对任意 $v\in \mathcal{P}$,
% \[
% \langle B\boldsymbol{\tau}, v\rangle =b(\boldsymbol{\tau}, v)=b(\boldsymbol \tau, I_hv)=\int_{\Omega}(\mathrm{div}\boldsymbol{\mathrm{div}})_h\boldsymbol{\tau}\, I_hv\,{\rm d}x=0.
% \]
% 因此 $\boldsymbol{\tau}\in \ker B$. 
% \end{proof}


% 接着定义 $\boldsymbol{\Pi}_K: \boldsymbol{H}^1(K, \mathbb{S})\to \boldsymbol{P}_{r-1}(K, \mathbb{S})$ 如下 (参见~\cite{BabuvskaOsbornPitkaranta1980,FalkOsborn1980, Comodi1989,Brezzi.F;Fortin.M1991}):给定 $\boldsymbol{\tau}\in \boldsymbol{H}^1(K, \mathbb{S})$,对于任意单元 $K\in \mathcal{T}_h$ 和 $K$ 的任意边 $e$,
% \begin{align*}
% \int_eM_n\left((\boldsymbol{\tau}-\boldsymbol{\Pi}_K\boldsymbol{\tau})|_K\right)\mu\, {\rm d}s &=0 \quad \forall~\mu\in P_{r-1}(e),\\
% \int_K(\boldsymbol{\tau}-\boldsymbol{\Pi}_K\boldsymbol{\tau}):\boldsymbol{\varsigma}\,{\rm d}x &=0 \quad \forall~\boldsymbol{\varsigma}\in \boldsymbol{P}_{r-2}(K, \mathbb{S}).
% \end{align*}
% 相应的全局插值算子 $\boldsymbol{\Pi}_h: \mathcal{V}\to \mathcal{V}_h$ 定义为
% \[
% (\boldsymbol{\Pi}_h)|_K:=\boldsymbol{\Pi}_K \quad\textrm{ 对所有 } K\in\mathcal{T}_h.
% \]
% 由 $\boldsymbol{\Pi}_h$ 的定义可知
% \begin{equation}\label{eq:bpi0}
% b(\boldsymbol{\tau}-\boldsymbol{\Pi}_h\boldsymbol{\tau}, v)=0 \quad \forall~\boldsymbol{\tau}\in \mathcal{V}, v\in \mathcal{P}_h.
% \end{equation}
% 即 $Q_h B = (\mathrm{div}\boldsymbol{\mathrm{div}})_h\boldsymbol\Pi_h$. 

% \smallskip
% \begin{lemma}\label{lem:hhjexactsequence}
% 如下 HHJ 混合方法的序列
% % \begin{equation}\label{exactsequence2}
% % \overline{\boldsymbol{P}}_1(\Omega; \mathbb{R}^2)\autorightarrow{$\subset$}{} \mathcal{S}_h\autorightarrow{$\nabla ^s\times$}{} \mathcal{V}_h \autorightarrow{$(\mathrm{div}\boldsymbol{\mathrm{div}})_h$}{} \mathcal{P}_h\autorightarrow{}{}0
% % \end{equation}
% 是一个正合序列. 
% \end{lemma}
% \begin{proof}
% 与 \eqref{exactsequence1} 类似,通过直接计算可知 \eqref{exactsequence2} 是一个复形. 随后我们证明 $\ker((\mathrm{div}\boldsymbol{\mathrm{div}})_h)=\nabla ^s\times\mathcal{S}_h$. 
% 取任意 $\boldsymbol{\tau}\in\ker((\mathrm{div}\boldsymbol{\mathrm{div}})_h)$. 由于 $\ker((\mathrm{div}\boldsymbol{\mathrm{div}})_h)\subset\ker B$,利用 \eqref{eq:temp8} 以及连续层面的正合序列 \eqref{eq:hhjexactsequencecontinuous},我们可以找到向量函数 $\boldsymbol{\phi}\in\boldsymbol{H}^{1}(\Omega; \mathbb{R}^2)$ 满足 $\boldsymbol{\tau}=\nabla ^s\times\boldsymbol{\phi}$. 通过直接计算,对每个 $K\in\mathcal{T}_h$ 成立
% \[
% \mathbf{curl}(\mathrm{div}(\boldsymbol{\phi}|_K))=2\mathbf{div}(\boldsymbol{\tau}|_K)\in \boldsymbol{P}_{r-2}(K, \mathbb{R}^2).
% \]
% 因此 $\mathrm{div}(\boldsymbol{\phi}|_K)\in P_{r-1}(K)$,结合 $\nabla ^s\times\boldsymbol{\phi}=\boldsymbol{\tau}\in  \boldsymbol{P}_{r-1}(K, \mathbb{S})$ 可知 $\boldsymbol{\nabla}(\boldsymbol{\phi}|_K)\in  \boldsymbol{P}_{r-1}(K, \mathbb{S})$. 
% 故 $\boldsymbol{\phi}|_K\in\boldsymbol{P}_{r}(K, \mathbb{R}^2)$,即 $\boldsymbol{\phi}\in\mathcal{S}_h$. 

% 利用与引理~\ref{lem:1} 类似的论证,我们有 $\ker(\nabla ^s\times)=\overline{\boldsymbol{P}}_1(\Omega; \mathbb{R}^2)$. 
% 为证明 \eqref{exactsequence2} 是正合的,我们需通过采用文献~\cite[p.~1056]{BabuvskaOsbornPitkaranta1980} 中的技巧证明 $(\mathrm{div}\boldsymbol{\mathrm{div}})_h(\mathcal{V}_h)=\mathcal{P}_h$. 

% 对于任意 $p\in \mathcal{P}_h$,令 $w_h\in \mathcal{P}_h$ 为下述方程的解:
% \[
% \int_{\Omega}\boldsymbol{\nabla}w_h\cdot \boldsymbol{\nabla}v\,{\rm d}x=-\int_{\Omega}pv\, {\rm d}x \quad \forall~v\in\mathcal{P}_h.
% \]
% 令 $\boldsymbol{\sigma}_0=\left(
% \begin{array}{cc}
% w_h & 0 \\
% 0 & w_h
% \end{array}
% \right)$. 由于
% $M_n(\boldsymbol{\sigma}_0)=\boldsymbol{n}^T\boldsymbol{\sigma}_0\boldsymbol{n}=w_h$ 且 $w_h\in \mathcal{P}_h$,故 $\boldsymbol{\sigma}_0\in\mathcal{V}$. 
% 令 $\boldsymbol{\sigma}_I=\boldsymbol{\Pi}_h\boldsymbol{\sigma}_0\in\mathcal{V}_h$. 利用 \eqref{eq:bpi0},两次分部积分,以及 $\boldsymbol{\sigma}_0$ 和 $w_h$ 的定义,对任意 $v\in\mathcal{P}_h$ 有
% \begin{align*}
% b(\boldsymbol{\sigma}_I, v)=&\, b(\boldsymbol{\sigma}_0, v)=\sum_{K\in\mathcal{T}_h}\int_{K}\boldsymbol{\sigma}_0:\nabla^2v \,{\rm d}x  - \sum_{K\in\mathcal{T}_h}\int_{\partial K}M_n(\boldsymbol{\sigma}_0)\partial_{\boldsymbol n}v \,{\rm d} s \\
% =&\sum_{K\in\mathcal{T}_h}\int_{K}w_h\Delta v \,{\rm d} s - \sum_{K\in\mathcal{T}_h}\int_{\partial K}w_h\partial_{\boldsymbol n}v \,{\rm d} s \\
% =& -\int_{\Omega}\boldsymbol{\nabla}w_h\cdot \boldsymbol{\nabla}v \,{\rm d}x=\int_{\Omega}pv \,{\rm d}x,
% \end{align*}
% 由此可见 $p=(\mathrm{div}\boldsymbol{\mathrm{div}})_h\boldsymbol{\sigma}_I$. 证明完毕. 
% \end{proof}

% \smallskip
% \begin{theorem}
% 我们得到如下 HHJ 混合方法的交换图:
% \[
% \begin{array}{c}
% \xymatrix{
%   \overline{\boldsymbol{P}}_1(\Omega; \mathbb{R}^2) \ar[r]^{\subset} & \boldsymbol{H}^{1}(\Omega; \mathbb{R}^2) \ar[d]^{\boldsymbol{I}_h} \ar[r]^-{\nabla ^s\times}
%                 & \boldsymbol{H}^{-1}(\mathrm{div}\boldsymbol{\mathrm{div}},\Omega; \mathbb{S}) \ar[d]^{\boldsymbol{\Pi}_h}   \ar[r]^-{\mathrm{div}\boldsymbol{\mathrm{div}}} & \ar[d]^{Q_h}H^{-1}(\Omega) \ar[r]^{} & 0 \\
%  \overline{\boldsymbol{P}}_1(\Omega; \mathbb{R}^2) \ar[r]^{\subset} & \mathcal{S}_h \ar[r]^{\nabla ^s\times}
%                 & \mathcal{V}_h   \ar[r]^{(\mathrm{div}\boldsymbol{\mathrm{div}})_h} &  \mathcal{P}_h \ar[r]^{}& 0    }
% \end{array}.
% \]
% \end{theorem}
% \begin{proof}
% 恒等式 $Q_h \mathrm{div}\boldsymbol{\mathrm{div}} = (\mathrm{div}\boldsymbol{\mathrm{div}})_h\boldsymbol \Pi_h$ 已在 \eqref{eq:bpi0} 中证明. 
% %First we prove that for any $\boldsymbol{\tau}\in\boldsymbol{C}^{\infty}(\Omega; \mathbb{S})$, $B_h\boldsymbol{\Pi}_h\boldsymbol{\tau}=T_hB\boldsymbol{\tau}$.
% %This can be achieved as follows by using
% % the definitions of $B_h$ and $B$, \eqref{eq:bpi0} and integration by parts
% %\[
% %\int_{\Omega}(B_h\boldsymbol{\Pi}_h\boldsymbol{\tau})v\, {\rm d}x=b(\boldsymbol{\Pi}_h\boldsymbol{\tau}, v)=b(\boldsymbol{\tau}, v)=\int_{\Omega}(B\boldsymbol{\tau})v\, {\rm d}x=\int_{\Omega}(T_hB\boldsymbol{\tau})v\, {\rm d}x\quad \forall~v\in\mathcal{P}_h.
% %\]

% 接下来证明:对于任意 $\boldsymbol{\phi}\in \boldsymbol{H}^{1}(\Omega; \mathbb{R}^2)\cap {\rm dom}(\boldsymbol I_h)$,有 $\nabla ^s\times(\boldsymbol{I}_h\boldsymbol{\phi})=\boldsymbol{\Pi}_h\nabla ^s\times\boldsymbol{\phi}$. 
% 对任意 $\boldsymbol{\varsigma}\in \boldsymbol{P}_{r-2}(K, \mathbb{S})$ 和 $K\in\mathcal{T}_h$,由分部积分以及 $\boldsymbol{\Pi}_h$ 和 $\boldsymbol{I}_h$ 的定义可得
% \begin{equation}\label{eq:temp1}
% \int_K(\nabla ^s\times(\boldsymbol{I}_h\boldsymbol{\phi})-\boldsymbol{\Pi}_h(\nabla ^s\times\boldsymbol{\phi})):\boldsymbol{\varsigma}\, {\rm d}x=\int_K\nabla ^s\times(\boldsymbol{I}_h\boldsymbol{\phi}-\boldsymbol{\phi}):\boldsymbol{\varsigma}\, {\rm d}x=0.
% \end{equation}
% 在每条边 $e\in \mathcal{E}_h(K)$ 上,由 $\boldsymbol{\Pi}_h$ 的定义,对任意 $\mu\in P_{r-1}(e)$ 成立
% \[
% \int_eM_n(\nabla ^s\times(\boldsymbol{I}_h\boldsymbol{\phi})-\boldsymbol{\Pi}_h(\nabla ^s\times\boldsymbol{\phi}))\mu ~\textrm{d}s=\int_eM_n(\nabla ^s\times(\boldsymbol{I}_h\boldsymbol{\phi}-\boldsymbol{\phi}))\mu~\textrm{d}s.
% \]
% 注意到事实 $M_n(\nabla ^s\times(\boldsymbol{I}_h\boldsymbol{\phi}-\boldsymbol{\phi}))=\partial_{\boldsymbol{t}}((\boldsymbol{I}_h\boldsymbol{\phi}-\boldsymbol{\phi})\cdot\boldsymbol{n})$. 
% 因此,由分部积分及 $\boldsymbol{I}_h$ 的定义可得
% \begin{align}
% \int_eM_n(\nabla ^s\times(\boldsymbol{I}_h\boldsymbol{\phi})-\boldsymbol{\Pi}_h(\nabla ^s\times\boldsymbol{\phi}))\mu ~\textrm{d}s=& \int_e\partial_{\boldsymbol{t}}((\boldsymbol{I}_h\boldsymbol{\phi}-\boldsymbol{\phi})\cdot\boldsymbol{n})\mu ~\textrm{d}s=0. \label{eq:temp2}
% \end{align}
% 由于 $(\nabla ^s\times(\boldsymbol{I}_h\boldsymbol{\phi})-\boldsymbol{\Pi}_h(\nabla ^s\times\boldsymbol{\phi}))|_K\in \boldsymbol{P}_{k-1}(K, \mathbb{S})$,式 \eqref{eq:temp1}-\eqref{eq:temp2} 结合 $\boldsymbol{\Pi}_h$ 的适定性意味着 $\nabla ^s\times(\boldsymbol{I}_h\boldsymbol{\phi})-\boldsymbol{\Pi}_h(\nabla ^s\times\boldsymbol{\phi})=\boldsymbol{0}$,即 $\nabla ^s\times(\boldsymbol{I}_h\boldsymbol{\phi})=\boldsymbol{\Pi}_h(\nabla ^s\times\boldsymbol{\phi})$. 
% \end{proof}

% \smallskip
% \begin{remark}\rm
% 值得一提的是,我们在交换图的顶层序列中使用了具有最小正则性的自然 Sobolev 空间. 然而,插值算子 $\boldsymbol I_h$ 和 $\boldsymbol \Pi_h$ 是针对更光滑的函数定义的,且在相应的 Sobolev 范数下并非有界的. 换言之,我们将这些插值算子视为稠密定义的无界算子. 可以使用文献 \cite{ArnoldFalkWinther2006} 中的平滑过程来定义稳定的拟插值算子,同时保持交换性质. $\Box$
% \end{remark}


% \section{可杂交混合有限元方法}
