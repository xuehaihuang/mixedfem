% !TEX root = lecture.tex
\chapter{微分复形在偏微分方程中的应用}


\section{Stokes Complex 及应用}
Complex 1.
\begin{equation}\label{CA10}
	\resizebox{0.8\hsize}{!}{$
		\begin{array}{c}
			\xymatrix{	
\mathbb{R} \ar[r] & H_0^1(\Omega)\ar[r]^-{\grad} & \boldsymbol{H}_0(\curl, \Omega)  \ar[r]^-{\curl}
				& \boldsymbol{H}_0(\div, \Omega)    \ar[r]^-{\div} & L_0^2(\Omega) \ar[r]^{} & 0 ,
 }
		\end{array}
		$}
	\end{equation}

Example 1:The Possion equation
\begin{equation}\label{Cw1}
\left\{
\begin{aligned}
-\Delta u &=f,\quad \mathrm{in}\,\Omega \\
 u &=0, \quad \mathrm{on}\,\partial\Omega .
\end{aligned}\right.
\end{equation}
引入新变量
\begin{equation}\label{Cq}
\left\{
\begin{aligned}
\boldsymbol{\sigma} &=\nabla u \\
\div\boldsymbol{\sigma} &=-f.
\end{aligned}\right.
\end{equation}
则方程 (\ref{Cw1})
变为
\begin{equation}\label{Cw2}
\left\{
\begin{aligned}
\div\boldsymbol{\sigma}+f &=0,\quad \mathrm{in}\,\Omega\\
\boldsymbol{\sigma} -\nabla u&=0,\quad \mathrm{in}\,\Omega\\
 u &=0, \quad \mathrm{on}\,\partial\Omega.
\end{aligned}\right.
\end{equation}
在结合复形(\ref{CA10})的后半部分, 则方程 (\ref{Cw2})的变分形式为: $\boldsymbol{\tau}\in\boldsymbol{H}_0(\div, \Omega),\,u\in L_0^{2}(\Omega)$, 使得
\begin{equation}\label{Cp1}
\left\{
\begin{aligned}
(\boldsymbol{\sigma},\boldsymbol{\tau})+(\div\boldsymbol{\tau},u)&=0
  &&\forall \,\, \boldsymbol{\tau}\in \boldsymbol{H}_0(\div, \Omega),
 \\
 ( \div\boldsymbol{\sigma},v) &=0 &&\forall \,\, v\in L_0^{2}(\Omega),
\end{aligned}\right.
\end{equation}
下面利用复形(\ref{CA10})来证明 inf-sup 条件和强制性 .\\
(1) inf-sup 条件: 对 $v\in L_0^{2}(\Omega)$, 根据复形(\ref{CA10})可知 $ \exists\,\boldsymbol{\tau}\in \boldsymbol{H}_0(\div, \Omega)$,
使得
$\div\boldsymbol{\tau}=v$, 则有
\begin{align*}
\|v\|_{ 0} =\frac{(\div\boldsymbol{\tau},v)}{\|\div\boldsymbol{\tau}\|_{0}}
\lesssim\sup_{\boldsymbol{\tau}\in\boldsymbol{H}_0(\div, \Omega)}\frac{(\div\boldsymbol{\tau},u)}{\|\boldsymbol{\tau}\|_{\boldsymbol{H}(\div)}}
\end{align*}
(2) 强制性
\begin{align*}
\ker \boldsymbol{B}
&=\{\boldsymbol{\tau}\in \boldsymbol{H}_0(\div, \Omega),\,(\div\boldsymbol{\tau},v)=0,\quad\forall \, v\in L_0^{2}(\Omega)\}\\
&=\{\boldsymbol{\tau}\in \boldsymbol{H}(\div, \Omega),\,\div\boldsymbol{\tau}=0\}
\end{align*}
所以
\begin{align*}
\|\boldsymbol{\tau}\|_{\boldsymbol{H}(\div)}^2=\|\boldsymbol{\tau}\|_{0}^2+\|\div\boldsymbol{\tau}\|_{0}^2
\lesssim\|\boldsymbol{\tau}\|_{0}^2=(\boldsymbol{\tau},\boldsymbol{\tau}),\quad\forall \,\boldsymbol{\tau}\in \ker \boldsymbol{B}.
\end{align*}

Example 2:The Maxwell equation
\begin{equation}\label{Ck}
\left\{
\begin{aligned}
\curl\curl \boldsymbol{u} &=\boldsymbol{f},\quad \mathrm{in}\,\Omega \\
\div u &=0, \quad \mathrm{in}\,\Omega \\
\boldsymbol{u}\times \boldsymbol{n}&=0,\quad \mathrm{on}\,\partial\Omega .
\end{aligned}\right.
\end{equation}
则方程 (\ref{Ck})的变分形式为: $\boldsymbol{u}\in\boldsymbol{H}_0(\curl, \Omega),\,p\in H_0^{1}(\Omega)$,使得
\begin{equation}\label{Cr}
\left\{
\begin{aligned}
(\curl\boldsymbol{u},\curl\boldsymbol{v})+(\boldsymbol{v},\nabla\boldsymbol{p})&=(\boldsymbol{f},\boldsymbol{v})
  &&\forall \,\, \boldsymbol{v}\in \boldsymbol{H}_0(\curl, \Omega),
 \\
 ( \boldsymbol{u},\nabla\boldsymbol{q}) &=0 &&\forall \,\, q\in H_0^{1}(\Omega),
\end{aligned}\right.
\end{equation}
下面利用复形(\ref{CA10})来证明 inf-sup 条件和强制性 .\\
(1) inf-sup 条件: 对 $q\in H_0^{1}(\Omega)$, 根据复形(\ref{CA10})可知 $ \exists\,\boldsymbol{v}=\nabla q\in \boldsymbol{H}_0(\curl, \Omega)$,
又由 poinc\'{a}re 不等式可得
\begin{align*}
\|q\|_{ H_0^{1}(\Omega)}\lesssim |q|_{ H_0^{1}(\Omega)}=\frac{(\nabla q,\nabla q)}{\|\nabla q\|_{0}}
=\frac{(\boldsymbol{v},\nabla q)}{\|\boldsymbol{v}\|_{\boldsymbol{H}(\curl)}}\lesssim\sup_{\boldsymbol{v}\in\boldsymbol{H}_0(\curl, \Omega)}\frac{(\boldsymbol{v},\nabla q)}{\|\boldsymbol{v}\|_{\boldsymbol{H}(\curl)}}
\end{align*}
(2) 强制性
\begin{align*}
\ker \boldsymbol{B}
&=\{\boldsymbol{v}\in \boldsymbol{H}_0(\curl, \Omega),\,(\boldsymbol{v},\nabla q)=0,\quad\forall \, q\in H_0^{1}(\Omega)\}\\
&=\{\boldsymbol{v}\in \boldsymbol{H}(\curl, \Omega),\,\div\boldsymbol{v}=0\}
\end{align*}
又由 $\|\boldsymbol{v}\|_{0}\lesssim\|\curl\boldsymbol{v}\|_{0}$,
所以
\begin{align*}
\|\boldsymbol{v}\|_{\boldsymbol{H}(\curl)}^2=\|\boldsymbol{v}\|_{0}^2+\|\curl\boldsymbol{v}\|_{0}^2
\lesssim\|\curl\boldsymbol{v}\|_{0}^2,\quad\forall \,\boldsymbol{v}\in \ker \boldsymbol{B}.
\end{align*}

Complex 2.
\begin{equation}\label{CA1}
	\resizebox{0.8\hsize}{!}{$
		\begin{array}{c}
			\xymatrix{	
\mathbb{R} \ar[r] & H^{3}(\Omega)\ar[r]^-{\grad} & \boldsymbol{H}^{2}(\Omega;\mathbb{R}^{3})  \ar[r]^-{\curl}
				& \boldsymbol{H}^{1}(\Omega;\mathbb{R}^{3})    \ar[r]^-{\div} & L^2(\Omega) \ar[r]^{} & 0 .
 }
		\end{array}
		$}
	\end{equation}
Example 3:The Stokes equation
\begin{equation}\label{eq-Stokes1}
\left\{
\begin{aligned}
-\Delta \boldsymbol{u} - \nabla p &=\boldsymbol{f},\quad \mathrm{in}\,\Omega,\\
\div \boldsymbol{u}&=0, \quad \mathrm{in}\,\Omega\\
\boldsymbol{u}&=0,\quad \mathrm{on}\,\partial\Omega
\end{aligned}
\right.
\end{equation}
则方程 (\ref{eq-Stokes1})的变分形式为: $\boldsymbol{u}\in\boldsymbol{H}_{0}^1(\Omega,\mathbb{R}^{3}),\,p\in L_0^{2}(\Omega)$,使得
\begin{equation}\label{Cq1}
\left\{
\begin{aligned}
(\nabla\boldsymbol{u},\nabla\boldsymbol{v})+(\div\boldsymbol{v},\boldsymbol{p})&=(\boldsymbol{f},\boldsymbol{v})
  &&\forall \,\, \boldsymbol{v}\in \boldsymbol{H}_{0}^1(\Omega,\mathbb{R}^{3}),
 \\
 ( \div\boldsymbol{u},q) &=0 &&\forall \,\, q\in L_0^{2}(\Omega),
\end{aligned}\right.
\end{equation}
下面利用复形(\ref{CA1})来证明 inf-sup 条件和强制性 .\\
(1) inf-sup 条件: 对 $q\in   L_0^{2}(\Omega)$, 根据复形(\ref{CA1})可知 $ \exists\,\boldsymbol{v}\in \boldsymbol{H}_{0}^1(\Omega,\mathbb{R}^{3})$,
$\div \boldsymbol{v}=q$, 且由 poinc\'{a}re 不等式可得 $ \|\boldsymbol{v}\|_{1}\lesssim\|\div\boldsymbol{v}\|_{0}=\|q\|_{0}$, 则
$$(\div \boldsymbol{v},q)=\|q\|_{0}^2\gtrsim\|q\|_{0}\|\boldsymbol{v}\|_{1},$$
则有
\begin{align*}
\|q\|_{0}\lesssim\frac{(\div \boldsymbol{v},q)}{\|\boldsymbol{v}\|_{1}}\lesssim\sup_{\boldsymbol{v}\in\boldsymbol{H}_0(\Omega,\mathbb{R}^{3})}\frac{(\div \boldsymbol{v},q)}{\|\boldsymbol{v}\|_{1}}
\end{align*}
(2) 强制性
又由 poinc\'{a}re 不等式可得
\begin{align*}
\|\boldsymbol{v}\|_{1}^2
\lesssim(\nabla\boldsymbol{v},\nabla\boldsymbol{v}),\quad\forall \,\boldsymbol{v}\in \boldsymbol{H}_{0}^1(\Omega,\mathbb{R}^{3}).
\end{align*}

Complex 3.
\begin{equation}\label{CA3}
	\resizebox{0.8\hsize}{!}{$
		\begin{array}{c}
			\xymatrix{	
\mathbb{R} \ar[r] & H_0^{1}(\Omega)\ar[r]^-{\grad} & \boldsymbol{H}_{0}(\grad\curl,\Omega)  \ar[r]^-{\curl}
				& \boldsymbol{H}_{0}^{1}(\Omega;\mathbb{R}^{3})    \ar[r]^-{\div} & L_{0}^2(\Omega) \ar[r]^{} & 0 .
 }
		\end{array}
		$}
	\end{equation}
Example 4:The quad-curl problems
\begin{equation}\label{eq1}
\left\{
\begin{aligned}
-\curl\triangle\curl \boldsymbol{u} + \nabla p &=\boldsymbol{f},\quad \mathrm{in}\,\Omega,\\
\div \boldsymbol{u}&=0, \quad \mathrm{in}\,\Omega\\
\boldsymbol{u}\times \boldsymbol{n}=\curl\boldsymbol{u}&=0,\quad \mathrm{on}\,\partial\Omega
\end{aligned}
\right.
\end{equation}
则方程 (\ref{eq1})的变分形式为: $\boldsymbol{u}\in\boldsymbol{H}_{0}(\grad\curl,\Omega),\,p\in H_0^{1}(\Omega)$,使得
\begin{equation}\label{Cq1}
\left\{
\begin{aligned}
(\grad\curl\boldsymbol{u},\grad\curl\boldsymbol{v})+(\boldsymbol{v},\nabla p)&=(\boldsymbol{f},\boldsymbol{v})
  &&\forall \,\, \boldsymbol{v}\in \boldsymbol{H}_{0}(\grad\curl,\Omega),
 \\
 ( \boldsymbol{u},\nabla  q) &=0 &&\forall \,\, q\in H_0^{1}(\Omega),
\end{aligned}\right.
\end{equation}
下面利用复形(\ref{CA3})来证明 inf-sup 条件和强制性 .\\
(1) inf-sup 条件: 对 $q\in H_0^{1}(\Omega)$, 根据复形(\ref{CA3})可知 $ \exists\,\boldsymbol{v}=\nabla q\in \boldsymbol{H}_{0}(\grad\curl,\Omega)$,
又由 poinc\'{a}re 不等式可得
\begin{align*}
\|q\|_{ H_0^{1}(\Omega)}\lesssim |q|_{ H_0^{1}(\Omega)}=\frac{(\nabla q,\nabla q)}{\|\nabla q\|_{0}}
=\frac{(\boldsymbol{v},\nabla q)}{\|\boldsymbol{v}\|_{\boldsymbol{H}(\grad\curl)}}\lesssim\sup_{\boldsymbol{v}\in\boldsymbol{H}_0(\grad\curl, \Omega)}\frac{(\boldsymbol{v},\nabla q)}{\|\boldsymbol{v}\|_{\boldsymbol{H}(\grad\curl)}}
\end{align*}
(2) 强制性
\begin{align*}
\ker \boldsymbol{B}
&=\{\boldsymbol{v}\in \boldsymbol{H}_0(\grad\curl, \Omega),\,(\boldsymbol{v},\nabla q)=0,\quad\forall \, q\in H_0^{1}(\Omega)\}\\
&=\{\boldsymbol{v}\in \boldsymbol{H}(\grad\curl, \Omega),\,\div\boldsymbol{v}=0\}
\end{align*}
又由 poinc\'{a}re 不等式可得 $\|\boldsymbol{v}\|_{0}^{2}+\|\curl\boldsymbol{v}\|_{0}^{2}\lesssim\|\curl\boldsymbol{v}\|_{0}^2\lesssim\|\grad\curl\boldsymbol{v}\|_{0}$,
所以
\begin{align*}
\|\boldsymbol{v}\|_{\boldsymbol{H}(\grad\curl)}^2=\|\boldsymbol{v}\|_{0}^2+\|\curl\boldsymbol{v}\|_{0}^2++\|\grad\curl\boldsymbol{v}\|_{0}^2
\lesssim\|\grad\curl\boldsymbol{v}\|_{0}^2,\quad\forall \,\boldsymbol{v}\in \ker \boldsymbol{B}.
\end{align*}
Complex 4.
 \begin{equation}\label{CA4}
	\resizebox{0.8\hsize}{!}{$
		\begin{array}{c}
			\xymatrix{	
\mathbb{R} \ar[r] & H^{1}(\Omega)\ar[r]^-{\grad} & \boldsymbol{H}(\mathrm{Hess}\curl,\Omega)  \ar[r]^-{\curl}
				& \boldsymbol{H}^{2}(\Omega;\mathbb{R}^{3})    \ar[r]^-{\div} & H^{1}(\Omega) \ar[r]^{} & 0 .
 }
		\end{array}
		$}
	\end{equation}
Example 5 :The strain gradient problems\\
这里引入应变梯度$\varepsilon(\boldsymbol{u})=\sym\grad\boldsymbol{u}$
则应变梯度问题的变分形式为: $\boldsymbol{u}\in\boldsymbol{H}_{0}^2(\Omega;\mathbb{R}^{3}),\,p\in H_0^{1}(\Omega)$,使得
\begin{equation}\label{Cq10}
\left\{
\begin{aligned}
(\varepsilon(\boldsymbol{u}),\varepsilon(\boldsymbol{v}))+(\div\boldsymbol{v},p)&=(\boldsymbol{f},\boldsymbol{v})
  &&\forall \,\, \boldsymbol{v}\in \boldsymbol{H}_{0}^2(\Omega;\mathbb{R}^{3}),
 \\
 ( \div\boldsymbol{u},\boldsymbol{q}) &=0 &&\forall \,\, q\in H_0^{1}(\Omega),
\end{aligned}\right.
\end{equation}
进一步改写成带参数 $l$ 的形式
\begin{equation}\label{Cq11}
\left\{
\begin{aligned}
(\varepsilon(\boldsymbol{u}),\varepsilon(\boldsymbol{v}))_{l}+(\div\boldsymbol{v},p)_{l}&=(\boldsymbol{f},\boldsymbol{v})
  &&\forall \,\, \boldsymbol{v}\in \boldsymbol{H}_{0}^2(\Omega;\mathbb{R}^{3}),
 \\
 ( \div\boldsymbol{u},\boldsymbol{q})_{l} &=0 &&\forall \,\, q\in H_0^{1}(\Omega),
\end{aligned}\right.
\end{equation}
这里
$$(p,q)_l=l^2(\nabla p,\nabla q)+(p,q).$$
还是下面利用复形(\ref{CA4})来证明 inf-sup 条件和强制性, 证明方法和上面类似.

Complex 5.
 \begin{equation}\label{CA2}
	\resizebox{0.8\hsize}{!}{$
		\begin{array}{c}
			\xymatrix{	
\mathbb{R} \ar[r] & H^{2}(\Omega)\ar[r]^-{\grad} & \boldsymbol{H}^{1}(\curl,\Omega)  \ar[r]^-{\curl}
				& \boldsymbol{H}^{1}(\Omega;\mathbb{R}^{3})    \ar[r]^-{\div} & L^2(\Omega) \ar[r]^{} & 0 .
 }
		\end{array}
		$}
	\end{equation}
Example 5 :The biharmonic equation
\begin{equation}
\left\{
\begin{aligned}
\Delta^{2}u&=f\qquad \,\,\,\mathrm{in}\,\,\Omega,\\
u=\partial_{\upsilon}u&=0 \quad \mathrm{on}\,\, \partial\Omega
\end{aligned}%
\right.   \label{eqtwo}
\end{equation}%
原始变分形式为
找 $u\in H_{0}^{2}(\Omega)$
 使得
\begin{align}\label{AA0}
(\nabla^{2}u,\nabla^{2}v)=\langle f,v\rangle \quad \forall\,\, v\in H_{0}^{2}(\Omega)
\end{align}