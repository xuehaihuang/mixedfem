
设$\Omega=(0,1)^2$为平面上的单位矩形. 考虑Poisson方程
$$
\begin{cases}
-\Delta u=f\quad  \textrm{在 }\, \Omega \textrm{ 内},\\
u|_{\partial \Omega}=0.
\end{cases}
$$
设真解 $u=\sin(\pi x)\sin(\pi y)$.

在一致网格剖分上分别用
%线性Lagrange元, 二次Lagrange元和非协调线性元
线性 BDM 元和二次 BDM 元方法
数值求解上述Poisson方程, 进行数值模拟, 列表给出误差 $\|u-u_h\|$, $\|Q_hu-u_h\|$, $\|\nabla_h(u-u_h)\|$, $\|\nabla_h(Q_hu-u_h)\|$ 和 $\|\boldsymbol{\sigma}-\boldsymbol{\sigma}_h\|$ 关于 $h$ 的收敛阶,
并画出这些误差关于 $h$ 的收敛阶的图.

\vskip1cm
\noindent{\bf 注}: 
% 提交电子版, 截止时间:12月31日.

\begin{enumerate}[(1)]
\item 报告组成: 原理分析, 程序代码, 数值实验结果及其分析总结.
\item 可以采用有限元软件包编写, 比如 iFEM (\href{https://www.math.uci.edu/~chenlong/programming.html}{https://www.math.uci.edu/\textasciitilde chenlong/programming.html}), MFEM (\href{https://mfem.org}{https://mfem.org}), FEALPy (\href{https://github.com/weihuayi/fealpy}{https://github.com/weihuayi/fealpy}); 也可以完全自己编写.

\item 提交电子版, 截止时间:12月31日.


\end{enumerate}



%\vskip1cm
%\noindent{\bf 实验报告要求}
%\begin{enumerate}[(1)]
%\item 报告组成: 原理分析, 算法设计, 程序代码, 数值实验结果及其分析总结.
%
%\item 打印上交, 上交时间为12周周三.
%\end{enumerate}
