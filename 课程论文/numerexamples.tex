%!TEX program=xelatex
%\documentclass[11pt,CJK]{book}
%\documentclass[11pt,a4paper]{book}
\documentclass{book}
\usepackage{beamerarticle}



\usepackage{fontspec}
%\setCJKmainfont{Songti SC Regular}
%\setmainfont{Kaiti SC Regular}
\setsansfont{Songti SC Regular}
\setmainfont{Songti SC Regular}
%\setmainfont{Times New Roman}
\usepackage{ctexcap}
\usepackage{enumerate}

%\usepackage{beamerthemesplit}   % 使用split风格

%\usepackage[colorlinks,bookmarksnumbered=true,linkcolor=black,citecolor=black]{hyperref}
\usepackage{hyperref}
\hypersetup{
    colorlinks=true,
    linkcolor=blue,
    filecolor=blue,      
    urlcolor=blue,
    citecolor=cyan,
}
\usepackage{amsmath,amssymb,amsfonts}
\usepackage{color,xcolor}
\usepackage{graphicx}
\usepackage{manfnt}

\usepackage{multicol,multienum}%多栏结构在文中用
\usepackage{multirow}
\usepackage{indentfirst} % 中文段落首行缩进


\usepackage{mathdots} % for \iddots
\usepackage{extarrows} % for \xlongequal

%\usepackage{pmat} % 分块矩阵

\usepackage{slashbox} % for \backslashbox
\usepackage{cite}

\usepackage{galois} % for \comp 复合函数小圆圈

\usepackage{wasysym} % for permil 千分之一符号

\usepackage{arcs}

%\usepackage{mnsymbol}

\usepackage{stmaryrd}

\usepackage[all]{xy}

%\usepackage{algorithm,algorithmic}
\usepackage{algorithm, algorithmicx, algpseudocode}


%\usepackage{ccmap}% 复制的中文不乱码

\usepackage{chemarrow}

%\usetikzlibrary{matrix}

\usepackage{pgf, tikz}
\usetikzlibrary{arrows, automata}
\usepackage{eso-pic}


%\usepackage{titletoc}
\usepackage{titlesec}

%%%%%%%%%%%%%%%%%%%%%%%%%%%%%%%%%%%%%%%%%%%%%%%%%%%%%%%%%%%%%%%%%%%%%%%%%%%%%%%%%%%%%%%%%%%%%%%%%%%%%%%%%
%                                           定制幻灯片---重定义字体、字号命令                           %
%%%%%%%%%%%%%%%%%%%%%%%%%%%%%%%%%%%%%%%%%%%%%%%%%%%%%%%%%%%%%%%%%%%%%%%%%%%%%%%%%%%%%%%%%%%%%%%%%%%%%%%%%
\newcommand{\song}{\CJKfamily{song}}    % 宋体   (Windows自带simsun.ttf)
\newcommand{\fs}{\CJKfamily{fs}}        % 仿宋体 (Windows自带simfs.ttf)
\newcommand{\kai}{\CJKfamily{kai}}      % 楷体   (Windows自带simkai.ttf)
\newcommand{\hei}{\CJKfamily{hei}}      % 黑体   (Windows自带simhei.ttf)
\newcommand{\li}{\CJKfamily{li}}        % 隶书   (Windows自带simli.ttf)
\newcommand{\you}{\CJKfamily{you}}      % 幼圆   (Windows自带simyou.ttf)
\newcommand{\chuhao}{\fontsize{42pt}{\baselineskip}\selectfont}     % 字号设置
\newcommand{\xiaochuhao}{\fontsize{36pt}{\baselineskip}\selectfont} % 字号设置
\newcommand{\yichu}{\fontsize{32pt}{\baselineskip}\selectfont}      % 字号设置
\newcommand{\yihao}{\fontsize{28pt}{\baselineskip}\selectfont}      % 字号设置
\newcommand{\erhao}{\fontsize{21pt}{\baselineskip}\selectfont}      % 字号设置
\newcommand{\xiaoerhao}{\fontsize{18pt}{\baselineskip}\selectfont}  % 字号设置
\newcommand{\sanhao}{\fontsize{15.75pt}{\baselineskip}\selectfont}  % 字号设置
\newcommand{\sihao}{\fontsize{14pt}{\baselineskip}\selectfont}      % 字号设置
\newcommand{\xiaosihao}{\fontsize{12pt}{\baselineskip}\selectfont}  % 字号设置
\newcommand{\wuhao}{\fontsize{10.5pt}{\baselineskip}\selectfont}    % 字号设置
\newcommand{\xiaowuhao}{\fontsize{9pt}{\baselineskip}\selectfont}   % 字号设置
\newcommand{\liuhao}{\fontsize{7.875pt}{\baselineskip}\selectfont}  % 字号设置
\newcommand{\qihao}{\fontsize{5.25pt}{\baselineskip}\selectfont}    % 字号设置


\newcommand{\dx}{\,{\rm d}x}
\newcommand{\dd}{\,{\rm d}}
\newcommand{\bs}{\boldsymbol}
\newcommand{\mcal}{\mathcal}
\newcommand{\dint}{\displaystyle\int} 

\DeclareMathOperator*{\img}{img}
%\DeclareMathOperator*{\span}{span}
\newcommand{\curl}{\operatorname{curl}}
\renewcommand{\div}{\operatorname{div}}
%\renewcommand{\grad}{\operatorname{grad}}
\newcommand{\grad}{\operatorname{grad}}
\DeclareMathOperator*{\tr}{tr}
\DeclareMathOperator*{\rot}{rot}
\newcommand{\var}{\operatorname{Var}}
\DeclareMathOperator*{\cov}{Cov}
\newcommand{\dev}{\operatorname{dev}}
\newcommand{\sym}{\operatorname{sym}}
\newcommand{\skw}{\operatorname{skw}}
\newcommand{\spn}{\operatorname{spn}}



%% 定义一些自选的模板,包括背景、图标、导航条和页脚等,修改要慎重
%\beamertemplateshadingbackground{red!10}{structure!10}
%\beamertemplatesolidbackgroundcolor{white!90!blue}
%\beamertemplatetransparentcovereddynamic \beamertemplateballitem
%\beamertemplatenumberedballsectiontoc

\setbeamertemplate{theorems}[numbered]


\pagestyle{plain} \topmargin -10mm \oddsidemargin 0mm
\evensidemargin 0mm \textheight 225mm \textwidth 155mm
\pagestyle{plain}
\parindent 8mm
\parskip 0mm
\hyphenpenalty=1500 \flushbottom

\begin{document}

%======================= 标题名称中文化 ============================%
\newtheorem{dingyi}{\hei 定义~}[section]
\newtheorem{dingli}{\hei 定理~}[section]
\newtheorem{yinli}[dingli]{\hei 引理~}
\newtheorem{tuilun}[dingli]{\hei 推论~}
\newtheorem{mingti}[dingli]{\hei 命题~}

\newtheorem{defn}{\hei 定义~}[section]
\newtheorem{thm}{\hei 定理~}[section]
\newtheorem{lem}[thm]{\hei 引理~}
\newtheorem{cor}[thm]{\hei 推论~}
\newtheorem{proposition}[thm]{\hei 命题~}
\newtheorem{property}[thm]{\hei 性质~}
\newtheorem{remark}[thm]{\hei 注~}
\newtheorem{exm}{\hei 例~}[section]
\newtheorem{exmex}{\hei 例~}
\newenvironment{sln}{\par\noindent\textbf{解}:}{}
\newenvironment{prf}{\par\noindent\textbf{证明}:}{}
\newtheorem{lianxi}{练习~}%[section]
\renewcommand{\proofname}{证明}


% \titlecontents{section}[2em]{\vspace{.25\baselineskip}}
%     {\S\thecontentslabel\quad}{}
%     {\hspace{.5em}\titlerule*[10pt]{$\cdot$}\contentspage}

%\titleformat{\chapter}{\raggedright\Huge\bfseries}{第\,\thechapter\,章}{1em}{}
% \titleformat{\chapter}{\raggedright\Huge\bfseries}{数值实验\,\thechapter\,}{1em}{}
\titleformat{\chapter}{\raggedright\Huge\bfseries}{}{1em}{}

\makeatletter
\renewcommand\theequation{\thesection.\arabic{equation}}
\@addtoreset{equation}{section}
\makeatother

\renewcommand{\thedefn}{\arabic{section}.\arabic{defn}}
\renewcommand{\thethm}{\arabic{section}.\arabic{thm}}
\renewcommand{\thelem}{\arabic{section}.\arabic{lem}}
\renewcommand{\thecor}{\arabic{section}.\arabic{cor}}
\renewcommand{\theproposition}{\arabic{section}.\arabic{proposition}}
\renewcommand{\theexm}{\arabic{section}.\arabic{exm}}
\renewcommand{\theremark}{\arabic{section}.\arabic{remark}}



%%%%%%%%%%%%%%%%%%%%%%%%%%%%%%%%%%%%%%%%%%%%%%%%%%%%%%%
% 选择题选项对齐格式的自定义
%%%%%%%%%%%%%%%%%%%%%%%%%%%%%%%%%%%%%%%%%%%%%%%%%%%%%%%%
%分一行.
\newcommand{\onech}[4]{
\makebox[92pt][l]{(A) #1} \hfill
\makebox[92pt][l]{(B) #2} \hfill
\makebox[92pt][l]{(C) #3} \hfill
\makebox[92pt][l]{(D) #4}\\}
%分两行.
\newcommand{\twoch}[4]{
\indent\makebox[110pt][l]{\qquad(A) #1} \hfill\makebox[220pt][l]{(B) #2}\\
\indent\makebox[110pt][l]{\qquad(C) #3} \hfill\makebox[220pt][l]{(D) #4}\\}
%分四行.
\newcommand{\fourch}[4]{
\indent\makebox[262pt][l]{(A) #1} \newline
\indent\makebox[262pt][l]{(B) #2} \newline
\indent\makebox[262pt][l]{(C) #3} \newline
\indent\makebox[262pt][l]{(D) #4} \newline}

\def\comp{\ensuremath\mathop{\scalebox{.6}{$\circ$}}}
\newcommand\arccot{\mathop{\mathrm{arccot}}}
\newcommand\sgn{\mathop{\mathrm{sgn}}}
\newenvironment{xenumerate}[1][]
{\enumerate[(1)]}{\endenumerate}



%-------------------------------------------------------------------
\newcommand\BackgroundPicture{%
\put(0,0){%
    \parbox[b][\paperheight]{\paperwidth}{%
      \vfill
      \centering%
\begin{tikzpicture}[remember picture,overlay]
\node [rotate=60,scale=4,text opacity=0.1] at (-2,+2) {黄学海};
\node [rotate=60,scale=4,text opacity=0.1] at (current page.center) {上海财经大学数学学院};
%\node [rotate=60,scale=4,text opacity=0.06] at (-2,+2) {黄学海};
%\node [rotate=60,scale=4,text opacity=0.06] at (current page.center) {温州大学数学与信息科学学院};
\end{tikzpicture}%
      \vfill
    }}}
%-------------------------------------------------------------------




\title{混合有限元方法数值实验}
\author{黄学海  \\{\small 上海财经大学数学学院}
\\{\small E-mail: huang.xuehai@sufe.edu.cn}
\\
\\
\\
\\
%2020-2021学年秋\\%16信息与计算
 }

\date{}
\vskip 0.5cm
\maketitle


\cleardoublepage
%\renewcommand\contentsname{目录}
\pdfbookmark[1]{目录}{anchor name}
\setcounter{tocdepth}{2}
\tableofcontents


%\AddToShipoutPicture{\BackgroundPicture}



\phantomsection
%\chapter{习题~\thechapter}
\chapter{数值试验}

设$\Omega=(0,1)^2$为平面上的单位矩形. 考虑Poisson方程
$$
\begin{cases}
-\Delta u=f\quad  \textrm{在 }\, \Omega \textrm{ 内},\\
u|_{\partial \Omega}=0.
\end{cases}
$$
设真解 $u=\sin(\pi x)\sin(\pi y)$.

在一致网格剖分上分别用
%线性Lagrange元, 二次Lagrange元和非协调线性元
线性 BDM 元和二次 BDM 元方法
数值求解上述Poisson方程, 进行数值模拟, 列表给出误差 $\|u-u_h\|$, $\|Q_hu-u_h\|$, $\|\nabla_h(u-u_h)\|$, $\|\nabla_h(Q_hu-u_h)\|$ 和 $\|\boldsymbol{\sigma}-\boldsymbol{\sigma}_h\|$ 关于 $h$ 的收敛阶,
并画出这些误差关于 $h$ 的收敛阶的图.

\vskip1cm
\noindent{\bf 注}: 
% 提交电子版, 截止时间:12月31日.

\begin{enumerate}[(1)]
\item 报告组成: 原理分析, 程序代码, 数值实验结果及其分析总结.
\item 可以采用有限元软件包编写, 比如 iFEM (https://www.math.uci.edu/~chenlong/programming.html), MFEM (https://mfem.org/), FEALPy (https://github.com/weihuayi/fealpy); 也可以完全自己编写.

\item 提交电子版, 截止时间:12月31日.
\end{enumerate}



%\vskip1cm
%\noindent{\bf 实验报告要求}
%\begin{enumerate}[(1)]
%\item 报告组成: 原理分析, 算法设计, 程序代码, 数值实验结果及其分析总结.
%
%\item 打印上交, 上交时间为12周周三.
%\end{enumerate}

\cleardoublepage

% \phantomsection
% %\chapter{习题~\thechapter}
% \chapter{课程论文}
% \input{coursepaperfem.tex}
% \cleardoublepage


% \phantomsection
% %\chapter{习题~\thechapter}
% \chapter{函数插值}
% \input{interpolation1.tex}
% \cleardoublepage

% \phantomsection
% %\chapter{习题~\thechapter}
% \chapter{快速傅立叶变换(FFT)}
% % \input{interpolationrunge.tex}
% \input{fft.tex}
% \cleardoublepage

% \phantomsection
% %\chapter{习题~\thechapter}
% \chapter{最佳一致逼近多项式算法}
% % \input{interpolationrunge.tex}
% \input{approximationmaxnorm.tex}
% \cleardoublepage


% \phantomsection
% %\chapter{习题~\thechapter}
% \chapter{曲线拟合}
% \input{fittingrunge.tex}
% \cleardoublepage


% \phantomsection
% %\chapter{习题~\thechapter}
% \chapter{数值积分}
% \input{quadrature.tex}
% \cleardoublepage

%\pdfbookmark[0]{参考文献}{reference}
%\nocite{*}

% \phantomsection
% \addcontentsline{toc}{chapter}{参考文献}
% \bibliographystyle{abbrv}
% \bibliography{refs}
% \cleardoublepage





\end{document}
