%!TEX program = xelatex
%CITATA{OutputFilter=latex2.dll}
%CITATA{Version=5.00.0.2552}
%CITATA{CSTFile=LaTeX article (bright).cst}
%CITATA{Created=Sunday, January 09, 2005 14:34:49}
%CITATA{LastRevised=Tuesday, February 01, 2005 13:57:41}
%CITATA{<META NAME="Graphicness" CONTENT="32">}
%CITATA{<META NAME="Salverform" CONTENT="1">}
%CITATA{<META NAME="Documentable" CONTENT="Standard LaTeX\Blank - Standard LaTeX Article">}

\documentclass[11pt]{book}
\usepackage{amsfonts}
\usepackage{latexsym,amssymb,amsmath}
%\usepackage{amsthm}
\usepackage[amsmath,thmmarks]{ntheorem}
\usepackage{bm}
\usepackage{graphicx}
\usepackage{subfigure}
\usepackage{indentfirst}
\usepackage{cite}
\usepackage{enumerate}
\usepackage{multicol,multienum}%多栏结构在文中用
\usepackage{mathrsfs}
\usepackage{extarrows} % 长等号上方或下方可输文本
\usepackage{verbatim}
\usepackage{overpic}
\usepackage{caption2}

\usepackage{fontspec}
%\setCJKmainfont{Songti SC Regular}
%\setmainfont{Kaiti SC Regular}
\setsansfont{Songti SC Regular}
\setmainfont{Songti SC Regular}
%\setmainfont{Times New Roman}
\usepackage{ctexcap}

%\usepackage[colorlinks,bookmarksnumbered=true,linkcolor=black,citecolor=black]{hyperref}
\usepackage{hyperref}
\hypersetup{
    colorlinks=true,
    linkcolor=blue,
    filecolor=blue,      
    urlcolor=blue,
    citecolor=cyan,
}

\usepackage{tikz}
\usepackage{eso-pic}
\usepackage{titletoc,titlesec}%章节标题设置
\usepackage{fancyhdr}%定义页眉和页脚 使用fancyhdr 宏包
\renewcommand{\captionlabeldelim}{}
\setcounter{MaxMatrixCols}{30}
%TCIDATA{OutputFilter=latex2.dll}
%TCIDATA{Version=5.00.0.2552}
%TCIDATA{LastRevised=Monday, June 13, 2005 13:13:31}
%TCIDATA{<META NAME="GraphicsSave" CONTENT="32">}
%TCIDATA{<META NAME="Salverform" CONTENT="1">}
\pagestyle{plain} \topmargin -10mm \oddsidemargin 0mm
\evensidemargin 0mm \textheight 225mm \textwidth 155mm
\pagestyle{plain}
\parindent 8mm
\parskip 0mm
\hyphenpenalty=1500 \flushbottom

\newcommand{\dx}{\,{\rm d}x}
\newcommand{\dd}{\,{\rm d}}
\newcommand{\bs}{\boldsymbol}
\newcommand{\mcal}{\mathcal}

\DeclareMathOperator*{\img}{img}
%\DeclareMathOperator*{\span}{span}
\newcommand{\curl}{\operatorname{curl}}
\renewcommand{\div}{\operatorname{div}}
%\renewcommand{\grad}{\operatorname{grad}}
\newcommand{\grad}{\operatorname{grad}}
\DeclareMathOperator*{\tr}{tr}
\DeclareMathOperator*{\rot}{rot}
\DeclareMathOperator*{\var}{Var}
\DeclareMathOperator*{\cov}{Cov}
\newcommand{\dev}{\operatorname{dev}}
\newcommand{\sym}{\operatorname{sym}}
\newcommand{\skw}{\operatorname{skw}}
\newcommand{\spn}{\operatorname{spn}}


\begin{document}
%\titleformat{\chapter}[hang]{\centering\large\sf}{\chaptername}{1em}{}
\titleformat{\chapter}[hang]{\LARGE\bfseries\center}{\chaptertitlename}{10pt}{}
\renewcommand{\chaptername}{\thechapter}
%\renewcommand{\chaptername}{第\thechapter 章}
%\renewcommand{\chaptername}{上机实验\CJKnumber{\thechapter}}
%\renewcommand{\chaptername}{Homework~\thechapter}
\renewcommand{\chaptername}{}
\renewcommand{\figurename}{Fig.}
\renewcommand{\tablename}{表}
\newtheorem{thm}{定理}
\newtheorem{remark}{注}[section]
\newtheorem{corollary}{Corollary}
\newtheorem{lem}{引理}[section]
\newtheorem{col}[lem]{推论}
%\newtheorem{lem}[thm]{Lemma}
%\newtheorem{example}{Example}[section]
\theoremstyle{plain}
\theoremseparator{}
\newtheorem{defn}{定义}
\newtheorem{exm}{例}[chapter]
\newtheorem{hardexm}[exm]{例$^{\star}$}
\numberwithin{equation}{section}

\newenvironment{tad}{\par\noindent\textbf{教学目的与要求}}{}
\newenvironment{idtp}{\par\noindent\textbf{教学重点、难点}}{}
%\newenvironment{exm}{\par\noindent\textbf{例}}{}
\newenvironment{sln}{\par\noindent\textbf{解:}}{}
\newenvironment{prf}{\par\noindent\textbf{证明:}}{}


% 缩小目录中各级标题之间的缩进
%\dottedcontents{part}[2em]{\CJKfamily{hei}\large}{4em}{1pc}
%\dottedcontents{chapter}[2.0em]{\CJKfamily{song}\normalsize}{4em}{1pc}
%\dottedcontents{chapter}[0em]{}{0em}{1pc}
%\dottedcontents{chapter}[0.0em]{\CJKfamily{song} \bf \vspace{0.5em}}{1.0em}{5pt}
%\makeatletter
%\renewcommand*\l@chapter{\@dottedtocline{1}{0em}{1.5em}}
%\makeatother
\titlecontents{chapter}[3.5em]{\CJKfamily{song} \bf}
{\contentslabel{4.5em}}{}
{\titlerule*[0.8pc]{.}\contentspage}[\addvspace{0.5ex}]
%\titlecontents{chapter}[2.8em]{\color{blue}\CJKfamily{song} \bf
%\addvspace{1.5ex}}
%{\contentslabel{2em}\hspace*{-1.0em}}{\hspace*{-2.3em}}
%{\color{black}\titlerule*[0.8pc]{.}\contentspage}[\addvspace{0.5ex}]




%-------------------------------------------------------------------
\newcommand\BackgroundPicture{%
\put(0,0){%
    \parbox[b][\paperheight]{\paperwidth}{%
      \vfill
      \centering%
\begin{tikzpicture}[remember picture,overlay]
\node [rotate=60,scale=4,text opacity=0.1] at (-2,+2) {混合有限元方法习题};
\node [rotate=60,scale=4,text opacity=0.1] at (current page.center) {上海财经大学数学学院};
\end{tikzpicture}%
      \vfill
    }}}
%-------------------------------------------------------------------






%%%%%%%%%%%%%%%%%%%%%%%%%%%%%%%%%%%%%%%%%%%%%%%%%%%%%%%
% 定义页眉和页脚 使用fancyhdr 宏包
%%%%%%%%%%%%%%%%%%%%%%%%%%%%%%%%%%%%%%%%%%%%%%%%%%%%%%%%

\newcommand{\makeheadrule}{%
\makebox[0pt][l]{\rule[.7\baselineskip]
{\headwidth}{\headrulewidth}}%


% 1 Line Modified by Lei Wang BaconChina
% XJTU Version
%    \rule[.6\baselineskip]{\headwidth}{0.4pt}\vskip-.8\baselineskip}
% THU Version
    \vskip-.8\baselineskip}

\makeatletter
\renewcommand{\headrule}{%
    {\if@fancyplain\let\headrulewidth\plainheadrulewidth\fi
     \makeheadrule}}

\pagestyle{fancyplain}

%去掉章节标题中的数字
%\renewcommand{\chaptermark}[1]{\markboth{第\chaptername 章 \ #1}{}}
\renewcommand{\chaptermark}[1]{\markboth{\chaptername \ #1}{}}
 \fancyhf{}

%在book文件类别下,\leftmark自动存录各章之章名,\rightmark记录节标题

%\fancyhead[CO]{\CJKfamily{kai} \fontsize{12pt}{12pt} \bfseries\leftmark}
\fancyhead[CO]{\CJKfamily{kai} \bfseries \leftmark}
\fancyhead[CE]{\CJKfamily{kai} \bfseries 混合有限元方法}
\fancyfoot[C,C]{--~\thepage~--}
%\fancyfoot[R]{Copyright \copyright 2010-2011 WZU.\\ All rights reserved.}

\fancypagestyle{plain}{%
\fancyhead{} % clear all header fields
\renewcommand{\headrulewidth}{0pt}
}




%%%%%%%%%%%%%%%%%%%%%%%%%%%%%%%%%%%%%%%%%%%%%%%%%%%%%%%
% 选择题选项对齐格式的自定义
%%%%%%%%%%%%%%%%%%%%%%%%%%%%%%%%%%%%%%%%%%%%%%%%%%%%%%%%
%分一行.
\newcommand{\onech}[4]{
\makebox[92pt][l]{(A) #1} \hfill
\makebox[92pt][l]{(B) #2} \hfill
\makebox[92pt][l]{(C) #3} \hfill
\makebox[92pt][l]{(D) #4}\\}
%分两行.
\newcommand{\twoch}[4]{
\indent\makebox[110pt][l]{\qquad(A) #1} \hfill\makebox[220pt][l]{(B) #2}\\
\indent\makebox[110pt][l]{\qquad(C) #3} \hfill\makebox[220pt][l]{(D) #4}\\}
%分四行.
\newcommand{\fourch}[4]{
\indent\makebox[262pt][l]{\qquad(A) #1}\\
\indent\makebox[262pt][l]{\qquad(B) #2}\\
\indent\makebox[262pt][l]{\qquad(C) #3}\\
\indent\makebox[262pt][l]{\qquad(D) #4}\\}


%%%%%%%%%%%%%%%%%%%%%%%%%%%%%%%%%%%%%%%%%%%%%%%%%%%%%%%
% zenumerate可以与table环境混排
%%%%%%%%%%%%%%%%%%%%%%%%%%%%%%%%%%%%%%%%%%%%%%%%%%%%%%%%
\newenvironment{zenumerate}{\newcounter{zitem}\setcounter{zitem}{0}}{}
\newcommand\zitem{\refstepcounter{zitem}(\thezitem) }
%\newcommand\zitem{\refstepcounter{zitem}\thezitem. }

\newenvironment{xenumerate}[1][]
{\enumerate[(1)]}{\endenumerate}




\title{混合有限元方法习题}
\author{黄学海  \\{\small 上海财经大学数学学院}
\\{\small E-mail: huang.xuehai@sufe.edu.cn }
\\
\\
\\
\\
%2010年秋\quad 10教育技术师范班
 }

\date{}
\vskip 0.5cm
\maketitle


\cleardoublepage
\renewcommand\contentsname{目录}
\pdfbookmark[1]{目录}{anchor name}
\tableofcontents

%\addtocontents{toc}{\addhspace{100pt}}
%\addcontentsline{toc}{chapter}



% \AddToShipoutPicture{\BackgroundPicture}



% \phantomsection
% \chapter{椭圆型方程的有限差分法}
% \input{ellipticfdm.tex}
% \cleardoublepage


\phantomsection
\chapter{混合变分问题习题}
% !TEX root = problems.tex
\begin{enumerate}
\item 
写出有界区域 $\Omega\subset\mathbb R^d$ 上 Darcy 定律问题
$$
\begin{cases}    
\boldsymbol{\sigma}=-K\nabla u, \quad \div\boldsymbol{\sigma}=f, \quad \textrm{ 在 $\Omega$ 内}, \\
u|_{\partial\Omega}=0
\end{cases}
$$
的混合变分形式, 其中 $K$ 是可逆矩阵.
\end{enumerate}


\cleardoublepage


\phantomsection
\chapter{有限元 de Rham 复形习题}
% !TEX root = problems.tex
\begin{enumerate}
\item 写出二维 de Rham 复形的 Poincar\'e算子,并参照讲义中与三维Poincar\'e算子有关的理论结果写出相应的二维理论结果,并证明.
\end{enumerate}


\cleardoublepage

\phantomsection
\chapter{Poisson 方程混合元方法习题}
% !TEX root = problems.tex
\begin{enumerate}
\item 设 $k=1$, $(\boldsymbol{\sigma}_h, u_h)$ 是 BDM 元方法的解. 利用对偶论证推导 $\|Q_hu-u_h\|$ 的误差估计.
\end{enumerate}


\cleardoublepage


% \phantomsection
% \chapter{数值实验一}
% \input{numerexperiment1.tex}
% \cleardoublepage



%\newpage
%\ClearShipoutPicture


\phantomsection
\addcontentsline{toc}{chapter}{参考文献}
\bibliographystyle{abbrv}
\bibliography{refs}
\cleardoublepage

\end{document}
